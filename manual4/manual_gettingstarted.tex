
\chapter{Getting Started}

\section{What is BayesX?}

\BayesX is a software tool for estimating structured additive regression models. Structured additive regression embraces several well-known regression models such as generalized additive models (GAM), generalized additive mixed models (GAMM), generalized geoadditive mixed models (GGAMM), dynamic models, varying coefficient models, and geographically weighted regression within a unifying framework. Besides exponential family regression, BayesX also supports non-standard regression situations such as regression for categorical responses, hazard regression for continuous survival times, continuous time multi-state models, quantile regression, distributional regression models and multilevel models.

\subsection{Inferential procedures}

Estimation of regression models can be achieved based on four different inferential procedures (implemented in different regression objects):
\begin{itemize}
 \item MCMC simulation techniques (mcmcreg objects). A fully Bayesian interpretation of structured additive regression models is obtained by specifying prior distributions for all unknown parameters. Estimation can then be facilitated using Markov chain Monte Carlo simulation techniques. Mcmcreg objects provide numerically efficient implementations of MCMC schemes for structured additive regression models in case of exponential family responses, categorical responses, distributional regression, quantile regression and multilevel models.
 \item  Mixed model based estimation (remlreg objects). Taking advantage of the close connection between penalised likelihood estimate and mixed models, the smoothing parameters of the penalties in structured additive regression can be interpreted as variance components of the random effects. Remlreg objects therefore employ mixed model methodology for the estimation of structured additive regression models. From a Bayesian perspective, this yields empirical Bayes / posterior mode estimates for the structured additive regression models. However, estimates can also merely be interpreted as penalized likelihood estimates from a frequentist perspective.
 \item Penalized least squares including model selection (stepwisereg objects). As a fourth alternative, BayesX provides a penalized likelihood approach for estimating structured additive regression models including model selection. The algorithms are able to
 \begin{itemize}
  \item decide whether particular effect types enter the model,
  \item decide whether a continuous covariate enters the model linearly or nonlinearly,
  \item select complex interaction effects (two dimensional surfaces, varying coefficient terms),
  \item select the degree of smoothness of nonlinear covariate, spatial or cluster specific heterogeneity effects.
 \end{itemize}
     Different models are compared via various goodness of fit criteria, e.g. AIC, BIC, GCV and 5 or 10 fold cross validation.
\end{itemize}

\subsection{Model classes and model terms}

BayesX provides functionality for the following types of responses:
\begin{itemize}
 \item Univariate exponential family: Supported response distributions are Gaussian, Poisson, Binomial and Gamma distribution.
 \item Distributional regression: A large number of univariate and multivariate continuous, discrete or mixed discrete-continuous responses can be treated within the framework of distributional regression. In this setting, potentially all parameters of these distributions can be related to structured additive predictors.
 \item Quantile regression: Bayesian quantile regression allows to study different quantiles of the response distribution without relying on a specific distributional assumption.
 \item Categorical responses with unordered categories: For categorical responses with unordered categories, BayesX supports multinomial logit and multinomial probit models. Both effects of category-specific and globally-defined covariates can be estimated. Category-specific offsets or non-availability indicators can be defined to account for varying availability and varying choice sets.
 \item Categorical responses with ordered responses: For ordered categorical responses, ordinal as well as sequential models can be specified. Effects can be requested to be category-specific or to be constant over the categories. Supported response functions include the logit and the probit transformation.
 \item Continuous time survival models: BayesX supports Cox-type hazard regression models with structured additive predictor for continuous time survival analysis. In contrast to the Cox model, the baseline hazard rate is estimated jointly with the remaining effects based on penalized splines. Furthermore, both time-varying effects and time-varying covariates can be included in the predictor. Arbitrary combinations of right, left and interval censored as well as left truncated observations can be analysed.
 \item Continuous time multi-state models: Multi-state models form a general class for the analysis of the evolution of discrete phenomena in continuous time. Transition intensities between the discrete states are specified in analogy to the hazard rate in continuous time survival models.
\end{itemize}

Structured additive regression models can be build from arbitrary combinations of the following model terms:
\begin{itemize}
 \item Nonlinear effects: Nonlinear effects can be estimated based on penalized splines.
 \item Seasonal effects: Specific autoregressive priors allow for the estimation of flexible, time-varying seasonal effects.
 \item Spatial effects: Spatial effects can be specified based on Markov random fields, stationary Gaussian random fields (kriging) or bivariate penalized splines. Both georeferenced regional data as well as point-referenced data based on coordinates are supported.
 \item Interaction surfaces: Bivariate extensions of penalised splines allow to estimate flexible interactions between continuous covariates. Stationary Gaussian random fields can also be considered a radial basis function approach and, hence, form a second possibility for the specification of interaction surfaces.
 \item Varying coefficients: Varying coefficient models with both continuous and spatial effect modifiers can be estimated. The latter case is also known as geographically weighted regression.
 \item Cluster-specific random effects: BayesX supports i.i.d. Gaussian random intercepts and random slopes.
 \item Regularized high-dimensional effects: High-dimensional vectors of regression coefficients can be assigned Bayesian regularization priors. Available alternatives are ridge regression, lasso regularization and normal mixture of inverse gamma (spike and slab) priors.
 \item User-defined model terms and general tensor products: User-defined model terms can be specified by supplying a design and penalty matrix as options. This also allows for the construction of general tensor product interactions when supplying the design and penalty matrices of the main effects for which a tensor product interaction shall be constructed.
 \item Multilevel models: In multilevel models, parameters of specific effects can themselves be assigned a structured additive predictor (e.g. in multilevel random effects specifications).
\end{itemize}

Note that parts of the functionality may be available for one of the regression objects only. For example, mcmcreg objects do not support continuous time duration and multilevel models while distributional regression models can not be estimated with remlreg objects. Details can be found in the corresponding chapters of this manual.

\section{Model Overview}

{\color{red}\bfseries Provide an overview on different models types in the style of Regressionsbuch (Ch. 2)?}

\section{Available Versions}
 \label{availableversions}
 \index{Command line version}
 \index{R2BayesX}
 \index{BayesXsrc}

\BayesX is available in two different versions: A simple command line version implemented in C++ including the computational kernel for all methods available and a second version within the R package {\it R2BayesX}. Most of this manual applies to the command line version of \BayesX. 

\subsection{Command Line Version}

The command line version of \BayesX provides all functionality for estimating various types of structured additive regression models. However, it only offers limited functionality concerning the handling of data sets or geographical information. It is therefore recommended to perform all necessary data manipulation steps prior to the statistical analysis with \BayesX in you favorite statistical computing environment and to use \BayesX only for the computation of structured additive regression models. Visualization is then most conveniently also done within statistical programming environments such as R or Stata.

The command line version has to be installed from the source code available from \href{http://www.bayesx.org}{http://www.bayesx.org}. The installation will require a C++ compiler as well as some additional tools. A makefile is provided that assumes that the GNU C++ compiler is available on your system. Hence, the command line version is suitable for any operating system that supports the GNU compiler family and has been successfully tested on Windows, Linux and Mac OS. In addition, the cmake toolchain can be used to generate customized makefiles.

In case you are not familiar with compiling software on your own computer, the easiest way is to install the R package {\it BayesXsrc} that provides the source code of \BayesX and installs it using R's package installation infrastructure (see the next point on {\it R2BayesX}).

\subsection{R2BayesX}

The R package {\it R2BayesX} provides convenient access to \BayesX from within R such that models can be estimated in the usual R formula syntax. {\it R2BayesX} relies on the package {\it BayesXsrc} that comprises all \BayesX sourcecode and uses R's package infrastructure for compilation. In particular, pre-compiled versions for Windows and Mac OS are directly available. While providing the natural access point via the package {\it R2BayesX}, users can also use the command line version compiled via {\it BayesXsrc} directly.

This manual mostly focuses on the pure command line version while documentation on {\it R2BayesX} is available from CRAN (\href{http://www.r-project.org}{http://www.r-project.org}). 

\section{General Usage}
\label{generalusage}

\subsection{Creating objects}
 \label{createobject} 
 \index{Objects} 
 \index{Objects!Create}

\BayesX is implemented in an object-oriented way, although the object-oriented concept does not go too far, i.e. inheritance or other concepts of object oriented programming languages such as C++ are not supported. As a consequence, the first thing to do during a session, is to create some objects. Currently, five different object types are available: {\em dataset objects}, {\em mcmcreg objects}, {\em remlreg objects}, {\em stepwisereg objects}, and {\em map objects}.

{\em Dataset objects} are used to store, handle, and manipulate data sets, see \autoref{datasetobj} for details. {\em Map objects} are used to handle geographical information and are covered in more detail in \autoref{map}. The main purpose of {\em map objects} is to serve as auxiliary objects for regression objects when estimating spatial effects. 

The most important object types are {\em mcmcreg objects}, {\em remlreg objects} and {\em stepwisereg objects}. These objects are used to estimate Bayesian structured additive regression models based on the different inferential approaches implemented in \BayesX (see \autoref{mcmcreg} for {\em mcmcreg objects}, \autoref{remlreg} for {\em remlreg objects}, and \autoref{stepwisereg} for {\em stepwisereg objects}.

The syntax for creating a new object is:

#># {\em objecttype objectname}

To create, for example, a {\em dataset object} with name #mydata#, type:

#> dataset mydata#

Note that some restrictions are imposed on the names of objects, i.e. not all object names are allowed. For example, object names have to begin with an uppercase or lowercase letter rather than a number. Section \ref{varnames} discusses valid variable names but the same rules apply also to object names.

\subsection{Applying methods to previously defined objects}

When an object has been created successfully, you can apply methods to that particular object. For instance, {\em dataset objects} may be used to read data stored in an ASCII file using method #infile#, to create new variables using method #generate#, to modify existing variables using method #replace# and so on. The syntax for applying methods to the objects is similar for all methods and independent of the particular object type. The general syntax is: \index{General syntax}\index{Syntax}

#># {\em objectname.methodname} [{\em model}] [#weight# {\em varname}] [#if# {\em boolean expression}] [, {\em options}] \\
\hspace*{4.8cm} [#using# {\em usingtext}]

\autoref{syntaxtable} explains the syntax parts in more detail.

\begin{table}[ht]
 \centering
 \begin{tabular}{|l|l|}
 \hline
 Syntax part & Description \\
 \hline
 {\em objectname} & the name of the object to apply the method to \\
 {\em methodname} & the name of the method \\
 {\em model} & a model specification (for example a regression model) \\
 {\em #weight# varname} & specifies {\em varname} as weight variable \\
 #if# {\em boolean expression} & indicates that the method should be applied only if a \\
 & certain condition holds \\
 , {\em options} & define (or modify) options for the method \\
 #using# {\em usingtext} & indicates that another object or file is required to \\
 & apply the particular method \\
 \hline
 \end{tabular}
 {\em \caption{\label{syntaxtable}Parts of the general BayesX syntax.}}
\end{table}

Note that $[\dots]$ indicates that this part of the syntax is optional and may be omitted. Moreover for most methods only some of the syntax parts above will be meaningful. The specification of
invalid syntax parts is not allowed and will cause an error message.

We illustrate the concept with some simple methods of {\em dataset objects}. Suppose that a {\em dataset object} with name #mydata# has already been created and that some variables should be
created. First of all, we have to tell \BayesX how many observations we want to create. This can be done with the #set obs# command, see also \autoref{setobs}. For example

#> mydata.set obs = 1000#

indicates that the data set #mydata# should have 1000 observations. In this case, the {\em methodname} is #set# and the {\em model} is #obs =# #1000#. Since no other syntax parts (for
example #if# statements) are meaningful for this method, they are not allowed. For instance, specifying an additional weight variable #x# by typing

#> mydata.set obs = 1000 weight x#

will cause the error message:

#ERROR: weight statement not allowed#

In a second step we can now create a new variable #X#, say, that contains Gaussian (pseudo) random numbers with mean 2 and standard deviation 0.5:

#> mydata.generate X = 2+0.5*normal()#

Here, #generate# is the {\em methodname} and #X = 2+0.5*normal()#  is the {\em model}. In this case the {\em model} consists of the specification of the new variable name, followed by the equal sign
'#=#' and a mathematical expression for the new variable. Similar as for the #set obs# command other syntax parts are not meaningful and therefore not allowed. If the negative values of #X# should be
replaced with the constant 0, this can be achieved using the #replace# command:

#> mydata.replace X = 0 if X < 0#

Obviously, the #if# statement is meaningful and is therefore allowed, but not required.

\section{Special Commands}

This section describes some commands that are not associated with a particular object type. Among others, there are commands for exiting \BayesX, opening and closing log files, saving program output, deleting objects etc.

\subsection{Exiting BayesX}
 \index{Exiting BayesX}

You can exit \BayesX by typing either

#> exit#

or

#> quit#

on the  command line. 

\subsection{Opening and closing log files} 
 \label{logfile}
 \index{Log files}

Program output and commands entered by the user can automatically be stored in a log file to make them available for editing in your favorite text editor. Another important application of log files is the documentation of your work. A log file is opened by the command:

#> logopen# [{\em, option}] #using# {\em filename}

Afterwards all commands entered and all program output will be saved in the file {\em filename}. If the log file specified in {\em filename} is already existing, new output is appended at the end of the file. To overwrite an existing log file, option #replace# has to be specified in addition. Note that it is not allowed to open more than one log file simultaneously.

An open log file can be closed by typing:

#> logclose#

Exiting \BayesX automatically closes the current log file.

\subsection{Using batch files}
 \label{batch} 
 \index{Batch files}

You can execute commands stored in a file just as if they were entered from the keyboard. This may be useful if you want to re-run a certain analysis more than once (possibly with some minor
changes) or if you want to run time-consuming statistical methods.

Execution of a batch file is started by typing

#> usefile# {\em filename}

This executes the commands stored in {\em filename} successively. \BayesX will not stop the execution if an error occurs in one or more commands. Note that it is allowed to invoke another batch
file within a currently running batch file. 

\subsection{Changing the delimiter}
 \label{delimiter} 
 \index{Delimiter}

By default, each line in a batch file is interpreted as a separate command. This can be inconvenient, in particular if your statements are long such that it may be favorable to split a
statement into several lines, and execute the command using a different delimiter. 

You can change the delimiter using the #delimiter# command. The syntax is

#> delimiter# = {\em newdel}

where {\em newdel} is the new delimiter. Only two different delimiters are currently allowed, namely the return key and the '#;#' (semicolon) key. To specify the semicolon as the delimiter, add

#> delimiter = ;#

to your batch filw. To return to the return key as the delimiter, add

#> delimiter = return;#

Note that this statement has to end with a semicolon, since this was previously set to be the current delimiter.

\subsubsection*{Comments}\index{Comments}

Comments in batch files are indicated by a  #%# sign, i.e. every line starting with #%# is ignored by the program.

\subsection{Deleting objects}
 \index{Objects!Deleting} 
 \index{Deleting objects}

You can delete objects by typing

#> drop# {\em objectlist}

This deletes the objects specified in {\em objectlist}. The names of the objects in {\em objectlist} must be separated by blanks.

\section{Some Example Data Sets}
 \label{datadescription} 
 \index{Data set examples}

{\color{red}\bfseries Add data sets that are used in the manual. Provide files on the website?}

This section describes some data sets used to illustrate many of the features of {\em BayesX} in the following chapters. All data sets are stored columnwise in
plain ASCII-format. The first row of each data set contains the variable names separated by blanks. Subsequent rows contain the observations, one observation per row.

\subsection{Rents for flats}
 \label{rentdata} 
 \index{Rents for flats} 
 \index{Data set examples!Rents for flats}

According to the German rental law, owners of apartments or flats can base an increase in the amount that they charge for rent on 'average rents' for flats comparable in type, size, equipment,
quality and location in a community. To provide information about these 'average rents', most of the larger cities publish 'rental guides', which can be based on regression analyses with rent as
the dependent variable. The file #rent94.raw# is a subsample of data collected in 1994 for the Munich rental guide. The variable of primary interest is the monthly rent per square meter in German Marks. Covariates characterizing the flat were constructed from almost 200 variables out of a questionnaire answered by tenants of flats. The present data set contains a small subset of these variables that are sufficient for demonstration purposes.

In addition to the data set, the map of Munich is available in the file #munich.bnd#. This map will be useful for visualizing effects of the location #L#. See \autoref{map} for a description on how to incorporate geographical maps into \BayesX.

\begin{table}

\centering
\begin{tabular}{|l|l|}
\hline
{\bf Variable} & {\bf Description} \\
\hline
R & monthly rent per square meter in German marks \\
$F$ & floor space in square meters \\
$A$ & year of construction \\
$L$ & location of the building in subquarters \\
 \hline
\end{tabular}
{\em \caption{\label{rentdatavar}Variables of the rent data set.}}
\end{table}

\subsection{Credit scoring}
 \label{creditdata} 
 \index{Credit scoring} 
 \index{Data set examples!Credit scoring}

The aim of credit scoring is to model and / or predict the probability that a client with certain covariates ('risk factors') will not pay back his credit. The data set contained in the file #credit.raw# consists of 1000 consumer credits from a bank in southern Germany. The response variable is 'creditability' in dichotomous form ($y=0$ for creditworthy, $y=1$ for not creditworthy). In addition, 20 covariates that are assumed to influence creditability were collected. The present data set (stored in the #examples# directory) contains a subset of these covariates that proved to be the  main influential variables on the response variable, see \citeasnoun[Ch.~2.1]{FahTut01}. \autoref{creditdatavar} gives a description of the variables of the data set. Usually a binary logit model is applied to estimate the effect of the covariates on the probability of being not creditworthy. 

\begin{table}[ht]

\begin{tabular}{|l|l|}
\hline
{\bf Variable} & {\bf Description} \\
\hline
$y$ & creditability, dichotomous with $y=0$ for creditworthy, $y=1$ for \\
    & not creditworthy \\
$account$ & running account, trichotomous with categories "no
running account" \\& ($=1$),
    "good running account"
($=2$),  "medium running account" \\&("less than 200 DM") ($=3$)  \\
$duration$ & duration of credit in months, continuous \\
$amount$ & amount of credit in 1000 DM, continuous \\
$payment$ & payment of previous credits, dichotomous with categories "good" ($=1$), \\ & "bad" ($=2$)  \\
$intuse$ & intended use, dichotomous with categories "private" ($=1$) or \\ & "professional" ($=2$)  \\
$marstat$ & marital status, with categories "married" ($=1$) and "living alone" ($=2$). \\
\hline
\end{tabular}
{\em \caption{\label{creditdatavar}Variables of the credit scoring
data set.}}
\end{table}

\subsection{Childhood undernutrition in Zambia}
 \label{zambia} 
 \index{Childhood undernutrition} 
 \index{Data set examples!Childhood undernutrition}

Acute and chronic undernutrition is considered to be one of the worst health problems in developing countries. Undernutrition among children is usually determined by assessing the
anthropometric status of the child relative to a reference standard. In our example undernutrition is measured through stunting (insufficient height for age), indicating chronic
undernutrition. Stunting for child $i$ is determined using the Z-score
\[
 Z_i = \frac{AI_i-MAI}{\sigma}
\]
where $AI$ refers to the child`s anthropometric indicator (height at a certain age in our example), MAI refers to the median of the reference population and $\sigma$ refers to the standard deviation
of the reference population. 

The data set #zambia.raw# contains the (standardized) Z-score for 4847 children together with several covariates that are supposed to influence undernutrition (e.g. the body mass index of the
mother, the age of the child, and the district the mother lives in). \autoref{zambiavar} gives more information on the covariates in the data set.

\begin{table}
\begin{center}
\begin{tabular}{|l|l|}
 \hline
 {\bf Variable} & {\bf Description}\\
 \hline
 $hazstd$ & standardized Z-score for stunting\\
 $bmi$ & body mass index of the mother\\
 $agc$ & age of the child\\
 $district$ & district where the mother lives\\
 $rcw$ & mother`s employment status with categories "working" (= 1) and "not working" \\
 & (= $-1$)\\
 $edu1$ & mother`s educational status with categories "complete primary but incomplete\\
 $edu2$ & secondary" ($edu1=1$), "complete secondary or higher" ($edu2=1$) and\\
 & "no education or incomplete primary" ($edu1=edu2=-1$)\\
 $tpr$ & locality of the domicile with categories "urban" (= 1) and "rural" (= $-1$)\\
 $sex$ & gender of the child with categories "male" (= 1) and
 "female" (= $-1$)\\
 \hline
\end{tabular}
{\em\caption{Variables in the undernutrition data set.
\label{zambiavar}}}
\end{center}
\end{table}
