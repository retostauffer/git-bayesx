\chapter{remlreg objects}\normalsize
\label{remlreg} \index{Remlreg object}

{\em Remlreg objects} are used to fit (multivariate) exponential family, hazard rate or multi-state models with {\em structured
additive predictor} subsumed in the class of {\em structured additive regression (STAR)} models, see \citeasnoun{FahKneLan04}.
Inference is based on a mixed model representation of the regression model and yields either penalised likelihood estimates
(from a frequentist perspective) or empirical Bayes / posterior mode estimates (from a Bayesian perspective). The
methodological background is provided in considerable detail in the methodology manual. More details on models for univariate
responses can be found in \citeasnoun{FahKneLan04}. \citeasnoun{KneFah06} describe models for categorical responses. Models for
continuous time survival analysis based on structured hazard regression can be found in \citeasnoun{KneFah07}. Interval
censoring and some further extensions are discussed in \citeasnoun{Kne06}.

\index{Generalized linear models} \index{Generalized additive
models} \index{Varying coefficients} \index{Bayesian semiparametric
regression} \index{MCMC} \index{Markov chain Monte Carlo}

\section{Method regress}\index{Regress function}\index{Remlreg object!Regress
function}\index{Mixed model based regression} \label{remlregregress}

\subsection{Syntax}\index{Regression syntax}\index{Remlreg object!Regression syntax}
\label{remlregregresssyntax}

 #># {\em objectname}.#regress# {\em model} [#weight# {\em weightvar}] [#if# {\em expression}] [{\em , options}] #using# {\em dataset}

Method #regress# estimates the regression model specified in {\em
model} using the data specified in {\em dataset}. {\em dataset}
has to be the name of a {\em dataset object} created before. The
details of correct model specification are covered in
\autoref{remlregmodelsyntax}. The distribution of the response
variable can be either Gaussian, binomial, multinomial, Poisson or
gamma. In addition, {\em BayesX} supports continuous time survival
and multi-state models. The response distribution is specified using
option #family#, see \autoref{remlregfamilysyntax} below. The
default is #family=binomial# with a logit link. An #if# statement
can be specified to analyze only parts of the data set, i.e. the
observations where {\em expression} is true.

\subsubsection{Optional weight variable}\index{Weighted regression}
\label{remlregweightspecification}

An optional weight variable {\em weightvar} can be specified to
estimate weighted regression models. For Gaussian responses, {\em
BayesX} assumes that $y_i|\eta_i,\sigma^2 \sim
N(\eta_i,\sigma^2/weightvar_i)$. Thus, for grouped Gaussian
responses the weights represent the number of observations in the
groups if the $y_i$'s are the average of individual responses. If
the $y_i$s are the sum of responses in every group, the weights
have to be the reciprocal of the number of observations in the
groups. Of course, estimation of usual weighted regression models
with heteroscedastic errors is also possible. In this case, the
weights should be proportional to the reciprocal of the
heteroscedastic variances. If the response distribution is
binomial, the weight variable should correspond to the number of
replications while the values of the response variable should
represent the number of successes. If weight is omitted, {\em
BayesX} assumes that the number of replications is one, i.e. the
values of the response must be either zero or one. For grouped
Poisson data, the weights have to specify the number of
observations in a group while the $y_i$s are assumed to be the
average of individual responses. Weights are not allowed for
models with categorical response, continuous survival time models
and multi-state models.

\subsubsection{Syntax of possible model terms}
\label{remlregmodelsyntax}\index{Model terms}\index{Remlreg
object!Model terms}

The general syntax of models for {\em remlreg objects} is:

$depvar = term_1 + term_2 + \cdots + term_r$

{\em depvar} specifies the dependent variable whereas
$term_1$,\dots,$term_r$ define the form of covariate influences.
The different terms must be separated by '+' signs. A constant
intercept is automatically included in the model and therefore has
not to be specified by the user.

This section reviews all possible model terms supported in the
current version of {\em remlreg objects} and provides some
specific examples. Note that all described terms may be combined
in arbitrary order. An overview about the capabilities of {\em
remlreg objects} is given in \autoref{remlregterms}.
\autoref{remlreginteractions} shows how interactions between
covariates are specified. Full details about all available options
are given in \autoref{remlreglocaloptions}.

Throughout this section Y denotes the dependent variable.

\begin{table}[ht] \footnotesize
\begin{center}
\begin{tabular}{|p{2.8cm}|p{3.6cm}|p{7.1cm}|}
\hline
{\bf Type} & {\bf Syntax example} & {\bf Description} \\
\hline \hline
Offset & #offs(offset)#  & Variable #offs# is an offset term. \\
\hline
Linear effect & #W1#  & Linear effect of #W1#. \\
\hline
Category-specific linear effect & #W1(catspecific)#  & Category-specific linear effect of #W1# (in cumulative or sequential models only). \\
\hline
First or second order random walk &   #X1(rw1)#  \newline  #X1(rw2)#  & Nonlinear effect of #X1#. \\
\hline
P-spline &  #X1(psplinerw1)#   \newline  #X1(psplinerw2)#  & Nonlinear effect of #X1#.  \\
\hline
Seasonal prior & #time(season,period=12)# & Varying seasonal effect of #time# with period 12. \\
\hline Markov random \newline field &  #region(spatial,map=m)#  &
Spatial effect of #region# where #region# indicates the region an
observation pertains to. The boundary information and the
neighborhood structure is stored in the {\em map object}
#m#. \\
\hline Two dimensional \newline P-spline &
#region(geospline,map=m)# & Spatial effect of #region#. Estimates
a two dimensional P-spline
based on the centroids of the regions. The centroids are stored in the {\em map object} #m#. \\
 \hline
 Stationary Gaussian random field & #region(geokriging)# & Spatial effect of #region#. Estimates
a stationary Gaussian random field
based on the centroids of the regions. The centroids are stored in the {\em map object} #m#. \\
\hline
 Random intercept &  #grvar(random)# & I.i.d. Gaussian (random) effect of the group indicator #grvar#,
 e.g.~#grvar# may be an individual indicator when analyzing longitudinal data.  \\
\hline
 Baseline in Cox or multi-state models & #time(baseline)# & Nonlinear shape
of the baseline effect $\lambda_0(time)$ of a Cox model. $\log(\lambda_0(time))$ is modelled by a P-spline with second order random walk penalty. \\
 \hline
\end{tabular}
{\em \caption {\label{remlregterms} Overview over different model
terms for remlreg objects.}}
\end{center}
\end{table}


\begin{table}[ht] \footnotesize
\begin{center}
\begin{tabular}{|p{3.5cm}|p{3.8cm}|p{5.9cm}|}
\hline
{\bf Type of interaction} & {\bf Syntax example} & {\bf Description} \\
 \hline
\hline Varying coefficient term & #X1*X2(rw1)# \newline
#X1*X2(rw2)#
\newline
 #X1*X2(psplinerw1) #
 \newline  #X1*X2(psplinerw2)# \newline #X1*time(season)#
 & Effect of #X1# varies smoothly over the range of the continuous covariate #X2# or #time#. \\
\hline Random slope & #X1*grvar(random)#  &  The regression
coefficient of #X1# varies with respect
to the unit- or cluster index variable #grvar#. \\
\hline Geographically weighted \newline regression &
#X1*region(spatial,map=m)#  & Effect of #X1# varies
geographically.
Covariate #region# indicates the region an observation pertains to. \\
\hline Two dimensional \newline surface &  #X1*X2(pspline2dimrw1)#
 & Two dimensional surface for the continuous
covariates #X1# and #X2#. \\
 \hline
 Stationary Gaussian random field &  #X1*X2(kriging)# & Stationary Gaussian random field for coordinates #X1# and #X2#. \\
 \hline
 Time-varying effect in Cox or multi-state models & #X1*time(baseline)# &
 Nonlinear, time-varying effect of #X1#.\\
 \hline
\end{tabular}
\caption {\label{remlreginteractions} \em Possible interaction
terms for remlreg objects.}
\end{center}
\end{table}

\subsubsection*{Offset}\index{Offset}

\begin{itemize}
\item[] {\em Description}: Adds an offset term to the predictor.
\item[] {\em Predictor}: $\eta =  \cdots + offs + \cdots$
\item[] {\em Syntax}:

#offs(offset)#
\item[] {\em Example}:

For example, the following model statement can be used to estimate
a poisson model with #offs# as offset term and #W1# and #W2# as
fixed effects (if #family=poisson# is specified in addition):

\texttt{Y = offs(offset) + W1 + W2}
\end{itemize}

\subsubsection*{Fixed effects}\index{Fixed effects}

\begin{itemize}
\item[] {\em Description}: Incorporates covariate #W1# as a fixed effect into the model.
\item[] {\em Predictor}: $\eta =  \cdots + \gamma_1 W1 + \cdots$
\item[] {\em Syntax}:

#W1#
\item[] {\em Example}:

The following model statement causes #regress# to estimate a model
with $q$ fixed (linear) effects:

\texttt{Y = W1 + W2 + $\cdots$ + Wq}
\end{itemize}

\subsubsection*{Category-specific fixed
effects}\index{Category-specific fixed effects}

\begin{itemize}
\item[] {\em Description}: In cumulative and sequential models for
ordered categorical responses, fixed effects may either be defined
globally or category-specific. To request the estimation of
category-specific fixed effects, the keyword #catspecific# has to
be specified. Category-specific effects can only be estimated for
the response families #cumlogit#, #cumprobit#, #seqlogit#, and
#seqprobit#.
\item[] {\em Predictor}: $\eta^{(j)} =  \cdots +
\gamma_1^{(j)} W1 + \cdots$ \item[] {\em Syntax}:

#W1(catspecific)#
\item[] {\em Example}:

The following model statement causes #regress# to estimate a model
with category-specific effect of covariate #W1# and a global
effect of covariate #W2#:

\texttt{Y = W1(catspecific) + W2}
\end{itemize}

\subsubsection*{Nonlinear effects of continuous covariates and time
scales}\index{Nonlinear effects}\index{Random
walks}\index{P-splines}

\begin{itemize}
\item[]{\bf\sffamily First or second order random walk}

\item[] {\em Description}: Defines a first or second order random walk prior for the effect of #X1#.
\item[] {\em Predictor}: $\eta = \cdots + f_1(X1) + \cdots $
\item[] {\em Syntax}:

#X1(rw1#[, {\em options}]#)#

#X1(rw2#[, {\em options}]#)#
\item[] {\em Example}:

Suppose that #X1# is a continuous covariate with possibly
nonlinear effect. The following model statement defines a second
order random walk prior for $f_1$:

#Y = X1(rw2)#

The term #X1(rw2,a=0.001,b=0.001)# indicates, that the effect of
#X1# should be incorporated nonparametrically using a second order
random walk prior. A first order random walk can be requested by
specifying #X1(rw1)# instead.

\item[] {\bf\sffamily P-spline with first or second order random
walk penalty}

\item[] {\em Description}: Defines a P-spline with first or second
order random walk penalty for the parameters of the spline.
\item[] {\em Predictor}: $\eta =  \cdots + f_1(X1) + \cdots$
\item[] {\em Syntax}:

#X1(psplinerw1#[, {\em options}]#)#

#X1(psplinerw2#[, {\em options}]#)#
\item[] {\em Example}:

For example, a P-spline with second order random walk penalty is
obtained using the following model statement:

#Y = X1(psplinerw2)#

By default, the degree of the spline is 3 and the number of inner
knots is 20. The following model term defines a quadratic P-spline
with 30 knots:

#Y = X1(psplinerw2,degree=2,nrknots=30)#

\item[]{\bf\sffamily Seasonal component for time scales}

\item[] {\em Description}: Defines a time-varying seasonal effect
of #time#. \item[] {\em Predictor}: $\eta =  \cdots +
f_{season}(time) + \cdots $ \item[] {\em Syntax}:

#time(season#[, {\em options}]#)#
\item[] {\em Example}:

A seasonal component for a time scale #time# is specified by

#Y = time(season,period=12)#

where the second argument indicates the period of the seasonal
effect. In the example above, the period is 12 corresponding to
monthly data.
\end{itemize}

\subsubsection*{Nonlinear baseline effect in continuous time survival or multi-state models}
\index{Baseline}\index{Cox model}\index{Multi-state model}

\begin{itemize}
\item[]{\bf\sffamily P-spline with second order random walk
penalty}

\item[] {\em Description}: Defines a P-spline with second order
random walk penalty for the parameters of the spline for the
log-baseline effect $\log(\lambda_0$(#time#)). \item[] {\em
Predictor}: $\eta = \log(\lambda_0(time)) + \cdots$ \item[] {\em
Syntax}:

#time(baseline#[, {\em options}]#) # \item[] {\em Example}:

Suppose continuous-time survival data (#time#, #delta#) with
additional covariates (#W1#, #X1#) are given, where #time# denotes
the vector of observed duration times, #delta# is the vector of
corresponding indicators of non-censoring, #W1# is a discrete
covariate, and #X1# a continuous covariate. The following Cox-type
model with hazard rate $\lambda$ and log-baseline effect
$\log(\lambda_0$(#time#))
\begin{eqnarray*}
 \lambda(time) & = & \lambda_0(time)\exp (\gamma_0 + \gamma_1 W1 + f(X1))\\
 & = & \exp\left(\log(\lambda_0(time)) + \gamma_0 + \gamma_1 W1 + f(X1)\right)
\end{eqnarray*}
can be estimated by the model statement

#delta = time(baseline) + W1 + X1(psplinerw2)#

Similarly, baseline effects for the transition intensities in
multi-state models can be specified.
\end{itemize}

\subsubsection*{Spatial Covariates}\index{Spatial effects}\index{Markov random fields}
\index{Two-dimensional P-spline}\index{Kriging}

\begin{itemize}
\item[]{\bf\sffamily Markov random field}

\item[] {\em Description}:

Defines a Markov random field prior for the spatial covariate
#region#. {\em BayesX} allows to incorporate spatial covariates
with geographical information stored in the {\em map object}
specified in option #map#.
\item[] {\em Predictor}: $\eta = \cdots
+ f_{spat}(region) + \cdots$ \item[] {\em Syntax}:

#region(spatial,map=#{\em characterstring}[, {\em options}]#)#
\item[] {\em Example}:

For the specification of a Markov random field prior, #map# is an
obligatory argument that represents the name of a {\em map object}
(see \autoref{map}) containing all necessary spatial information
about the geographical map, i.e.~the neighbors of each region and
the weights associated with the neighbors. For example the
statement

#Y = region(spatial,map=germany)#

defines a Markov random field prior for #region# where the
geographical information is stored in the {\em map object}
#germany#. An error will be raised if #germany# is not existing.

\item[]{\bf\sffamily Two-dimensional P-spline with first order
random walk penalty}

\item[] {\em Description}:

Defines a two-dimensional P-spline for the spatial covariate
#region# with a two-dimensional first order random walk penalty
for the parameters of the spline. Estimation is based on the
coordinates of the centroids of the regions. The centroids are
computed using the geographical information stored in the {\em map
object} specified in the option #map#.
\item[] {\em Predictor}:
$\eta= \cdots + f(centroids) + \cdots$ \item[] {\em Syntax}:

#region(geospline,map=#{\em characterstring}[, {\em options}]#)#
\item[] {\em Example}:

For the specification of a two-dimensional P-spline ({\em
geospline}) #map# is an obligatory argument indicating the name of
a {\em map object} (see \autoref{map}) that contains all necessary
spatial information about the geographical map, i.e.~the neighbors
of each region and the weights associated with the neighbors. The
model term

#Y = region(geospline,map=germany)#

specifies a two-dimensional cubic P-spline with first order random
walk penalty where the geographical information is stored in the
{\em map object} #germany#.

\item[]{\bf\sffamily Stationary Gaussian random field}

\item[] {\em Description}:

Defines a stationary Gaussian random field for the spatial
covariate #region#. Estimation is based on the coordinates of the
centroids of the regions an observation pertains to. The centroids
are computed using the geographical information stored in the {\em
map object} specified in option #map#. \item[] {\em Predictor}:
$\eta= \cdots + f(centroids) + \cdots$ \item[] {\em Syntax}:

#region(geokriging,map=#{\em characterstring}[, {\em options}]#)#
\item[] {\em Example}:

For the specification of a stationary Gaussian random field
(#geokriging#), #map# is an obligatory argument indicating the
name of a {\em map object} (see \autoref{map}). The model term

#Y = region(geokriging,map=germany)#

specifies a stationary Gaussian random field where the geographical
information is stored in the {\em map object} #germany#.
\end{itemize}

\subsubsection*{Unordered group indicators}\index{Unordered group
indicators}\index{Random effects}\index{Random intercept}

\begin{itemize}
\item[]{\bf\sffamily Unit- or cluster specific unstructured
effect}

\item[] {\em Description}: Defines an unstructured (uncorrelated)
random effect with respect to grouping variable #grvar#. \item[]
{\em Predictor}: $\eta = \cdots + f(grvar) + \cdots$ \item[] {\em
Syntax}:

#grvar(random#[, {\em options}]#)#
\item[] {\em Example}:

Gaussian i.i.d.~random effects allow to cope with unobserved
heterogeneity among units or clusters of observations. Suppose the
analyzed data set contains a group indicator #grvar# that gives
information about the individual or cluster a particular
observation belongs to. Then an individual-specific uncorrelated
random effect is defined by

#Y = grvar(random)#

The inclusion of more than one random effect term in the model is
possible, allowing the estimation of multilevel models. However,
we have only limited experience with multilevel models so that it
is not clear how well these models can be estimated using {\em
remlreg objects}.
\end{itemize}

\subsubsection*{Varying coefficients with continuous covariates as
effect modifier}\index{Varying coefficients}

\begin{itemize}
\item[]{\bf\sffamily First or second order random walk}

\item[] {\em Description}:

Defines a varying coefficient term, where the effect of #X1#
varies smoothly over the range of #X2#. Therefore covariate #X2#
is called the effect modifier. The smoothness prior for $f(X2)$ is
a first or second order random walk.
\item[] {\em Predictor}:
$\eta= \cdots + f(X2)X1 + \cdots$ \item[] {\em Syntax}:

#X1*X2(rw1#[, {\em options}]#)#

#X1*X2(rw2#[, {\em options}]#)#
\item[] {\em Example}:

For example, a varying coefficient term with a second order random
walk smoothness prior is defined as follows:

#Y = X1*X2(rw2)#

\item[]{\bf\sffamily P-spline with first or second order random
walk penalty}

\item[] {\em Description}:

Defines a varying coefficient term, where the effect of #X1#
varies smoothly over the range of #X2#. The smoothness prior for
$f$ is a P-spline with first or second order random walk penalty.
\item[] {\em Predictor}: $\eta= \cdots + f(X2)X1 + \cdots$ \item[]
{\em Syntax}:

#X1*X2(psplinerw1#[, {\em options}]#)#

#X1*X2(psplinerw2#[, {\em options}]#)#
\item[] {\em Example}:

For example, a varying coefficient term with a second order random
walk smoothness prior is defined as follows:

#Y = X1*X2(psplinerw2)#

If the effect of a covariate should vary according to different
types of effect modifiers, this leads to similar identification
problems as in usual additive models. To avoid such problems,
option #center# can be specified to request the estimation of
centered effects. For example, if both #X2# and #Z2# are assumed
to modify the effect of #X1#, the specification of

#Y = X1*X2(psplinerw2) + X1*Z2(psplinerw2)#

yields a non-identifiable model. In contrast

#Y = X1 + X1*X2(psplinerw2, center) + X1*Z2(psplinerw2, center)#

is well-identified. Note that the main effect of #X1# has to be
included separately. Equivalently, we could absorb the main effect
into the first term, yielding

#Y = X1*X2(psplinerw2) + X1*Z2(psplinerw2, center)#

However, the former specification has the advantage that the model
terms are clearly separated.

Models of the type just discussed arise for example if #X1# is a
binary dummy-variable indicating two different groups of data. In
this case the model

 #Y = X1 + X2(psplinerw2) + X1*X2(psplinerw2, center) + Z2(psplinerw2) + X1*Z2(psplinerw2, center)#

assumes different effects of both #X2# and #Z2# in the groups.

\item[]{\bf\sffamily Seasonal prior}

\item[] {\em Description}:

Defines a varying coefficients term where the effect of #X1#
varies over the range of the effect modifier #time#. A seasonal
prior is assumed for the effect of #time#.

\item[] {\em Predictor}: $\eta= \cdots + f_{season}(time)X1 +
\cdots $ \item[] {\em Syntax}:

#X1*time(season#[, {\em options}]#)#
\item[] {\em Example}:

The inclusion of a varying coefficients term with a seasonal prior
may be meaningful if we expect a different seasonal effect with
respect to a binary variable #X1#. In this case we can include an
additional seasonal effects for observations with #X1#=1 by

#Y = X1*time(season) #

\end{itemize}

\subsubsection*{Time-varying effects in continuous-time or multi-state models}
\index{Time-varying effects}

\begin{itemize}
\item[]{\bf\sffamily P-spline with second order random walk
penalty}

\item[] {\em Description}: Defines a varying coefficients term
where the effect of #X1# varies over the range of the effect
modifier #time#, i.e. variable #X1# is assumed to have a
time-varying effect. The smoothness prior for $f($#time#$)$ is a
P-spline with second order random walk penalty.

 \item[] {\em Predictor}: $\eta = \log(\lambda_0(time)) +
f(time)X1 \cdots$ \item[] {\em Syntax}:

 #X1*time(baseline#[, {\em options}]#) #
 \item[] {\em Example}:

Suppose continuous-time survival data (#time#, #delta#) together
with an additional covariate #X1# are given, where #time# denotes
the vector of observed duration times and #delta# is the vector of
corresponding indicators of non-censoring. The following Cox model
with hazard rate
\begin{eqnarray*}
 \lambda(time) & = & \lambda_0(time)\exp(\gamma_0 + f(time)X1)\\
 & = & \exp\left(\log(\lambda_0(time)) + \gamma_0 + f(time)X1\right)
\end{eqnarray*}
is estimated by the model statement

#delta = time(baseline) + X1*time(baseline)#

Similarly, time-varying effects on the transition intensities in
multi-state models can be specified.
\end{itemize}

\subsubsection*{ Varying coefficients with spatial covariates as
effect modifiers}

\begin{itemize}
\item[]{\bf\sffamily Markov random field}

\item[] {\em Description}:

Defines a varying coefficient term where the effect of #X1# varies
smoothly over the range of the spatial covariate #region#. A
Markov random field is estimated for $f_{spat}$. The geographical
information is stored in the {\em map object} specified through the
option #map#.
\item[] {\em Predictor}: $\eta = \cdots + f_{spat}(region)X1 + \cdots$
\item[] {\em Syntax}:

#X1*region(spatial,map=#{\it characterstring} #[,# {\it options}#])#
\item[] {\em Example}:

For example the statement

#Y = X1*region(spatial,map=germany)#

defines a varying coefficient term with the spatial covariate
#region# as the effect modifier and a Markov random field as spatial
smoothness prior. Weighted Markov random fields can be estimated by
including an appropriate weight definition when creating the {\em
map object} #germany# (see \autoref{mapinfile}).

Similarly as for varying coefficient terms with continuous effect
modifiers, varying coefficients with spatial effect modifier can be
centered to avoid identifiability problems:

#Y = X1*region(spatial, map=germany, center)#

\end{itemize}

\subsubsection*{Varying coefficients with unordered group indicators
as effect modifiers (random slopes)}\index{Random
effects}\index{Random slope}

\begin{itemize}
\item[]{\bf\sffamily Unit- or cluster specific unstructured
effect}

\item[] {\em Description}:

Defines a varying coefficient term where the effect of #X1# varies
over the range of the group indicator #grvar#. Models of this type
are usually referred to as models with random slopes. A Gaussian
i.i.d.~random effect with respect to grouping variable #grvar# is
assumed for $f$.
\item[] {\em Predictor}: $\eta = \cdots + f(grvar)X1 + \cdots$
\item[] {\em Syntax}:

#X1*grvar(random#[, {\em options}]#)#
\item[] {\em Example}:

For example, a random slope is specified as follows:

#Y = X1*grvar(random)#

Note, that in contrast to {\em bayesreg objects}, the main effects
are {\em not} included automatically. If main effects should be
included in the model, they have to be specified as additional
fixed effects. The syntax for obtaining the predictor

$\eta = \cdots + \gamma X1 + f(grvar)X1 + \cdots$

would be

#X1 + X1*grvar(random#[, {\em options}]#)#

\end{itemize}

\subsubsection*{Surface estimators}\index{Surface
estimators}\index{Two-dimensional P-spline}\index{Kriging}

\begin{itemize}
\item[]{\bf\sffamily Two-dimensional P-spline with first order
random walk penalty}

\item[] {\em Description}:

Defines a two-dimensional P-spline based on the tensor product of
one-dimensional P-splines with a two-dimensional first order
random walk penalty for the parameters of the spline. \item[] {\em
Predictor}: $\eta= \cdots + f(X1,X2) + \cdots$ \item[] {\em
Syntax}:

#X1*X2(pspline2dimrw1#[, {\em options}]#)#
\item[] {\em Example}:

The model term

#Y = X1*X2(pspline2dimrw1)#

specifies a tensor product cubic P-spline with first order random
walk penalty.

In many applications it is favorable to additionally incorporate
the one-dimensional main effects of #X1# and #X2# into the models.
In this case the two-dimensional surface can be seen as the
deviation from the main effects. Note, that in contrast to {\em
bayesreg objects} the number of inner knots and the degree of the
spline may be different for the main effects and for the
interaction. For example, a model with 20 inner knots for the main
effects and 10 inner knots for the two-dimensional P-spline is
estimated by

 #Y = X1(psplinerw2,nrknots=20) + X2(psplinerw2,nrknots=20)#\\
 #    + X1*X2(pspline2dimrw1,nrknots=10)#

\item[]{\bf\sffamily Stationary Gaussian random field}

\item[] {\em Description}:

Defines that the parameters of the locations follow a stationary
Gaussian random field. Depending on the options chosen, locations
are given either by the distinct pairs of #X1# and #X2# or by a
subset of these pairs, which we will also refer to as knots. Note
that in principle stationary Gaussian random fields can be used to
estimate surfaces depending on arbitrary variables #X1# and #X2#,
but they are defined based on {\em isotropic} correlation
functions. This means that correlations between sites that have
the same distance also have the same correlation, regardless of
direction and the sites location. Therefore, if Gaussian random
fields shall be used to estimate interactions between variables
that do not represent longitude and latitude, these variables have
to be standardized appropriately.

\item[] {\em Predictor}: $\eta= \cdots + f(X1,X2) + \cdots$
\item[] {\em Syntax}:

#X1*X2(kriging#[, {\em options}]#)# \item[] {\em Example}:

The model term

#Y = X1*X2(kriging,nrknots=100)#

specifies a stationary Gaussian random field for the effect of
#X1# and #X2# with 100 knots, which are computed based on a
space filling algorithm. If all distinct pairs of #X1# and #X2# shall
be used as knots, we have to specify

#Y = X1*X2(kriging,full)#

Note, that the knots computed by the space filling algorithm will
be stored in a file in the outfile directory of the {\em remlreg
object} (the file name will be printed in the output window with
the estimated effects). These knots can be read into a {\em
dataset object} which may be passed to the kriging term if we want
to use the same knots as in previous calls:

 #dataset kn#\\
 #kn.infile using #{\em knotfile}\\
 #Y = X1*X2(kriging,knotdata=kn)#

To determine the actual number of knots, the options are
interpreted in a specific sequence. If option #full# is specified,
both #nrknots# and #knotdata# are ignored. Similarly, #nrknots# is
ignored if #knotdata# is specified.

\end{itemize}

\subsubsection{Description of additional options for terms of {\em remlreg objects}}
\label{remlreglocaloptions}

All arguments described in this section are optional and may be
omitted. Generally, options are specified by adding the option name
to the specification of the model term type in the parentheses,
separated by commas. All options may be specified in arbitrary
order. \autoref{remlregoptions} provides explanations and the
default values of all possible options. All reasonable combinations
of model terms and options can be found in
\autoref{remlregtermsoptions}.

\begin{table}[ht] \footnotesize \centering
\begin{tabular}{|p{0.1\linewidth}|p{0.6\linewidth}|p{0.2\linewidth}|}
 \hline
 optionname & description & default\\
 \hline\hline
 #lambdastart# & Starting value for the smoothing parameter $\lambda$. & #lambdastart=10# \\
 \hline
 #degree# & Degree of B-spline basis functions. & #degree=3# \\
 \hline
 #nrknots# & Number of inner knots for a P-spline term or number of knots for a kriging term. & #nrknots=20# (P-splines)\newline #nrknots=100# (kriging)  \\
 \hline
 #knotdata# & {\em Dataset object} containing the knots to be used
 with the kriging term & no default.\\
 \hline
 #full# & Specifies that all distinct locations should be used as
 knots in the kriging term. & -\\
 \hline
 #nu# & The smoothness parameter $\nu$ of the Mat\`{e}rn correlation function for kriging terms. & #nu=1.5# \\
 \hline
 #maxdist# & Specifies the value $c$ that is used to determine the scale parameter $\rho$ of the Mat\`{e}rn correlation function for kriging terms.
  & default depends on #nu#\\
 \hline
 #p# & Parameter $p$ of the coverage criterion for the space filling algorithm that determines the knots of a kriging term. & #p=-20#\\
 \hline
 #q# & Parameter $q$ of the coverage criterion for the space filling algorithm that determines the knots of a kriging term. & #q=20#\\
 \hline
 #maxsteps# & Maximum number of steps to be performed by the space filling algorithm. & #maxsteps=300#\\
 \hline
 #gridchoice# & How to choose grid points for numerical integration in Cox and multi-state models. May be either '#quantiles#', '#equidistant#' or '#all#'. & #gridchoice=quantiles# \\
 \hline
 #tgrid# & Number of equidistant time points to be used for numerical integration in Cox and multi-state models. Only meaningful if #gridchoice=equidistant#. & #tgrid=100#\\
 \hline
 #nrquantiles# & Number of quantiles that are used to define the grid points for numerical integration in Cox and multi-state models. First a grid of #nrquantiles# quantiles is computed, then the grid for integration is defined by #nrbetween# equidistant points between each quantile. Only meaningful if #gridchoice=quantiles#. & #nrquantiles=50#\\
 \hline
 #nrbetween# & Number of points between quantiles that are used to define the grid points for numerical integration in Cox and multi-state models. First a grid of #nrquantiles# quantiles is computed, then the grid for integration is defined by #nrbetween# equidistant points between each quantile. Only meaningful if #gridchoice=quantiles#.& #nrbetween=5#\\
 \hline
 #map# & {\em Map object} for spatial effects. & no default\\
 \hline
 #period# & Period of a seasonal effect. The default (#period=12#) corresponds to monthly data. & #period=12# \\
 \hline
 #catspecific# & Requests that the corresponding effect should be modelled category-specific. Can only be used in cumulative and sequential models for categorical responses, i.e. with response families #cumlogit#, #cumprobit#, #seqlogit# and #seqprobit#. & - \\
 \hline
 #center# & For varying coefficient terms this option requests that the effect should be centered to avoid identifiability problems& - \\
 \hline
\end{tabular}
{\em \caption{\label{remlregoptions} Optional arguments for {\em
remlreg object} terms.}}
\end{table}

\begin{sidewaystable} \footnotesize
\begin{tabular}{|l||c|c|c|c|c|c|c|c|c|c|}

\hline
            & rw1/rw2       & season    & psplinerw1/psplinerw2    & spatial & random & geospline & pspline2dimrw1 & kriging  & geokriging & baseline\\
 \hline\hline
 #lambdastart#$^*$  & realvalue   & realvalue   & realvalue   & realvalue   & realvalue   & realvalue   & realvalue & realvalue  & realvalue & realvalue\\
 \hline
 #degree#       & $\times$   & $\times$   &  integer   & $\times$ & $\times$ &  integer &  integer &  $\times$ & $\times$ & integer\\
 \hline
 #nrknots#      & $\times$   & $\times$   &  integer   & $\times$ & $\times$ &  integer &  integer &  integer & $\times$ & integer\\
 \hline
 #knotdata#     & $\times$   & $\times$   &  $\times$   & $\times$ & $\times$ &  $\times$ &  $\times$ & {\em dataset object}& {\em dataset object} & $\times$\\
 \hline
 #full#     & $\times$   & $\times$   &  $\times$   & $\times$ & $\times$ &  $\times$ &  $\times$ &  $\triangle$ & $\triangle$ & $\times$\\
 \hline
 #nu#     & $\times$   & $\times$   &  $\times$   & $\times$ & $\times$ &  $\times$ &  $\times$ &  $\bullet$ &  $\bullet$ & $\times$\\
 \hline
 #maxdist#$^*$     & $\times$   & $\times$   &  $\times$   & $\times$ & $\times$ &  $\times$ &  $\times$ &  realvalue &  realvalue &  $\times$\\
 \hline
 #p#$^{**}$     & $\times$   & $\times$   &  $\times$   & $\times$ & $\times$ &  $\times$ &  $\times$ &  realvalue &  realvalue &  $\times$\\
 \hline
 #q#$^*$     & $\times$   & $\times$   &  $\times$   & $\times$ & $\times$ &  $\times$ &  $\times$ &  realvalue &  realvalue &  $\times$\\
 \hline
 #maxsteps#     & $\times$   & $\times$   &  $\times$   & $\times$ & $\times$ &  $\times$ &  $\times$ &  integer  &  integer & $\times$\\
 \hline
 #gridchoice#   & $\times$  & $\times$  & $\times$  & $\times$  & $\times$  & $\times$  & $\times$  & $\times$  & $\times$ & $\circ$\\
 \hline
 #tgrid#   & $\times$  & $\times$  & $\times$  & $\times$  & $\times$  & $\times$  & $\times$  & $\times$  & $\times$ & integer\\
 \hline
 #nrquantiles#   & $\times$  & $\times$  & $\times$  & $\times$  & $\times$  & $\times$  & $\times$  & $\times$  & $\times$ & integer\\
 \hline
 #nrbetween#   & $\times$  & $\times$  & $\times$  & $\times$  & $\times$  & $\times$  & $\times$  & $\times$  & $\times$ & integer\\
 \hline
 #period#      & $\times$   & integer     & $\times$  & $\times$      & $\times$  & $\times$ & $\times$ & $\times$  & $\times$ & $\times$\\
 \hline
 #map#      & $\times$   & $\times$     & $\times$  & {\em map object}  & $\times$  & {\em map object} & $\times$ & $\times$ & {\em map object} & $\times$ \\
 \hline
 #catspecific#      & $\triangle$   & $\triangle$     & $\triangle$  & $\triangle$ & $\triangle$  & $\triangle$ & $\triangle$ & $\triangle$ & $\triangle$ & $\times$ \\
 \hline
 #center#      & $\triangle$   & $\triangle$     & $\triangle$  & $\triangle$ & $\times$  & $\triangle$ & $\times$ & $\times$ & $\triangle$ & $\times$ \\
 \hline \hline
 $^*$ & \multicolumn{10}{l|}{positive values only}\\
 \hline
 $^{**}$ & \multicolumn{10}{l|}{negative values only}\\
 \hline
 $\times$    & \multicolumn{10}{l|}{not available} \\
 \hline
 $\bullet$  & \multicolumn{10}{l|}{admissible values are #0.5,1.5,2.5,3.5#} \\
 \hline
 $\triangle$   & \multicolumn{10}{l|}{available as boolean option (specified without supplying a value)} \\
 \hline
 $\circ$  & \multicolumn{10}{l|}{admissible values are #quantiles#, #equidistant# and #all#} \\
 \hline
\end{tabular}
{\em\centering \caption{\label{remlregtermsoptions} Terms and
options for remlreg objects.}}
\end{sidewaystable}

\subsubsection{Specifying the response distribution}\index{Response
distribution} \label{remlregfamilysyntax}

Supported univariate distributions are Gaussian, binomial (with
logit, probit or cumulative log-log link), Poisson and gamma. For
multivariate responses, {\em BayesX} supports multinomial logit
models for categorical responses with unordered categories and
cumulative as well as sequential logit and probit models for
categorical responses with ordered categories. Continuous survival
times as well as multi-state models can be analysed based on
semiparametric models with Cox-type hazard rates. An overview over
the supported models is given in \autoref{remlregfamilyopt}. The
distribution of the response is specified by adding the additional
option #family# to the (global) options list of the regression call.
For instance, #family=gaussian# defines the response to be Gaussian
distributed. In some cases, one or more additional options
associated with the specified response distribution can be
specified. An example is the #reference# option for multinomial
responses, which defines the reference category. In the following we
give detailed instructions on how to specify the various models.

\begin{table}[ht]
\begin{center}
\begin{tabular} {|l|p{5cm}|p{2.7cm}|p{1.7cm}|}
 \hline
 value of #family# & response distribution & link & options\\
 \hline
 \hline
 #family=gaussian#            & Gaussian              & identity & \\
 \hline
 #family=binomial#            & binomial              & logit & \\
 #family=binomialprobit#      & binomial              & probit & \\
 #family=binomialcomploglog#      & binomial              & complementary log-log & \\
 \hline
 #family=multinomial#         & unordered multinomial & logit & #reference#\\
 #family=multinomialcatsp#    & unordered multinomial (with category-specific covariates) & logit & #reference#\\
 \hline
 #family=cumprobit#           & cumulative multinomial   & probit & \\
 #family=cumlogit#            & cumulative multinomial   & logit & \\
 \hline
 #family=seqprobit#           & sequential multinomial   & probit & \\
 #family=seqlogit#            & sequential multinomial   & logit & \\
 \hline
 #family=poisson#             & Poisson               & log & \\
 \hline
 #family=gamma#               & gamma                 & log & \\
 \hline
 #family=cox#                 & continuous-time survival data & & #leftint#, #lefttrunc#\\
 \hline
 #family=multistate#                 & continuous-time multi-state data & & #state#, #lefttrunc#\\
 \hline
\end{tabular}
{\em \caption {\label{remlregfamilyopt} Summary of supported
response distributions.}}
\end{center}
\end{table}

\subsubsection*{Gaussian responses}

For Gaussian responses {\em BayesX} assumes $y_i | \eta_i,\sigma^2
\sim N(\eta_i,\sigma^2/weightvar_i)$ or, equivalently, in matrix
notation $y | \eta, \sigma^2 \sim N(\eta,\sigma^2C^{-1})$, where
$C=diag(weightvar_1,\dots,weightvar_n)$ is a known weight matrix.
Gaussian regression models are obtained by adding

#family=gaussian#

to the options list.

An optional weight variable {\em weightvar} can be specified to
estimate weighted regression models. For grouped Gaussian
responses, the weights represent the number of observations in the
groups if the $y_i$'s are the average of individual responses. If
the $y_i$s are the sum of responses in every group, the weights are
given by the reciprocal of the number of observations in the groups.
Of course, estimation of usual weighted regression models with
heteroscedastic errors is also possible. In this case, the weights
should be proportional to the reciprocal of the heteroscedastic
variances. If no weight variable is specified, {\em BayesX} assumes
$weightvar_i = 1$, $i=1,\dots,n$.

\subsubsection*{Binomial logit, probit and complementary log-log
models}

A binomial logit model is requested by the option

#family=binomial#

while a probit model is obtained with

#family=binomialprobit#

and a complementary log-log model with

#family=binomialcomploglog#

A additional weight variable may be specified, see
\autoref{remlregregresssyntax} for the syntax. {\em BayesX} assumes
that the weight variable corresponds to the number of replications
and the response variable to the number of successes. If the weight
variable is omitted, {\em BayesX} assumes that the number of
replications is one, i.e.~the values of the response must be either
zero or one.

\subsubsection*{Multinomial logit models}\index{Category-specific
covariates}\index{Multinomial logit model!Category-specific
covariates}

So far, {\em remlreg objects} support only multinomial logit models
and no probit models. A multinomial logit model without
category-specific covariates is specified by adding the option

#family=multinomial#

to the options list.

If there are category-specific covariates, the option has to be
altered to

#family=multinomialcatsp#

Category-specific covariates are included as follows: Suppose that
covariate #x# has been observed for a response variable with the
three categories 1, 2 and 3. Then you have to include the three
variable #x1#, #x2# and #x3# into your dataset. Within the
regression syntax, you have to specify

#x_catspecific#

to request a parametric effect of #x#

#x_catspecific(psplinerw2)#

for a nonparametric effect of #x# and

#x_catspecific*id(psplinerw2)#

to obtain a random effect of #x# with respect to the grouping
variable #id#. Currently BayesX only supports these three term types
for effects of category-specific effects.

For both #family=multinomial# and #family=multinomialcatsp# a second
option (#reference#) can be added to the options list to define the
reference category.  If the response variable has three categories
1, 2 and 3, the reference category can be set to 2, by adding

#reference=2#

to the options list. If the option is omitted, the {\em smallest}
number will be used as the reference category.

\index{Availability indicators}\index{Multinomial logit
Model!Availability indicators} If some categories are not available
for some observations, {\em BayesX} can account for this by
including either category-specific offsets or non-availability
indicators. Suppose again that the response has the three categories
1, 2 and 3. Then offset terms #o1#, #o2# and #o3# can be used to
account for varying choice sets. If a category is available, the
offset is simply set to zero, while a large negative value (i.e.
-1000) has to be assigned to the offset term of categories which are
not available. To account for the offset term,

#o_catspecific(offset)#

has to be added to the model specification. Of course, you can also
assign different values to the offsets, e.g. to account for a priori
differences in the availability of some categories.

The usage of offset terms to account for non-availability may in
some cases be numerically unstable (e.g. if several categories are
not available or if the reference category is not available).
Therefore an alternative possibility is to include non-availability
indicators #na1#, #na2# and #na3#. Each of the indicators is
assigned the value one if the corresponding category is not
available and zero otherwise. Within the regression syntax, the
non-availability indicator has to be specified as a global option,
i.e.

# ... , family=multinomialcatsp naindicator=na_catspecific#


\subsubsection*{Cumulative logit and probit models}\index{Cumulative
logit model}\index{Cumulative probit model}\index{Category-specific
effects}\index{Cumulative logit model!Category-specific
effects}\index{Cumulative probit model!Category-specific effects}

A cumulative logit model is specified by adding

#family=cumlogit#

to the options list, a cumulative probit is obtained by

#family=cumprobit#

In both cases, the reference category will always be the largest
value of the response.

Note, that in contrast to {\em bayesreg objects} {\em remlreg
objects} can deal with an arbitrary number of ordered categories.
However, for more than about 5 categories estimation may become
rather computer intensive and time demanding (depending on the size
of your data set).

By default, all effects in cumulative logit and probit models are
considered to be defined globally. To obtain category-specific
effects, the additional keyword #catspecific# has to be specified.
For example, the specification of the predictor

 #Y = W1(catspecific) + W2 + X1(psplinerw2, catspecific) + X2(psplinerw2)#

requests category-specific effects for the covariates #W1# and #X1#,
and global effects for the covariates #W2# and #X2#. Note that
complicated ordering restrictions have to be fulfilled for the
covariate-dependent thresholds defined implicitly by
category-specific effects. Therefore numerical problems are likely
to be observed in models with sparse data or a lot of
category-specific effects.

\subsubsection*{Sequential logit and probit models}\index{Sequential
logit model}\index{Sequential probit model}\index{Category-specific
effects}\index{Sequential logit model!Category-specific
effects}\index{Sequential probit model!Category-specific effects}

A sequential logit model is specified by adding

#family=cumlogit#

to the options list, while a sequential probit can be requested by

#family=cumprobit#

The reference category will always be the largest value of the
response.

Similar as in cumulative models, all effects in sequential logit and
probit models are considered to be defined globally by default. To
obtain category-specific effects, the additional keyword
#catspecific# has to be specified. For example, the specification of
the predictor

 #Y = W1(catspecific) + W2 + X1(psplinerw2, catspecific) + X2(psplinerw2)#

requests category-specific effects for the covariates #W1# and #X1#,
and global effects for the covariates #W2# and #X2#. In contrast to
cumulative models no ordering restrictions are imposed in sequential
models.

\subsubsection*{Poisson regression}

A Poisson regression model is specified by adding

#family=poisson#

to the options list.

A weight variable may be specified in addition, see
\autoref{remlregregresssyntax} for the syntax. For grouped Poisson
data, the weights must be the number of observations in a group and
the responses are assumed to be the average of individual responses.

\subsubsection*{Gamma distributed responses}

In the literature, the density function of the gamma distribution is
parameterized in various ways. In the context of regression
analysis, the density is usually parameterized in terms of the mean
$\mu$ and the scale parameter $s$. Then, the density of a gamma
distributed random variable $y$ is given by
\begin{equation}
\label{remlgammapar1} p(y) \propto y^{s-1}\exp(-\frac{s}{\mu} y)
\end{equation}
for $y > 0$. For the mean and the variance we obtain $E(y) = \mu$
and $Var(y) = \mu^2/s$. We write $y \sim G(\mu,s)$.

A second parameterization is typically employed for hyperparameters
#a# and #b# of priors for variance parameters in the context of
Bayesian hierarchical models. In this case, the density is given by
\begin{equation}
\label{remlgammapar2} p(y) \propto y^{a-1}\exp(-b y)
\end{equation}
for $y>0$. In this parameterization we obtain $E(y) = a/b$ and
$Var(y) = a/b^2$ for the mean and the variance, respectively. We
write $y \sim G(a,b)$

In {\em BayesX} a gamma distributed response variable is
parameterised in the first form (\ref{remlgammapar1}). For the $r$th
observation {\em BayesX} assumes  $y_r | \eta_r,\nu \sim
G(\exp(\eta_r),\nu/weightvar_r)$ where $\mu_r = \exp(\eta_r)$ is the
mean and $s=\nu/weightvar_r$ is the scale parameter. A gamma
distributed response is specified by adding

#family=gamma#

to the options list. An optional weight variable {\em weightvar} can
be specified to estimate weighted regression models, see
\autoref{remlregregresssyntax} for the syntax.

\subsubsection*{Continuous time survival analysis}

\textit{BayesX} offers two alternatives of estimating continuous
time survivals models with semiparametric predictor. The first alternative is to assume that all
time-dependent values are piecewise constant, leading to the so
called \textit{piecewise exponential model} (p.e.m.). The second
alternative is to estimate the log-baseline effect
$\log(\lambda_0(t))=f_0(t)$ based on a P-spline with second order
random walk penalty.

\subsubsection*{Piecewise exponential model
(p.e.m.)}\index{Piecewise exponential model}

Suppose that the data have been set up appropriately for a piecewise exponential model such that the
following modified data set is available
\vspace{0.5cm}\\
\begin{tabular}{c|c|c|c|c|c|c}
#y# & #indnr# & #a# & $\delta$ &  $\Delta$ &   #x1# &
#x#2\\\hline\hline
0 &  1 &   0.1 &   1  &  log(0.1) & 0  & 3\\
0  & 1   & 0.2  &  1  &  log(0.1) & 0 &  3\\
1  & 1   & 0.3  &  1  &  log(0.05)& 0  & 3\\\hline
0 &  2 &   0.1 &   0 &   log(0.1) & 1 &  5\\
0  & 2  &  0.2 &   0  &  log(0.02)& 1 &  5\\\hline
$\vdots$ & $\vdots$ & $\vdots$ & $\vdots$ & $\vdots$ & $\vdots$& $\vdots$\\
\end{tabular}
\vspace{0.5cm}\\
with indicator #y#, interval limit #a#, indicator of non-censoring
$\delta$ and offset $\Delta$. Let #x1# be a
covariate with linear effect and #x2# a continuous covariate with
nonlinear effect. Then the correct syntax for estimating a
p.e.m.~with a {\em remlreg object} named #r# is e.g.~as follows:

 #> r.regress y = a(rw1) + Delta(offset) + x1 + x2(psplinerw2), family=poisson# $\ldots$

or

 #> r.regress y = a(rw2) + Delta(offset) + x1 + x2(psplinerw2), family=poisson# $\ldots$

Note that a time-varying effect of an additional covariate #X# may
be estimated by simply adding the term

#X*a(rw1) or X*a(rw2)#

to the model statement.

\subsubsection*{Specifying a P-spline prior for the
log-baseline}\index{Cox model}\index{Continuous time survival
analysis}\index{Survival analysis}\index{Baseline}

For a continuous time survival model with a P-spline prior with
second order random walk penalty for the baseline effect,

#family=cox#

has to be specified in the options list. The number of knots and
degree of the P-spline prior for $f_0(t)$ can be specified as
additional options for the baseline term. Note that it is obligatory
that there is a baseline term specified for the vector of observed
duration times. The indicator of non-censoring $\delta_i$ has to be
specified as the dependent variable in the model statement. Data
augmentation and the specification of an offset term are not
required here. In the example above with survival data

\vspace{0.5cm}

\begin{tabular}{c|c|c|c}
  #t# &   $\delta$ &  #x1# &  #x2#\\\hline\hline
0.25  &  1  &    0  &  3\\\hline 0.12  &  0  &    1  &  5\\\hline
$\vdots$ & $\vdots$ & $\vdots$ & $\vdots$ \\
\end{tabular}
\vspace{0.5cm}\\
a continuous time survival model with a quadratic P-spline prior
with 15 knots for the log-baseline would be estimated as follows:

 #> r.regress delta = t(baseline,degree=2,nrknots=15)+ x1 + x2(psplinerw2),#\\
 #  family=cox# \ldots

Again a time-varying effect of a covariate #X# can be estimated by
simply adding the term

#X*time(baseline)#

to the model statement.

\subsubsection*{Interval censoring and left truncation}
\index{Interval censoring} \index{Left truncation} \index{Left
censoring}

Interval censoring and left truncation can be incorporated using the
additional options #leftint# and #lefttrunc# of {\em remlreg
objects}. These two variables represent the lower interval boundary
$T_{lo}$ and the left truncation time $T_{tr}$ as discussed in
section \ref*{intervalcensoring} of the methodology manual. The time
variable specified in the baseline statement corresponds to
$T_{up}$, the upper boundary of the interval. In general an
observation can now be described completely by the quadruple
$(T_{tr},T_{lo},T_{up},\delta)$, with
\begin{center}
\begin{tabular}{ll}
$T_{lo}=T_{up}$, $\delta=1$ & if the observation is uncensored,\\
$T_{lo}=T_{up}$, $\delta=0$ & if the observation is right censored,\\
$T_{lo}<T_{up}$, $\delta=0$ & if the observation is interval censored.\\
\end{tabular}
\end{center}
For left truncated observations we have $T_{tr}>0$ while $T_{tr}=0$
for observations which are not truncated.

An example for a statement that estimates a model with left
truncation and interval censoring is given by

 #> r.regress delta = tup(baseline)+ x1 + x2(psplinerw2), family=cox#\\
 #  lefttrunc=ttr leftint=tlo# \ldots

\subsubsection*{Continuous time multi-state
models}\index{Multi-state model}

Multi-state models describe the temporal development of discrete
phenomena in continuous time based on transition intensities for
each of the observable transition types. Consider for example a
multi-state model for human sleep as depicted in
\autoref{msmsleep_illustration_reml} and that the transition
intensities for the four possible transitions are specified as
\begin{center}
\begin{tabular}{rcl}
 $\lambda_{AS,i}(t)$ & $=$ & $\exp\left[g_0^{(AS)}(t) + b_i^{(AS)}\right],$\\[2mm]
 $\lambda_{SA,i}(t)$ & $=$ & $\exp\left[g_0^{(SA)}(t) + b_i^{(SA)}\right],$\\[2mm]
 $\lambda_{NR,i}(t)$ & $=$ & $\exp\left[g_0^{(NR)}(t) + c_i(t)g_1^{(NR)}(t) + b_i^{(NR)}\right]$\\[2mm]
 $\lambda_{RN,i}(t)$ & $=$ & $\exp\left[g_0^{(RN)}(t) + c_i(t)g_1^{(RN)}(t) + b_i^{(RN)}\right]$
\end{tabular}
\end{center}
Each of the transitions is parameterised in terms of a baseline
effect $g_0^{(h)}(t)$ and a transition specific frailty term (random
effect) $b_i^{(h)}$. In addition, time-varying effects
$g_1^{(h)}(t)$ of binary indicators $c_i(t)$ for a high blood level
of cortisol are introduced for the transitions between REM and
Non-REM.

\begin{figure}
\begin{center}
\setlength{\unitlength}{0.7cm}
\begin{picture}(15,7)
 \put(0,0) {\framebox(15,7){ }}
 \put(0.5,0.5) {\framebox(14,3){ }}

 \put(5.5,5.5) {\framebox(4,1){\sf Awake}}
 \put(1.5,1.5) {\framebox(4,1){\sf Non-REM}}
 \put(9.5,1.5) {\framebox(4,1){\sf REM}}

 \put(0.7,2.8) {\makebox(2,0.5)[lt]{\sf Sleep}}

 \put(7,5.25) {\vector(0,-1){1.5}}
 \put(8,3.75) {\vector(0,1){1.5}}

 \put(6.25,2.25) {\vector(1,0){2.5}}
 \put(8.75,1.75) {\vector(-1,0){2.5}}

 \put(6.5,0.9) {\makebox(2,0.75){\small$\lambda_{RN}(t)$}}
 \put(6.5,2.35) {\makebox(2,0.75){\small$\lambda_{NR}(t)$}}

 \put(5,4) {\makebox(1.5,1){\small$\lambda_{AS}(t)$}}
 \put(8.5,4) {\makebox(1.5,1){\small$\lambda_{SA}(t)$}}

\end{picture}
\caption{Schematic representation of sleep stages and transitions of
interest.\label{msmsleep_illustration_reml}}
\end{center}
\end{figure}

The corresponding data set should be arranged as follows:

\begin{verbatim}
 id  st  beg    end  tas tsa trn tnr cort corthigh
 1   2   0      1    0   1   0   0   52.6  0
 1   1   1      5    1   0   0   0   52.6  0
 1   2   5      8    0   1   0   0   52.6  0
 1   1   8      10   1   0   0   0   52.6  0
 1   2   10     36   0   0   0   0   52.6  0
 1   2   36     76   0   0   0   0   46.9  0
 1   2   76     108  0   0   0   1   47.5  0
 1   3   108    109  0   0   1   0   47.5  0
 1   2   109    110  0   0   0   1   47.5  0
 1   3   110    111  0   0   1   0   47.5  0
 1   2   111    115  0   0   0   1   47.5  0
 1   3   115    116  0   0   0   0   47.5  0
 1   3   116    126  0   0   1   0   37.4  0
 .   .    .      .   .   .   .   .     .   .
 .   .    .      .   .   .   .   .     .   .
 .   .    .      .   .   .   .   .     .   .
 2   2   0      12    0   1   0   0   22.5  0
 2   1   12     15    1   0   0   0   22.5  0
 2   1   15     28    0   1   0   0   88.6  1
 .   .    .      .   .   .   .   .     .   .
 .   .    .      .   .   .   .   .     .   .
 .   .    .      .   .   .   .   .     .   .
\end{verbatim}

Each path observed for the multi-state model is transformed into
several lines in the data set, where #id# identifies the original
paths. In the above example, parts of the first two observations are
displayed. Each line of the data set represents a time interval
identified by the variables #beg# and #end#. Variable #st# indicates
the current state of the process. Note that the states have to be
numbered consecutively from 1 to $H$. Since we are considering
continuous time scales, an observation should start at $t=0$ (unless
the observation is left truncated) and the variables #beg# and #end#
should be generated so that within each observation process #beg#
equals the value of #end# in the previous row (unless observations
are fragmentary only).

The variables #tas#, #tsa#, #trn# and #tnr# are binary indicators
for the four transitions sleep $\rightarrow$ awake (#tsa#), awake
$\rightarrow$ sleep (#tas#), Non-REM $\rightarrow$ REM (#tnr#) and
REM $\rightarrow$ Non-REM (#trn#). Such an indicator equals one if
the corresponding transition is observed at the end of the interval
and zero otherwise. Note that there are lines in the data set, where
none of the transitions is observed. These correspond to intervals
where the value of the time-varying covariate #cort#
(cortisol-level) changes. The variable #corthigh# is a dichotomized
version of #cort# which indicates a high level of cortisol
(#cort#$>$60).

The model specified above is estimated by entering the following
command

\label{msm_code}
\begin{verbatim}
> remlreg msm
> msm.mregress tas = end(baseline) + id(random):
               tsa = end(baseline) + id(random):
               trn = end(baseline) + corthigh*end(baseline) + id(random):
               tnr = end(baseline) + corthigh*end(baseline) + id(random),
 family=multistate lefttrunc=beg state=st using sleep
\end{verbatim}

Note that a separate model equation has to specified for each
transition with the binary transition indicator as response. Instead
of method #regress#, method #mregress# has to be called since
multiple model equations are combined. The right and the left
boundary of the time intervals have to specified as covariate for
the the baseline effect and as global option #lefttrunc#,
respectively. Similarly, the state variable has to specified via the
global option #state#.

\subsection{Options}
\label{remlregregressoptions}

\subsubsection*{Options for controlling the estimation process}
\label{remlest_options}

\begin{itemize}
\item #eps = #{\em realvalue } \\
Defines the termination criterion of the estimation process. If both
the relative changes in the regression coefficients and the variance
parameters are less than #eps#, the estimation process is
assumed to have converged.\\
DEFAULT: #eps = 0.00001#

\item #lowerlim = #{\em realvalue } \\
Since small variances are close to the boundary of their parameter
space, the usual Fisher-scoring algorithm for their determination
has to be modified. If the fraction of the penalized part of an
effect relative to the total effect is less than #lowerlim#, the
estimation of the corresponding variance is stopped and the
estimator is defined to be the current value of the variance (see
section \ref*{glmmmeth} of the methodology manual for details).\\
DEFAULT: #lowerlim = 0.001#

\item #maxit = #{\em integer } \\
Defines the maximum number of iterations to be used in estimation.
Since the estimation process will not necessarily converge, it may
be useful to define an upper bound for the number of iterations.
Note, that {\it BayesX} returns results based on the current values
of all parameters even if no convergence could be achieved within
#maxit# iterations, but a warning message will be printed
in the {\it output window}.\\
DEFAULT: #maxit=400#

\item #maxchange = #{\em realvalue } \\
Defines the maximum value that is allowed for relative changes in
parameters in one iteration to prevent the program from crashing
because of numerical problems. Note, that {\it BayesX} produces
results based on the current values of all parameters even if the
estimation procedure is stopped due to numerical problems, but an
error message will be printed in the {\it output window}.\\
DEFAULT: #maxchange=1000000#
\end{itemize}

\subsubsection*{Options for the analysis of survival times and multi-state models}
\label{remlest_survival_options}

\begin{itemize}
\item #leftint = #{\em variablename}\\
Gives the name of the variable that contains the lower (left)
boundary $T_{lo}$ of the interval $[T_{lo},T_{up}]$ for an interval
censored observation. For right censored or uncensored observations
we have to specify $T_{lo}=T_{up}$. If #leftint# is missing, all
observations are assumed to be right censored or uncensored,
depending on the corresponding value of the censoring indicator.

\item #lefttrunc = #{\em variablename}\\
Option #lefttrunc# specifies the name of the variable containing the
left truncation time $T_{tr}$. For observations that are not
truncated, we have to specify $T_{tr}=0$. If #lefttrunc# is missing,
all observations are assumed to be not truncated. For multi-state
models variable #lefttrunc# specifies the left endpoint of the
corresponding time interval (compare page~\pageref{msm_code}).

\item #state = #{\em variablename}\\
For multi-state models, #state# specifies the current state of the
process (compare page~\pageref{msm_code}).
\end{itemize}

\subsubsection*{Further options} \label{remlreg_further_options}

\index{Credible intervals} \index{Credible intervals!Changing the
nominal level} \index{Changing the nominal level of credible
intervals}\index{Remlreg object!Credible intervals}
\begin{itemize}
\item \label{remlreglevel1} #level1 = #{\em integer} \\
Besides the posterior mode, #regress# provides (approximate)
pointwise posterior credible intervals for every effect in the
model. By default, {\em BayesX} computes credible intervals for
nominal levels of 80\% and 95\%. The option #level1# allows to
redefine one of the nominal levels (95\%). Adding, for instance,

#level1=99 #

to the options list leads to the computation of credible intervals
for a nominal level of 99\% rather than 95\%.
\item \label{remlreglevel2} #level2 = #{\em integer} \\
Besides the posterior mode, #regress# provides (approximate)
pointwise posterior credible intervals for every effect in the
model. By default, {\em BayesX} computes credible intervals for
nominal levels of 80\% and 95\%. The option #level2# allows to
redefine one of the nominal levels (80\%). Adding, for instance,

#level2=70#

to the options list leads to the computation of credible intervals
for a nominal level of 70\% rather than 80\%.
\end{itemize}

\subsection{Estimation output}

The way the estimation output is presented depends on the estimated
model. Estimation results for fixed effects are displayed in a
tabular form in the {\em output window} and/or in a log file (if
created before). This table will contain the posterior mode, the
standard deviation, p-values and an approximate 95\% credible
interval. Other credible intervals may be obtained by specifying the
#level1# option, see \autoref{remlregregressoptions} for details.
Additionally, a file replicating results for the fixed effects is
created. The name of this file is supplied in the {\em output
window} and/or in a log file.

Estimated nonparametric effects are presented in a different way.
Here, results are stored in external ASCII-files that can be read
into any general purpose statistics program (e.g. STATA, R, S-plus)
to further analyze and/or visualize the results. The structure of
these files is as follows: There will be one file for every
nonparametric effect in the model. The names of the files and the
storing directory are displayed in the {\em output window} and/or a
log file. The files contain ten columns (for main effects) or eleven
columns (for interaction effects). The first column contains a
parameter index (starting with one), the second column (and the
third column if the estimated effect is an interaction) contain the
values of the covariate(s) whose effect has been estimated. In the
following columns the estimation results are given in form of the
posterior mode, the lower boundaries of the (approximate) 95\% and
80\% credible intervals, the standard deviation and the upper
boundaries of the 80\% and 95\% credible intervals. The last two
columns contain approximations to the posterior probabilities based
on nominal levels of 95\% and 80\%. A value of 1 corresponds to a
strictly positive 95\% or 80\% credible interval while a value of -1
to a strictly negative credible interval. A value of 0 indicates
that the corresponding credible interval contains zero. Other
credible intervals and posterior probabilities may be obtained by
specifying the #level1# and/or #level2# option, see
\autoref{remlregregressoptions} for details. As an example, compare
the following lines, which are the beginning of a file containing
the results for a nonparametric effect of a particular covariate, x
say:

\footnotesize
 intnr \,\, x \,\, pmode \,\, ci95lower \,\, ci80lower \,\, std \,\, ci80upper \,\, ci95upper \,\, pcat95 \,\, pcat80\\
 1 \,\, -2.87694 \,\, -0.307921 \,\, -0.886815 \,\, -0.686408 \,\, 0.295295 \,\, 0.070567   \,\, 0.270973 \,\, 0 \,\, 0\\
 2 \,\, -2.86203 \,\, -0.320479 \,\, -0.885375 \,\, -0.689815 \,\, 0.288154 \,\, 0.0488558  \,\, 0.244416 \,\, 0 \,\, 0\\
 3 \,\, -2.8515  \,\, -0.329367 \,\, -0.88473  \,\, -0.69247  \,\, 0.283292 \,\, 0.0337362  \,\, 0.225997 \,\, 0 \,\, 0\\
 4 \,\, -2.85066 \,\, -0.330072 \,\, -0.884692 \,\, -0.692689 \,\, 0.282913 \,\, 0.0325457  \,\, 0.224549 \,\, 0 \,\, 0\\
 5 \,\, -2.82295 \,\, -0.3535   \,\, -0.884544 \,\, -0.700703 \,\, 0.270887 \,\,-0.00629671 \,\, 0.177545 \,\, 0 \,\, -1\\
 6 \,\, -2.79856 \,\, -0.37418  \,\, -0.886192 \,\, -0.708939 \,\, 0.261178 \,\,-0.0394208  \,\, 0.137832 \,\, 0 \,\, -1\\
 7 \,\, -2.79492 \,\, -0.377272 \,\, -0.886579 \,\, -0.710263 \,\, 0.259798 \,\,-0.0442813  \,\, 0.132035 \,\, 0 \,\, -1\\
 8 \,\, -2.79195 \,\, -0.379788 \,\, -0.886921 \,\, -0.711358 \,\, 0.258689 \,\,-0.0482183  \,\, 0.127345 \,\, 0 \,\, -1\\
 9 \,\, -2.78837 \,\, -0.382834 \,\, -0.887367 \,\, -0.712704 \,\, 0.257363 \,\,-0.0529641  \,\, 0.1217   \,\, 0 \,\, -1
\normalsize

Note that the first row of the files always contains the names of
the columns.

The estimated nonlinear effects can be visualized by using either the graphics capabilities of {\em BayesX} or the {\it BayesX}
R package, see \autoref{bayesxplot} and \autoref{rpackage}, respectively. Of course, any other (statistics) software package
with plotting facilities may be used as well.

Estimation results for the variances and the smoothing parameters of nonparametric effects are printed in the {\em output
window} and/or a log file. Additionally, a file is created containing the same information. For example, the file corresponding
to the nonparametric effect presented above contains:

\footnotesize
 variance \,\, smoothpar \,\, stopped\\
 0.0492324 \,\, 20.3118 \,\, 0
\normalsize

The value in the last row indicates whether the estimation of the
variance has been stopped before convergence. A value of 1
corresponds to a 'stopped' variance.

\subsection{Examples}

Suppose that we have a data set #test# with a binary response
variable #y#, and covariates #x1#, #x2#, #x3#, #t# and #region#,
where #t# is assumed to be a time scale measured in months and
#region# indicates the geographical region an observation belongs
to. Suppose further that we have already created a {\em remlreg
object} #r#.

\subsubsection*{Fixed effects}

We first specify a model with #y# as the response variable and
fixed effects for the covariates #x1#, #x2# and #x3#. Hence the
predictor is

$$
\eta = \gamma_0 + \gamma_1 x1 + \gamma_2 x2 + \gamma_3 x3
$$

This model is estimated by typing:

#> r.regress y = x1 + x2 + x3, family=binomial using test#

By specifying option #family=binomial#, a binomial logit model is
estimated. A probit model can be obtained by specifying
#family=binomialprobit#.

\subsubsection*{Additive models}

Suppose now that we want to allow for possibly nonlinear effects
of #x2# and #x3#. Defining cubic P-splines with second order
random walk penalty as smoothness priors, we obtain

 #> r.regress y = x1 + x2(psplinerw2) + x3(psplinerw2), family=binomial using test#

which corresponds to the predictor

$$
\eta = \gamma_0 + \gamma_1 x1 + f_1(x2) + f_2(x3).
$$

If the response is not binary but categorical with unordered
categories 1, 2 and 3, we can estimate a multinomial logit model by
typing:

 #> r.regress y = x1 + x2(psplinerw2) + x3(psplinerw2), family=multinomial#\\
 #  reference=2 using test#

In this case, #family=binomial# has to be altered to
#family=multinomial#, and the option #reference=2# was added to
define the value 2 as the reference category.

\subsubsection*{Time scales}

In the next step we extend the model by incorporating an additional
trend and a flexible seasonal component for the time scale #t#:

 #> r.regress y = x1 + x2(psplinerw2) + x3(psplinerw2) +  #\\
 #  t(psplinerw2) + t(season,period=12), family=binomial using test#

Note that we passed the period of the seasonal component as a
second argument.

\subsubsection*{Spatial covariates}

To incorporate a structured spatial effect, we have to create a {\em
map object} first. Afterwards we read the boundary information of
the different regions (polygons that form the regions, neighbors
etc.). If you are unfamiliar with {\em map objects} please read
\autoref{map} first.

#> map m# \\
#> m.infile using c:\maps\map.bnd#

Since we usually need the map again in further sessions, we store
it in {\em graph file} format, because reading {\em graph files}
is much faster than reading {\em boundary files}.

#> m.outfile , graph using c:\maps\mapgraph.gra#

We can now augment our predictor with a spatial effect:

 #> r.regress y = x1 + x2(psplinerw2) + x3(psplinerw2) + t(psplinerw2)#\\
 #  + t(season,period=12) + region(spatial,map=m), family=binomial using test#

In some situations it may be reasonable to incorporate  an
additional unstructured  random effect into the model in order to
split the total spatial effect into a structured and an unstructured
component. This is achieved by

#> r.regress y = x1 + x2(psplinerw2) + x3(psplinerw2) + t(psplinerw2)#\\
#  + t(season,period=12) + region(spatial,map=m) + region(random),#\\
#  family=binomial using test#

\section{Global options}
\label{remlregglobopt} \index{Remlreg object!Global options}

The purpose of global options is to affect the global behavior of
a {\em remlreg object}. The main characteristic of global options
is, that they are not associated with a certain method.

The syntax for specifying global options is

{\em objectname}.{\em optionname} = {\em newvalue}

where {\em newvalue} is the new value of the option. The type of
the value depends on the respective option.

Currently only one global option is available for {\em remlreg
objects}:

\begin{itemize}
\item #outfile = #{\em filename} \\
By default, the estimation output produced by the #regress#
procedure will be written to the default output directory, which
is

{\em$<$INSTALLDIRECTORY$>$}#\output#

The default file name is composed of the name of the {\em remlreg
object} and the type of the file. For example, if you estimated a
nonparametric effect for a covariate #X#, say, using a P-spline,
then the estimation output will be written to

{\em$<$INSTALLDIRECTORY$>$}#\output\r_f_X_pspline.res#

where #r# is the name of the {\em remlreg object}. In most cases,
however, it may be necessary to save estimation results into a
different directory and/or under a different file name than the
default. This can be achieved using the #outfile# option. Here, you
have to specify the directory where the output should be stored and
a base file name. This base file name should not be a complete file
name. For example specifying

#outfile = c:\data\res1#

would cause {\em BayesX} to store the estimation result for the
nonparametric effect of #X# in file

#c:\data\res1_f_X_pspline.res#
\end{itemize}

\section{Visualizing estimation results}

Visualization of estimation results is described in
\autoref{visualization}
