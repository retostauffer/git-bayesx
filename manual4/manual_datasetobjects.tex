\chapter{dataset objects} 
 \label{chap_data}
 \label{datasetobj} 
 \index{Dataset objects} 
 \index{Dataset}

In \BayesX, {\em dataset objects} are used to store, manage, and manipulate data. A new {\em dataset object} is created by typing

 #> dataset# {\em objectname}

where {\em objectname} is the name to be assigned to the data set. After the creation of a {\em dataset object}, one can apply the various methods discussed below.

Note that in the current version of \BayesX only numerical variables are allowed while character-valued variables, for example, are not yet supported by \BayesX and the attempt to read such variables will raise an error message.

\section{Method descriptive} 
 \label{descriptive} 
 \index{Summary statistics} 
 \index{Descriptives} 
 \index{Dataset!Descriptive command}

\begin{stanza}{Description}

Method #descriptive# calculates and displays a number of univariate summary statistics. To be more specific, the method computes the number of observations, the mean, median, standard deviation, minimum and maximum of the specified variables.
\end{stanza}

\begin{stanza}{Syntax}
 #> #{\em objectname}#.descriptive# {\em varlist} [#if# {\em expression}]

Method #descriptive# computes summary statistics for the variables in {\em varlist}. An optional #if# statement can be added to analyze only a part of the data.
\end{stanza}

\begin{stanza}{Options}
 not allowed
\end{stanza}

\begin{stanza}{Example}
The statement

#> d.descriptive x y#

computes summary statistics for the variables #x# and #y#. The statement

#> d.descriptive x y if x>0#

restricts the analysis to observations with #x>0#.
\end{stanza}


\section{Method drop}
 \label{drop} 
 \index{Dropping variables} 
 \index{Dropping observations} 
 \index{Dataset!Drop command}

\begin{stanza}{Description}
Method #drop# deletes variables or observations from the data set.
\end{stanza}

\begin{stanza}{Syntax}
#> #{\em objectname}.#drop# {\em varlist}

#> #{\em objectname}.#drop if# {\em expression}

The first command is used to delete the variables specified in {\em varlist} from the data set. The second statement deletes certain observations, i.e. all observations for which {\em expression} is true will be deleted.
\end{stanza}

\begin{stanza}{Options}
not allowed
\end{stanza}

\begin{stanza}{Example}
The statement

#> credit.drop account duration#

deletes the variables #account# and #duration# from the credit scoring data set. With the statement

#> credit.drop if marstat = 2#

all observations with #marstat = 2#, i.e. all persons living alone, will be deleted. The statement

#> credit.drop account duration if marstat = 2#

will raise the error

#ERROR: dropping variables and observations in one step not allowed#

because it is not allowed to drop variables and observations in one single command.
\end{stanza}


\section{Functions and Expressions}
 \label{expression} 
 \index{Expressions}

The primary use of expressions is to generate new variables or change existing variables, see \autoref{generate} and \autoref{replace}, respectively. Expressions can also be used in #if# statements to force \BayesX to apply a method only to
observations where the boolean expression in the #if# statement is true. The following statements are all examples of expressions:

 #2+2# 
 
 #log(amount)# 
 
 #1*(age <= 30)+2*(age > 30 & age <= 40)+3*(age > 40)# 
 
 #age=30# 
 
 #age+3.4*age^2+2*age^3# 
 
 #amount/1000#


\subsection{Operators}
 \index{Expressions!Operators} 
 \index{Operators}

Three different types of operators can be included in expressions: arithmetic, relational and logical operators.

\subsubsection{Arithmetic operators}
 \index{Operators!Arithmetic}

The arithmetic operators are #+# (addition), #-# (subtraction), #*# (multiplication), #/# (division), #^# (raise to a power) and the prefix #-# (negation). Any arithmetic operation on a missing value or an undefined arithmetic operation (such as division by zero) yields a missing value.

\begin{stanza}{Example}
The expression

#(x+y^(3-x))/(x*y)#

implements the formula
$$
\frac{x+y^{3-x}}{x\cdot y}
$$
and evaluates to missing if #x# or #y# is missing or zero.
\end{stanza}


\subsubsection{Relational operators}
 \index{Operators!Relational}

The relational operators are #># (greater than), #<# (less than), #>=# (greater than or equal), #<=# (less than or equal), #=# (equal) and #!=# (not equal). Relational expressions are either 1 (if the expression is true) or 0 (if the expression is false).

\begin{stanza}{Example}
Relational operators can, for example, be used to create indicator variables. The following statement generates a new variable #amountcat#, with value 1 if #amount<=10# and value 2 if #amount>10#.

#> credit.generate amountcat = 1*(amount<=10)+2*(amount>10)#

Another useful application of relational operators is in #if# statements. For example, changing an existing variable only when a certain condition holds can be achieved by the following command:

#> credit.replace amount = NA if amount <= 0#

This sets all observations missing where #amount<=0#.
\end{stanza}


\subsubsection{Logical operators}
 \index{Operators!Logical}

The logical operators are #&# (and) and #|# (or).

\subsubsection*{Example}

Suppose you want to generate a variable #amountind# whose value is 1 for married people with amount greater than 10 and 0 otherwise. This can be achieved by typing

#> credit.generate amountind = 1*(marstat=1 & amount>10)#


\subsubsection{Order of evaluation of the operators}
 \index{Operators!Order of evaluation}

The order of evaluation (from first to last) of operators is

 #^# 
 
 #/#, #*#
 
 #-#, #+#
 
 #!=#, #>#, #<#, #<=#, #>=#, #=#
 
 #&#, #|#

Brackets can be used to change the order of evaluation.


\subsection{Functions}
 \index{Functions}

Functions are a further component of expressions. Any function is indicated by the function name followed by an opening and a closing parenthesis. Inside the parentheses, one or more arguments can be specified. The argument(s) of a function can again be expressions, including calls to further functions. Multiple arguments of a function are separated by commas. All functions return missing values when given missing values as arguments or when the result is undefined.

All mathematical function currently available in \BayesX are referenced in \autoref{mathfunc}. Statistical functions can be
found in \autoref{statfunc}.

 \index{Functions!abs} 
 \index{Functions!cos} 
 \index{Functions!sin}
 \index{Functions!exp} 
 \index{Functions!floor} 
 \index{Functions!lag}
 \index{Functions!logarithm} 
 \index{Functions!square root}
 \index{Functions!Bernoulli distributed random numbers}
 \index{Functions!Binomial distributed random numbers}
 \index{Functions!Exponential distributed random numbers}
 \index{Functions!Normally distributed random numbers}
 \index{Functions!Uniformly distributed random numbers}

\begin{table}[ht]
\begin{center}
\begin{tabular}{|l|l|}
\hline
{\bf Function} & {\bf Description} \\
\hline \hline
#abs(#x#)# & absolute value \\
#cos(#x#)# & cosine of radians \\
#exp(#x#)# & exponential \\
#floor(#x#)# & returns the integer obtained by truncating $x$. \\
& Thus #floor(5.2)# evaluates to 5 as does #floor(5.8)#. \\
#lag(#x#)# & lag operator \\
#log(#x#)# & natural logarithm \\
#log10(#x#)# & log base 10 of $x$ \\
#sin(#x#)# & sine of radians \\
#sqrt(#x#)# & square root \\
\hline
\end{tabular}
{\em\caption{\label{mathfunc} List of mathematical functions.}}
\end{center}
\end{table}



\begin{table}[ht]
\begin{center}
\begin{tabular}{|l|p{11cm}|}
\hline
{\bf Function} & {\bf Description} \\
 \hline \hline
 #bernoulli(#$p$#)# & returns Bernoulli distributed random numbers with probability of success $p$. If $p$ is not
 within the interval $[0;1]$, a missing value will be returned. \\
 \hline
 #binomial(#$n$#,#$p$#)# & returns $B(n;p)$ distributed random numbers. Both, the number of trials $n$ and the probability of success $p$ may be expressions. If $n < 1$, a missing value will be returned. If $n$ is not integer valued, the number of trials will be $[n]$. If $p$ is not within the interval $[0;1]$, a  missing value will be returned. \\
 \hline
 #exponential(#$\lambda$#)# & returns exponentially distributed  random numbers with parameter $\lambda$.  If $\lambda \leq 0$, a missing value will be returned. \\
 \hline
 #normal()# & returns standard normally distributed random  numbers; $N(\mu,\sigma^2)$ distributed random numbers may be generated with $\mu + \sigma$*normal(). \\
 \hline
 #uniform()# & returns uniformly distributed pseudo-random numbers on the interval $(0,1)$ \\
 \hline
\end{tabular}
{\em \caption{\label{statfunc} List of statistical functions}}
\end{center}
\end{table}


\subsection{Constants}
 \index{Expressions!Constants} 
 \index{Number of observations}
 \index{Missing values} 
 \index{$\pi$} 
 \index{Current observation}

\autoref{constant} lists all constants that can be used in expressions.


\begin{table}[ht]
\begin{center}
\begin{tabular}{|l|l|}
\hline
Constant & Description \\
\hline \hline
\texttt{\_n} & contains the number of the current observation.  \\
\texttt{\_N }& contains the total number of observations in the data set. \\
\texttt{\_pi} & contains the value of $\pi$. \\
\texttt{NA} & indicates a missing value \\
.  & indicates a missing value \\
\hline
\end{tabular}
{\em \caption{\label{constant} List of constants}}
\end{center}
\end{table}


\subsubsection*{Example}

The following statement generates a variable #obsnr# whose value is 1 for the first observation, 2 for the second and so on.

#> credit.generate obsnr = _n#

The command

#> credit.generate nrobs = _N#

generates a new variable {\em nrobs} whose values are all equal to the total number of observations, say 1000, for all observations.

\subsection{Explicit Subscribing}
 \index{Expressions!Explicit subscribing} 
 \index{Subscribing}

Individual observations on variables can be referenced by subscribing the variables. Explicit subscripts are specified by
the variable name with square brackets that contain an expression. The result of the subscript expression is truncated to an integer, and the value of the variable for the indicated observation is returned. If  the value of the subscript expression is less than 1 or greater than the number of observations in the data set, a missing value is returned.

\subsubsection*{Example}

Explicit subscribing combined with the variable #_n# (see \autoref{constant}) can be used to create lagged values of a
variable. For example, the lagged value of a variable #x# in a data set #data# can be created by

#> data.generate xlag = x[_n-1]#

Note that #xlag# can also be generated using the #lag# function

#> data.generate xlag = lag(x)#


\section{Method generate}
 \label{generate} 
 \index{Generating new variables}
 \index{Dataset!Generate command}

\begin{stanza}{Description}
Method #generate# is used to create a new variable in an existing {\em dataset object}.
\end{stanza}

\begin{stanza}{Syntax}
#> #{\em objectname}.#generate# {\em newvar} = {\em expression}

Method #generate# creates a new variable with name {\em newvar} (compare \autoref{varnames} for valid variable names). The values of the new variable are specified by {\em expression}.
\end{stanza}

\begin{stanza}{Options}
not allowed
\end{stanza}

\begin{stanza}{Example}
The following command generates a new variable called #amount2# which contains the squared amount in the credit scoring data set.

#> credit.generate amount2 = amount^2#

If you try to change the variable currently generated, for example by typing

#> credit.generate amount2 = amount^0.5#

the error message

#ERROR: variable amount2 is already existing #

will occur. This prevents unintentional changes of an existing variable. Existing variables can be changed with method
#replace#, see \autoref{replace}.

If you want to generate an indicator variable #largeamount# for amounts exceeding a certain value, say 3.5, you have to type

#> credit.generate largeamount = 1*(amount>3.5)#

The variable #largeamount# takes the value 1 if #amount# is larger than 3.5 and 0 otherwise.
\end{stanza}


\section{Method infile}
 \label{infile} 
 \index{Reading data from ASCII files}
 \index{Dataset!Infile command}

\begin{stanza}{Description}
Reads data from an ASCII file.
\end{stanza}

\begin{stanza}{Syntax}
#> #{\em objectname}.#infile #[{\em varlist}] [{\em , options}] #using# {\em filename}

Reads the data stored in {\em filename}. The variable names can either be specified in {\em varlist} or be extracted from the first line of the file. To be more specific, if {\em varlist} is empty, it is assumed that the first row of the file contains the variable names separated by blanks or tabs. The observations do not have to be stored in a special format, except that successive observations should be separated by one or more blanks or tabs. The first value read from the file will be the value of the first variable of the first observation, the second value will be the value of the second variable of the first observation, and so on. An error message will occur if no values can be read for the last observation for some variables.

The period '.' or '#NA#' should be used to indicate missing values.

Note that in the current version of \BayesX only numerical variables are allowed. Thus, the attempt to read string valued variables, for example, will cause an error.
\end{stanza}

\begin{stanza}{Options}
\begin{itemize}
\item #missing = # {\em missingsigns} \\
By default, the period '.' or '#NA#' indicate missing values. If the missing values are indicated by a different sign, you can specify them in the #missing# option. For example #missing = MIS# defines '#MIS#' as an indicator for a missing value. Note that '.' and '#NA#' remain valid indicators for missing values, even if the missing option is specified.

\item #maxobs = #{\em integer} \\
If you work with large data sets, you may observe the problem that reading a data set using the #infile# command is time consuming. The reason for this problem is that \BayesX does not know the number of observations in advance and therefore is unable to allocate enough memory. Therefore new memory has to be allocated whenever a certain amount of memory is exceeded. Specifying option #maxobs# allows to circumvent this problem and to considerably reduce the computing time. When #maxobs# is specified, \BayesX can allocate enough memory to store at least {\em integer} observations before new memory must be reallocated. Suppose for example that your data set consists of approximately 100,000 observations. Then specifying #maxobs=105000# allocates enough memory to read in the data set quickly. Note that #maxobs=105000# does not require the data set to have exactly 105,000 observations. It only means that new memory will have to be allocated when the number of observations exceeds the 105,000 observations limit.
\end{itemize}
\end{stanza}

\begin{stanza}{Example}
Suppose we want to read a data set stored in #c:\data\testdata.raw# containing the two variables #var1# and #var2#. The first few rows of the datafile could look like this:

var1 var2 \\
2 2.3 \\
3 4.5 \\
4 6 \\
...

To read the data set, we have to create a new {\em dataset object} first, say #testdata#. We proceed by reading the data using the #infile# command:

#> dataset testdata# \\
#> testdata.infile using c:\data\testdata.raw#

If the first row of the data set file contains no variable names, the second command has to be altered to:

#> testdata.infile var1 var2 using c:\data\testdata.raw#

If furthermore the data set is large with about 100,000 observations, the #maxobs# option is very useful to
reduce reading time. Typing for example

#> testdata.infile var1 var2 , maxobs=101000 using c:\data\testdata.raw#

will be much faster than the command without option #maxobs#.
\end{stanza}


\section{Method outfile}
 \label{outfile} 
 \index{Writing data to a file} 
 \index{Saving data in an ASCII file} 
 \index{Dataset!Outfile command}

\begin{stanza}{Description}
Method #outfile# writes data to a file in ASCII format.
\end{stanza}

\begin{stanza}{Syntax}
#> #{\em objectname}.#outfile# [{\em varlist}] [#if# {\em expression}] [{\em , options}] #using# {\em filename}

#outfile# writes the variables specified in {\em varlist} to the file with name {\em filename}. If {\em varlist} is omitted in the outfile statement, {\em all} variables in the data set will be included. Each row in the data file will correspond to one observation. Different variables will be separated by blanks. Optionally, an #if# statement can be used to store only those observations where a certain boolean expression is true.
\end{stanza}

\begin{stanza}{Options}
\begin{itemize}
\item #header# \\
Specifying the #header# option causes \BayesX to write the variable names to the first row of the created data file.
\item #replace# \\
The #replace# option allows \BayesX to overwrite an already existing data file. If #replace# is omitted and the file specified in {\em filename} is already existing, an error will be raised. This prevents you from unintentionally overwriting an existing file.
\end{itemize}
\end{stanza}

\begin{stanza}{Example}
The statement

#> credit.outfile using c:\data\cr.dat#

writes the complete credit scoring data set to #c:\data\cr.dat#. To generate two different ASCII data sets for married people and people living alone, you could type

#> credit.outfile if marstat = 1 using c:\data\crmarried.dat# \\
#> credit.outfile if marstat = 2 using c:\data\cralone.dat#

If you want to store only the two variables #y# and #amount#, you could type

#> credit.outfile y amount using c:\data\cr.dat#

This will raise the error message

#ERROR: file c:\data\cr.dat is already existing#

because #c:\data\cr.dat# has already been created. You can overwrite the file using the #replace# option

#> credit.outfile y amount, replace using c:\data\cr.dat#
\end{stanza}


\section{Method rename}
 \index{Renaming variables}
 \index{Dataset!Rename command}
 \label{rename}

\begin{stanza}{Description}
#rename# is used to change variable names.
\end{stanza}

\begin{stanza}{Syntax}
#> #{\em objectname}.#rename# {\em varname} {\em newname}

#rename# changes the name of {\em varname} to {\em newname}. {\em newname} must be a valid variable name, see \autoref{varnames} for valid variable names.
\end{stanza}

\begin{stanza}{Options}
not allowed
\end{stanza}


\section{Method replace}
 \label{replace} 
 \index{Changing existing variables}
 \index{Dataset!Replace command}

\begin{stanza}{Description}
 #replace# changes values of an existing variable.
\end{stanza}

\begin{stanza}{Syntax}
 #> #{\em objectname}.#replace# {\em varname} = {\em expression} [#if# {\em expression}]

#replace# changes some or all values of the variable {\em varname} to the values specified in {\em expression}. If {\em varname} is not existing, an error will be raised. An optional #if# statement can be used to change the values of the variable only if the boolean expression {\em expression} is true.
\end{stanza}

\begin{stanza}{Options}
not allowed
\end{stanza}

\begin{stanza}{Example}
The statement

#> credit.replace amount = NA if amount<0#

changes the values of the variable #amount# in the credit scoring data set to missing if #amount<0#.
\end{stanza}


\section{Method set obs}
 \label{setobs}

\begin{stanza}{Description}
 #set obs# changes the number of observations in a data set.
\end{stanza}

\begin{stanza}{Syntax}
#> #{\em objectname}.#set obs# = {\em intvalue}

#set obs# raises the number of observations in the data set to {\em intvalue}. The new number of observations has to be greater or equal than the current number to prevent you from unintentionally deleting parts of the data currently in memory. Observations may be eliminated using the #drop# statement, see \autoref{drop}. The values of newly created observations will be set to the missing value.
\end{stanza}


\section{Method tabulate}
 \label{tabulate} 
 \index{Tabulate} 
 \index{Table of frequencies}
 \index{One way table of frequencies} 
 \index{Dataset!Tabulate command}

\begin{stanza}{Description}
Method #tabulate# calculates and displays a frequency tables.
\end{stanza}

\begin{stanza}{Syntax}
#># {\em objectname}.#tabulate# {\em varlist} [#if# {\em expression}]

Method #tabulate# computes and displays frequency tables of the variables specified in {\em varlist}. An optional #if# statement can be added to restrict the analysis to a part of the data.
\end{stanza}

\begin{stanza}{Options}
not allowed
\end{stanza}

\begin{stanza}{Example}
The statement

#> d.tabulate x y#

displays frequency tables for the variables #x# and #y#. The statement

#> d.tabulate x y if x>0#

restricts the analysis to observations with #x>0#.
\end{stanza}


\section{Variable Names}
 \label{varnames} 
 \index{Variables names}

A valid variable name is a sequence of letters (#A#-#Z# and #a#-#z#), digits (#0#-#9#), and underscores (#_#). The first
character of a variable name must be either a letter or an underscore. \BayesX is case-sensitive, i.e. #myvar#, #Myvar#
and #MYVAR# are three distinct variable names.

\section{Examples: The credit scoring data set}

This section illustrates some of the methods described in this chapter using the credit scoring data set (see \autoref{creditdata} for a description).

We start by illustrating how to code categorical variables according to either dummy or effect coding. This will be useful in regression models, where all categorical covariates must be coded in dummy or effect coding before they can be added to the model.

We start by creating the {\em dataset object} #credit# and proceed by reading the data using the #infile# command.

#> dataset credit# \\
#> credit.infile using c:\bayesx\examples\credit.raw#

We can now generate new variables to obtain dummy coded versions of the categorical covariates #account#, #payment#, #intuse# and #marstat#:

#> credit.generate account1  = 1*(account=1)# \\
#> credit.generate account2  = 1*(account=2)# \\
#> credit.generate payment1 = 1*(payment=1)# \\
#> credit.generate intuse1 = 1*(intuse=1)# \\
#> credit.generate marstat1 = 1*(marstat=1)#

The reference categories are chosen to be 3 for #account# and 2 for the other variables. Alternatively, we could code the
variables according to effect coding. This is achieved as follows:

#> credit.generate account_eff1  = 1*(account=1)-1*(account=3)# \\
#> credit.generate account_eff2  = 1*(account=2)-1*(account=3)# \\
#> credit.generate payment_eff1 = 1*(payment=1)-1*(payment=2)# \\
#> credit.generate intuse_eff1 = 1*(intuse=1)-1*(intuse=2)# \\
#> credit.generate marstat_eff1 = 1*(marstat=1)-1*(marstat=2)#


