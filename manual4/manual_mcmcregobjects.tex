\chapter{mcmcreg objects}
\label{mcmcreg} \index{mcmcreg object}

\begin{itemize}
 \item variable selection priors
 \item alternative priors for the smoothing variance (scale-dependent, half cauchy, half normal, uniform, \ldots) plus reference to sdPrior R package
 \item modular regression / user defined terms (incl. tensor product)
 \item check list of implemented distributions
\end{itemize}

{\em mcmcreg objects} are used to fit Bayesian quantile regression and (multivariate) distributional regression models with {\em
structured additive predictor}, see \citeasnoun{kleknelan14a}. Hierarchical data structures may be considered using
hierarchical or multilevel structured additive predictors. For multilevel structured additive models  see
\citeasnoun{LanUml14}. Inference is based on a fully Bayesian approach implemented via Markov Chain Monte Carlo (MCMC)
simulation techniques. The methodology manual provides a brief introduction to (multilevel) structured additive regression and MCMC-based
inference. More details can be found in the references cited above and in the book by \citeasnoun{fahkne13}.

\section{Linear and Structured Additive Mean Regression Models}

\subsection{Methodological Background}

\subsection{Syntax and Options}

\subsection{Examples}

\section{Distributional Regression Models}

\subsection{Methodological Background}

\subsection{Syntax and Options}

\subsection{Examples}

\section{Hierarchical Regression Models}

\subsection{Methodological Background}

\subsection{Syntax and Options}

\subsection{Examples}

\section{Overview on Model Terms}

incl. userdefined

constraints, centering

\section{Effect Selection Priors}

\section{Hyperprior Specification}

\section{MCMC Options and Convergence Diagnostics}


