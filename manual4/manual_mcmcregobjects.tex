\chapter{mcmcreg objects}
\label{mcmcreg} \index{mcmcreg object}

\begin{itemize}
 \item variable selection priors
 \item alternative priors for the smoothing variance (scale-dependent, half cauchy, half normal, uniform, \ldots) plus reference to sdPrior R package
 \item modular regression / user defined terms (incl. tensor product)
 \item check list of implemented distributions
\end{itemize}

{\em mcmcreg objects} are used to fit Bayesian quantile regression and (multivariate) distributional regression models with {\em
structured additive predictor}, see \citeasnoun{kleknelan14a}. Hierarchical data structures may be considered using
hierarchical or multilevel structured additive predictors. For multilevel structured additive models  see
\citeasnoun{LanUml14}. Inference is based on a fully Bayesian approach implemented via Markov Chain Monte Carlo (MCMC)
simulation techniques. The methodology manual provides a brief introduction to (multilevel) structured additive regression and MCMC-based
inference. More details can be found in the references cited above and in the book by \citeasnoun{fahkne13}.

\section{Linear and Additive Mean Regression Models}

\subsection{Methodological Background}

Gaussian linear models

$$
y \sim (\eta,\sigma^2)  
$$
where
$$
 \eta = E(y) = \beta_0 + \beta_1 x_1 + \ldots + \beta_k x_k
$$

Binary regression
$$
y \sim B(1,\pi)
$$
where
$$
\pi = E(y) = F(\eta) = F(\beta_0 + \beta_1 x_1 + \ldots + \beta_k x_k)
$$
and 
$$
F(\eta) = \Phi(\eta) \mbox{ or } F(\eta) = \frac{\exp(\eta)}{1+\exp(\eta)} 
$$

Poisson regression
$$
y \sim Po(\lambda)
$$
where
$$
\lambda = E(y) = \exp(\eta) = \exp(\beta_0 + \beta_1 x_1 + \ldots + \beta_k x_k)
$$

Other distributions see ??

Additive predictor. 

Replace 
$$
\eta = \beta_0 + \beta_1 x_1 + \ldots + \beta_k x_k
$$
by
$$
\eta = f(z_{1}) + \ldots + f(z_{q}) + \beta_0 + \beta_1 x_{1} + \ldots + \beta_k x_{k}
$$

For modeling $f_j$ see ??

\subsection{Syntax and Options}

All models are estimated using method #hregress#. The general syntax is \\

#> #{\em objectname}.#hregress# {\em model} [#weight# {\em weightvar}] [#if# {\em expression}] [{\em , options}] #using# {\em dataset} \\

Method #hregress# estimates the regression model specified in {\em
model} using the data specified in {\em dataset}, which is 
the name of a {\em dataset object} created before. 

The distribution of the response variable can be chosen from a wide
range of uni- and even multivariate distributions. It is
specified using option
#family#, e.g. #family=gaussian# for Gaussian linear or additive models. 
For a comprehensive list of available distributions compare \autoref{mcmcreg_distributions}.
The default value is #family=gaussian#
with an identity link.  

An #if# statement may be specified to analyze
only parts of the data set, i.e.~the observations where {\em
expression} is true.

An optional weight variable {\em weightvar} may be specified to
estimate weighted regression models.


The general syntax of models is: \\

$depvar = term_1 + term_2 + \cdots + term_r$ \\

where {\em depvar} specifies the dependent variable in the model
and $term_1$,\dots,$term_r$ define the specific form of covariate
effects on the dependent variable. The different terms have to be
separated by '+' signs. A constant intercept is NOT automatically
included in the models and must be specified using the term #const#. 
All terms may be combined
in arbitrary order. 
The following model statement specifies a model with $q$ linear effects and an intercept: \\

\texttt{Y = const + W1 + W2 + $\cdots$ + Wq} \\

A P-spline is obtained by \\

#Y = X1(pspline)# \\

By default, a second order random walk prior is used, the degree of the spline is 3 and the number of inner
knots is 20. The following model term defines a quadratic P-spline
with 30 knots and a first order random walk prior: \\

#Y = X1(pspline,degree=2,nrknots=30,difforder=1)# \\



A comprehensive overview about  possible model terms including 
spatial effects, two dimensional nonlinear effects, etc. is given in \autoref{mcmcreg_modelterms}.


Examples..









 

\subsection{Examples}

\section{Distributional Regression Models}

\subsection{Methodological Background}

\subsection{Syntax and Options}

\subsection{Examples}

\section{Hierarchical Regression Models}

\subsection{Methodological Background}

\subsection{Syntax and Options}

\subsection{Examples}

\subsection{Overview on available Distributions}
\label{mcmcreg_distributions}


\section{Overview on Model Terms}
\label{mcmcreg_modelterms}

incl. userdefined

constraints, centering

\section{Effect Selection Priors}

\section{Hyperprior Specification}

\section{MCMC Options and Convergence Diagnostics}


