\documentclass[10pt,a4paper]{article}
\usepackage[dvips]{graphicx}
\usepackage{rawfonts}
\usepackage{german}
\sloppy
\parindent0em
\parskip0.2em
\topmargin-1.5cm \textheight24cm \textwidth15cm
\oddsidemargin0.5cm
%\pagestyle{empty}

\renewcommand{\baselinestretch}{2}



\def \re {{\bf R}}
\def \beq {\begin{equation}}
\def \eeq {\end{equation}}
\def \bdis {\begin{displaymath}}
\def \edis {\end{displaymath}}
\def \ds {\displaystyle}

\newcommand{\reell}{{I\hspace{-0.18cm} R}}


\begin{document}



{\bf \Huge TESTDATENS"ATZE}

\tableofcontents

\section{Hinweise zum Testen}

Folgendes ist beim Testen zu beachten:

\begin{itemize}
\item {\bf Ergebnisse Korrekt} \\
"Uberpr"ufen, ob die Ergebnisse grunds"atzlich ok sind. Soweit
m"oglich sollte man die Ergebnisse mit anderen Programmen
vergleichen (z.B. fixe Effekte mit Stata, S-Plus; Random effects mit
S-Plus, SAS; Nichtparametrische Effekte mit S-Plus, SAS oder auch mit dem S-Plus Programm
von Thomas Kneib.)
\item {\bf Posterior Mode} \\
"Uberpr"ufen, ob der Posteriori Modus korrekt berechnet wird (mit anderen Programmen vergleichen!!!).
Bei nichtparametrischen Termen (z.B. P-splines) kann man den Gl"attungsparameter (lambda) angeben.
Der Posterori Modus kann im Moment nur im Gaussfall berechnet werden.
\item {\bf Hyperp. a} \\
In vielen F"allen kann man Hyperparameter f"ur die Priori der Varianz angeben (z.B. random walks, P-splines).
"Uberpr"uft werden soll, ob die im Handbuch angegebene Syntax zur Festlegung des Hyperparameters $a$
vom Programm akzeptiert wird und ob "Anderungen in BayesX beachtet werden.
\item {\bf Hyperp. b} \\
In vielen F"allen kann man Hyperparameter f"ur die Priori der Varianz angeben (z.B. random walks, P-splines).
"Uberpr"uft werden soll, ob die im Handbuch angegebene Syntax zur Festlegung des Hyperparameters $b$
vom Programm akzeptiert wird und ob "Anderungen in BayesX beachtet werden.
\item {\bf min} \\
Es soll "uberpr"uft werden, ob die im Handbuch beschriebene Syntax zur Festlegung der minimalen Blockgr"ossen
von BayesX beachtet werden.
\item {\bf max} \\
Es soll "uberpr"uft werden, ob die im Handbuch beschriebene Syntax zur Festlegung der maximalen Blockgr"ossen
von BayesX beachtet werden.
\item {\bf nrknots} \\
Es soll "uberpr"uft werden, ob die im Handbuch beschriebene Syntax zur Festlegung der Anzahl der inneren Knoten
(bei P-splines) von BayesX beachtet werden.
\item {\bf degree} \\
Es soll "uberpr"uft werden, ob die im Handbuch beschriebene Syntax zur Festlegung des Splinegrades
(bei P-splines) von BayesX beachtet werden.
\item {\bf gridsize} \\
Es soll "uberpr"uft werden, ob die im Handbuch beschriebene Syntax von BayesX beachtet wird und ob alles wie im Handbuch
beschrieben funktioniert.
\item {\bf 2 Effekte} \\
Hier soll bei nichtparametrischen Ausdr"ucken "uberpr"uft werden, ob die Angabe von Parametern (z.B. Hyperparameter a)
immer noch funktioniert. Gelegentlich ist es vorgekommen, dass z.B. bei 2 P-splines im Modell die Parametereinstellungen
vom ersten Effekt im zweiten Effekt "ubernommen werden.
(Beispiel: ... + x1(psplinerw2,a=2) + x2(psplinerw2), hier wurde des "ofteren f"ur x2 auch a=2 angenommen, obwohl
ja der Default a=1 ist.)
\item {\bf Filenamen} \\
Hier soll "uberpr"uft werden, ob die Ergebnisfiles die richtigen Filenamen haben (siehe Kapitel \ref{filenamen}).
Ausserdem soll "uberpr"uft werden, ob die Files die in Kapitel 7.1.4 des Handbuchs beschriebene Struktur besitzen.
\item {\bf Grafiken (Java)}  \\
Hier soll "uberpr"uft werden, ob nichtparametrische Effekte bzw. Landkarten korrekt gezeichnet werden (d.h.
so wie Handbuch beschrieben). Auch soll "uberpr"uft werden, ob alle beschriebenen Optionen existieren und
funktionieren. Siehe dazu Kapitel 7.4.1 im Handbuch.
\item {\bf Grafiken (S-plus)} \\
Hier soll "uberpr"uft werden, ob nichtparametrische Effekte bzw. Landkarten korrekt gezeichnet werden (d.h.
so wie Handbuch beschrieben). Auch soll "uberpr"uft werden, ob alle beschriebenen Optionen existieren und
funktionieren. Siehe dazu Kapitel 7.4.2 im Handbuch.
\item {\bf Postestimation C.} \\
Hier soll "uberpr"uft werden, ob die Postestimation commands alle funktionieren (Berechnen und Plotten
der Autokorrelationen, Berechnen und Plotten der Samples). Es geht um die Befehle plotsample (bei graph objekten)
und plotautocor, getsample und getautocor bei bayesreg Objekten. Das muss man f"ur jeden Effekt ausprobieren,
nicht nur f"ur fixe Effekte. Auch sollen die erzeugten Files die im Handbuch angegebene Struktur besitzen.
\end{itemize}

Dar"uberhinaus sollten alle im Handbuch beschriebenen Optionen (Kapitel 7.1.3) "uberpr"uft werden.

Grunds"atzlich sollte man auch Fehleingaben versuchen und "uberpr"ufen ob die im Programm richtig abgefangen
werden.

Es soll auch zumindest mit fixen Effekten ausprobiert werden, ob die Sch"atzung mit Offset funktioniert.
(Kann man z.B. mit Stata "uberpr"ufen)

Fehlermeldungen bitte an:

Andi Brezger bei P-splines, Smoothingsplines etc. sowie bei Grafikfehlern (sowohl Java als auch S-Plus)

Thomas Kneib bei Problemen mit der Javaoberfl�che.

Stefan Lang beim Rest.

Fehlermeldungen bitte immer auch an Stefan Lang, wegen Fehlerliste im Internet.

{\bf Verbesserungsvorschl"age sind willkommen!!!}

\section{Gaussian response}

\subsection{Eine nichtlineare Funktion}

{\bf Datensatz}

gaussian\_nonp\_1fkt.raw

{\bf Modell}

$y  \sim N(\sin(x),0.3^2)$

{\bf Batch-file}

gaussian\_nonp\_1fkt.prg

\subsection{Zwei nichtlineare Funktionen}

{\bf Datensatz}

gaussian\_nonp\_2fkt.raw

{\bf Modell}

$y  \sim N(\sin(x1)+0.8x2,0.3^2)$

{\bf Batch-file}

gaussian\_nonp\_2fkt.prg


\subsection{Variierende Koeffizienten, eine nichtlineare Funktion}

{\bf Datensatz}

gaussian\_varcoeff\_1fkt.raw

{\bf Modell}

$y  \sim N(\sin(x1)x2,0.3^2)$

{\bf Batch-file}

gaussian\_varcoeff\_1fkt.prg


\subsection{Random Effects}

{\bf Datens"atze}

500 Beobachtungen: \\
gaussian\_randomeffects.raw \\
gaussian\_randomeffects.dta

2500 Beobachtungen: \\
gaussian\_randomeffects2.raw \\
gaussian\_randomeffects2.dta \\

Daten wurden erzeugt mit den Stata do-files sim\_randomeffects.do und sim\_randomeffects2.do.


{\bf Modell}

$y1  \sim N(0.7+b0_{ind},0.3^2)$ \\
$y2  \sim N(0.4+b0_{ind}+b1_{ind}x1+b2_{ind}x2,0.3^2)$ \\
$y3  \sim N(-0.4+0.4x1-0.4x2+b0_{ind}+b1_{ind}x1+b2_{ind}x2,0.3^2)$

$b0_{ind} \sim N(0,0.7^2)$, i.i.d. \\
$b1_{ind} \sim N(0,0.5^2)$, i.i.d. \\
$b2_{ind} \sim N(0,0.5^2)$, i.i.d. \\

ind = Gruppenindikator (von 1 bis 25) \\

500 Beobachtungen oder 2500 Beobachtungen \\

Wahre Werte gespeichert unter

true\_randomeffects.raw \\
true\_randomeffects.dta

SAS-Ergebnisse gespeichert unter

gaussian\_randomeffectsy1\_sas.raw \\
gaussian\_randomeffectsy1\_var\_sas.raw \\
gaussian\_randomeffectsy1\_fixed\_sas.raw \\

bzw.

gaussian\_randomeffectsy2\_sas.raw \\
gaussian\_randomeffectsy2\_var\_sas.raw \\
gaussian\_randomeffectsy2\_fixed\_sas.raw \\


bzw.

gaussian\_randomeffectsy3\_sas.raw \\
gaussian\_randomeffectsy3\_var\_sas.raw \\
gaussian\_randomeffectsy3\_fixed\_sas.raw \\


{\bf Batch-file}

gaussian\_randomeffectsy1.prg \\
gaussian\_randomeffectsy2.prg \\
gaussian\_randomeffectsy3.prg \\


\subsection{Lattice Daten}

{\bf Datensatz}

gaussian\_lattice\_1fkt.raw

{\bf Modell}

$y  \sim N(0.4(xcenter+ycenter),0.3^2)$

$xcenter$ standardisierte Version von $xc \in \{1,\dots,10\}$

$ycenter$ standardisierte Version von $yc \in \{1,\dots,10\}$

Die wahre Funktion $f(xc,yc) = 0.4(xcenter+ycenter)$ ist in Abbildung
\ref{gaussianlatticetruef} zu finden.


{\bf Batch-file}

gaussian\_lattice\_1fkt.prg

\begin{figure}[h]
\centering
\includegraphics[scale = 0.32] {gaussian_lattice_1fkt_truefunction.ps}

\caption{\label{gaussianlatticetruef} {\em Gaussian response: Wahre r"aumliche Funktion bei lattice Daten.}}
\end{figure}

\subsection{Eine r"aumliche Funktion}

{\bf Datensatz}

gaussian\_spatial\_1fkt.raw

{\bf Modell}

$y1  \sim N(0.4(xcenter+ycenter),0.3^2)$

wobei $xcenter$ und $ycenter$ die (zentrierten) Zentroide der Regionen sind.
Die wahre Funktion ist in Abbildung \ref{cum3truef} zu finden.
Die Zentroide findet man in der Datei kreisecentroid.raw, das bnd-file unter
kreisesim.bnd. Das bandweitenoptimal sortierte graph-file ist unter dem Namen kreisesimgraphsort.bnd
gespeichert.

{\bf batch-file}

gaussian\_spatial\_1fkt.prg


\section{Binomialverteilter Response mit Logit-Link}

\subsection{Eine nichtlineare Funktion}

{\bf Datensatz}

binomial\_logit\_nonp\_1fkt.raw

{\bf Modell}

$y  \sim B(1,\pi)$

$\pi = \frac{\exp(\eta)}{1+\exp(\eta)}$

$\eta = sin(x)$

{\bf Batch-file}

binomial\_logit\_nonp\_1fkt.prg


\subsection{Random Effects}

{\bf Datens"atze}

500 Beobachtungen \\
binomial\_logit\_randomeffects.raw \\
binomial\_logit\_randomeffects.dta \\

2500 Beobachtungen  \\
binomial\_logit\_randomeffects2.raw \\
binomial\_logit\_randomeffects2.dta \\

Daten wurden mit Stata do-files sim\_randomeffects.do und sim\_randomeffects2.do
erzeugt. \\


{\bf Modell}

$y1  \sim B(1,\pi)$ \\
$\pi1 = \frac{\exp(\eta1)}{1+\exp(\eta1)}$ \\
$\eta1 = 0.7+b0_{ind}$

$y2  \sim B(1,\pi)$ \\
$\pi2 = \frac{\exp(\eta2)}{1+\exp(\eta2)}$ \\
$\eta2 = 0.4+b0_{ind}+b1_{ind}x1+b2_{ind}x2$

$y3  \sim B(1,\pi)$ \\
$\pi3 = \frac{\exp(\eta3)}{1+\exp(\eta3)}$ \\
$\eta3 = -0.4+0.4x1-0.4x2+b0_{ind}+b1_{ind}x1+b2_{ind}x2$ \\

$b0_{ind} \sim N(0,0.7^2)$, i.i.d. \\
$b1_{ind} \sim N(0,0.5^2)$, i.i.d. \\
$b2_{ind} \sim N(0,0.5^2)$, i.i.d. \\

ind = Gruppenindikator (von 1 bis 25) \\

Wahre Werte gespeichert unter

true\_randomeffects.raw \\
true\_randomeffects.dta

SAS-Ergebnisse gespeichert unter

binomial\_logit\_randomeffectsy1\_sas.raw \\
binomial\_logit\_randomeffectsy1\_var\_sas.raw \\
binomial\_logit\_randomeffectsy1\_fixed\_sas.raw \\

bzw.

binomial\_logit\_randomeffectsy2\_sas.raw \\
binomial\_logit\_randomeffectsy2\_var\_sas.raw \\
binomial\_logit\_randomeffectsy2\_fixed\_sas.raw \\

bzw.

binomial\_logit\_randomeffectsy3\_sas.raw \\
binomial\_logit\_randomeffectsy3\_var\_sas.raw \\
binomial\_logit\_randomeffectsy3\_fixed\_sas.raw \\


{\bf Batch-file}

binomial\_\_logit\_randomeffects1.prg \\
binomial\_logit\_randomeffects2.prg \\
binomial\_logit\_randomeffects3.prg


\section{Binomialverteilter Response mit Probit Link}

\subsection{Fixe Effekte}

{\bf Datensatz}

binomial\_probit\_fixed.raw \\
binomial\_probit\_fixed.dta (Stata-Format)

{\bf Modell}

$y  \sim B(1,\pi)$

$\pi = \Phi(\eta)$

$\eta = -0.5+0.7x1-0.6x2$

500 Beobachtungen

{\bf Batch-file}

binomial\_probit\_fixed.prg

{\bf Ergebnisse mit Stata}


\begin{table}[h]
\begin{tabular} {|l|l|l|l|}
\hline
 & Coef. & Std. Err.  & 95\% Conf. Interval \\
\hline
x1 &   .7827467 &  .0684006    & .648684 -- .9168093 \\
x2 &  -.6572396 &  .0630554    & -.7808259 --  -.5336533 \\
constant &  -.6590516 &  .0896717 &  -.8348049 --  -.4832984 \\
\hline
\end{tabular}
\end{table}

\subsection{Eine nichtlineare Funktion}

{\bf Datensatz}

binomial\_probit\_nonp\_1fkt.raw \\
binomial\_probit\_nonp\_1fkt.dta

{\bf Modell}

$y  \sim B(1,\pi)$

$\pi = \Phi(\eta)$

$\eta = sin(x)$

500 Beobachtungen

{\bf Batch-file}

binomial\_probit\_nonp\_1fkt.prg

\subsection{Zwei nichtlineare Funktionen}

{\bf Datensatz}

binomial\_probit\_nonp\_2fkt.raw \\
binomial\_probit\_nonp\_2fkt.dta

{\bf Modell}

$y  \sim B(1,\pi)$

$\pi = \Phi(\eta)$

$\eta = sin(x1)+0.7x2$

500 Beobachtungen

{\bf Batch-file}

binomial\_probit\_nonp\_2fkt.prg

\subsection{Random Effects}

{\bf Datensatz}

binomial\_probit\_randomeffects.raw \\
binomial\_probit\_randomeffects.dta \\


{\bf Modell}

$y1  \sim B(1,\pi)$ \\
$\pi1 = \Phi(\eta1)$ \\
$\eta1 = -0.7+b0_{ind}$

$y2  \sim B(1,\pi)$ \\
$\pi2 = \Phi(\eta1)$ \\
$\eta2 = -0.7+b0_{ind}+b1_{ind}x1+b2_{ind}x2$

$b0_{ind} \sim N(0,0.7^2)$, i.i.d. \\
$b1_{ind} \sim N(0,0.5^2)$, i.i.d. \\
$b2_{ind} \sim N(0,0.5^2)$, i.i.d.

ind = Gruppenindikator (von 1 bis 25)

500 Beobachtungen

Wahre Werte gespeichert unter

true\_randomeffects.raw \\
true\_randomeffects.dta

{\bf Batch-file}

binomial\_probit\_randomeffects1.prg \\
binomial\_probit\_randomeffects2.prg


\section{Kumulative Schwellenwertmodelle mit Probit-Link}

\subsection{Fixe Effekte}

{\bf Datensatz}

cum3\_probit\_fixed.raw

cum3\_probit\_fixed.dta (Stata Format)

{\bf Modell}

3 Kategorien ($y=3$ Referenz)

$\theta_1 = -0.6$

$\theta_2 = 0.6$

$\eta = 0.6x1-0.5x2$


{\bf Batch-file}

cum3\_probit\_fixed.prg


\subsection{Eine nichtlineare Funktion}

{\bf Datensatz}

cum3\_probit\_nonp\_1fkt.raw

{\bf Modell}

3 Kategorien ($y=3$ Referenz)

$\theta_1 = -0.6$

$\theta_2 = 0.6$

$\eta = \sin(x1)$


{\bf Batch-file}

cum3\_probit\_nonp\_1fkt.prg

\subsection{Eine r"aumliche Funktion}

{\bf Datensatz}

cum3\_probit\_spatial\_1fkt.raw

{\bf Modell}

3 Kategorien ($y=3$ Referenz)

$\theta_1 = -0.6$

$\theta_2 = 0.6$

$\eta = 0.4xy$ wobei $x$ und $y$ die (zentrierten) Zentroide der Regionen sind.
Die wahre Funktion ist in Abbildung \ref{cum3truef} zu finden.

Die Zentroide findet man in der Datei kreisecentroid.raw. Das bnd-file heisst
kreisesim.bnd. Das bandweitenoptimal sortierte graph-file heisst kreisesimgraphsort.bnd.

{\bf batch-file}

\begin{figure}[h]
\centering
\includegraphics[scale = 0.32] {cum3_probit_spatial_1fkt_trufunction.ps}

\caption{\label{cum3truef} {\em Kumulatives Probitmodell: Wahre r"aumliche Funktion.}}
\end{figure}

\section{Multinomiale Logitmodelle}

\subsection{Fixe Effekte}

{\bf Datensatz}

multinom\_fixed.raw \\
multinom\_fixed.dta \\
sim\_multinom\_fixed.do (Stata do file zur Simulation der Daten)


{\bf Modell}

$y = 1,2,3$, 3 Referenzkategorie

$y  \sim MN(1,(\pi_1,\pi_2,\pi_3))$

$\pi_1 = \frac{\exp(\eta_1)}{1+\exp(\eta_1)+\exp(\eta_2)}$

$\pi_2 = \frac{\exp(\eta_2)}{1+\exp(\eta_1)+\exp(\eta_2)}$

$\pi_3 = 1-\pi_1-\pi_2$


$\eta_1 = 0.4+0.6 x1-0.7 x2$

$\eta_2 = 0.6-0.7x1+0.5 x2$

x1 und x2 gleichverteilt auf $(0,1)$.

500 Beobachtungen

{\bf Batch-file}

multinom\_fixed.prg



{\bf Ergebnisse mit Stata}

\begin{verbatim}
------------------------------------------------------------------------------
           y |      Coef.   Std. Err.      z    P>|z|     [95% Conf. Interval]
-------------+----------------------------------------------------------------
1            |
          x1 |    .820069   .1645014     4.99   0.000     .4976521    1.142486
          x2 |  -1.065848   .1670865    -6.38   0.000    -1.393332   -.7383646
       _cons |   .2214211   .1446839     1.53   0.126    -.0621542    .5049963
-------------+----------------------------------------------------------------
2            |
          x1 |  -.7420497   .1479948    -5.01   0.000    -1.032114   -.4519852
          x2 |   .3721598   .1486553     2.50   0.012     .0808009    .6635188
       _cons |   .4646242   .1353302     3.43   0.001     .1993818    .7298665
------------------------------------------------------------------------------
(Outcome y==3 is the comparison group)
\end{verbatim}


\subsection{Eine nichtlineare Funktion}

{\bf Datensatz}

multinom\_logit\_1fkt.raw \\
multinom\_logit\_1fkt.dta \\
multinom\_logit\_1fkt.do (Stata do file zur Simulation der Daten)


{\bf Modell}

$y = 1,2,3$, 3 Referenzkategorie

$y  \sim MN(1,(\pi_1,\pi_2,\pi_3))$

$\pi_1 = \frac{\exp(\eta_1)}{1+\exp(\eta_1)+\exp(\eta_2)}$

$\pi_2 = \frac{\exp(\eta_2)}{1+\exp(\eta_1)+\exp(\eta_2)}$

$\pi_3 = 1-\pi_1-\pi_2$

$\eta_1 = 1.3 \sin(x)$

$\eta_2 = -1.3 \sin(x)$

x gleichverteilt zwischen $(-3,3)$.

1000 Beobachtungen

{\bf Batch-file}

multinom\_logit\_1fkt.prg






\section{Poissonverteilter Response}

\subsection{Fixe Effekte}

{\bf Datensatz}

poisson\_fixed.raw

{\bf Modell}

$y  \sim Po(\lambda)$

$\lambda = exp(\eta)$

$\eta = 0.8x1-0.3x2$

500 Beobachtungen

{\bf Batch-file}

poisson\_fixed.prg
\newpage

{\bf Ergebnisse mit Stata}


\begin{table}[h]
\begin{tabular} {|l|l|l|l|}
\hline
 & Coef. & Std. Err.  & 95\% Conf. Interval \\
\hline
x1 &   .8288586 &  .0246793    & .7804881 -- .8772291 \\
x2 &  -.309638 &  .017039    & -.3430338 --  -.2762423 \\
constant &  -.0584579 &  .0569696 &  -.1701164 --  .0532005 \\
\hline
\end{tabular}
\end{table}

\subsection{Eine nichtlineare Funktion}

{\bf Datensatz}

poisson\_nonp\_1fkt.raw

{\bf Modell}

$y  \sim Po(\lambda)$

$\lambda = exp(\eta)$

$\eta = sin(x)$

500 Beobachtungen

{\bf Batch-file}

poisson\_nonp\_1fkt.prg

\subsection{Zwei nichtlineare Funktionen}

{\bf Datensatz}

poisson\_nonp\_2fkt.raw

{\bf Modell}

$y  \sim Po(\lambda)$

$\lambda = exp(\eta)$

$\eta = 0.45sin(x1)+0.55sin(x2)$

500 Beobachtungen

{\bf Batch-file}

poisson\_nonp\_2fkt.prg

\subsection{Variierende Koeffizienten, eine nichtlineare Funktion}

{\bf Datensatz}

poisson\_varcoeff\_1fkt.raw

{\bf Modell}

$y  \sim Po(\lambda)$

$\lambda = exp(\eta)$

$\eta = x2*sin(x1)$

{\bf Batch-file}

poisson\_varcoeff\_1fkt.prg

\subsection{Variierende Koeffizienten, zwei nichtlineare Funktion}

{\bf Datensatz}

poisson\_varcoeff\_2fkt.raw

{\bf Modell}

$y  \sim Po(\lambda)$

$\lambda = exp(\eta)$

$\eta = dummy_1*0.5*sin(x)+dummy_2*0.5*cos(x)$

{\bf Batch-file}

poisson\_varcoeff\_2fkt.prg


\subsection{Lattice Daten}

{\bf Datensatz}

poisson\_lattice\_1fkt.raw

{\bf Modell}

$y  \sim Po(\lambda)$

$\lambda = exp(\eta)$

$\eta = 0.4(xcenter+ycenter)$

$xcenter$ standardisierte Version von $xc \in \{1,\dots,10\}$

$ycenter$ standardisierte Version von $yc \in \{1,\dots,10\}$

{\bf Batch-file}

poisson\_lattice\_1fkt.prg

\subsection{Eine r"aumliche Funktion}

{\bf Datensatz}

poisson\_spatial\_1fkt.raw

{\bf Modell}

$y  \sim Po(\lambda)$

$\lambda = exp(\eta)$

$\eta = 0.4(xcenter+ycenter)$

wobei $xcenter$ und $ycenter$ die (zentrierten) Zentroide der
Regionen sind. Die wahre Funktion ist in Abbildung \ref{cum3truef}
zu finden. Die Zentroide findet man in der Datei
kreisecentroid.raw, das bnd-file unter kreisesim.bnd. Das
bandweitenoptimal sortierte graph-file ist unter dem Namen
kreisesimgraphsort.bnd gespeichert.

{\bf batch-file}

poisson\_spatial\_1fkt.prg

\subsection{Random Effects}

{\bf Datens\"atze}

2500 Beobachtungen:\\
poisson\_randomeffects.raw\\


10000 Beobachtungen:\\
poisson\_randomeffects10000.raw\\



{\bf Modell}

$y1  \sim Po(\lambda)$

$\lambda = exp(\eta)$

$\eta = 0.7+b0_{ind}$


$b0_{ind} \sim N(0,0.7^2)$, i.i.d. \\

ind = Gruppenindikator (von 1 bis 25) \\

2500 bzw. 10000 Beobachtungen \\


{\bf Batch-file}

poisson\_randomeffects.prg

\subsection{Random Slope}

{\bf Datensatz}

poisson\_randomslope.raw


{\bf Modell}

$y2  \sim Po(\lambda)$

$\lambda = exp(\eta)$

$\eta = 0.4+b0_{ind}+b1_{ind}x_1+b2_{ind}x_2$


$b0_{ind} \sim N(0,0.7^2)$, i.i.d. \\
$b1_{ind} \sim N(0,0.5^2)$, i.i.d. \\
$b2_{ind} \sim N(0,0.5^2)$, i.i.d. \\


ind = Gruppenindikator (von 1 bis 25) \\

2500 Beobachtungen \\


{\bf Batch-file}

poisson\_randomslope.prg

\subsection{Saisonaler Effekt}

{\bf Datensatz}

poisson\_season\_1000\_1fkt.raw


{\bf Modell}

$y, y_{vcm}  \sim Po(\lambda)$

$\lambda = exp(\eta)$ bzw. $\lambda=exp(\eta_{vcm})$

\[
\eta=\left\{
 \begin{tabular}{cc}
0.7 & $season_1$\\0.5&$season_2$\\-1.5&$season_3$\\0.3&$season_4$\\
 \end{tabular}
 \right.
\]

\[
\eta_{vcm}=\left\{
 \begin{tabular}{cc}
x$\cdot$ 0.7 & $season_1$\\x $\cdot$0.5&$season_2$\\x$\cdot$(-1.5)&$season_3$\\x$\cdot$0.3&$season_4$\\
 \end{tabular}
 \right.
\]

$x\in \{0,0.1,\ldots,0.9 \}$

$t=1,\ldots,100$

je 10 Beobachtungen pro Zeitpunkt

$\Rightarrow$ 1000 Beobachtungen



{\bf Batch-file}

poisson\_seanson\_1fkt.prg


\subsection{Offset}

{\bf Datensatz}

poisson\_offset.raw

{\bf Modell}

$y  \sim Po(\lambda)$

$\lambda = exp(\eta)$

$\eta = 0.5x1-0.5x2+\delta$

$\delta=log(\Delta)$, $\Delta \in \{1,2\}$

500 Beobachtungen

{\bf Batch-file}

poisson\_offset.prg

{\bf Ergebnisse mit Stata}


\begin{table}[h]
\begin{tabular} {|l|l|l|l|}
\hline
 & Coef. & Std. Err.  & 95\% Conf. Interval \\
\hline
x1 &   .483348  &   .0176726    &  .4487104  -- .5179856 \\
x2 &  -.5109228  &  .0186664     &  -.5475083  --   -.4743373 \\
constant &  -.0287701 &   .0435811  &  -.1141875 --  .0566474\\
\hline
\end{tabular}
\end{table}



\section{Negative Binomial}

\subsection{Fixe Effekte}

\subsubsection{Negative Binomial, Poisson--Gamma}

{\bf Datensatz}

nbinomial\_poga\_fixed.raw

{\bf Modell}

$y  \sim Po(\nu * \lambda)$

$\lambda = exp(\eta)$

$\eta = 2 + 0.5*x1-1*x2$

$x1 \sim Ber(0.3), x2 \sim Ber(0.5)$

$\nu \sim Gamma(\delta, \delta)$

$\delta = 1$

1000 Beobachtungen

{\bf Batch-file}

nbinomial\_poga\_fixed.prg

\subsubsection{Poisson--Inverse Gauss}

{\bf Datensatz}

nbinomial\_poig\_fixed.raw

{\bf Modell}

$y  \sim Po(\nu * \lambda)$

$\lambda = exp(\eta)$

$\eta = 2 + 0.5*x1-1*x2$

$x1 \sim Ber(0.3), x2 \sim Ber(0.5)$

$\nu \sim InvGauss(1, \delta)$

$\delta = 1$

1000 Beobachtungen

{\bf Batch-file}

nbinomial\_poig\_fixed.prg


\subsubsection{Poisson--LogNormal}

{\bf Datensatz}

nbinomial\_poln\_fixed.raw

{\bf Modell}

$y  \sim Po(\nu * \lambda)$

$\lambda = exp(\eta)$

$\eta = 2 + 0.5*x1-1*x2$

$x1 \sim Ber(0.3), x2 \sim Ber(0.5)$

$\nu \sim exp\{N(-0.5*\tilde\delta, \sqrt{\tilde\delta})\} \sim logN(1, \sqrt\delta)$

$\tilde\delta=log(1+1/\delta)$

$\delta = 1$

1000 Beobachtungen

{\bf Batch-file}

nbinomial\_poln\_fixed.prg


\subsection{Eine nichtlineare Funktion}

\subsubsection{Negative Binomial, Poisson--Gamma}

{\bf Datensatz}

nbinomial\_poga\_spline.raw

{\bf Modell}

$y  \sim Po(\nu * \lambda)$

$\lambda = exp(\eta)$

$\eta = 2 + sin(x)$

$x \in [-3, 3]$

$\nu \sim Gamma(\delta, \delta)$

$\delta = 1$

1000 Beobachtungen

{\bf Batch-file}

nbinomial\_poga\_spline.prg

\subsubsection{Poisson--Inverse Gauss}

{\bf Datensatz}

nbinomial\_poig\_spline.raw

{\bf Modell}

$y  \sim Po(\nu * \lambda)$

$\lambda = exp(\eta)$

$\eta = 2 + sin(x)$

$x \in [-3, 3]$

$\nu \sim InvGauss(1, \delta)$

$\delta = 1$

1000 Beobachtungen

{\bf Batch-file}

nbinomial\_poig\_spline.prg


\subsubsection{Poisson--LogNormal}

{\bf Datensatz}

nbinomial\_poln\_spline.raw

{\bf Modell}

$y  \sim Po(\nu * \lambda)$

$\lambda = exp(\eta)$

$\eta = 2 + sin(x)$

$x \in [-3, 3]$

$\nu \sim exp\{N(-0.5*\tilde\delta, \sqrt{\tilde\delta})\} \sim logN(1, \sqrt\delta)$

$\tilde\delta=log(1+1/\delta)$

$\delta = 1$

1000 Beobachtungen

{\bf Batch-file}

nbinomial\_poln\_spline.prg

\subsection{Random Effects}

\subsubsection{Negative Binomial, Poisson--Gamma}

{\bf Datensatz}

nbinomial\_poga\_random.raw

{\bf Modell}

$y1  \sim Po(\nu*\lambda)$

$\lambda = exp(\eta)$

$\eta = 2 + b0_{ind}$

$b0_{ind} \sim N(0,0.707)$, i.i.d.

ind = Gruppenindikator (von 1 bis 25)

$\nu \sim Gamma(\delta, \delta)$

$\delta = 1$

1000 Beobachtungen

{\bf Batch-file}

nbinomial\_poga\_random.prg


\subsubsection{Poisson--Inverse Gauss}

{\bf Datensatz}

nbinomial\_poig\_random.raw

{\bf Modell}

$y1  \sim Po(\nu*\lambda)$

$\lambda = exp(\eta)$

$\eta = 2 + b0_{ind}$

$b0_{ind} \sim N(0,0.707)$, i.i.d.

ind = Gruppenindikator (von 1 bis 25)

$\nu \sim InvGauss(1, \delta)$

$\delta = 1$

1000 Beobachtungen

{\bf Batch-file}

nbinomial\_poig\_random.prg


\subsubsection{Poisson--LogNormal}

{\bf Datensatz}

nbinomial\_poln\_random.raw

{\bf Modell}

$y1  \sim Po(\nu*\lambda)$

$\lambda = exp(\eta)$

$\eta = 2 + b0_{ind}$

$b0_{ind} \sim N(0,0.707)$, i.i.d.

ind = Gruppenindikator (von 1 bis 25)

$\nu \sim exp\{N(-0.5*\tilde\delta, \sqrt{\tilde\delta})\} \sim logN(1, \sqrt\delta)$

$\tilde\delta=log(1+1/\delta)$

$\delta = 1$

1000 Beobachtungen

{\bf Batch-file}

nbinomial\_poln\_random.prg


\section{Korrekte Filenamen und Titel}
\label{filenamen}

{\bf Pfad Resultate}

{\em grundname}\_f\_{\em Variablenname}\_{\em Priorityp}.res

{\em grundname}\_f\_{\em Kategorie}\_{\em Variablenname}\_{\em Priorityp}.res

{\bf Titel}

f\_{\em Variablenname}

f\_Kategorie\_{\em Variablenname}

f\_{\em Variablenname}\_variance

f\_Kategorie\_{\em Variablenname}\_variance

{\em Interaktionsvariable}\_f\_{\em Effektmodifier}


\end{document}
