\chapter{remlreg objects}\normalsize
\label{remlreg} \index{remlreg object}

{\em Authors: Andreas Brezger, Thomas Kneib and Stefan Lang} \\
{\em email:
\href{mailto:andib@stat.uni-muenchen.de}{andib@stat.uni-muenchen.de},
\href{mailto:thomas.kneib@stat.uni-muenchen.de}{thomas.kneib@stat.uni-muenchen.de},
and\newline
\href{mailto:stefan.lang@stat.uni-muenchen.de}{stefan.lang@stat.uni-muenchen.de}} \\
\vspace{0.3cm}

{\em Remlreg objects} are used to fit generalized linear models with
a {\em structured additive predictor (STAR)}, see Fahrmeir, Kneib
and Lang (2004). Inference is based on mixed model representations
of the regression models and may be seen as empirical Bayes /
posterior mode estimation. The methodological background is provided
in considerable detail in the methodology manual. More details on
models for univariate responses can be found in Fahrmeir, Kneib and
Lang (2004), Kneib and Fahrmeir (2005) deals with models for
multicategorical responses. A description of models for continuous
time survival analysis based on the Cox model can be found in Kneib
and Fahrmeir (2004). Good introductions into generalized linear
models are the monographs of Fahrmeir and Tutz (2001) and McCullagh
and Nelder (1989). Introductions to semi- and nonparametric models
are given in Green and Silverman (1994), Hastie and Tibshirani
(1990), Hastie and Tibshirani (1993) and Hastie, Tibshirani and
Friedman (2001).

First steps with {\em remlreg objects} can be done with the example
in chapter \ref*{remlregzambiaanalysis} of the tutorial manual.

\section{Method regress}
\label{remlregregress}

\subsection{Syntax}
\label{remlregregresssyntax}

 #># {\em objectname}.#regress# {\em model} [#weight# {\em weightvar}] [#if# {\em expression}] [{\em , options}] #using# {\em dataset}

Method #regress# estimates the regression model specified in {\em
model} using the data specified in {\em dataset}. {\em dataset}
must be the name of a {\em dataset object} created before. The
details of correct models are covered in
\autoref{remlregmodelsyntax}. The distribution of the response
variable can be either Gaussian, binomial, multinomial, Poisson or
gamma, see also \autoref{remlregfamilyopt} for an overview about
the models supported by {\em remlreg objects}. The response
distribution is specified using option #family#, see
\autoref{remlregfamilysyntax} below. The default is
#family=binomial# with a logit link. An #if# statement may be
specified to analyze only a part of the data set, i.e. the
observations where {\em expression} is true.

\subsubsection{Optional weight variable }
\label{remlregweightspecification}

An optional weight variable {\em weightvar} may be specified to
estimate weighted regression models. For Gaussian responses {\em
BayesX} assumes that $y_i|\eta_i,\sigma^2 \sim
N(\eta_i,\sigma^2/weightvar_i)$. Thus, for grouped Gaussian
responses the weights must be the number of observations in the
groups if the $y_i$'s are the average of individual responses. If
the $y_i$'s are the sum of responses in every group, the weights
must be the reciprocal of the number of observations in the
groups. Of course, estimation of usual weighted regression models
if the errors are heteroscedastic is also possible. In this case
the weights should be proportional to the reciprocal of the
heteroscedastic variances. If the response distribution is
binomial, it is assumed that the values of the weight variable
correspond to the number of replications and that the values of
the response variable correspond to the number of successes. If
weight is omitted, {\em BayesX} assumes that the number of
replications is one, i.e. the values of the response must be
either zero or one. For grouped Poisson data the weights must be
the number of observations in a group and the $y_i$'s are assumed
to be the average of individual responses. Weights are not allowed
for models with multicategorical response.

\subsubsection{Syntax of possible model terms}
\label{remlregmodelsyntax}

The general syntax of models for {\em remlreg objects} is:

$depvar = term_1 + term_2 + \cdots + term_r$

{\em depvar} specifies the dependent variable in the model and
$term_1$,\dots,$term_r$ define in which way the covariates
influence the dependent variable. The different terms must be
separated by '+' signs. A constant intercept is automatically
included in the models and must not be specified by the user. This
section reviews all possible model terms that are supported in the
current version of {\em remlreg objects} and provides some
specific examples. Note that all described terms may be combined
in arbitrary order. An overview about the capabilities of {\em
remlreg objects} is given in \autoref{remlregterms}.
\autoref{remlreginteractions} shows how interactions between
covariates are specified. Full details about all available options
are given in \autoref{remlreglocaloptions}.

Throughout this section Y denotes the dependent variable.

\begin{table}[ht] \footnotesize
\begin{center}
\begin{tabular}{|p{2.8cm}|p{3.6cm}|p{7.1cm}|}
\hline
{\bf Type} & {\bf Syntax example} & {\bf Description} \\
\hline \hline
offset & #offs(offset)#  & Variable #offs# is an offset term. \\
\hline
linear effect & #W1#  & Linear effect for #W1#. \\
\hline
first or second order random walk &   #X1(rw1)#  \newline  #X1(rw2)#  & Nonlinear effect of #X1#. \\
\hline
P-spline &  #X1(psplinerw1)#   \newline  #X1(psplinerw2)#  & Nonlinear effect of #X1#.  \\
\hline
seasonal prior & #time(season,period=12)# & Varying seasonal effect of #time# with period 12. \\
\hline Markov random \newline field &  #region(spatial,map=m)#  &
Spatial effect of #region# where #region# indicates the region an
observation pertains to. The boundary information and the
neighborhood structure is stored in the {\em map object}
#m#. \\
\hline Two dimensional \newline P-spline &
#region(geospline,map=m)# & Spatial effect of #region#. Estimates
a two dimensional P-spline
based on the centroids of the regions. The centroids are stored in the {\em map object} #m#. \\
 \hline
 Stationary Gaussian random field & #region(geokriging)# & Spatial effect of #region#. Estimates
a stationary Gaussian random field
based on the centroids of the regions. The centroids are stored in the {\em map object} #m#. \\
\hline
 random intercept &  #grvar(random)# & I.i.d.~(random) Gaussian effect of the group indicator #grvar#,
 e.g.~#grvar# may be an individuum indicator when analyzing longitudinal data.  \\
\hline
 baseline in Cox models & #time(baseline)# & Nonlinear shape
of the baseline effect $\lambda_0(time)$ of a Cox model. $\log(\lambda_0(time))$ is modelled by a P-spline with second order penalty. \\
 \hline
\end{tabular}
{\em \caption {\label{remlregterms} Overview over different model
terms for remlreg objects.}}
\end{center}
\end{table}


\begin{table}[ht] \footnotesize
\begin{center}
\begin{tabular}{|p{3.5cm}|p{3.8cm}|p{5.9cm}|}
\hline
{\bf Type of interaction} & {\bf Syntax example} & {\bf Description} \\
 \hline
\hline Varying coefficient term & #X1*X2(rw1)# \newline
#X1*X2(rw2)#
\newline
 #X1*X2(psplinerw1) #
 \newline  #X1*X2(psplinerw2)# \newline #X1*time(season)#
 & Effect of #X1# varies smoothly over the range of the continuous covariate #X2# or #time#, respectively. \\
\hline random slope & #X1*grvar(random)#  &  The regression
coefficient of #X1# varies with respect
to the unit- or cluster index variable #grvar#. \\
\hline Geographically weighted \newline regression &
#X1*region(spatial,map=m)#  & Effect of #X1# varies
geographically.
Covariate #region# indicates the region an observation pertains to. \\
\hline Two dimensional \newline surface &  #X1*X2(pspline2dimrw1)#
 & Two dimensional surface for the continuous
covariates #X1# and #X2#. \\
 \hline
 Stationary Gaussian random field &  #X1*X2(kriging)# & Stationary Gaussian random field for coordinates #X1# and #X2#. \\
 \hline
 Time-varying effect in Cox Models & #X1*time(baseline)# &
 Nonlinear, time-varying effect of #X1#.\\
 \hline
\end{tabular}
\caption {\label{remlreginteractions} \em Possible interaction
terms for remlreg objects.}
\end{center}
\end{table}

\subsubsection*{Offset}

\begin{itemize}
\item[] {\em Description}: Adds an offset term to the predictor.
\item[] {\em Predictor}: $\eta =  \cdots + offs + \cdots$
\item[] {\em Syntax}:

#offs(offset)#
\item[] {\em Example}:

For example, the following model statement can be used to estimate
a poisson model with #offs# as offset term and #W1# and #W2# as
fixed effects (if #family=poisson# is specified in addition):

\texttt{Y = offs(offset) + W1 + W2}
\end{itemize}

\subsubsection*{Fixed effects}

\begin{itemize}
\item[] {\em Description}: Incorporates covariate #W1# as a fixed effect into the model.
\item[] {\em Predictor}: $\eta =  \cdots + \gamma_1 W1 + \cdots$
\item[] {\em Syntax}:

#W1#
\item[] {\em Example}:

The following model statement causes #regress# to estimate a model
with $q$ fixed (linear) effects:

\texttt{Y = W1 + W2 + $\cdots$ + Wq}
\end{itemize}


\subsubsection*{Nonlinear effects of metrical covariates and time
scales}

\begin{itemize}
\item[]{\bf\sffamily First or second order random walk}

\item[] {\em Description}: Defines a first or second order random walk prior for the effect of #X1#.
\item[] {\em Predictor}: $\eta = \cdots + f_1(X1) + \cdots $
\item[] {\em Syntax}:

#X1(rw1#[, {\em options}]#)#

#X1(rw2#[, {\em options}]#)#
\item[] {\em Example}:

Suppose we have a continuous covariate #X1#, whose effect is
assumed to be nonlinear. The following model statement defines a
second order random walk prior for $f_1$:

#Y = X1(rw2)#

Here, the expression #X1(rw2)# indicates, that the effect of #X1#
should be incorporated nonparametrically into the model using a
second order random walk prior. A first order random walk is
specified in the model statement by modifying #rw2# to #rw1# which
yields the term #X1(rw1)#.

\item[] {\bf\sffamily P-spline with first or second order random
walk penalty}

\item[] {\em Description}: Defines a P-spline with a first or second order random walk penalty for
the parameters of the spline.
\item[] {\em Predictor}: $\eta =  \cdots + f_1(X1) + \cdots$
\item[] {\em Syntax}:

#X1(psplinerw1#[, {\em options}]#)#

#X1(psplinerw2#[, {\em options}]#)#
\item[] {\em Example}:

For example, a P-spline with second order random walk penalty is
obtained using the following model statement:

#Y = X1(psplinerw2)#

By default, the degree of the spline is 3 and the number of inner
knots is 20. The following model term defines a quadratic P-spline
with 30 knots:

#Y = X1(psplinerw2,degree=2,nrknots=30)#

\item[]{\bf\sffamily Seasonal component for time scales}

\item[] {\em Description}: Defines a seasonal effect of #time#.
\item[] {\em Predictor}: $\eta =  \cdots + f_{season}(time) + \cdots $
\item[] {\em Syntax}:

#time(season#[, {\em options}]#)#
\item[] {\em Example}:

A seasonal component for a time scale #time# is specified for
example by

#Y = time(season,period=12)#

Here, the second argument specifies the period of the seasonal
effect. In the example above the period is 12, corresponding to
monthly data.
\end{itemize}

\subsubsection*{Nonlinear baseline effect in Cox models}

\begin{itemize}
\item[]{\bf\sffamily P-spline with second order random walk
penalty}

\item[] {\em Description}: Defines a P-spline with second order
random walk penalty for the parameters of the spline for the
log-baseline effect $\log(\lambda_0$(#time#)). \item[] {\em
Predictor}: $\eta = \log(\lambda_0(time)) + \cdots$ \item[] {\em
Syntax}:

#time(baseline#[, {\em options}]#) # \item[] {\em Example}:

Suppose continuous-time survival data (#time#, #delta#) together
with additional covariates (#W1#, #X1#) are given, where #time#
denotes the vector of observed duration times, #delta# is the vector
of corresponding indicators of non-censoring, #W1# is a discrete
covariate and #X1# a continuous one. The following Cox model with
hazard rate $\lambda$ and log-baseline effect
$\log(\lambda_0$(#time#))
\begin{eqnarray*}
 \lambda(time) & = & \lambda_0(time)\exp (\gamma_0 + \gamma_1 W1 + f(X1))\\
 & = & \exp\left(\log(\lambda_0(time)) + \gamma_0 + \gamma_1 W1 + f(X1)\right)
\end{eqnarray*}
is estimated by the model statement

#delta = time(baseline) + W1 + X1(psplinerw2)#

\end{itemize}

\subsubsection*{Spatial Covariates}

\begin{itemize}
\item[]{\bf\sffamily Markov random field}

\item[] {\em Description}:

Defines a Markov random field prior for the spatial covariate
#region#. {\em Remlreg objects} allow an appropriate incorporation
of spatial covariates with geographical information stored in the
{\em map object} specified through the option #map#.
\item[] {\em Predictor}: $\eta = \cdots + f_{spat}(region) + \cdots$
\item[] {\em Syntax}:

#region(spatial,map=#{\em characterstring}[, {\em options}]#)#
\item[] {\em Example}:

The specification of a Markov random field prior for spatial data
has #map# as a required argument which must be the name of a {\em
map object} (see \autoref{map}) that contains all necessary
spatial information about the geographical map, i.e.~the neighbors
of each region and the weights associated with the neighbors. For
example the statement

#Y = region(spatial,map=germany)#

defines a Markov random field prior for #region# where the
geographical information is stored in the {\em map object}
#germany#. An error will be raised if #germany# is not existing.

\newpage

\item[]{\bf\sffamily 2 dimensional P-spline with first order
random walk penalty}

\item[] {\em Description}:

Defines a 2 dimensional P-spline for the spatial covariate
#region# based on the tensor product of 1 dimensional P-splines
with a 2 dimensional first order random walk penalty for the
parameters of the spline. Estimation is based on the coordinates
of the centroids of the regions an observation pertains to. The
centroids are computed using the geographical information stored
in the {\em map object} specified through the option #map#.
\item[] {\em Predictor}: $\eta= \cdots + f(centroids) + \cdots$
\item[] {\em Syntax}:

#region(geospline,map=#{\em characterstring}[, {\em options}]#)#
\item[] {\em Example}:

The specification of a 2 dimensional P-spline  (#geospline#) for
spatial data has #map# as a required argument which must be the
name of a {\em map object} (see \autoref{map}) that contains all
necessary spatial information about the geographical map, i.e.~the
neighbors of each region and the weights associated with the
neighbors. The model term

#Y = region(geospline,map=germany)#

specifies a tensor product cubic P-spline with first order random
walk penalty where the geographical information is stored in the
{\em map object} #germany#.

\item[]{\bf\sffamily Stationary Gaussian random field}

\item[] {\em Description}:

Defines a stationary Gaussian random field for the spatial covariate
#region#. Estimation is based on the coordinates of the centroids of
the regions an observation pertains to. The centroids are computed
using the geographical information stored in the {\em map object}
specified through the option #map#.
\item[] {\em Predictor}: $\eta= \cdots + f(centroids) + \cdots$
\item[] {\em Syntax}:

#region(geokriging,map=#{\em characterstring}[, {\em options}]#)#
\item[] {\em Example}:

The specification of a stationary Gaussian random field
(#geokriging#) for spatial data has #map# as a required argument
which must be the name of a {\em map object} (see \autoref{map})
that contains all necessary spatial information about the
geographical map, i.e.~the neighbors of each region and the weights
associated with the neighbors. The model term

#Y = region(geokriging,map=germany)#

specifies a stationary Gaussian random field where the geographical
information is stored in the {\em map object} #germany#.
\end{itemize}

\subsubsection*{Unordered group indicators}

\begin{itemize}
\item[]{\bf\sffamily Unit- or cluster specific unstructured
effect}

\item[] {\em Description}: Defines an unstructured (uncorrelated)
random effect with respect to grouping variable #grvar#. \item[]
{\em Predictor}: $\eta = \cdots + f(grvar) + \cdots$ \item[] {\em
Syntax}:

#grvar(random#[, {\em options}]#)#
\item[] {\em Example}:

Method regress supports Gaussian i.i.d.~random effects to cope with
unobserved heterogeneity among units or clusters of observations.
Suppose the analyzed data set contains a group indicator #grvar#
that gives information about the individual or cluster a particular
observation belongs to. Then an individual or cluster specific
uncorrelated random effect is incorporated through the term

#Y = grvar(random)#

The inclusion of more than one random effect term in the model is
possible, allowing the estimation of multilevel models. However,
we have only limited experience with multilevel models so that it
is not clear how well these models can be estimated using {\em
remlreg objects}.
\end{itemize}

\subsubsection*{Varying coefficients with metrical covariates as
effect modifier}

\begin{itemize}
\item[]{\bf\sffamily First or second order random walk}

\item[] {\em Description}:

Defines a varying coefficient term, where the effect of #X1# varies
smoothly over the range of #X2#. Covariate #X2# is the effect
modifier. The smoothness prior for $f$ is a first or second order
random walk.
\item[] {\em Predictor}: $\eta= \cdots + f(X2)X1 + \cdots$
\item[] {\em Syntax}:

#X1*X2(rw1#[, {\em options}]#)#

#X1*X2(rw2#[, {\em options}]#)#
\item[] {\em Example}:

For example, a varying coefficient term with a second order random
walk smoothness prior is defined as follows:

#Y = X1*X2(rw2)#

\item[]{\bf\sffamily P-spline with first or second order random
walk penalty}

\item[] {\em Description}:

Defines a varying coefficient term, where the effect of #X1# varies
smoothly over the range of #X2#. Covariate #X2# is the effect
modifier. The smoothness prior for $f$ is a P-spline with first or
second order random walk penalty.
\item[] {\em Predictor}: $\eta= \cdots + f(X2)X1 + \cdots$
\item[] {\em Syntax}:

#X1*X2(psplinerw1#[, {\em options}]#)#

#X1*X2(psplinerw2#[, {\em options}]#)#
\item[] {\em Example}:

For example, a varying coefficient term with a second order random
walk smoothness prior is defined as follows:

#Y = X1*X2(psplinerw2)#

\item[]{\bf\sffamily Seasonal prior}

\item[] {\em Description}:

Defines a varying coefficients term where the effect of #X1# varies
over the range of the effect modifier #time#. For #time# a seasonal
prior is used.
\item[] {\em Predictor}: $\eta= \cdots + f_{season}(time)X1 + \cdots $
\item[] {\em Syntax}:

#X1*time(season#[, {\em options}]#)#
\item[] {\em Example}:

The inclusion of a varying coefficients term with a seasonal prior
may be meaningful if we expect a different seasonal effect with
respect to grouping variable #X1#. In this case we can include
additional seasonal effects for each category of #X1# by

#Y = X1*time(season)#

\end{itemize}

\subsubsection*{Time-varying effects in Cox models}

\begin{itemize}
\item[]{\bf\sffamily P-spline with second order random walk
penalty}

\item[] {\em Description}: Defines a varying coefficients term
where the effect of #X1# varies over the range of the effect
modifier #time#, i.e. variable #X1# has time-varying effect. The
smoothness prior for $f($#time#$)$ is a P-spline with second order
random walk penalty.

 \item[] {\em Predictor}: $\eta = \log(\lambda_0(time)) +
f(time)X1 \cdots$ \item[] {\em Syntax}:

 #X1*time(baseline#[, {\em options}]#) #
 \item[] {\em Example}:

Suppose continuous-time survival data (#time#, #delta#) together
with an additional covariate #X1# are given, where #time# denotes
the vector of observed duration times, #delta# is the vector of
corresponding indicators of non-censoring. The following Cox model
with hazard rate $\lambda$
\begin{eqnarray*}
 \lambda(time) & = & \lambda_0(time)\exp(\gamma_0 + f(time)X1)\\
 & = & \exp\left(\log(\lambda_0(time)) + \gamma_0 + f(time)X1\right)
\end{eqnarray*}
is estimated by the model statement

#delta = time(baseline) + X1*time(baseline)#

\end{itemize}

\subsubsection*{ Varying coefficients with spatial covariates as
effect modifiers}

\begin{itemize}
\item[]{\bf\sffamily Markov random field}

\item[] {\em Description}:

Defines a varying coefficient term where the effect of #X1# varies
smoothly over the range of the spatial covariate #region#. A
Markov random field is estimated for $f_{spat}$. The geographical
information is stored in the {\em map object} specified through the
option #map#.
\item[] {\em Predictor}: $\eta = \cdots + f_{spat}(region)X1 + \cdots$
\item[] {\em Syntax}:

#X1*region(spatial,map=#{\it characterstring} #[,# {\it options}#])#
\item[] {\em Example}:

For example the statement

#Y = X1*region(spatial,map=germany)#

defines a varying coefficient term with the spatial covariate
#region# as the effect modifier and a Markov random field as spatial
smoothness prior. Weighted Markov random fields can be estimated by
including an appropriate weight definition when creating the {\em
map object} #germany# (see \autoref{mapinfile}).
\end{itemize}

\subsubsection*{Varying coefficients with unordered group indicators
as effect modifiers (random slopes)}

\begin{itemize}
\item[]{\bf\sffamily Unit- or cluster specific unstructured
effect}

\item[] {\em Description}:

Defines a varying coefficient term where the effect of #X1# varies
over the range of the group indicator #grvar#. Models of this type
are usually referred to as models with random slopes. A Gaussian
i.i.d.~random effect with respect to grouping variable #grvar# is
assumed for $f$.
\item[] {\em Predictor}: $\eta = \cdots + f(grvar)X1 + \cdots$
\item[] {\em Syntax}:

#X1*grvar(random#[, {\em options}]#)#
\item[] {\em Example}:

For example, a random slope is specified as follows:

#Y = X1*grvar(random)#

Note, that in contrast to {\em bayesreg objects} no main effects
are included automatically. If main effects should be included in
the model, they have to be specified as additional fixed effects.
The syntax for obtaining the predictor

$\eta = \cdots + \gamma X1 + f(grvar)X1 + \cdots$

would be

#X1 + X1*grvar(random#[, {\em options}]#)#

\end{itemize}

\subsubsection*{Surface estimators}

\begin{itemize}
\item[]{\bf\sffamily 2 dimensional P-spline with first order
random walk penalty}

\item[] {\em Description}:

Defines a 2 dimensional P-spline based on the tensor product of 1
dimensional P-splines with a 2 dimensional first order random walk
penalty for the parameters of the spline.
\item[] {\em Predictor}: $\eta= \cdots + f(X1,X2) + \cdots$
\item[] {\em Syntax}:

#X1*X2(pspline2dimrw1#[, {\em options}]#)#
\item[] {\em Example}:

The model term

#Y = X1*X2(pspline2dimrw1)#

specifies a tensor product cubic P-spline with first order random
walk penalty.

In many applications it is favorable to additionally incorporate the
1 dimensional main effects of #X1# and #X2# into the models. In this
case the 2 dimensional surface can be seen as the deviation from the
main effects. Note, that in contrast to {\em bayesreg objects} the
number of inner knots and the degree of the spline may be different
for the main effects and for the interaction. For example, a model
with 20 inner knots for the main effects and 10 inner knots for the
2 dimensional P-spline is estimated by

 #Y = X1(psplinerw2,nrknots=20) + X2(psplinerw2,nrknots=20)#\\
 #    + X1*X2(pspline2dimrw1,nrknots=10)#

\item[]{\bf\sffamily Stationary Gaussian random field}

\item[] {\em Description}:

Defines that the parameters of the locations follow a stationary
Gaussian random field. Depending on the chosen options, locations
are given either by the distinct pairs of #X1# and #X2# or by a
subset of these pairs, which we will also refer to as knots. Note
that, although stationary Gaussian random fields can be used to
estimate surfaces depending on arbitrary variables #X1# and #X2#,
they are defined based on {\em isotropic} correlation functions.
This means that correlations between sites that have the same
distance also have the same correlation, regardless of direction
and the sites location. Therefore, if Gaussian random fields shall
be used to estimate interactions between variables that do not
represent longitude and latitude, these variables have to be
standardized appropriately.

\item[] {\em Predictor}: $\eta= \cdots + f(X1,X2) + \cdots$
\item[] {\em Syntax}:

#X1*X2(kriging#[, {\em options}]#)# \item[] {\em Example}:

The model term

#Y = X1*X2(kriging,nrknots=100)#

specifies a stationary Gaussian random field for the effect of #X1#
and #X2# with 100 knots, which are computed based on the space
filling algorithm described in section \ref*{spatial} of the
methodology manual. If all distinct pairs of #X1# and #X2# shall be
used as knots, we have to specify

#Y = X1*X2(kriging,full)#

Note, that the knots computed by the space filling algorithm are
stored in {\it knotfile} in the outfile directory of the {\em
remlreg object}. These knots can be read into a {\em dataset object}
which may be passed in the call of method #regress# if we want to
use the same knots as in previous calls:

 #dataset kn#\\
 #kn.infile using #{\em knotfile}\\
 #Y = X1*X2(kriging,knotdata=kn)#

To determine the actual number of knots, the options are
interpreted in a specific sequence. If option #full# is specified,
both #nrknots# and #knotdata# are ignored. Similarly, #nrknots# is
ignored if #knotdata# is specified.

\end{itemize}

\subsubsection{Description of additional options for terms of {\em remlreg objects}}
\label{remlreglocaloptions}

All arguments described in this section are optional and may be
omitted. Generally, options are specified by adding the option
name to the specification of the model term type in the
parentheses, separated by comma. Note that all options may be
specified in arbitrary order. \autoref{remlregoptions} provides
explanations and the default values of all possible options. In
\autoref{remlregtermsoptions} all reasonable combinations of model
terms and options can be found.

\begin{table}[ht] \footnotesize \centering
\begin{tabular}{|p{0.1\linewidth}|p{0.6\linewidth}|p{0.2\linewidth}|}
 \hline
 optionname & description & default\\
 \hline\hline
 #lambdastart# & Provides a starting value for the smoothing parameter $\lambda$. & #lambdastart=10# \\
 \hline
 #degree# & Specifies the degree of the B-spline basis functions. & #degree=3# \\
 \hline
 #nrknots# & Specifies the number of inner knots for a P-spline term or the number of knots for a kriging term. & #nrknots=20# (P-splines)\newline #nrknots=100# (kriging)  \\
 \hline
 #knotdata# & {\em Dataset object} containing the knots to be used
 with the kriging term & no default.\\
 \hline
 #full# & Specifies that all distinct locations should be used as
 knots with the kriging term. & -\\
 \hline
 #nu# & The smoothness parameter $\nu$ of the Mat\`{e}rn correlation function. & #nu=1.5# \\
 \hline
 #maxdist# & Specifies the value $c$ that is used to determine the scale parameter $\rho$ of the Mat\`{e}rn correlation function. Compare section \ref*{spatial} of the methodology manual. & default depends on #nu#\\
 \hline
 #p# & Defines the parameter $p$ used in the coverage criterion of the space filling algorithm. & #p=-20#\\
 \hline
 #q# & Defines the parameter $q$ used in the coverage criterion of the space filling algorithm. & #q=20#\\
 \hline
 #maxsteps# & Specifies the maximum number of steps to be performed by the space filling algorithm. & #maxsteps=300#\\
 \hline
 #gridchoice# & How to choose grid points for numerical integration in Cox models. May be either '#quantiles#' or '#equidistant#'. & #gridchoice=quantiles# \\
 \hline
 #tgrid# & Number of equidistant time points to be used for numerical integration in Cox models. Only meaningful if #gridchoice=equidistant#. & #tgrid=100#\\
 \hline
 #nrquantiles# & Number of quantiles that are used to define the grid points for numerical integration in Cox models. First a grid of #nrquantiles# quantiles is computed, then the grid for integration is defined by #nrbetween# equidistant points between each quantile. Only meaningful if #gridchoice=quantiles#. & #nrquantiles=50#\\
 \hline
 #nrbetween# & Number of points between quantiles that are used to define the grid points for numerical integration in Cox models. First a grid of #nrquantiles# quantiles is computed, then the grid for integration is defined by #nrbetween# equidistant points between each quantile. Only meaningful if #gridchoice=quantiles#.& #nrbetween=5#\\
 \hline
 #map# & {\em Map object} for spatial effects. & no default\\
 \hline
 #period# & The period of the seasonal effect can be specified with the option #period#. The default is #period=12# which corresponds to monthly data. & #period=12# \\
 \hline
\end{tabular}
{\em \caption{\label{remlregoptions} Optional arguments for {\em
remlreg object} terms}}
\end{table}

\begin{sidewaystable} \footnotesize
\begin{tabular}{|l||c|c|c|c|c|c|c|c|c|c|}

\hline
            & rw1/rw2       & season    & psplinerw1/psplinerw2    & spatial & random & geospline & pspline2dimrw1 & kriging  & geokriging & baseline\\
 \hline\hline
 #lambdastart#$^*$  & realvalue   & realvalue   & realvalue   & realvalue   & realvalue   & realvalue   & realvalue & realvalue  & realvalue & realvalue\\
 \hline
 #degree#       & $\times$   & $\times$   &  integer   & $\times$ & $\times$ &  integer &  integer &  $\times$ & $\times$ & integer\\
 \hline
 #nrknots#      & $\times$   & $\times$   &  integer   & $\times$ & $\times$ &  integer &  integer &  integer & $\times$ & integer\\
 \hline
 #knotdata#     & $\times$   & $\times$   &  $\times$   & $\times$ & $\times$ &  $\times$ &  $\times$ & {\em dataset object}& {\em dataset object} & $\times$\\
 \hline
 #full#     & $\times$   & $\times$   &  $\times$   & $\times$ & $\times$ &  $\times$ &  $\times$ &  $\triangle$ & $\triangle$ & $\times$\\
 \hline
 #nu#     & $\times$   & $\times$   &  $\times$   & $\times$ & $\times$ &  $\times$ &  $\times$ &  $\bullet$ &  $\bullet$ & $\times$\\
 \hline
 #maxdist#$^*$     & $\times$   & $\times$   &  $\times$   & $\times$ & $\times$ &  $\times$ &  $\times$ &  realvalue &  realvalue &  $\times$\\
 \hline
 #p#$^{**}$     & $\times$   & $\times$   &  $\times$   & $\times$ & $\times$ &  $\times$ &  $\times$ &  realvalue &  realvalue &  $\times$\\
 \hline
 #q#$^*$     & $\times$   & $\times$   &  $\times$   & $\times$ & $\times$ &  $\times$ &  $\times$ &  realvalue &  realvalue &  $\times$\\
 \hline
 #maxsteps#     & $\times$   & $\times$   &  $\times$   & $\times$ & $\times$ &  $\times$ &  $\times$ &  integer  &  integer & $\times$\\
 \hline
 #gridchoice#   & $\times$  & $\times$  & $\times$  & $\times$  & $\times$  & $\times$  & $\times$  & $\times$  & $\times$ & $\circ$\\
 \hline
 #tgrid#   & $\times$  & $\times$  & $\times$  & $\times$  & $\times$  & $\times$  & $\times$  & $\times$  & $\times$ & integer\\
 \hline
 #nrquantiles#   & $\times$  & $\times$  & $\times$  & $\times$  & $\times$  & $\times$  & $\times$  & $\times$  & $\times$ & integer\\
 \hline
 #nrbetween#   & $\times$  & $\times$  & $\times$  & $\times$  & $\times$  & $\times$  & $\times$  & $\times$  & $\times$ & integer\\
 \hline
 #period#      & $\times$   & integer     & $\times$  & $\times$      & $\times$  & $\times$ & $\times$ & $\times$  & $\times$ & $\times$\\
 \hline
 #map#      & $\times$   & $\times$     & $\times$  & {\em map object}  & $\times$  & {\em map object} & $\times$ & $\times$ & {\em map object} & $\times$ \\
 \hline \hline
 $^*$ & \multicolumn{10}{l|}{positive values only}\\
 \hline
 $^{**}$ & \multicolumn{10}{l|}{negative values only}\\
 \hline
 $\times$    & \multicolumn{10}{l|}{not available} \\
 \hline
 $\bullet$  & \multicolumn{10}{l|}{admissible values are #0.5,1.5,2.5,3.5#} \\
 \hline
 $\triangle$   & \multicolumn{10}{l|}{available as boolean option (specified without supplying a value)} \\
 \hline
 $\circ$  & \multicolumn{10}{l|}{admissible values are '#quantiles#' and '#equidistant#'} \\
 \hline
\end{tabular}
{\em\centering \caption{\label{remlregtermsoptions} Terms and
options for remlreg objects}}
\end{sidewaystable}

\subsubsection{Specifying the response distribution}
\label{remlregfamilysyntax}

The current version of {\em BayesX} supports the most common
univariate distributions of the response for the use with {\em
remlreg objects}. These are Gaussian, binomial (with logit or
probit link), Poisson and gamma. An overview over the supported
models is given in \autoref{remlregfamilyopt}. In {\em BayesX} the
distribution of the response is specified by adding the additional
option #family# to the options list of the regress command. For
instance, #family=gaussian# defines the responses to be Gaussian.
In the following we give detailed instructions on how to specify
the different models:

\begin{table}[ht]
\begin{center}
\begin{tabular} {|l|l|p{2.7cm}|l|}
 \hline
 value of #family# & response distribution & link & options\\
 \hline
 \hline
 #family=gaussian#            & Gaussian              & identity & \\
 \hline
 #family=binomial#            & binomial              & logit & \\
 #family=binomialprobit#      & binomial              & probit & \\
 #family=binomialcomploglog#      & binomial              & complementary log-log & \\
 \hline
 #family=multinomial#         & unordered multinomial & logit & #reference#\\
 \hline
 #family=cumprobit#           & cumulative multinomial   & probit & \\
 #family=cumlogit#            & cumulative multinomial   & logit & \\
 \hline
 #family=seqprobit#           & sequential multinomial   & probit & \\
 #family=seqlogit#            & sequential multinomial   & logit & \\
 \hline
 #family=poisson#             & Poisson               & log & \\
 \hline
 #family=gamma#               & gamma                 & log & \\
 \hline
 #family=cox#                 & continuous-time survival data & & \\
 \hline
\end{tabular}
{\em \caption {\label{remlregfamilyopt} Summary of supported
response distributions.}}
\end{center}
\end{table}

\subsubsection*{Gaussian responses}

For Gaussian responses {\em BayesX} assumes $y_i | \eta_i,\sigma^2
\sim N(\eta_i,\sigma^2/weightvar_i)$ or equivalently in matrix
notation $y | \eta, \sigma^2 \sim N(\eta,\sigma^2C^{-1})$. Here
$C^{-1}=diag(weightvar_1,\dots,weightvar_n)$ is a known covariance
matrix. A Gaussian response is specified by adding

#family=gaussian#

to the options list.

An optional weight variable {\em weightvar} may be specified to
estimate weighted regression models, see
\autoref{remlregregresssyntax} for the syntax. For grouped
Gaussian responses the weights must be the number of observations
in the groups if the $y_i$'s are the average of individual
responses. If the $y_i$'s are the sum of responses in every group,
the weights must be the reciprocal of the number of observations
in the groups. Of course, estimation of usual weighted regression
models if the errors are heteroscedastic is also possible. In this
case the weights should be proportional to the reciprocal of the
heteroscedastic variances. If a weight variable is not specified,
{\em BayesX} assumes $weightvar_i = 1$, $i=1,\dots,n$.

\subsubsection*{Binomial logit, probit and complementary log-log
models}

A binomial logit model is specified by adding the option

#family=binomial#

a probit model by adding

#family=binomialprobit#

and a complementary log-log model by adding

#family=binomialcomploglog#

to the options list.

A weight variable may be additionally specified, see
\autoref{remlregregresssyntax} for the syntax. {\em BayesX}
assumes that the weight variable corresponds to the number of
replications and the response variable to the number of successes.
If the weight variable is omitted, {\em BayesX} assumes that the
number of replications is one, i.e.~the values of the response
must be either zero or one.

\subsubsection*{Multinomial logit models}

So far {\em remlreg objects} support only multinomial logit
models. A multinomial logit model is specified by adding the
option

#family=multinomial#

to the options list.

Usually a second option must be added to the options list to
define the reference category. This is achieved by specifying the
#reference# option. Suppose that the response variable has three
categories 1,2 and 3. To define, for instance, the reference
category to be 2, simply add

#reference=2 #

to the options list. If this option is omitted, the {\em smallest}
number will be used as the reference category.

\subsubsection*{Cumulative logit and probit models}

A cumulative logit model is specified by adding

#family=cumlogit#

to the options list, a cumulative probit model by adding

#family=cumprobit#

to the options list. The reference category will always be the
largest value of the response.

Note, that in contrast to {\em bayesreg objects} {\em remlreg
objects} can deal with an arbitrary number of ordered categories.
However, for more than 5 categories estimation will become rather
computer intensive and time demanding.

\subsubsection*{Sequential logit and probit models}

A sequential logit model is specified by adding

#family=cumlogit#

to the options list, a sequential probit model by adding

#family=cumprobit#

to the options list. The reference category will always be the
largest value of the response.

\subsubsection*{Poisson regression}

A Poisson regression is specified by adding

#family=poisson#

to the options list.

A weight variable may be additionally specified, see
\autoref{remlregregresssyntax} for the syntax. For grouped Poisson
data the weights must be the number of observations in a group and
the responses are assumed to be the average of individual
responses.

\subsubsection*{Gamma distributed responses}

In the literature, the density function of the gamma distribution
is parameterized in various ways. In the context of regression
analysis the density is usually parameterized in terms of the mean
$\mu$ and the scale parameter $s$. Then, the density of a gamma
distributed random number $y$ is given by
\begin{equation}
\label{remlgammapar1} p(y) \propto y^{s-1}\exp(-\frac{s}{\mu} y)
\end{equation}
for $y > 0$. For the mean and the variance we obtain $E(y) = \mu$
and $Var(y) = \mu^2/s$. We write $y \sim G(\mu,s)$.

A second parameterization is based on hyperparameters $a$ and $b$
and is usually used in the context of Bayesian hierarchical models
to specify hyperpriors for variance components. The density is
then given by
\begin{equation}
\label{remlgammapar2} p(y) \propto y^{a-1}\exp(-b y)
\end{equation}
for $y>0$. In this parameterization we obtain $E(y) = a/b$ and
$Var(y) = a/b^2$ for the mean and the variance, respectively. We
write $y \sim G(a,b)$

In {\em BayesX} a gamma distributed response is defined in the first
parameterization (\ref{remlgammapar1}). For the $r$th observation
{\em BayesX} assumes  $y_r | \eta_r,\nu \sim
G(\exp(\eta_r),\nu/weightvar_r)$ where $\mu_r = \exp(\eta_r)$ is the
mean and $s=\nu/weightvar_r$ is the scale parameter. A Gamma
distributed response is specified by adding

#family=gamma#

to the options list. An optional weight variable {\em weightvar}
may be specified to estimate weighted regression models. In this
case the weights should be proportional to the reciprocal of the
heteroscedastic variances, see \autoref{remlregregresssyntax} for
the syntax.

\subsubsection*{Continuous time survival analysis}

\textit{BayesX} offers two alternatives of estimating continuous
time Cox models with semiparametric predictor $\eta$, which are
described in section \ref*{continuoustime} of the methodology
manual. The first alternative is to assume that all time-dependent
effects are piecewise constant, which leads to the so called
\textit{piecewise exponential model} (p.e.m.), and the second one is
to estimate the log-baseline effect $\log(\lambda_0(t))=f_0(t)$ by a
P-spline with second order random walk penalty.

\subsubsection*{Piecewise exponential model (p.e.m.)}

In section \ref*{continuoustime} of the methodology manual we
demonstrated how continuous time survival data has to be manipulated
such that a Poisson model may be used for estimation. Suppose now we
have the modified data set
\vspace{0.5cm}\\
\begin{tabular}{c|c|c|c|c|c|c}
#y# & #indnr# & #a# & $\delta$ &  $\Delta$ &   #x1# &
#x#2\\\hline\hline
0 &  1 &   0.1 &   1  &  log(0.1) & 0  & 3\\
0  & 1   & 0.2  &  1  &  log(0.1) & 0 &  3\\
1  & 1   & 0.3  &  1  &  log(0.05)& 0  & 3\\\hline
0 &  2 &   0.1 &   0 &   log(0.1) & 1 &  5\\
0  & 2  &  0.2 &   0  &  log(0.02)& 1 &  5\\\hline
$\vdots$ & $\vdots$ & $\vdots$ & $\vdots$ & $\vdots$ & $\vdots$& $\vdots$\\
\end{tabular}
\vspace{0.5cm}\\
with indicator #y#, interval limit #a#, indicator of non-censoring
$\delta$ and offset $\Delta$ defined as in section
\ref*{continuoustime} of the methodology manual. Let #x1# be a
covariate with linear effect and #x2# a continuous one with a
nonlinear effect. Then the correct syntax for estimating a
p.e.m.~with a {\em remlreg object} named #r# is e.g.~as follows:

 #> r.regress y = a(rw1) + Delta(offset) + x1 + x2(psplinerw2), family=poisson# $\ldots$

or

 #> r.regress y = a(rw2) + Delta(offset) + x1 + x2(psplinerw2), family=poisson# $\ldots$

Note that a time-varying effect of a covariate #X# may be
estimated in the p.e.m.~by adding the term

#X*a(rw1) or X*a(rw2)#

to the model statement.

\subsubsection*{Specifying a P-spline prior for the log-baseline}

For the estimation of a Cox model with a P-spline prior with
second order random walk penalty

#family=cox#

has to be specified in the options list. The number of knots and
degree of the P-spline prior for $f_0(t)$ may be specified in the
baseline term. The indicator of non-censoring $\delta_i$ has to be
specified as the dependent variable in the model statement. Data
augmentation and the specification of an offset term are not
required here.

In the example above with survival data

\vspace{0.5cm}

\begin{tabular}{c|c|c|c}
  #t# &   $\delta$ &  #x1# &  #x2#\\\hline\hline
0.25  &  1  &    0  &  3\\\hline 0.12  &  0  &    1  &  5\\\hline
$\vdots$ & $\vdots$ & $\vdots$ & $\vdots$ \\
\end{tabular}
\vspace{0.5cm}\\
a Cox model with a quadratic P-spline prior with 15 knots for the
log-baseline would be estimated as follows:

 #> r.regress delta = t(baseline,degree=2,nrknots=15)+ x1 + x2(psplinerw2),#\\
 #  family=cox# \ldots

Note, that we assume that a {\em remlreg object} #r# has been
created before executing the command.

\subsection{Options}
\label{remlregregressoptions}

\subsubsection*{Options for controlling the estimation process}
\label{remlest_options}

Options for controlling estimation process are listed in
alphabetical order.

\begin{itemize}
\item #eps = #{\em realvalue } \\
\\Defines the termination criterion of the estimation process. If
both the relative changes in the regression coefficients and the
variance parameters are less than #eps#, the estimation process is
assumed to have converged.\\
DEFAULT: #eps = 0.00001#

\item #lowerlim = #{\em realvalue } \\
Since small variances are near to the boundary of their parameter
space, the usual Fisher-scoring algorithm for their determination
has to be modified. If the fraction of the penalized part of an
effect relative to the total effect is less than #lowerlim#, the
estimation of the corresponding variance is stopped and the
estimator is defined to be the current value of the variance (see
section \ref*{glmmmeth} of the methodology manual for details).\\
DEFAULT: #lowerlim = 0.001#

\item #maxit = #{\em integer } \\
Defines the maximum number of iterations to be used in estimation.
Since the estimation process will not necessarily converge, it may
be useful to define an upper bound for the number of iterations.
Note, that {\it BayesX} produces results based on the current
values of all parameters even if no convergence could be achieved
within #maxit# iterations, but a warning message will be printed
in the {\it output window}.\\
DEFAULT: #maxit=400#

\item #maxchange = #{\em realvalue } \\
Defines the maximum value that is allowed for relative changes in
parameters in one iteration to prevent the program from crashing
because of numerical problems. Note, that {\it BayesX} produces
results based on the current values of all parameters even if the
estimation procedure is stopped due to numerical problems, but an
error message will be printed in the {\it output window}.\\
DEFAULT: #maxchange=1000000#

\end{itemize}

\subsubsection*{Further options} \label{remlreg_further_options}

\index{credible intervals} \index{credible intervals!changing the
nominal level} \index{changing the nominal level of credible
intervals}\index{remlreg object!credible intervals}
\begin{itemize}
\item \label{remlreglevel1} #level1 = #{\em integer} \\
Besides the posterior mode, #regress# provides (approximate)
pointwise posterior credible intervals for every effect in the
model. By default, {\em BayesX} computes credible intervals for
nominal levels of 80\% and 95\%. The option #level1# allows to
redefine one of the nominal levels (95\%). Adding, for instance,

#level1=99 #

to the options list leads to the computation of credible intervals
for a nominal level of 99\% rather than 95\%.
\item \label{remlreglevel2} #level2 = #{\em integer} \\
Besides the posterior mode, #regress# provides (approximate)
pointwise posterior credible intervals for every effect in the
model. By default, {\em BayesX} computes credible intervals for
nominal levels of 80\% and 95\%. The option #level2# allows to
redefine one of the nominal levels (80\%). Adding, for instance,

#level2=70#

to the options list leads to the computation of credible intervals
for a nominal level of 70\% rather than 80\%.
\end{itemize}

\subsection{Estimation output}

The way the estimation output is presented depends on the estimated
model. Estimation results for fixed effects are displayed in a
tabular form in the {\em output window} and/or in a log file (if
created before). Shown will be the posterior mode, the standard
deviation, p-values and an approximate 95\% credible interval. Other
credible intervals may be obtained by specifying the #level1#
option, see \autoref{remlregregressoptions} for details.
Additionally a file is created where estimation results for fixed
effects are replicated. The name of the file is given in the {\em
output window} and/or in a log file.

Estimation effects for nonparametric effects are presented in a
different way. Here, results are stored in external ASCII-files
whose contents can be read into any general purpose statistics
program (e.g. STATA, S-plus) to further analyze and/or visualize the
results. The structure of these files is as follows: There will be
one file for every nonparametric effect in the model. The names of
the files and the storing directory are displayed in the {\em output
window} and/or a log file. The files contain ten or eleven columns,
depending on whether the corresponding model term is an interaction
effect. The first column contains a parameter index (starting with
one), the second column (and the third column if the estimated
effect is a 2 dimensional P-spline) contain the values of the
covariate(s) whose effect is estimated. In the following columns the
estimation results are given in form of the posterior mode, the
lower boundaries of the (approximate) 95\% and 80\% credible
intervals, the standard deviation and the upper boundaries of the
80\% and 95\% credible intervals. The last two columns contain
approximations to the posterior probabilities based on nominal
levels of 95\% and 80\%. A value of 1 corresponds to a strictly
positive 95\% or 80\% credible interval and a value of -1 to a
strictly negative credible interval. A value of 0 indicates that the
corresponding credible interval contains zero. Other credible
intervals and posterior probabilities may be obtained by specifying
the #level1# and/or #level2# option, see
\autoref{remlregregressoptions} for details. As an example compare
the following lines, which are the beginning of a file containing
the results for a nonparametric effect of a particular covariate, x
say:

\footnotesize
 intnr \,\, x \,\, pmode \,\, ci95lower \,\, ci80lower \,\, std \,\, ci80upper \,\, ci95upper \,\, pcat95 \,\, pcat80\\
 1 \,\, -2.87694 \,\, -0.307921 \,\, -0.886815 \,\, -0.686408 \,\, 0.295295 \,\, 0.070567   \,\, 0.270973 \,\, 0 \,\, 0\\
 2 \,\, -2.86203 \,\, -0.320479 \,\, -0.885375 \,\, -0.689815 \,\, 0.288154 \,\, 0.0488558  \,\, 0.244416 \,\, 0 \,\, 0\\
 3 \,\, -2.8515  \,\, -0.329367 \,\, -0.88473  \,\, -0.69247  \,\, 0.283292 \,\, 0.0337362  \,\, 0.225997 \,\, 0 \,\, 0\\
 4 \,\, -2.85066 \,\, -0.330072 \,\, -0.884692 \,\, -0.692689 \,\, 0.282913 \,\, 0.0325457  \,\, 0.224549 \,\, 0 \,\, 0\\
 5 \,\, -2.82295 \,\, -0.3535   \,\, -0.884544 \,\, -0.700703 \,\, 0.270887 \,\,-0.00629671 \,\, 0.177545 \,\, 0 \,\, -1\\
 6 \,\, -2.79856 \,\, -0.37418  \,\, -0.886192 \,\, -0.708939 \,\, 0.261178 \,\,-0.0394208  \,\, 0.137832 \,\, 0 \,\, -1\\
 7 \,\, -2.79492 \,\, -0.377272 \,\, -0.886579 \,\, -0.710263 \,\, 0.259798 \,\,-0.0442813  \,\, 0.132035 \,\, 0 \,\, -1\\
 8 \,\, -2.79195 \,\, -0.379788 \,\, -0.886921 \,\, -0.711358 \,\, 0.258689 \,\,-0.0482183  \,\, 0.127345 \,\, 0 \,\, -1\\
 9 \,\, -2.78837 \,\, -0.382834 \,\, -0.887367 \,\, -0.712704 \,\, 0.257363 \,\,-0.0529641  \,\, 0.1217   \,\, 0 \,\, -1
\normalsize

Note that the first row of the files always contains the names of
the columns.

The estimated nonlinear effects can be visualized using either the
graphics capabilities of {\em BayesX} (Java based version only) or a
couple of S-plus functions,  see \autoref{bayesxplot} and
\autoref{splus}, respectively. Of course, any other (statistics)
software package with plotting facilities may be used as well.

Estimation results for the variances and the smoothing parameters
of nonparametric effects are printed in the {\em output window}
and/or a log file. Additionally, a file is created containing the
same information. For example, the file corresponding to the
nonparametric effect presented above contains:

\footnotesize
 variance \,\, smoothpar \,\, stopped\\
 0.0492324 \,\, 20.3118 \,\, 0
\normalsize

The value in the last row indicates whether the estimation of the
variance has been stopped before convergence. A value of 1
corresponds to a 'stopped' variance.

\subsection{Examples}

Here we give only a few examples about the usage of method
#regress#. A more detailed, tutorial like example can be found in
chapter \ref*{remlregzambiaanalysis} of the tutorial manual.

Suppose that we have a data set #test# with a binary response
variable #y#, and covariates #x1#, #x2#, #x3#, #t# and #region#,
where #t# is assumed to be a time scale measured in months and
#region# indicates the geographical region an observation belongs
to. Suppose further that we have already created a {\em remlreg
object} #r#.

\subsubsection*{Fixed effects}

We first specify a model with #y# as the response variable and
fixed effects for the covariates #x1#, #x2# and #x3#. Hence the
predictor is

$$
\eta = \gamma_0 + \gamma_1 x1 + \gamma_2 x2 + \gamma_3 x3
$$

This model is estimated by typing:

#> r.regress y = x1 + x2 + x3, family=binomial using test#

By specifying option #family=binomial#, a binomial logit model is
estimated. A probit model can be estimated by specifying
#family=binomialprobit#.

\subsubsection*{Additive models}

Suppose now that we want to allow for possibly nonlinear effects
of #x2# and #x3#. Defining cubic P-splines with second order
random walk penalty as smoothness priors, we obtain

 #> r.regress y = x1 + x2(psplinerw2) + x3(psplinerw2), family=binomial using test#

which corresponds to the predictor

$$
\eta = \gamma_0 + \gamma_1 x1 + f_1(x2) + f_2(x3).
$$

Suppose now for a moment that the response is not binary but
multicategorical with unordered categories 1, 2 and 3. In that
case we can estimate a multinomial logit model. Such a model is
estimated by typing:

 #> r.regress y = x1 + x2(psplinerw2) + x3(psplinerw2), family=multinomial#\\
 #  reference=2 using test#

That is, #family=binomial# was altered to #family=multinomial#,
and the option #reference=2# was added in order to define the
value 2 as the reference category.

\subsubsection*{Time scales}

In our next step we extend the model by incorporating an
additional trend and a flexible seasonal component for the time
scale #t#:

 #> r.regress y = x1 + x2(psplinerw2) + x3(psplinerw2) +  #\\
 #  t(psplinerw2) + t(season,period=12), family=binomial using test#

Note that we passed the period of the seasonal component as a
second argument.

\subsubsection*{Spatial covariates}

To incorporate a structured spatial effect, we first have to
create a {\em map object} and read in the boundary information of
the different regions (polygons that form the regions, neighbors
etc.). If you are unfamiliar with {\em map objects} please read
\autoref{map} first.

#> map m# \\
#> m.infile using c:\maps\map.bnd#

Since we usually need the map again in further sessions, we store
it in {\em graph file} format, because reading {\em graph files}
is much faster than reading {\em boundary files}.

#> m.outfile , graph using c:\maps\mapgraph.gra#

We can now extend our predictor with a spatial effect:

 #> r.regress y = x1 + x2(psplinerw2) + x3(psplinerw2) + t(psplinerw2)#\\
 #  + t(season,period=12) + region(spatial,map=m), family=binomial using test#

In some situations it may be reasonable to incorporate  an
additional unstructured  random effect into the model in order to
split the total spatial effect into a structured and an
unstructured component. This is done by typing

#> r.regress y = x1 + x2(psplinerw2) + x3(psplinerw2) + t(psplinerw2)#\\
#  + t(season,period=12) + region(spatial,map=m) + region(random),#\\
#  family=binomial using test#

\section{Global options}
\label{remlregglobopt} \index{remlreg object!global options}

The purpose of global options is to affect the global behavior of
a {\em remlreg object}. The main characteristic of global options
is, that they are not associated with a certain method.

The syntax for specifying global options is

{\em objectname}.{\em optionname} = {\em newvalue}

where {\em newvalue} is the new value of the option. The type of
the value depends on the respective option.

Currently only one global option is available for {\em remlreg
objects}:

\begin{itemize}
\item #outfile = #{\em filename} \\
By default, the estimation output produced by the #regress#
procedure will be written to the default output directory, which
is

{\em$<$INSTALLDIRECTORY$>$}#\output#

The default file name is composed of the name of the {\em remlreg
object} and the type of the file. For example, if you estimated a
nonparametric effect for a covariate #X#, say, using P-spline then
the estimation output will be written to

{\em$<$INSTALLDIRECTORY$>$}#\output\r_f_X_pspline.res#

where #r# is the name of the {\em remlreg object}. In most cases,
however, it may be necessary to save estimation results into a
different directory and/or under a different file name than the
default. This can be done using the #outfile# option. With the
outfile option you have to specify the directory where the output
should be stored to and in addition a base file name. The base
file name should not be a complete file name. For example
specifying

#outfile = c:\data\res1#

would force {\em BayesX} to store the estimation result for the
nonparametric effect of #X# in file

#c:\data\res1_f_X_pspline.res#
\end{itemize}

\section{Visualizing estimation results}

Visualization of estimation results is described in
\autoref{visualization}

\section{References}
\label{remlregreferences}

\begin{description}

\item[Fahrmeir, L., Kneib, T. and Lang, S. (2004):] Penalized
structured additive regression for space-time data: A Bayesian
perspective. {\it Statistica Sinica}, 14, 715-745.

\item[Fahrmeir, L. and Tutz, G. (2001):] {\em Multivariate
Statistical Modelling based on Generalized Linear Models.} New
York: Springer--Verlag.

\item[Green, P.J. and Silverman, B. (1994):] {\em Nonparametric Regression and Generalized Linear Models.} Chapman
and Hall, London.

\item[Hastie, T. and Tibshirani, R. (1990):] {\em Generalized additive models.} Chapman and
Hall, London.

\item[Hastie, T. and Tibshirani, R. (1993):] Varying-coefficient Models.
{\em Journal of the Royal Statistical Society B} , 55, 757-796.

\item[Hastie, T., Tisbshirani, R. and Friedman, J. (2001):] {\em The Elements of Statistical Learning: Data Mining,
Inference and Prediction.} New York: Springer--Verlag.

\item[Kneib, T. and Fahrmeir, L. (2005):] Structured additive
regression for multicategorical space-time data: A mixed model
approach. {\it Biometrics}, to appear.

\item[Kneib, T. and Fahrmeir, L. (2004):] A mixed model approach
for structured hazard regression. SFB 386 discussion Paper 400,
University of Munich. Available from
\href{http://www.stat.uni-muenchen.de/~kneib/papers.html}{www.stat.uni-muenchen.de/$\sim$kneib/papers.html}

\item[Brezger, A. (2000):]
\href{http://www.stat.uni-muenchen.de/~andib} {\em Bayesianische
P-splines.} Master thesis, University of Munich.

\item[Lin, X. and Zhang, D. (1999):] Inference in generalized additive mixed models by using
smoothing splines. {\it Journal of the Royal Statistical Society
B}, 61, 381--400.

\item[McCullagh, P. and Nelder, J.A. (1989):] {\em Generalized Linear Models.} Chapman and Hall, London.

\end{description}
