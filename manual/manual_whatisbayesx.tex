\chapter{What is BayesX?}

\begin{stanza}{General scope}

{\it BayesX} is a software tool for estimating structured additive
regression models. Structured additive regression embraces
several well-known regression models such as generalized additive
models (GAM), generalized additive mixed models (GAMM),
generalized geoadditive mixed models (GGAMM), dynamic models,
varying coefficient models, and geographically weighted regression
within a unifying framework. Besides exponential family
regression, {\em BayesX} also supports non-standard regression
situations such as regression for categorical responses, hazard
regression for continuous survival times, continuous time
multi-state models, quantile regression, distributional regression models and multilevel models.
\end{stanza}

\begin{stanza}{Inferential procedures}

Estimation of regression models can be achieved based on four
different inferential concepts that have been implemented in
separate regression objects:

\begin{itemize}
\item {\bf\sffamily MCMC simulation techniques (bayesreg objects):} A fully Bayesian interpretation of structured additive
    regression models is obtained by specifying prior distributions for all unknown parameters. Estimation can be
    facilitated using Markov chain Monte Carlo simulation techniques, a general and versatile concept for Bayesian
    inference. Bayesreg objects provide numerically efficient implementations of MCMC schemes for structured additive
    regression models in case of exponential family responses, categorical responses, hazard regression and multi-state models.
    Suitable proposal densities have been developed to obtain rapidly mixing, well-behaved sampling
    schemes without the need for manual tuning.
\item {\bf\sffamily MCMC simulation techniques (mcmcreg objects):} Mcmcreg objects provide similar functionality for fully Bayesian inference as bayesreg objects but implement distributional regression models for responses beyond simple exponential families (distributional regression), quantile regression and multilevel models. Generally, estimation is more efficient (in terms of computing time) than with bayesreg objects. Therefore mcmcreg objects should be preferred to bayesreg objects if possible.
\item{\bf\sffamily Mixed model based estimation (remlreg objects):} An increasingly popular way to estimate semiparametric
    regression models is the representation of penalisation approaches as mixed models. Within {\em BayesX }this concept
    has been extended to structured additive regression  models and several types of non-standard regression situations.
    The general idea is to take advantage of the close  connection between penalty concepts and corresponding random
    effects distributions. The smoothing parameters of the  penalties then transform to variance components in the random
    effects (mixed) model. While the selection of smoothing  parameters has been a difficult task for a long time, several
    estimation procedures for variance components in mixed models are already available since the 1970's. The most popular
    one is restricted maximum likelihood in Gaussian mixed models with marginal likelihood as the non-Gaussian counterpart.
    Remlreg objects employ mixed model methodology for the estimation of structured additive regression models. While
    regression coefficients are estimated based on penalised likelihood, restricted maximum likelihood or marginal
    likelihood estimation forms the basis for the determination of smoothing parameters. From a Bayesian perspective, this
    yields empirical Bayes / posterior mode estimates for the structured additive regression models. However, estimates can
    also merely be interpreted as penalised likelihood estimates from a frequentist perspective.
\item{\bf \sffamily Penalized least squares including model selection (stepwisereg objects):}
As a fourth alternative {\em BayesX} provides a penalized least squares (respectively penalized likelihood) approach for
estimating structured additive regression tools.
In addition to
the previously described estimation alternatives, a powerful variable and model selection tool is included.  Model choice and estimation of
the parameters is done simultaneously. The algorithms are able to
\begin{itemize}
\item decide whether a particular covariate enters the model,
\item decide whether a continuous covariate enters the model linearly or nonlinearly,
\item decide whether a spatial effect enters the model,
\item decide whether a unit- or cluster specific heterogeneity effect enters the model,
\item select complex interaction effects (two dimensional surfaces, varying coefficient terms),
\item select the degree of smoothness of  nonlinear covariate, spatial or cluster specific heterogeneity effects.
\end{itemize}
Inference is based on penalized likelihood in combination with fast
algorithms for selecting relevant covariates and model terms.
Different models are compared via various goodness of fit criteria,
e.g. AIC, BIC, GCV and 5 or 10 fold cross validation.
\end{itemize}
\end{stanza}


\begin{stanza}{Model classes and model terms}

{\em BayesX} provides functionality for the following types of
responses:

\begin{itemize}
\item{\bf\sffamily Univariate exponential family:} Supported response distributions are Gaussian, Poisson, Binomial and Gamma distribution as well as some simple versions of the
    negative binomial, zero-inflated Poisson, and zero-inflated negative binomial.

\item {\bf\sffamily Distributional regression:} A large number of univariate and multivariate continuous, discrete or mixed discrete-continuous responses can be treated within the framework of distributional regression. In this setting, potentially all parameters of these distributions can be related to structured additive predictors.

\item {\bf\sffamily Quantile Regression:} Bayesian quantile regression allows to study specific quantiles of the response distribution without relying on a specific distributional assumption.

\item{\bf\sffamily Categorical responses with unordered responses:} For categorical responses with unordered categories,
    {\em BayesX} supports multinomial logit and multinomial probit models. Both effects of category-specific and
    globally-defined covariates can be estimated. Category-specific offsets or non-availability indicators can be defined
    to account for varying choice sets.

\item{\bf\sffamily Categorical responses with ordered responses:} For ordered categorical responses, ordinal as well as
    sequential models can be specified. Effects can be requested to be category-specific or to be constant over the
    categories. Supported response functions include the logit and the probit transformation.

\item{\bf\sffamily Continuous time survival models:} {\em BayesX} supports Cox-type hazard regression models with
    structured additive predictor for continuous time survival analysis. In contrast to the Cox model, the baseline hazard
    rate is estimated jointly with the remaining effects based on penalized splines. Furthermore, both time-varying effects
    and time-varying covariates can be included in the predictor. Arbitrary combinations of right, left and interval
    censored as well as left truncated observations can be analysed.

\item{\bf\sffamily Continuous time multi-state models:} Multi-state models form a general class for the analysis of the
    evolution of discrete phenomena in continuous time. Transition intensities between the discrete states are specified in
    analogy to the hazard rate in continuous time survival models.
\end{itemize}

Structured additive regression models can be build from arbitrary
combinations of the following model terms:

\begin{itemize}
\item{\bf\sffamily Nonlinear effects:} Nonlinear effects can be estimated based on either penalised spline or random walk
    models.

\item{\bf\sffamily Seasonal effects:} Specific autoregressive priors allow for the estimation of flexible, time-varying
    seasonal effects.

\item{\bf\sffamily Spatial effects:} Spatial effects can be specified based on Markov random fields, stationary Gaussian
    random fields (kriging) or bivariate penalised splines. Both georeferenced regional data as well as point-referenced
    data based on coordinates are supported.

\item{\bf\sffamily Interaction surfaces:} Bivariate extensions of penalised splines allow to estimate flexible interactions
    between continuous covariates. Stationary Gaussian random fields can also be considered a radial basis function
    approach and, hence, form a second possibility for the specification of interaction surfaces.

\item{\bf\sffamily Varying coefficients:} Varying coefficient models with both continuous and spatial effect modifiers can
    be estimated. The latter case is also known as geographically weighted regression.

\item{\bf\sffamily Cluster-specific random effects:} {\em BayesX} supports i.i.d. Gaussian random intercepts and random
    slopes.

\item {\bf\sffamily Regularised high-dimensional effects:} High-dimensional vectors of regression coefficients can be
    assigned Bayesian regularisation priors. Available alternatives are ridge, lasso, and normal mixture of inverse gamma
    (spike and slab) priors.

\item {\bf\sffamily Multilevel models:} In multilevel models, parameters of specific effects can themselves be assigned a structured additive predictor (e.g. in multilevel random effects specifications).
\end{itemize}

Note that parts of the functionality may be available for one of the regression objects only. For example, bayesreg objects do
not support interval censored survival times while multinomial probit models can not be estimated with remlreg objects. Details
can be found in the chapters corresponding to the specific object types.
\end{stanza}

\begin{stanza}{Further functionality}

\begin{itemize}
\item{\bf\sffamily Handling and manipulation of data sets:} {\em
BayesX} provides a number of functions for handling and
manipulating data sets, e.g. for reading ASCII data sets, creating
new variables, obtaining summary statistics etc. Compare
\autoref{datasetobj} for details.

\item{\bf\sffamily Handling and manipulation of geographical
maps:} {\em BayesX} is able to manipulate and draw geographical
maps. The regions of the map may be colored according to some
numerical characteristics. In {\em BayesX} version 1.5, a new
color scheme based on HCL colors has been added to obtain a better
representation of colored maps independent of the display device.
Details can be found in \autoref{map} and \autoref{graphobj}.

\item{\bf\sffamily Visualizing data:} {\em BayesX} provides
functions for drawing scatter plots and geographical maps. A
number of additional options are provided to customize the graphs
according to the personal needs of the user. Details can be found
in \autoref{graphobj}.

\item{\bf\sffamily Model selection for Gaussian and non-Gaussian
dags:} This tool estimates Gaussian and non-Gaussian directed
acyclical graphs (dag) via reversible jump MCMC. Details can be
found in \autoref{dag}.
\end{itemize}
\end{stanza}

%\begin{stanza}{Recommendations for further reading}
%If you are interested in using {\em BayesX}, it is of course not
%necessary to read the complete manual. \autoref{recomm} provides a
%rough guideline for reading this manual and other sources,
%depending on the features of {\em BayesX} you are interested in
%and your statistical background. In any case, you should read
%sections \ref{availableversions} -- \ref{generalusage} to make
%yourself familiar with the general usage of {\em BayesX}.

%\begin{table}[ht] \footnotesize
%\hspace{1cm}\begin{tabular}{ |p{7cm}|p{7.7cm}|}
% \hline
% {\bf Intended use and background} & {\bf Guideline} \\
% \hline\hline
% Bayesian semiparametric regression based on MCMC
% simulation techniques. No experience with MCMC.
% & Read an introductory text about MCMC first. A brief introduction is given
% for example in \citeasnoun{Gre01}. Read the methodology manual to make yourself
% familiar with STAR models.
% Proceed with chapter \ref*{zambiaanalysis} of the tutorial manual. \\
% \hline Bayesian semiparametric regression based on MCMC simulation
% techniques. At least basic knowledge about MCMC.
% & Read the methodology manual to make yourself familiar with STAR
% models. Proceed with chapter \ref*{zambiaanalysis} of the tutorial manual. \\
% \hline
% Semiparametric regression based on mixed model methodology. &
% Read the methodology manual to make yourself familiar with STAR
% models. Proceed with chapter \ref*{remlregzambiaanalysis} of the tutorial manual. \\
%\hline
%Semiparametric regression based on penalized least squares including
%variable and model selection. &
% Read the methodology manual to make yourself familiar with STAR
% models. Proceed with chapter \ref*{zambia_step_analysis} of the tutorial manual. \\
% \hline Model selection for DAGs. No experience with reversible jump
% MCMC. & Read an introductory text about reversible jump MCMC first.
% A brief introduction is given for example in \citeasnoun{Gre01}. Proceed with \autoref{dag}. \\
% \hline
% Draw and color geographical maps & Read \autoref{map} and \autoref{graphdrawmap} of the reference manual. \\
% \hline
%\end{tabular}
%\begin{center}
%{\em \caption {\label{recomm} Recommendations for further
%reading.}}
%\end{center}
%\end{table}
%\end{stanza}
