\chapter{What is BayesX?}

{\em BayesX} is a software tool for performing complex Bayesian
inference. The main features of {\em BayesX} are:
\begin{itemize}
\item {\bf Bayesian semiparametric regression based on MCMC simulation techniques} \\
{\em BayesX} provides a powerful regression tool for analyzing
regression and survival models with {\em structured additive
predictor} (STAR). STAR models cover a number of well known model
classes as special cases, e.g. {\em generalized additive models},
{\em generalized additive mixed models}, {\em geoadditive models},
{\em dynamic models}, {\em varying coefficient models}, and {\em
geographically weighted regression}. {\em BayesX} is able to
estimate nonlinear effects of continuous covariates, trends and
flexible seasonal patterns of time scales, correlated and/or
uncorrelated spatial effects (of geographical data) and
unstructured i.i.d. Gaussian effects of unordered group
indicators. The regression tool  supports the most common
distributions for the response variable. Supported distributions
for univariate responses are Gaussian, binomial, Poisson, negative
binomial and gamma. For multicategorial responses, both
multinomial logit or probit models for ordered categories of the
responses as well as cumulative threshold models for unordered
categories may be estimated. Recently complex models for
continuous time survival analysis based on the Cox model have been
added. At least some basic knowledge about Bayesian inference with
MCMC techniques is strongly recommended if you are interested in
using this tool. Details can be found in \autoref{star} and
\autoref{bayesreg}.
\item {\bf Inference for STAR models based on methodology for mixed models} \\
{\em BayesX} provides a second regression tool for estimating STAR
models with comparable functionality as the first tool based on
MCMC. This tool represents STAR models as {\em variance components
mixed models}. Inference is then based on estimation procedures
for mixed models, particularly {\em restricted maximum likelihood}
(REML). From a Bayesian perspective this yields empirical Bayes or
posterior mode estimates. Details can be found in \autoref{star}
and \autoref{remlreg}.
\item {\bf Model selection for Gaussian and non-Gaussian dag's} \\
This tool estimates Gaussian and non-Gaussian directed acyclical
graphs (dag) via reversible jump MCMC. Details are given in
\autoref{dag}
\item {\bf Handling and manipulation of data sets} \\
{\em BayesX} provides a growing number of functions for handling
and manipulating data sets, e.g. for reading ASCII data sets,
creating new variables, obtaining summary statistics etc. Details
are given in \autoref{datasetobj}.
\item {\bf Handling and manipulation of geographical maps} \\
{\em BayesX} is able to manipulate and draw geographical maps. The
regions of the map may be colored according to some numerical
characteristics. For details compare \autoref{map} and
\autoref{graphobj}.
\item {\bf Visualizing data} \\
{\em BayesX} provides functions for drawing scatter plots and
geographical maps. A number of additional options are provided to
customize the graphs according to the personal needs of the user.
Details can be found in \autoref{graphobj}.
\end{itemize}

\vspace{0.5cm} {\bf Recommendations for further reading}

\vspace{0.2cm} If you are interested in using {\em BayesX} it is
not necessary to read the complete manual. \autoref{recomm}
provides a guideline for reading this manual and other sources
depending on your purpose and background. In any case, you should
read \autoref{availableversions}-\autoref{generalusage} of
\autoref{gettingstarted} to make yourself familiar with {\em
BayesX}.


\begin{table}[ht] \footnotesize
\begin{center}
\begin{tabular}{ |p{7cm}|p{8.5cm}|}
\hline
{\bf Intended use and background} & {\bf Guideline} \\
\hline\hline Bayesian semiparametric regression based on MCMC
simulation techniques. No experience with MCMC techniques. & Read
first an introductory text about MCMC. A nice introduction is
given e.g. in Green (2001). Read \autoref{star} to make yourself
familiar with STAR regression models.
Proceed then with the tutorial like \autoref{zambiaanalysis} of \autoref{bayesreg}. \\
\hline Bayesian semiparametric regression based on MCMC simulation
techniques. At least a basic knowledge about MCMC techniques
exists. & Read \autoref{star} to make yourself familiar
with STAR regression models. Proceed then with the tutorial like \autoref{zambiaanalysis} of \autoref{bayesreg}. \\
\hline Semiparametric regression based on mixed model methodology.
& Read \autoref{star} to make yourself familiar with STAR
regression models. Proceed then with the tutorial like
\autoref{remlregzambianalysis} of \autoref{remlreg}. \\
\hline Model selection for dag's. No experience with reversible
jump MCMC. & Read first an introductory text about reversible jump
MCMC. A nice introduction is
given e.g. in Green (2001). Proceed with \autoref{dag}. \\
\hline
Draw and color geographical maps & Read \autoref{map} and \autoref{graphdrawmap} of \autoref{graphobj}. \\
\hline
\end{tabular}
{\em \caption {\label{recomm} Recommendations for further
reading}}
\end{center}
\end{table}

\subsubsection*{References}

\begin{description}
\item[Green, P.J. (2001):] A Primer in Markov Chain Monte Carlo. In: Barndorff-Nielsen, O.E.,
Cox, D.R. and Kl{\"u}ppelberg, C. (eds.), {\em Complex Stochastic
Systems}. Chapman and Hall, London, 1-62.
\end{description}
