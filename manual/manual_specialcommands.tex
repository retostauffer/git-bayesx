\chapter{Special Commands}

This chapter describes some commands that are not associated with
a particular object type. Among others, there are commands for
exiting {\em BayesX}, opening and closing log files, saving
program output, deleting objects etc.

\section{Exiting BayesX}
\index{exiting BayesX}

You can exit {\em BayesX} by typing either

#> exit#

or

#> quit#

in the {\em command window}. Of course, {\em BayesX} can also be
closed using the 'exit' entry from the file menu or by clicking on
the cross in the upper right corner.

\section{Opening and closing log files} \label{logfile}
\index{log files}

Program output and commands entered by the user can automatically
be stored in a log file to make them available for editing in your
favorite text editor. Another important application of log files
is the documentation of your work. A log file is opened by the
command:

#> logopen# [{\em, option}] #using# {\em filename}

Afterwards all commands entered and all program output will be
saved in the file {\em filename}. If the log file specified in
{\em filename} is already existing, new output is appended at the
end of the file. To overwrite an existing log file, option
#replace# has to be specified in addition. Note that it is not
allowed to open more than one log file simultaneously.

An open log file can be closed by typing:

#> logclose#

Exiting {\em BayesX} automatically closes the current log file.

\section{Saving the contents of the output window}
\index{output window!saving the contents} \index{saveoutput}

You can save the contents of the {\em output window} not only with
the {\em file$\rightarrow$save output} or {\em
file$\rightarrow$save output as} menu, but also using the
#saveoutput# command. This is particularly useful for automatic
saving in batch files, see \autoref{batch}. The syntax for saving
the {\em output window} is

#> saveoutput# [{\em , options}] #using# {\em filename}

where {\em filename} is the file (including path) in which the
contents of the output will be stored.

\subsection*{Options}

\begin{itemize}
\item # replace# \\
By default an error will be raised if you try to store the
contents of the {\em output window} in a file that is already
existing. This preserves you from overwriting a file
unintentionally. An already existing file can be overwritten by
explicitly specifying the #replace# option.
\item #type = rtf #$|$# txt # \\
The {\em output window} can be saved in two different file types,
namely rich text format (the default) and plain ASCII (requested
by specifying #type=txt# as an option.
\end{itemize}

\section{Changing the delimiter}
\label{delimiter} \index{delimiter}

By default, commands entered using the {\em command window} will
be executed after pressing the return key. This can be
inconvenient, in particular if your statements are long or in
batch files. In this case it may be favorable to split a statement
into several lines, and execute the command using a different
delimiter.

You can change the delimiter using the #delimiter# command. The
syntax is

#> delimiter# = {\em newdel}

where {\em newdel} is the new delimiter. Only two different
delimiters are currently allowed, namely the return key and the
'#;#' (semicolon) key. To specify the semicolon as the delimiter,
type

#> delimiter = ;#

and press return. To return to the return key as the delimiter, type

#> delimiter = return;#

Note that this statement has to end with a semicolon, since this
was previously set to be the current delimiter.


\section{Using batch files}
\label{batch} \index{batch files}

You can execute commands stored in a file just as if they were
entered from the keyboard. This may be useful if you want to
re-run a certain analysis more than once (possibly with some minor
changes) or if you want to run time consuming statistical methods.

Execution of a batch file is started by typing

#> usefile# {\em filename}

This executes the commands stored in {\em filename} successively.
{\em BayesX} will not stop the execution if an error occurs in one
or more commands. Note that it is allowed to invoke another batch
file within a currently running batch file.


\subsubsection*{Comments}\index{comments}

Comments in batch files are indicated by a  #%#
sign, i.e. every line
starting with #%# is ignored by the program.

\subsubsection*{Changing the delimiter}

In particular in batch files, the readability of your program code
may be improved if some (long) commands are split up into several
lines. By default this will cause errors, because {\em BayesX}
interprets each line in your program as one statement. To overcome
this problem one has to change the delimiter using the #delimiter#
command, see \autoref{delimiter}.

\section{Dropping objects}
\index{objects!dropping} \index{dropping objects}

You can delete objects by typing

#> drop# {\em objectlist}

This drops the objects specified in {\em objectlist}. The names of
the objects in {\em objectlist} must be separated by blanks.
