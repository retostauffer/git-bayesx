\chapter{Special Commands}

This chapter describes some commands that are not connected with a
particular object type. Among others, there are commands for
exiting {\em BayesX}, opening and closing log files, saving
program output, dropping objects etc..

\section{Exiting BayesX}
\index{exiting BayesX}

You can exit {\em BayesX} by simply typing either

#> exit#

or

#> quit#

in the {\em command window}.

\section{Opening and closing log files} \label{logfile}
\index{log files}

In a log file, program output and commands entered by the user,
are stored in plain ASCII format. This makes it easy to further
use the program output, for example results of statistical
procedures, in your favorite text editor. Another important
application of log files is
the documentation of your work. You open a log file by typing:

#> logopen# [{\em, option}] #using# {\em filename}

This opens a log file that will be saved in {\em filename}. After
opening a log file, all commands entered and all program output
appearing on the screen will be saved in this file. If the
log file specified in  {\em filename} is already existing, new
output is appended at the end of the file. To overwrite an
existing log file option #replace# must be specified in addition.
Note that it is not allowed to open more than one log file
simultaneously.

An open log file can be closed by simply typing:

#> logclose#

Note that exiting {\em BayesX} automatically closes the currently
open logfile.

\section{Saving the contents of the output window}
\index{output window!saving the contents} \index{saveoutput}

You can save the contents of the {\em output window} not only with
the {\em file-$>$save output} or {\em file-$>$save output as}
menu, but also using the #saveoutput# command. Saving the {\em
output window} with the #saveoutput# command is  particularly
useful in batch files, see \autoref{batch}. The syntax for saving
the
{\em output window} is

#> saveoutput# [{\em , options}] #using# {\em filename}

where {\em filename} is the file (including path) in which the
contents of the output will be saved.


\subsection*{Options}

\begin{itemize}
\item {\tt replace} \\
By default, an error will be raised if one tries to store the
contents of the {\em output window} in a file that is already
existing. This preserves you from overwriting a file
unintentionally. An already existing file can be overwritten by
explicitly specifying the #replace# option.
\item {\tt type = rtf $|$ txt } \\
The {\em output window} can be saved under two different file
types. By default, the contents of the window will be saved in
rich-text format. The second possibility is to store the {\em
output window} in plain ASCII--format. This can be done by
specifying #type = txt#. To explicitly store the file in rich-text
format {\em #type = rtf#} must be specified. \\
DEFAULT: #type = rtf#
\end{itemize}

\section{Changing the delimiter}
\label{delimiter} \index{delimiter}

By default, commands entered using the {\em command window} will
be executed by pressing the return key. This can be inconvenient,
in particular if your statements are long. In that case it may be
more favorable to split a statement into several lines, and
execute the command using a different delimiter than the return
key. You can change the delimiter using the #delimiter# command. The syntax is

#> delimiter# = {\em newdel}

where {\em newdel} is the new delimiter. There are only two
different delimiters allowed, namely the
return key and the ';' (semicolon) key. To specify the ';' key as the delimiter, type

#> delimiter = ;#

and press return. To return to the return key as the delimiter, type

#> delimiter = return;#

Note that the above statement must end with a semicolon, since
this was previously set to the current delimiter.


\section{Using batch files}
\label{batch} \index{batch files}

You can execute commands stored in a file just as if they were
entered from the keyboard. This may be useful if you want to
re-run a certain analysis more than once (possibly with some minor
changes) or if you want to run time consuming statistical methods
such as Bayesian regression based on MCMC simulation techniques
(see \autoref{bayesreg}).
You can run such batch files by simply typing

#> usefile# {\em filename}

This executes the commands stored in {\em filename} successively.
{\em BayesX} will not stop the execution if an error occurs in one
or more commands. Note that it is allowed to invoke
another batch file within a batch file currently running.


\subsubsection*{Comments}\index{comments}

Comments in batch files are allowed and are indicated by a  #%#
sign, that is every line
starting with a #%# sign is ignored by the program.

\subsubsection*{Changing the delimiter}

In particular in batch files, the readability of your program code
may be improved if some (long) commands are split up into several
lines. Normally this will cause errors, because {\em BayesX}
interprets each line in your program as one statement. To overcome
this problem one simply
has to change the delimiter using the #delimiter# command, see \autoref{delimiter}.


\section{Dropping objects}
\index{objects!dropping} \index{dropping objects}

You can delete objects by typing

#> drop# {\em objectlist}

This drops the objects specified in {\em objectlist}. The names of
the objects in {\em objectlist} must be separated by blanks.
