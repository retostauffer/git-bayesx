\chapter{dataset objects} \label{chap_data}
\label{datasetobj} \index{dataset objects} \index{dataset}


{\em Dataset objects} are used to manage and manipulate data. A new {\em dataset object} is created by typing

#> dataset# {\em objectname}

where {\em objectname} is the name of the data set. After the
creation of a {\em dataset object} you can apply the methods for
manipulating and managing data sets discussed below.

Note that in the current version of {\em BayesX} {\bf only
numerical variables are allowed}. Hence, string valued variables,
for example,  are not yet supported by {\em BayesX}.



\clearpage



\section{Method descriptive}
\label{descriptive} \index{summary statistics}
\index{descriptives} \index{dataset!descriptive command}

\begin{stanza}{Description}

{Method #descriptive# calculates and displays univariate summary
statistics. The method computes the number of observations, the
mean, median, standard deviation, minimum and maximum of
variables.}
\end{stanza}

\begin{stanza}{Syntax}

{#> #{\em objectname}.#descriptive# {\em varlist} [#if# {\em expression}]

Method #descriptive# computes summary statistics for the variables
in {\em varlist}. An optional #if# statement may be added to
analyze only a part of the data.}
\end{stanza}

\begin{stanza}{Options}

{not allowed}
\end{stanza}


\begin{stanza}{Example}

{The statement

#> d.descriptive x y#

computes summary statistics for the variables #x# and #y#.
The statement

#> d.descriptive x y if x>0#

restricts the analysis to observations with #x>0#.}
\end{stanza}

\clearpage

\section{Method drop}
\label{drop} \index{dropping variables} \index{dropping
observations} \index{dataset!drop command}

%\bigskip
%{\bf \em  Description} \\

\begin{stanza}{Description}

{Method #drop# deletes variables or observations from the data set.}
\end{stanza}

%\bigskip
\begin{stanza}{Syntax}

{#> #{\em objectname}.#drop# {\em varlist}

#> #{\em objectname}.#drop if# {\em expression}

The first command may be used to eliminate the variables specified
in {\em varlist} from the data set. The second statement may be
used to eliminate certain observations. An observation will be
removed from the data set if {\em expression} is true.}
\end{stanza}


\begin{stanza}{Options}

{not allowed}
\end{stanza}


\begin{stanza}{Example}

The statement

#> credit.drop account duration#

drops the variables #account# and #duration# from the credit scoring data set. With the statement

#> credit.drop if marstat = 2#

all observations with #marstat = 2#, i.e. all persons living
alone, will be dropped from the credit scoring data set. The
following statement

#> credit.drop account duration if marstat = 2#

will raise the error

#ERROR: dropping variables and observations in one step not allowed#

It is not allowed to drop variables and certain observations in
one single command.
\end{stanza}



\clearpage



\section{Functions and Expressions}
\label{expression} \index{expressions}

The primary use of expressions is to generate new variables or
change existing variables, see \autoref{generate} and
\autoref{replace}, respectively. Expressions may also be used in
#if# statements to force {\em BayesX} to apply a method only to
observations where the boolean expression in the #if# statement
is true. The following are all examples of expressions:

#2+2# \\
#log(amount)# \\
#1*(age <= 30)+2*(age > 30 & age <= 40)+3*(age > 40)# \\
#age=30# \\
#age+3.4*age^2+2*age^3# \\
#amount/1000#


\subsection{Operators}
\index{expressions!operators} \index{operators}


{\em BayesX} knows three different types of operators: arithmetic,
relational and logical. Each of the types is discussed below.

\subsubsection{Arithmetic operators}
\index{operators!arithmetic}

The arithmetic operators are #+# (addition), #-# (subtraction),
#*# (multiplication), #/# (division), #^# (raise to a power) and
the prefix #-# (negation). Any arithmetic operation on a missing
value or an impossible arithmetic operation (such as division by
zero) yields a missing value.\\

\begin{stanza}{Example}

The expression

#(x+y^(3-x))/(x*y)#

denotes the formula

$$
\frac{x+y^{3-x}}{x\cdot y}
$$

and evaluates to missing if #x# or #y# is missing or zero.
\end{stanza}

\subsubsection{Relational operators}
\index{operators!relational}

The relational operators are #># (greater than), #<# (less than),
#>=# (greater than or equal), #<=# (less than or equal), #=#
(equal) and #!=# (not equal). Relational expressions are either 1
(i.e. the expression is true) or 0 (i.e. the expression is
false).\\

\begin{stanza}{Example}

Relational operators may be used to create indicator variables.
The following statement generates a new variable #amountcat# (out
of the already existing variable #amount#), whose
value is 1 if #amount<=10# and 2 if #amount>10#.

#> credit.generate amountcat = 1*(amount<=10)+2*(amount>10)#

Another useful application of relational operators is in #if#
statements. For example, changing
an existing variable only when a certain condition holds can be done by the following command:

#> credit.replace amount = NA if amount <= 0#

This sets all observations missing where #amount<=0#.
\end{stanza}

\subsubsection{Logical operators}
\index{operators!logical}

The logical operators are #&# (and) and #|# (or).

\subsubsection*{Example}

Suppose you want to generate a variable #amountind# whose value is
1 for married people with
amount greater than 10 and 0 otherwise. This can be done by typing

#> credit.generate amountind = 1*(marstat=1 & amount > 10)#

\subsubsection{Order of evaluation of the operators}
\index{operators!order of evaluation}

The order of evaluation (from first to last) of operators is

#^# \\
#/#, #*#\\
#-#, #+#\\
#!=#, #>#, #<#, #<=#, #>=#, #=#\\
#&#, #|#.

Brackets may be used to change the order of evaluation.


\subsection{Functions}
\index{functions}

Functions may appear in expressions. Functions are indicated by
the function name, an opening and a closing parenthesis. Inside
the parentheses one or more arguments may be specified. The
argument(s) of a function may be any expression, including other
functions. Multiple arguments are separated by commas. All
functions return missing values when given missing values as
arguments or when the result is undefined.

\autoref{mathfunc} references all mathematical functions;
\autoref{statfunc} references all statistical functions.
\index{functions!abs} \index{functions!cos} \index{functions!sin}
\index{functions!exp} \index{functions!floor}
\index{functions!lag} \index{functions!logarithm}
\index{functions!square root} \index{functions!bernoulli
distributed random numbers} \index{functions!binomial distributed
random numbers} \index{functions!cumulative distribution function}
\index{Functions!gamma distributed random numbers}
\index{functions!exponential distributed random numbers}
\index{functions!normally distributed random numbers}
\index{functions!uniformly distributed random numbers}
\index{functions!poisson distributed random numbers}
\index{functions!weibull distributed random numbers}


\begin{table}[ht]
\begin{center}
\begin{tabular}{|l|l|}
\hline
{\bf Function} & {\bf Description} \\
\hline \hline
abs(x) & absolute value \\
cos(x) & cosine of radians \\
exp(x) & exponential \\
floor(x) & returns the integer obtained by truncating $x$. \\
& Thus floor(5.2) evaluates to 5 as floor(5.8). \\
lag(x) & lag operator \\
log(x) & natural logarithm \\
log10(x) & log base 10 of $x$ \\
sin(x) & sine of radians \\
sqrt(x) & square root \\
\hline
\end{tabular}
{\em\caption{\label{mathfunc} List of mathematical functions.}}
\end{center}
\end{table}



\begin{table}[ht]
\begin{center}
\begin{tabular}{|l|p{11cm}|}
\hline
{\bf Function} & {\bf Description} \\
 \hline \hline
 bernoulli($p$) & returns Bernoulli distributed
 random numbers with probability of success $p$. If $p$ is not
 within the interval $[0;1]$, a
 missing value will be returned. \\
 \hline
 binomial($n,p$) & returns $B(n;p)$ distributed random
 numbers. Both, the number of trials $n$ and the probability of
 success $p$ may be expressions. If $n < 1$, a missing value will
 be returned. If $n$ is not integer valued, the number of trials
 will be $[n]$. If $p$ is not within the interval $[0;1]$, a
 missing value will be returned. \\
 \hline
 cumul($x$) & cumulative distribution function \\
 \hline
 cumulnorm($x$) & cumulative distribution function $\Phi$ of the standard normal distribution. \\
 \hline
 exponential($\lambda$) & returns exponentially distributed
 random numbers with parameter $\lambda$.
 If $\lambda \leq 0$, a missing value will be returned. \\
 \hline
 gamma($\mu$,$\nu$) & returns gamma distributed random
 numbers with mean $\mu$ and variance $\mu^2/ \nu$.
 If $\mu$ and/or $\nu$ are less than zero, a missing value will be returned.  \\
 \hline
 normal() & returns standard normally distributed random
 numbers;
 $N(\mu,\sigma^2)$ distributed random numbers may be generated with $\mu + \sigma$*normal(). \\
 \hline
 poisson($\lambda$) & returns poisson distributed random
 numbers with parameter $\lambda$.  If $\lambda \leq 0$, a missing value will be returned.\\
 \hline
 uniform() & uniform
 pseudo random number function; returns uniformly distributed
 pseudo-random numbers on the interval $(0,1)$ \\
 \hline
 weibull($\alpha$,$\lambda$) & returns weibull distributed
 random numbers with density
 $f(x)=\alpha\lambda^\alpha x^{\alpha-1}\exp(-\lambda x^\alpha)$.
 If $\alpha \leq 0$ and/or $\lambda \leq 0$, a missing value will be returned.\\
\hline
\end{tabular}
{\em \caption{\label{statfunc} List of statistical functions}}
\end{center}
\end{table}


\subsection{Constants}
\index{expressions!constants} \index{number of observations}
\index{missing values} \index{$\pi$} \index{current observation}

\autoref{constant} lists all constants that may be used in expressions.


\begin{table}[ht]
\begin{center}
\begin{tabular}{|l|l|}
\hline
Constant & Description \\
\hline \hline
\texttt{\_n} & contains the number of the current observation.  \\
\texttt{\_N }& contains the total number of observations in the data set. \\
\texttt{\_pi} & contains the value of $\pi$. \\
\texttt{NA} & indicates a missing value \\
.  & indicates a missing value \\

\hline
\end{tabular}
{\em \caption{\label{constant} List of constants}}
\end{center}
\end{table}


\subsubsection*{Example}

The following statement generates a variable #obsnr# whose value
is 1 for the first observation, 2 for the second and so on.

#> credit.generate obsnr = _n#

The command

#> credit.generate nrobs = _N#

generates a new variable {\em nrobs} whose values are all equal to
the total number of observations, say 1000, for all observations.

\subsection{Explicit subscribing}
\index{expressions!explicit subscribing} \index{subscribing}

Individual observations on variables can be referenced by
subscribing the variables. Explicit subscripts are specified by
the variable name with square brackets that contain an expression.
The result of the subscript expression is truncated to an integer,
and the value of the variable for the indicated observation is
returned. If  the value of the subscript expression is less than 1
or greater than the number of observations in the data set,
a missing value is returned.

\subsubsection*{Example}

Explicit subscribing combined with the constant #_n# (see
\autoref{constant}) can be used to create lagged values
on a variable. For example the lagged value of a variable #x# in a data set #data# can be created by

#> data.generate xlag = x[_n-1]#

Note that #xlag# can also be generated using the #lag# function

#> data.generate xlag = lag(x)#


\clearpage

\section{Method generate}
\label{generate} \index{generating new variables}
\index{dataset!generate command}

\begin{stanza}{Description}

Method #generate# is used to create a new variable in an existing
{\em dataset object}.
\end{stanza}


\begin{stanza}{Syntax}

{#> #{\em objectname}.#generate# {\em newvar} = {\em expression}

Method #generate# creates a new variable with name {\em newvar}.
See \autoref{varnames} for valid variable names. The values of the
new variable are specified by {\em expression}. The details of
valid expressions are covered in \autoref{expression}.}
\end{stanza}


\begin{stanza}{Options}

{not allowed}
\end{stanza}


\begin{stanza}{Example}

{The following command generates a new variable called #amount2#
whose values are the square of amount in the credit
scoring data set.

#> credit.generate amount2 = amount^2#

If you try to change the variable currently generated, for example by typing

#> credit.generate amount2 = amount^0.5#

the error message

#ERROR: variable amount2 is already existing #

will occur. This prevents you to change an existing variable
unintentionally. An existing variable may be changed with method
#replace#, see \autoref{replace}.

If you want to generate an indicator variable #largeamount# whose
value is 1 if #amount# exceeds a certain value, say 3.5, and 0
otherwise, the following will
produce the desired result:

#> credit.generate largeamount = 1*(amount>3.5)#}
\end{stanza}

\clearpage

\section{Method infile}
\label{infile} \index{reading data from ASCII files}
\index{dataset!infile command}

\begin{stanza}{Description}

{Reads in data saved in an ASCII file.}
\end{stanza}

\begin{stanza}{Syntax}

{#> #{\em objectname}.#infile #[{\em varlist}] [{\em , options}] #using# {\em filename}

Reads in data stored in {\em filename}. The variables are given
names specified in {\em varlist}. If {\em varlist} is empty, i.e.
there is no {\em varlist} specified, it is assumed that the first
row of the datafile contains the variable names separated by
blanks or tabs. It is not required that the observations in the
datafile are stored in a special format, except that successive
observations should be separated by one or more blanks (or tabs).
The first value read from the file will be the first observation
of the first variable, the second value will be the first
observation of the second variable, and so on. An error will occur
if for some variables no values can be read for the last
observation.

It is assumed that  a period '.' or 'NA' indicates a missing
value.

Note that in the current version of {\em BayesX} {\bf only
numerical variables are allowed}. Thus, the attempt to read in
string valued variables, for example, will cause an error.}
\end{stanza}

\subheader{Options}

\begin{itemize}
\item #missing = # {\em missingsigns} \\
By default a dot '.' or 'NA' indicates a missing value. If you
have a data set where missing values are indicated by different
signs than '.' or 'NA', you can force {\em BayesX} to recognize
these signs as missing values by specifying the #missing# option.
For example #missing = MIS# defines 'MIS' as an indicator for a
missing value. Note that
 '.' and 'NA' remain valid indicators for missing values, even if the missing
option is specified.

\item #maxobs = #{\em integer} \\
If you work with large data sets, you may observe the problem that
reading in a data set using the #infile# command is very time
consuming. The reason for this problem is that {\em BayesX} does
not know the number of observations and thus the memory needed in
advance. The effect is that new memory must be allocated whenever
a certain amount of memory is used. To avoid this problem the
#maxobs# option may be used, leading to a considerable reduction
of computing time. This option forces {\em BayesX} to allocate in
advance enough memory  to store at least {\em integer}
observations before new memory must be reallocated. Suppose for
example that your data set consists of approximately 100,000
observations. Then specifying #maxobs=105000# allocates enough
memory to read in the data set quickly. Note that #maxobs=105000#
does not mean that your data set cannot hold more than 105,000
observations. This only means that new memory will/must be
allocated when the number of observations of your data set exceeds
the 105,000 observations limit.
\end{itemize}



\begin{stanza}{Example}

{Suppose we want to read a data set stored in
#c:\data\testdata.raw# containing two
variables #var1# and #var2#.
The first few rows of the datafile could look like this:

var1 var2 \\
2 2.3 \\
3 4.5 \\
4 6 \\
...


To read in this data set, we first have to create a new {\em dataset
object}, say #testdata#, and then read the data using
the #infile# command. The following two commands will produce the desired result.

#> dataset testdata# \\
#> testdata.infile using c:\data\testdata.raw#

If the first row of the data set file contains no variable names,
the second command must be modified to:

#> testdata.infile var1 var2 using c:\data\testdata.raw#

Suppose furthermore that the data set you want to read in is a
pretty large data set with 100,000
observations. In that case the #maxobs# option is very useful to reduce reading time.  Typing for example

#> testdata.infile var1 var2 , maxobs=101000 using c:\data\testdata.raw#

will produce the desired result.}
\end{stanza}

\clearpage



\section{Method outfile}
\label{outfile} \index{writing data to a file} \index{saving data
in an ASCII file} \index{dataset!outfile command}

\begin{stanza}{Description}

{Method #outfile# writes data to a file in ASCII format. The saved
data can be read back using the #infile# command, see
\autoref{infile}.}
\end{stanza}



\begin{stanza}{Syntax}

{#> #{\em objectname}.#outfile# [{\em varlist}] [#if# {\em expression}] [{\em , options}] #using# {\em filename}

#outfile# writes the variables specified in {\em varlist} to the
file with name {\em filename}. If {\em varlist} is omitted in the
outfile statement, {\em all} variables in the data set are written
to disk. Each row in the data file corresponds to one observation.
Different variables are separated by blanks. Optionally, an #if#
statement may be used to write only those observations to disk
where a certain boolean expression, specified in {\em expression},
is true.}
\end{stanza}


\subheader{Options}


\begin{itemize}
\item #header# \\
Specifying the #header# option forces {\em BayesX} to write the
variable names in the first row of the created data file.
\item #replace# \\
The #replace# option allows {\em BayesX} to overwrite an already
existing data file. If #replace# is omitted in the option list and
the file specified in {\em filename} is already existing, an error
will be raised.
This prevents you from overwriting an existing file unintentionally.
\end{itemize}


\begin{stanza}{Example}

The statement

#> credit.outfile using c:\data\cr.dat#

writes the complete credit scoring data set to
#c:\data\cr.dat#. To generate two different
ASCII
data sets for married people and people living alone, you could type

#> credit.outfile if marstat = 1 using c:\data\crmarried.dat# \\
#> credit.outfile if marstat = 2 using c:\data\cralone.dat#

Suppose you only want to write the two variables #y# and #amount#
to disk. You could type

#> credit.outfile y amount using c:\data\cr.dat#

This will raise the error message

#ERROR: file c:\data\cr.dat is already existing#

%\newpage

because #c:\data\cr.dat# has already been created. You can
overwrite the file using the #replace# option

#> credit.outfile y amount , replace using c:\data\cr.dat#
\end{stanza}



\clearpage



\section{Method pctile}
\label{pcitle} \index{percentiles of variables}
\index{dataset!pctile command}

\begin{stanza}{Description}

{Method #pctile# computes and displays the
1\%,5\%,25\%,50\%,75\%,95\% and 99\%  percentiles of a variable.}
\end{stanza}

\begin{stanza}{Syntax}

{#># {\em objectname}.#pctile# {\em varlist} [#if# {\em expression}]

Method #pctile# computes and displays the percentiles of the
variables specified in {\em varlist}. An optional #if# statement
may be added to compute the percentiles only for a part of the
data.}
\end{stanza}

\begin{stanza}{Options}

{not allowed}
\end{stanza}


\begin{stanza}{Example}

{The statement

#> d.pctile x y#

computes percentiles for the variables #x# and #y#.
The statement

#> d.pctile x y if x>0#

restricts the analysis to observations with x$>$0.}
\end{stanza}



\clearpage



\section{Method rename}
\index{renaming variables} \index{dataset!rename command}


\label{rename}


\begin{stanza}{Description}

{#rename# is used to change variable names.}
\end{stanza}


\begin{stanza}{Syntax}

{#> #{\em objectname}.#rename# {\em varname} {\em newname}

#rename# changes the name of {\em varname} to {\em newname}. {\em
newname} must be a valid variable name, see \autoref{varnames} for
valid variable names.}
\end{stanza}

\begin{stanza}{Options}

not allowed

\end{stanza}

\clearpage



\section{Method replace}
\label{replace} \index{changing existing variables}
\index{dataset!replace command}



\begin{stanza}{Description}

{#replace# changes values of an existing variable.}
\end{stanza}


\begin{stanza}{Syntax}

 #> #{\em objectname}.#replace# {\em varname} = {\em expression} [#if# {\em expression}]

#replace# changes the values of the existing variable {\em
varname}. If {\em varname} is not existing, an error will be
raised. The new values of the variable are specified in {\em
expression}. Expressions are covered in \autoref{expression}. An
optional #if# statement may be used to change the values of the
variable only if the boolean expression {\em expression} is true.
\end{stanza}


\begin{stanza}{Options}

{not allowed}
\end{stanza}


\begin{stanza}{Example}

{The statement

#> credit.replace amount = NA if amount<0#

changes the values of the variable #amount# in the credit scoring
data set to missing if #amount<0#.}
\end{stanza}



\clearpage



\section{Method set obs}
\label{setobs}


\begin{stanza}{Description}

{#set obs# changes the current number of observations in a data
set.}
\end{stanza}


\begin{stanza}{Syntax}

{#> #{\em objectname}.#set obs# = {\em intvalue}

#set obs# raises the number of observations in the data set to
{\em intvalue}, which must be greater or equal to the current
number of observations. This prevents you from deleting parts of
the data currently in memory. Observations may be eliminated using
the #drop# statement, see \autoref{drop}. The values of the
additionally created observations will be set to the missing
value.}
\end{stanza}



\clearpage



\section{Method sort}
\label{sort} \index{sorting variables} \index{dataset!sort
command}


\begin{stanza}{Description}

{Sorts the data set.}
\end{stanza}


\begin{stanza}{Syntax}

{#> #{\em objectname}.#sort# {\em varlist}  [{\em , options}]

Sorts the data set with respect to the variables specified in {\em
varlist}. Missing values are interpreted to be larger than any
other number and are thus placed last.}
\end{stanza}


\subheader{Options}

\begin{itemize}
\item #descending# \\
If this option is specified, the data set will be sorted in
descending order. The default is ascending order.
\end{itemize}



\clearpage



\section{Method tabulate}
\label{tabulate} \index{tabulate} \index{table of frequencies}
\index{one way table of frequencies} \index{dataset!tabulate
command}

\begin{stanza}{Description}

{Method #tabulate# calculates and displays a frequency table for a
variable.}
\end{stanza}

\begin{stanza}{Syntax}

{#># {\em objectname}.#tabulate# {\em varlist} [#if# {\em expression}]

Method #tabulate# computes and displays frequency tables of the
variables specified in {\em varlist}. An optional #if# statement
may be added to restrict the analysis to a part of the data.}
\end{stanza}

\begin{stanza}{Options}

{not allowed}
\end{stanza}

\begin{stanza}{Example}

{The statement

#> d.tabulate x y#

displays  frequency tables for the variables #x# and #y#.
The statement

#> d.tabulate x y if x>0#

restricts the analysis to observations with #x>0#.}
\end{stanza}



\clearpage



\section{Variable names}
\label{varnames} \index{variables names}


A valid variable name is a sequence of letters (#A#-#Z# and #a#-#z#),
digits (#0#-#9#), and underscores (#_#). The first character of a
variable name must  either be a letter or an underscore. {\em
BayesX} respects upper and lower case letters, that is #myvar#,
#Myvar# and #MYVAR# are three distinct variable names.

\section{Example}

This section contains two examples on how to work with {\em
dataset objects}. The first example illustrates some of the
methods described above, using one of the example data sets stored
in the #examples# directory, the credit scoring data set. A
description of this data set can be found in \autoref{creditdata}.
The second example shows how to simulate complex statistical
models.

\subsection{The credit scoring data set}

In this section we illustrate how to code categorical variables
according to one of the coding schemes, dummy or effect coding.
This will be useful in regression models, where all categorical
covariates must be coded in dummy or effect coding before they can
be added to the model.

We first create a {\em dataset object} #credit# and read in the data using the #infile# command.

#> dataset credit# \\
#> credit.infile using c:\bayesx\examples\credit.raw#

We can now generate new variables to obtain dummy coded versions
of the categorical covariates
#account#, #payment#, #intuse# and #marstat#:

#> credit.generate account1  = 1*(account=1)# \\
#> credit.generate account2  = 1*(account=2)# \\
#> credit.generate payment1 = 1*(payment=1)# \\
#> credit.generate intuse1 = 1*(intuse=1)# \\
#> credit.generate marstat1 = 1*(marstat=1)#

The reference categories are chosen to be 3 for #account# and 2
for the other variables. Alternatively, we could code the
variables according to effect coding. This is achieved with the
following program code:

#> credit.generate account_eff1  = 1*(account=1)-1*(account=3)# \\
#> credit.generate account_eff2  = 1*(account=2)-1*(account=3)# \\
#> credit.generate payment_eff1 = 1*(payment=1)-1*(payment=2)# \\
#> credit.generate intuse_eff1 = 1*(intuse=1)-1*(intuse=2)# \\
#> credit.generate marstat_eff1 = 1*(marstat=1)-1*(marstat=2)#


\subsection{Simulating complex statistical models}
\index{dataset!simulation of} \index{simulation of artificial data
sets}

In this section we illustrate how to simulate complex regression
models. Suppose first we want to simulate data according to the
following Gaussian regression model:

\begin{eqnarray}
y_i & = & 2 + 0.5 x_{i1} + \sin(x_{i2}) + \epsilon_i, \quad i = 1,\dots,1000 \\
x_{i1} & \sim & U(-3,3) \quad i.i.d.  \\
x_{i2} & \sim & U(-3,3) \quad i.i.d.  \\
\epsilon_i & \sim & N(0,0.5^2) \quad i.i.d.
\end{eqnarray}

We first have to create a new data set #gsim#, say, and specify the desired number of observations:

#> dataset gsim# \\
#> gsim.set obs = 1000#

In a second step the covariates #x1# and #x2# have to be created. In
this first example we
assume that the covariates are uniformly distributed between -3 and 3. To generate them, we must type:

#> gsim.generate x1 = -3+6*uniform()# \\
#> gsim.generate x2 = -3+6*uniform()#

In a last step we can now create the response variable by typing

#> gsim.generate y = 2 + 0.5*x1+sin(x2)+0.5*normal()#

You could now (if you wish) estimate a Gaussian regression model
with the generated data set using one of the regression tools of {\em BayesX}, see
\autoref{bayesreg} or \autoref{remlreg}. Of course, more refined models could be
simulated. We may for example drop the assumption of a constant
variance of $0.5^2$ in the error term. Suppose the variance is
heteroscedastic and growing with order log($i$)
where $i$ is the observation index. We can simulate such a heteroscedastic model by typing:

#> gsim.replace y = 2 + 0.5*x1+sin(x2)+0.1*log(_n+1)*normal()#

In this model the standard deviation is

$$
\sigma_i = 0.1*\log(i+1), \quad  i = 1,\dots,1000.
$$

Suppose now that we want to simulate data from a logistic
regression model. In a logistic regression model it is assumed
that (given the  covariates) the response variable $y_i$,
$i=1,\dots,n$, is binomially distributed with parameters $n_i$ and
$\pi_i$ where $n_i$ is the number of replications and $\pi_i$ is
the probability of success. For $\pi_i$ one assumes that it is
related to a linear predictor $\eta_i$ via the logistic
distribution function, that is

$$
\pi_i = \frac{\exp(\eta_i)}{1+\exp{(\eta_i)}}.
$$

To simulate such a model we have to specify the linear predictor
$\eta_i$ and the number of replications $n_i$. We specify a
similar linear predictor as in the example above for Gaussian
response, namely

$$
\eta_i = -1+0.5x_{i1}-\sin(x_{i2}).
$$


For simplicity, we set $n_i=1$ for the number of replications. The
following commands generate a data set 'bin'
according to the specified model:


#> dataset bin #\\
#> bin.set obs = 1000# \\
#> bin.generate x1 = -3+6*uniform()# \\
#> bin.generate x2 = -3+6*uniform()# \\
#> bin.generate eta = -1+0.5*x1-sin(x2)# \\
#> bin.generate pi = exp(eta)/(1+exp(eta))# \\
#> bin.generate y = binomial(1,pi)#

Note that the last three statements can be combined into a single command:

#> bin.generate y = binomial(1,exp(-1+0.5*x1-sin(x2))/(1+exp(-1+0.5*x1-sin(x2)))#


The first version however is much easier to read and should
therefore be preferred.
