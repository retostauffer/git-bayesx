\chapter{Getting started}
\label{gettingstarted}

This chapter provides information on the different versions of
{\em BayesX}, how to install {\em BayesX} on your computer and
explains the purpose of the different windows that appear after
having started {\em BayesX}. A fourth section covers the general
usage of the program and the features of the current version.
Finally we describe three data sets, which serve as the basis for
most of the illustrating examples in this manual.

\section{Available versions of BayesX}
\label{availableversions} \index{Java based version}
\index{non-Java based version} \index{versions}
\index{versions!Java based} \index{versions!non-Java based} {\em
BayesX} is currently available in two versions, a {\em Java based
version} and a {\em non-Java based version}. Both versions run
only under the various versions of the Windows operating system
(e.g. Windows 95, 98, 2000, NT, XP). The Java based version is
implemented partly in Java, only the computerintensive parts of
the program are implemented in C++. The non-Java based version is
written completely in C++. The non-Java based version runs
slightly faster and uses less disk space than the Java based
version. The Java based version has additional features. In
particular, functions for visualizing data and estimation results
are available only in the Java based version. We recommend the
installation of the Java based version. Both versions of {\em
BayesX} can be downloaded at
\href{http://www.stat.uni-muenchen.de/~lang/bayesx/bayesx.html}
{http://www.stat.uni-muenchen.de/\~{}lang/bayesx/bayesx.html}.


\section{Installing BayesX}
\index{installation} \index{installation directories} Installing
{\em BayesX} is very easy. Suppose you have already downloaded one
of the installation files, #bayesx.zip# for the non-Java based
version or #bayesxjava.exe# for the Java based version. To
install the non-Java based version unzip  the file
#bayesx.zip# using the {\em winzip} program and store the unzipped
files in a temporary directory. Start the program #setup.exe#,
for example by double clicking it in the {\em Windows explorer}.
Now follow the instructions of the setup routine to install {\em
BayesX}. To install the Java based version simply execute the file
#installBayesX.exe# and follow the installation instructions.

After the successful installation of {\em BayesX} your installation
directory contains five additional subdirectories, namely the
directories #doc#, #examples#, #output#, #sfunctions# and #temp#.
The #doc# directory contains the program documentation, that is this
manual. The #examples# directory contains three data sets,
#credit.raw#, #rents.raw# and #zambia.raw#. These data sets
exemplify many of the statistical functions and routines described
in the following chapters. A detailed description of the three data
sets is given in \autoref{datadescription}. The #examples# directory
also contains some tutorial programs that illustrate the usage of
most of the functions of {\em BayesX}, see the tutorials manual. The
#output# directory is the default directory for the program output.
The #output# directory can be redefined by the user. The
#sfunctions# directory contains some S-plus functions for
visualizing estimation results obtained with {\em bayesreg objects}
or {\em remlreg objects}, see \autoref{splus} for a detailed
description of the S-plus functions. However, by using the Java
based version of {\em BayesX} the S-plus functions are obsolete
because this version has its own capabilities for visualizing data
and results, see \autoref{graphobj} and \autoref{visualization} for
details. Finally in the #temp# directory some temporary files will
be stored. Normally you will never use this directory.

The created directories and their contents are briefly summarized
in \autoref{dirtable}.

\begin{table}[ht]
\begin{center}
\begin{tabular}{|l|l|}
\hline
Directory & Contents \\
\hline
#doc# & contains the manuals \\
#examples# & contains data set examples and tutorial programs \\
#output# & default directory for estimation output \\
#sfunctions# & contains some S-plus functions for visualizing output \\
#temp# & stores temporary files \\
\hline
\end{tabular}
{\em\caption{ \label{dirtable} Subdirectories of the installation
directory and their content}}
\end{center}
\end{table}

After a successful installation, {\em BayesX} can be started using
the {\em Windows Start} button.

\section{Manuals}

{\em BayesX} is shipped together with three different manuals. The
reference manual gives detailed information on the syntax of {\em
BayesX} commands and the different objects used by {\em BayesX}. The
methodology manual provides background information on the
statistical methodology that is implemented in {\em BayesX}. In this
manual you will also find more references on the methodological
background. The tutorial manual can be used to make yourself
familiar with the usage of {\em BayesX}. It contains some
self-contained tutorials, describing how to perform semiparametric
regression analyses using {\em BayesX}. In the Java based version of
{\em BayesX} the manuals are available from the help menu. In both
versions they can be found in the #doc# directory, which is a
subdirectory of the installation directory.

\section{Windows in BayesX}
\index{windows}

After starting {\em BayesX} you can see a main window with a menu
bar and four additional windows within the main window. The four
windows are the {\em command window}, the {\em output window}, the
{\em review window} and the {\em object browser}. The purpose of
these windows is described in the following four subsections.
Below the menu bar is a second bar containing three buttons named
BREAK, PAUSE and SUPPRESS OUTPUT. These buttons are described in
the last subsection of this section.

\subsection{The command window}

\index{command window} \index{windows!command} The {\em command window}
is used to enter and execute commands. By default, a command will
be executed if you press the return key. You can change this
default delimiter using the #delimiter# command, see
\autoref{delimiter}.

\subsection{The output window}
\index{output window} \index{windows!output}

In the {\em output window} all commands entered in the {\em
command window} or executed through a batch file (see
\autoref{batch}) are printed together with the program output.

\index{saving the output} The contents of the {\em output window} can be
saved and processed with your favorite text editor. For saving the
output, enter the {\em file menu} and click on {\em Save output}
or {\em Save output as}. The contents of the {\em output window} can be
saved in two different file formats. The default is the rich-text
format. The second choice is to store the {\em output window} in plain
ASCII format. The ASCII format however has the disadvantage that
all text highlights (for example bold letters) will disappear in
the saved file.

Through the {\em file menu} you can also clear the {\em output window}
(i.e. delete the contents of the window) or open an already
existing file.

Depending on the screen resolution of your computer, letters
appearing in the {\em output window} may be very small or too large. The
font size can be changed  using the {\em preferences menu}.



\subsection{The review window}
\index{review window} \index{windows!review}

In many cases subsequent commands change only slightly. The {\em
review window} gives you a convenient way to bring back and edit
past commands. In the {\em review window} the last 100 past
commands entered during a session are shown. Click once (double
click in the Java based version) on one of these past commands and
it is automatically copied to the {\em command window}, where the
command or a slightly modified version can be executed again.

\subsection{The object browser}
\index{object browser}

{\em BayesX} is object oriented although the concept goes not too
far. The {\em object browser} is used to view the contents of the
objects currently in memory. The window is split into two parts.
The left part shows the different object types currently supported
by {\em BayesX}. These are for the moment {\em dataset objects},
{\em bayesreg objects}, {\em remlreg objects}, {\em map objects},
{\em dag objects} and {\em graph objects}. By clicking on one of
the object types the names of all objects of this type will appear
in the right panel of the {\em object browser}. Double clicking on one
of the names gives a visualization of the object. The
visualization method depends on the respective object type. Double
clicking on {\em dataset objects}, for example, will open a
spreadsheet where you can inspect the variables and the
observations of the data set. Clicking on {\em map objects} opens
another window that contains a graphical representation of the
map.

\subsection{BREAK, PAUSE and SUPPRESS OUTPUT buttons}
\index{PAUSE button} \index{BREAK button} \index{SUPPRESS OUTPUT
button} \index{buttons} \index{buttons!PAUSE}
\index{buttons!BREAK} \index{buttons!SUPPRESS OUTPUT}

As already mentioned, below the menu bar a second panel can be
found which contains three buttons, a BREAK button, a PAUSE button
and the button SUPPRESS OUTPUT. The BREAK button is used to
interrupt the process that is currently executed. Clicking on the
PAUSE button interrupts the current process temporarily until the
button is pressed again. If a process is paused, the caption PAUSE
of the button is replaced by CONTINUE indicating that a second
click on the button will continue the current process. Pausing a
current process can be used to increase the execution speed of
other programs currently running on your computer. Pressing the
SUPPRESS OUTPUT button suppresses all output in the {\em output window}.
The button caption changes to SHOW OUTPUT to indicate that an
additional click on the button will cause the  program to print
the output again. Suppressing the output usually increases the
execution speed of {\em BayesX} and saves memory.


\section{General usage of BayesX}
\label{generalusage}

\subsection{Creating objects}
\label{createobject}

\index{objects} \index{objects!create} {\em BayesX} is object
oriented, that is the first thing to do during a session is to
create some objects. Currently there are six different object
types available: {\em dataset objects}, {\em bayesreg objects},
{\em remlreg objects}, {\em map objects}, {\em dag objects} and {\em
graph objects}. {\em Dataset objects} are used to handle and
manipulate data sets, see \autoref{datasetobj} for details. {\em Map
objects} are used to handle geographical maps and are covered in
more detail in \autoref{map}. The main purpose of {\em map objects} is
to serve as auxiliary objects for estimating spatial covariate
effects with {\em bayesreg objects} or {\em remlreg objects}. {\em
graph objects} are used to visualize data (e.g. create
scatterplots or color geographical maps according to some
numerical characteristics), see \autoref{graphobj} for details.
Probably the most important object types are {\em bayesreg
objects} and {\em remlreg objects}. These objects are used to
estimate Bayesian semiparametric regression models based on either
Markov Chain Monte Carlo simulation techniques ({\em bayesreg
objects}) or mixed model representations of the regression model
({\em remlreg objects}). See \autoref{bayesreg} for a detailed
description of {\em bayesreg objects} and \autoref{remlreg} for a
detailed description of {\em remlreg objects}. Another important
object type is the {\em dag object}. {\em Dag objects} are used to
estimate Gaussian or non-Gaussian dags (direct acyclic graphs)
using reversible jump MCMC simulation techniques. A detailed
description of {\em dag objects} can be found in \autoref{dag}.
The object oriented concept does not go too far, that is
inheritance or other concepts of object oriented programs or
languages such as S-plus or C++ are not supported.

Creating a new object during a session is very easy. The syntax for creating a new object is:

#># {\em objecttype objectname}

To create for example a {\em dataset object} with name #mydata#, simply type:

#> dataset mydata#

Note that there are restrictions to the naming of objects, that is
some object names are not allowed. For example, one rule is that
object names must begin with a (uppercase or lowercase) letter
rather than a number; see \autoref{varnames} for valid object
names. The section is about valid variable names for data sets,
but the same rules apply to object names.

\subsection{Applying methods of previously defined objects}

After the successful creation of an object you can apply methods
for that particular object. For instance, {\em dataset objects}
may be used to read in data stored in an ASCII file using method
#infile#, to create new variables using method #generate#, to modify
existing variables using method #replace# and so on. The syntax
for applying methods of the objects is similar for all methods
and independent of the particular object type. The general syntax is: \\
\index{general syntax} \index{syntax}

#># {\em objectname. methodname} [{\em model}] [#weight# {\em varname}] [#if# {\em boolean expression}]  \\
\hspace*{4.8cm}  [, {\em options}] [#using# {\em usingtext}]

\autoref{syntaxtable} explains the syntax parts in more detail.


\begin{table}[ht]
 \centering
\begin{tabular}{|l|l|}
\hline
Syntax part & Description \\
\hline
{\em objectname} & the name of the object to apply the method \\
{\em methodname} & the name of the method \\
{\em model} & a model specification (for example a regression model) \\
{\em #weight# varname} & specifies {\em varname} as a weight variable \\
#if# {\em boolean expression} & indicates that the method should be applied only if a \\
& certain condition holds \\
, {\em options} & define (or modify) options for the method \\
#using# {\em usingtext} & indicates that another object or file should be used to \\
& apply the particular method \\
\hline
\end{tabular}
{\em \caption{\label{syntaxtable}Syntax parts of methods for
objects}}
\end{table}


Note that $[\dots]$ indicates that this part of the syntax is
optional and may be omitted. Moreover for most methods only some
(or even none) of the syntax parts above are meaningful. Note that
the specification of invalid syntax parts is not allowed and will
cause an error message.

We illustrate the concept with some simple methods of {\em dataset
objects}. Suppose we have already created a {\em dataset object} with
name #mydata# and want to create some variables for our data set.
We first have to tell {\em BayesX} how many observations
we want to create. This can be done with the #set# command, see also \autoref{setobs}. For example

#> mydata.set obs = 1000#

indicates that the data set #mydata# should have 1000
observations. Here, the {\em methodname} is #set# and the {\em model}
is #obs =# #1000#. Since no other syntax parts (for example #if#
statements) are meaningful
for this method, they are not allowed. For instance, specifying an additional weight variable #x# by typing

#> mydata.set obs = 1000 weight x#

will cause the error message:

#ERROR: weight statement not allowed#

In a second step we can now create a new variable #X#, say, that
contains Gaussian (pseudo)random numbers with mean
2 and standard deviation 0.5:

#> mydata.generate X = 2+0.5*normal()#

Here, #generate# is the {\em methodname} and #X = 2+0.5*normal()# is
the {\em model}. In this case the {\em model} consists of the specification of
the new variable name, followed by the equal sign '=' and a
mathematical expression for the new variable. As is the case with
the #set# command other syntax parts are not meaningful and
therefore not allowed. Suppose now we want to replace the negative
values of #X# with the constant 0. This can
be done using the #replace# command by typing:

#> mydata.replace X = 0 if X < 0#

Obviously, an additional #if# statement is meaningful and is
therefore allowed, but not required.



\section{Description of data set examples}
\label{datadescription} \index{data set examples}

This section describes the three data sets used to illustrate many
of the features of {\em BayesX}. The three data sets are stored
columnwise in plain ASCII-format. The first row of each data set
contains the variable names separated by blanks. Subsequent rows
contain the observations, one observation per row.

\subsection{Rents for flats}
\label{rentdata} \index{rents for flats} \index{data set
examples!rents for flats}

According to the German rental law, owners of apartments or flats
can base an increase in the amount that they charge for rent on
'average rents' for flats comparable in type, size, equipment,
quality and location in a community. To provide information about
these 'average rents', most of the larger cities publish 'rental
guides', which can be based on regression analysis with rent as
the dependent variable. The #rent94.raw# file stored in the
#examples# directory is a subsample of data collected in 1994 for
the rental guide in Munich. The variable of primary interest is
the monthly rent per square meter in German Marks. Covariates
characterizing the flat were constructed from almost 200 variables
out of a questionnaire answered by tenants of flats. The present
data set contains a small subset of these variables that are
sufficient for demonstration purposes. \autoref{rentdatavar} describes the
variables of the data set. The data set will be used in the
following chapters to demonstrate the usage of {\em BayesX}.

Additional to the data set, the #examples# directory contains
the file #munich.bnd# that contains a map of Munich. This map
proves to be useful for visualizing regression results for the
explanatory variable location #L# in the data set. See
\autoref{map} for a description on how to incorporate geographical
maps into {\em BayesX}.

\begin{table}

\centering
\begin{tabular}{|l|l|}
\hline
{\bf Variable} & {\bf Description} \\
\hline
R & monthly rent per square meter in German marks \\
$F$ & floor space in square meters \\
$A$ & year of construction \\
$L$ & location of the building in subquarters \\
 \hline
\end{tabular}
{\em \caption{\label{rentdatavar}Variables of the rent data set}}
\end{table}


\subsubsection*{References}

\begin{description}

\item[Lang, S. and Brezger, A. (2002):] \href{http://www.stat.uni-muenchen.de/~lang/publications.html}
{Bayesian P-splines.} {\it Journal of Computational and Graphical
Statistics}, 13, 183-212.

\end{description}


\subsection{Credit scoring}
\label{creditdata} \index{credit scoring} \index{data set
examples!credit scoring}

The aim of credit scoring is to model or predict the probability
that a client with certain covariates ('risk factors') is to be
considered as a potential risk, and therefore will probably not
pay back his credit as agreed upon by contract. The data set
consists of 1000 consumers' credits from a South German bank. The
response variable is 'creditability', which is given in
dichotomous form ($y=0$ for creditworthy, $y=1$ for not
creditworthy). In addition, 20 covariates assumed to influence
creditability were collected. The present data set (stored in the
#examples# directory)  contains a subset of these covariates
that proved to be the main influential variables on the response
variable, see Fahrmeir and Tutz (2001, ch. 2.1). \autoref{creditdatavar}
contains a description of the variables of the data set. Usually a
binary logit model is applied to estimate the effect of the
covariates on the probability of being not creditworthy. As in the
case of the rents for flats example, this data set is used to
demonstrate the usage of certain features of {\em BayesX} , see
primarily \autoref{creditanalyse} for a Bayesian regression
analysis of the data set.

\begin{table}[ht]

\begin{tabular}{|l|l|}
\hline
{\bf Variable} & {\bf Description} \\
\hline
$y$ & creditability, dichotomous with $y=0$ for creditworthy, $y=1$ for \\
    & not creditworthy \\
$account$ & running account, trichotomous with categories "no
running account" \\& ($=1$),
    "good running account"
($=2$),  "medium running account" \\&("less than 200 DM") ($=3$)  \\
$duration$ & duration of credit in months, metrical \\
$amount$ & amount of credit in 1000 DM, metrical \\
$payment$ & payment of previous credits, dichotomous with categories "good" ($=1$), \\ & "bad" ($=2$)  \\
$intuse$ & intended use, dichotomous with categories "private" ($=1$) or \\ & "professional" ($=2$)  \\
$marstat$ & marital status, with categories "married" ($=1$) and "living alone" ($=2$). \\
\hline
\end{tabular}
{\em \caption{\label{creditdatavar}Variables of the credit scoring
data set}}
\end{table}

\subsubsection*{References}

\begin{description}
\item [Fahrmeir, L., Tutz, G. (2001):] {\it Multivariate Statistical
Modelling based on Generalized Linear Models.} New York:
Springer--Verlag.
\end{description}

\subsection{Childhood undernutrition in Zambia}
\label{zambia} \index{childhood undernutrition} \index{data set
examples!childhood undernutrition}

Acute and chronic undernutrition is considered to be one of the
worst health problems in developing countries. Undernutrition
among children is usually determined by assessing the
anthropometric status of the child relative to a reference
standard. In our example undernutrition is measured through
stunting or insufficient height for age, indicating chronic
undernutrition. Stunting for a child $i$ is determined using a
Z-score which is defined as
\[Z_i = \frac{AI_i-MAI}{\sigma}\]
where $AI$ refers to the child`s anthropometric indicator (height
at a certain age in our example), MAI refers to the median of the
reference population and $\sigma$ refers to the standard deviation
of the reference population.

The data set contains the (standardized) Z-score for 4847 children
together with several covariates that are supposed to have
influence on undernutrition including the body mass index of the
child`s mother, the age of the child and the district the child
lives in. \autoref{zambiavar} gives more information on the
covariates in the data set.

This data set is used in chapter \ref*{zambiaanalysis} and
\ref*{remlregzambiaanalysis} of the tutorial manual.

\begin{table}[|h|t|]
\begin{center}
\begin{tabular}{|l|l|}
 \hline
 {\bf Variable} & {\bf Description}\\
 \hline
 $hazstd$ & standardized Z-score of stunting\\
 $bmi$ & body mass index of the mother\\
 $agc$ & age of the child\\
 $district$ & district where the child lives\\
 $rcw$ & mother`s employment status with categories "working" (= 1) and "not working" \\
 & (= $-1$)\\
 $edu1$ & mother`s educational status with categories "complete primary but incomplete\\
 $edu2$ & secondary" ($edu1=1$), "complete secondary or higher" ($edu2=1$) and\\
 & "no education or incomplete primary" ($edu1=edu2=-1$)\\
 $tpr$ & locality of the domicile with categories "urban" (= 1) and "rural" (= $-1$)\\
 $sex$ & gender of the child with categories "male" (= 1) and
 "female" (= $-1$)\\
 \hline
\end{tabular}
{\em\caption{Variables in the undernutrition data set.
\label{zambiavar}}}
\end{center}
\end{table}

\subsubsection*{References}

\begin{description}
\item [Kandala, N. B., Lang, S., Klasen, S. and Fahrmeir, L. (2001):] Semiparametric Analysis of
the Socio-Demographic and Spatial Determinants of Undernutrition
in Two African Countries.{\it Research in Official Statistics}, 1,
81-100.
\end{description}
