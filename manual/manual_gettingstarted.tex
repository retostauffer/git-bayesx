\chapter{Getting started}
\label{gettingstarted}

This chapter provides some useful information for first-time users
of {\em BayesX}: Which versions of BayesX are currently available
(\autoref{availableversions}), how is BayesX installed
(\autoref{installbayesx}), what types of manuals exist
(\autoref{bayesxmanuals}), and how is the graphical user interface
organized (\autoref{bayesxwindows}). Section \ref{generalusage}
describes the general usage of {\em BayesX} and the structure of
{\em BayesX} syntax. The final section contains a description of
three data sets that will be used for demonstrating purposes in
the later chapters.

\section{Available versions of BayesX}
\label{availableversions} \index{Java based version} \index{Non-Java
based version} \index{Versions} \index{Versions!Java based}
\index{Versions!Non-Java based}

In its current form, {\em BayesX} runs only under the various
versions of the Windows operating system (e.g. Windows 95, 98,
2000, NT, XP). The graphical user interface and the visualisation
tools are implemented in Java, while the computerintensive parts
of the program have been implemented in C++. The current version
of {\em BayesX} can be downloaded from
\href{http://www.stat.uni-muenchen.de/~bayesx}
{http://www.stat.uni-muenchen.de/\~{}bayesx}.

Up to version 1.3, {\em BayesX} has been distributed in two
versions, the {\em Java based version} described above and a {\em
non-Java based version}, which was written completely in C++.
Since the Java based version has additional features, the non-Java
based version is no longer supported.

\section{Installing BayesX}\label{installbayesx}
\index{Installation} \index{Installation directories}

After you have downloaded the file #installBayesX.exe# from the
{\em BayesX} homepage, proceed by executing this file. The
installation process is quite simple and comparable to most
standard installations. The installation routine will request all
necessary information.

When {\em BayesX} has been installed successfully, it can be
started using the {\em Windows Start} button or the icon created
on the desktop (depending on your specifications during the
installation process). The installation directory contains the
five subdirectories #doc#, #examples#, #output#, #sfunctions# and
#temp#. The #doc# directory contains the program documentation,
i.e. the three {\em BayesX} manuals (see the following
subsection). The #examples# directory contains the three data
sets, #credit.raw#, #rents.raw# and #zambia.raw#, which will be
used for demonstrating purposes throughout the manual. A detailed
description of these data sets is given in
\autoref{datadescription}. The #examples# directory also contains
some tutorial programs that illustrate the usage of {\em BayesX},
see the tutorials manual. The #output# directory is the default
directory for program output stored in files. Of course, the
output window can be redefined by the user, compare
\autoref{bayesregglobopt} and \autoref{remlregglobopt}. The
#sfunctions# directory contains some R and S-plus functions for
visualizing estimation results obtained with {\em bayesreg
objects} or {\em remlreg objects}, see \autoref{splus} for a
detailed description of these functions. Note, however, that the
current {\em BayesX} version has its own graphics capabilities,
see \autoref{graphobj} and \autoref{visualization} for details.
Finally, temporary files created when estimating regression models
will be stored in the #temp# directory. Usually you will never use
this directory.

The subdirectories and their content are briefly summarized in
\autoref{dirtable}.

\begin{table}[ht]
\begin{center}
\begin{tabular}{|l|l|}
\hline
Directory & Content \\
\hline
#doc# & the {\em BayesX} manuals \\
#examples# & data set examples and tutorial programs \\
#output# & default directory for estimation output \\
#sfunctions# & R and S-plus functions for visualizing output \\
#temp# & temporary files \\
\hline
\end{tabular}
{\em\caption{ \label{dirtable} Subdirectories of the installation
directory and their content.}}
\end{center}
\end{table}

\section{Manuals}\label{bayesxmanuals}
\index{Manuals}

{\em BayesX} is shipped with three different manuals. The
reference manual (i.e. the manual you are just reading) gives
detailed information on the general usage of BayesX, the syntax of
{\em BayesX} commands and the different objects used by {\em
BayesX}. The methodology manual provides background information on
the statistical methodology that is implemented in {\em BayesX}.
In this manual you will also find more references on the
methodological background. The tutorial manual is intended to make
new users familiar with the usage of {\em BayesX} by demonstrating
examples. It contains two self-contained tutorials, describing how
to perform semiparametric regression analyses using {\em BayesX}.
The manuals are also available from the help menu and can be found
in the #doc# directory (a subdirectory of the installation
directory).

\section{Windows and buttons in BayesX}\label{bayesxwindows}
\index{Windows}

After starting {\em BayesX} you will see a main window with a menu
bar and four additional subwindows. The four windows are the {\em
command window}, the {\em output window}, the {\em review window}
and the {\em object browser}. The purpose of these windows is
described in the following four subsections. Below the menu bar
there is a menu bar containing the buttons BREAK, PAUSE and
SUPPRESS OUTPUT, and the priority menu. Their functionality is
described in subsection \ref{buttons} and \ref{prioritymenu}.

\subsection{The command window}
\index{Command window} \index{Windows!Command}

Allmost all {\em BayesX} commands are entered and executed in the
{\em command window}. By default, a command will be executed if
you press the return key. You can change this default delimiter
using the #delimiter# command, see \autoref{delimiter}.

\subsection{The output window}
\index{Output window} \index{Windows!Output}

In the {\em output window}, all commands entered in the {\em
command window} or executed through a batch file (see
\autoref{batch}) are printed together with the program output.

\index{Saving the output} The content of the {\em output window} can
be saved and processed with your favorite text editor. For saving
the output, enter the {\em file menu} and click on {\em Save output}
or {\em Save output as}. The file save dialog will allow you to
choose between two different file formats. The default is the
rich-text format but it is also possible to store the {\em output
window} in plain ASCII format. This, however, has the disadvantage
that all text highlights (for example bold letters) will disappear
in the saved file.

The {\em file menu} also allows to clear the {\em output window}
(i.e. delete the content of the window) or to open an already
existing file.

Depending on the screen resolution of your computer, letters
appearing in the {\em output window} may be very small or too
large. The font size can be varied in the {\em preferences menu}.

\subsection{The review window}
\index{Review window} \index{Windows!Review}

In many cases, subsequent commands change only slightly. The {\em
review window} gives you convenient access to the last 100 past
commands entered during a session. Double click on one of these
past commands and it is automatically copied to the {\em command
window}, where it can be modified and / or executed again.

\subsection{The object browser}
\index{Object browser}

{\em BayesX} is object oriented, i.e. different types of objects
are used to store data, estimate regression models, etc. The {\em
object browser} provides an overview of the objects currently
defined and about their contents. The {\em object browser} window
is split into two parts. The left part displays the different
object types currently supported by {\em BayesX} ({\em dataset
objects}, {\em bayesreg objects}, {\em remlreg objects}, {\em map
objects}, {\em dag objects} and {\em graph objects})s. By clicking
on one of the object types, the names of all objects of this type
will appear in the right panel of the {\em object browser}. Double
clicking on one of the names gives a visualization of the object
and / or a short summary in the {\em output window}, depending on
the object type. Double clicking on {\em dataset objects}, for
example, will open a spreadsheet where the variables and the
observations of the data set can be inspected. Clicking on {\em
map objects} opens a window that contains a graphical
representation of the map.

\subsection{BREAK, PAUSE and SUPPRESS OUTPUT button}
\label{buttons} \index{PAUSE button} \index{BREAK button}
\index{SUPPRESS OUTPUT button} \index{Buttons} \index{Buttons!PAUSE}
\index{Buttons!BREAK} \index{Buttons!SUPPRESS OUTPUT}

The {\em BayesX} button panel contains the BREAK button, the PAUSE
button and the SUPPRESS OUTPUT button. The purpose of the BREAK
button is to interrupt the process that is currently executed
(this may take some time). Clicking on the PAUSE button interrupts
the current process temporarily until the button is pressed again.
If a process is paused, the button caption PAUSE is replaced by
CONTINUE, indicating that a second click on the button will
continue the current process. Pausing a current process will
increase the execution speed of other programs currently running
on your computer. Clicking the SUPPRESS OUTPUT button suppresses
printing of output in the {\em output window}. The button caption
changes to SHOW OUTPUT to indicate that an additional click on the
button will cause the program to print the output again.
Suppressing the output increases the execution speed of {\em
BayesX} and saves memory. Note, that you can store your output in
a log-file even if printing of the output is suppressed (see
\autoref{logfile}).

\subsection{Priority menu}
\label{prioritymenu} \index{Priority menu}

When running extensive computations, it may be desirable to reduce
the priority of BayesX since otherwise all further programs may be
executed very slowly. The priority menu allows you to change the
priority of your computations from within BayesX. Usually there
should be no need to increase the priority (although it is
possible). To pause the current computations use the PAUSE button.

\section{General usage of BayesX}
\label{generalusage}

\subsection{Creating objects}
\label{createobject} \index{Objects} \index{Objects!Create}

{\em BayesX} is implemented in an object oriented way, although
the object oriented concept does not go too far, i.e. inheritance
or other concepts of object oriented programming languages such as
S or C++ are not supported. As a consequence, the first thing to
do during a session, is to create some objects. Currently, six
different object types are available: {\em dataset objects}, {\em
bayesreg objects}, {\em remlreg objects}, {\em map objects}, {\em
dag objects} and {\em graph objects}. {\em Dataset objects} are
used to store, handle, and manipulate data sets, see
\autoref{datasetobj} for details. {\em Map objects} are used to
handle geographical information and are covered in more detail in
\autoref{map}. The main purpose of {\em map objects} is to serve
as auxiliary objects for {\em bayesreg objects} or {\em remlreg
objects} when estimating spatial effects. {\em Graph objects} are
used to visualize data (e.g. to create scatterplots or to color
geographical maps according to some numerical characteristics),
see \autoref{graphobj} for details. The most important object
types are {\em bayesreg objects} and {\em remlreg objects}. These
objects are used to estimate Bayesian semiparametric regression
models based on either Markov Chain Monte Carlo simulation
techniques ({\em bayesreg objects}) or a mixed model
representation of the regression model ({\em remlreg objects}).
See \autoref{bayesreg} for a detailed description of {\em bayesreg
objects} and \autoref{remlreg} for a detailed description of {\em
remlreg objects}. {\em Dag objects} are used to estimate Gaussian
or non-Gaussian DAGs (direct acyclic graphs) based on reversible
jump MCMC simulation techniques (see \autoref{dag} for details).

The syntax for creating a new object is:

#># {\em objecttype objectname}

To create for example a {\em dataset object} with name #mydata#, simply type:

#> dataset mydata#

Note that some restrictions are imposed on the names of objects,
i.e. not all object names are allowed. For example, object names
have to begin with an uppercase or lowercase letter rather than a
number. Section \ref{varnames} discusses valid variable names but
the same rules apply also to object names.

\subsection{Applying methods to previously defined objects}

When an object has been created successfully, you can apply methods
to that particular object. For instance, {\em dataset objects} may
be used to read data stored in an ASCII file using method #infile#,
to create new variables using method #generate#, to modify existing
variables using method #replace# and so on. The syntax for applying
methods to the objects is similar for all methods and independent of
the particular object type. The general syntax is: \index{General
syntax} \index{Syntax}

#># {\em objectname.methodname} [{\em model}] [#weight# {\em varname}] [#if# {\em boolean expression}] [, {\em options}] \\
\hspace*{4.8cm} [#using# {\em usingtext}]

\autoref{syntaxtable} explains the syntax parts in more detail.


\begin{table}[ht]
 \centering
\begin{tabular}{|l|l|}
\hline
Syntax part & Description \\
\hline
{\em objectname} & the name of the object to apply the method to \\
{\em methodname} & the name of the method \\
{\em model} & a model specification (for example a regression model) \\
{\em #weight# varname} & specifies {\em varname} as weight variable \\
#if# {\em boolean expression} & indicates that the method should be applied only if a \\
& certain condition holds \\
, {\em options} & define (or modify) options for the method \\
#using# {\em usingtext} & indicates that another object or file is required to \\
& apply the particular method \\
\hline
\end{tabular}
{\em \caption{\label{syntaxtable}Parts of the general BayesX
syntax.}}
\end{table}

Note that $[\dots]$ indicates that this part of the syntax is
optional and may be omitted. Moreover for most methods only some
of the syntax parts above will be meaningful. The specification of
invalid syntax parts is not allowed and will cause an error
message.

We illustrate the concept with some simple methods of {\em dataset
objects}. Suppose that a {\em dataset object} with name #mydata#
has already been created and that some variables should be
created. First of all, we have to tell {\em BayesX} how many
observations we want to create. This can be done with the
 #set obs# command, see also \autoref{setobs}. For example

#> mydata.set obs = 1000#

indicates that the data set #mydata# should have 1000
observations. In this case, the {\em methodname} is #set# and the
{\em model} is #obs =# #1000#. Since no other syntax parts (for
example #if# statements) are meaningful for this method, they are
not allowed. For instance, specifying an additional weight
variable #x# by typing

#> mydata.set obs = 1000 weight x#

will cause the error message:

#ERROR: weight statement not allowed#

In a second step we can now create a new variable #X#, say, that
contains Gaussian (pseudo) random numbers with mean 2 and standard
deviation 0.5:

#> mydata.generate X = 2+0.5*normal()#

Here, #generate# is the {\em methodname} and #X = 2+0.5*normal()#
is the {\em model}. In this case the {\em model} consists of the
specification of the new variable name, followed by the equal sign
'#=#' and a mathematical expression for the new variable. Similar
as for the #set obs# command other syntax parts are not meaningful
and therefore not allowed. If the negative values of #X# should be
replaced with the constant 0, this can be achieved using the
#replace# command:

#> mydata.replace X = 0 if X < 0#

Obviously, the #if# statement is meaningful and is therefore
allowed, but not required.

\section{Description of data set examples}
\label{datadescription} \index{Data set examples}

This section describes three data sets used to illustrate many of
the features of {\em BayesX} in the following chapters as well as
in the tutorial manual. All data sets are stored columnwise in
plain ASCII-format. The first row of each data set contains the
variable names separated by blanks. Subsequent rows contain the
observations, one observation per row.

\subsection{Rents for flats}
\label{rentdata} \index{Rents for flats} \index{Data set
examples!Rents for flats}

According to the German rental law, owners of apartments or flats
can base an increase in the amount that they charge for rent on
'average rents' for flats comparable in type, size, equipment,
quality and location in a community. To provide information about
these 'average rents', most of the larger cities publish 'rental
guides', which can be based on regression analyses with rent as
the dependent variable. The file #rent94.raw# (stored in the
#examples# directory) is a subsample of data collected in 1994 for
the Munich rental guide. The variable of primary interest is the
monthly rent per square meter in German Marks. Covariates
characterizing the flat were constructed from almost 200 variables
out of a questionnaire answered by tenants of flats. The present
data set contains a small subset of these variables that are
sufficient for demonstration purposes (see \autoref{rentdatavar}).

In addition to the data set, the #examples# directory contains a
map of Munich in the file #munich.bnd#. This map will be useful
for visualizing effects of the location #L#. See \autoref{map} for
a description on how to incorporate geographical maps into {\em
BayesX}.

\begin{table}

\centering
\begin{tabular}{|l|l|}
\hline
{\bf Variable} & {\bf Description} \\
\hline
R & monthly rent per square meter in German marks \\
$F$ & floor space in square meters \\
$A$ & year of construction \\
$L$ & location of the building in subquarters \\
 \hline
\end{tabular}
{\em \caption{\label{rentdatavar}Variables of the rent data set.}}
\end{table}


\subsubsection*{References}

\begin{description}

\item[Lang, S. and Brezger, A. (2004):]
\href{http://www.stat.uni-muenchen.de/~lang/publications.html}
{Bayesian P-splines.} {\it Journal of Computational and Graphical
Statistics}, 13, 183-212.

\end{description}


\subsection{Credit scoring}
\label{creditdata} \index{Credit scoring} \index{Data set
examples!Credit scoring}

The aim of credit scoring is to model and / or predict the
probability that a client with certain covariates ('risk factors')
will not pay back his credit. The data set contained in the file
#credit.raw# consists of 1000 consumer credits from a bank in
southern Germany. The response variable is 'creditability' in
dichotomous form ($y=0$ for creditworthy, $y=1$ for not
creditworthy). In addition, 20 covariates that are assumed to
influence creditability were collected. The present data set
(stored in the #examples# directory) contains a subset of these
covariates that proved to be the main influential variables on the
response variable, see Fahrmeir and Tutz (2001, ch. 2.1).
\autoref{creditdatavar} gives a description of the variables of
the data set. Usually a binary logit model is applied to estimate
the effect of the covariates on the probability of being not
creditworthy. As in the case of the rents for flats example, this
data set is used to demonstrate the usage of certain features of
{\em BayesX}, see primarily \autoref{creditanalyse} for a Bayesian
regression analysis of the data set.

\begin{table}[ht]

\begin{tabular}{|l|l|}
\hline
{\bf Variable} & {\bf Description} \\
\hline
$y$ & creditability, dichotomous with $y=0$ for creditworthy, $y=1$ for \\
    & not creditworthy \\
$account$ & running account, trichotomous with categories "no
running account" \\& ($=1$),
    "good running account"
($=2$),  "medium running account" \\&("less than 200 DM") ($=3$)  \\
$duration$ & duration of credit in months, continuous \\
$amount$ & amount of credit in 1000 DM, continuous \\
$payment$ & payment of previous credits, dichotomous with categories "good" ($=1$), \\ & "bad" ($=2$)  \\
$intuse$ & intended use, dichotomous with categories "private" ($=1$) or \\ & "professional" ($=2$)  \\
$marstat$ & marital status, with categories "married" ($=1$) and "living alone" ($=2$). \\
\hline
\end{tabular}
{\em \caption{\label{creditdatavar}Variables of the credit scoring
data set.}}
\end{table}

\subsubsection*{References}

\begin{description}
\item [Fahrmeir, L., Tutz, G. (2001):] {\it Multivariate Statistical
Modelling based on Generalized Linear Models.} New York:
Springer--Verlag.
\end{description}

\subsection{Childhood undernutrition in Zambia}
\label{zambia} \index{Childhood undernutrition} \index{Data set
examples!Childhood undernutrition}

Acute and chronic undernutrition is considered to be one of the
worst health problems in developing countries. Undernutrition
among children is usually determined by assessing the
anthropometric status of the child relative to a reference
standard. In our example undernutrition is measured through
stunting (insufficient height for age), indicating chronic
undernutrition. Stunting for child $i$ is determined using the
Z-score
\[Z_i = \frac{AI_i-MAI}{\sigma}\]
where $AI$ refers to the child`s anthropometric indicator (height
at a certain age in our example), MAI refers to the median of the
reference population and $\sigma$ refers to the standard deviation
of the reference population.

The data set #zambia.raw# contains the (standardized) Z-score for
4847 children together with several covariates that are supposed
to influence undernutrition (e.g. the body mass index of the
mother, the age of the child, and the district the mother lives
in). \autoref{zambiavar} gives more information on the covariates
in the data set.

This data set is used in chapter \ref*{zambiaanalysis} and
\ref*{remlregzambiaanalysis} of the tutorial manual.

\begin{table}[|h|t|]
\begin{center}
\begin{tabular}{|l|l|}
 \hline
 {\bf Variable} & {\bf Description}\\
 \hline
 $hazstd$ & standardized Z-score for stunting\\
 $bmi$ & body mass index of the mother\\
 $agc$ & age of the child\\
 $district$ & district where the mother lives\\
 $rcw$ & mother`s employment status with categories "working" (= 1) and "not working" \\
 & (= $-1$)\\
 $edu1$ & mother`s educational status with categories "complete primary but incomplete\\
 $edu2$ & secondary" ($edu1=1$), "complete secondary or higher" ($edu2=1$) and\\
 & "no education or incomplete primary" ($edu1=edu2=-1$)\\
 $tpr$ & locality of the domicile with categories "urban" (= 1) and "rural" (= $-1$)\\
 $sex$ & gender of the child with categories "male" (= 1) and
 "female" (= $-1$)\\
 \hline
\end{tabular}
{\em\caption{Variables in the undernutrition data set.
\label{zambiavar}}}
\end{center}
\end{table}

\subsubsection*{References}

\begin{description}
\item [Kandala, N. B., Lang, S., Klasen, S. and Fahrmeir, L. (2001):] Semiparametric Analysis of
the Socio-Demographic and Spatial Determinants of Undernutrition
in Two African Countries.{\it Research in Official Statistics}, 1,
81-100.
\end{description}
