\chapter{stepwisereg objects}
\normalsize
\label{stepwisereg} \index{Stepwisereg object}

{\em stepwisereg objects} are used to fit models with {\em structured
additive predictor} subsumed in the class of {\em structured
additive regression (STAR)} models, see Belitz and Lang (2007),
Fahrmeir, Kneib, and Lang (2004, 2007). In contrary to {\em bayesreg} and
{\em remlreg} objects described in the previous two chapters, {\em stepwisereg}
objects are able perform simultaneous selection of relevant covariates, choice of the
smoothing parameters and  estimation of the parameters. The algorithms are able to
\begin{itemize}
\item decide whether a particular covariate enters the model,
\item decide whether a continuous covariate enters the model linearly or nonlinearly,
\item decide whether a spatial effect enters the model,
\item decide whether a unit- or cluster specific heterogeneity effect enters the model,
\item select complex interaction effects (two dimensional surfaces, varying coefficient terms),
\item select the degree of smoothness of  nonlinear covariate, spatial or cluster specific heterogeneity effects.
\end{itemize}
Inference is based on penalized likelihood in combination with fast algorithms for selection relevant covariates
and model terms. Different models are compared via various goodness of fit criteria., e.g. $AIC$, $BIC$, $GCV$ and 5 or 10 fold cross
validation. Models with structured additive predictor are described in considerable detail in
the methodology manual. Details on the algorithms for model choice and variable selection are given in Belitz and Lang (2007) and Belitz (2007).

\index{Generalized linear models} \index{Generalized additive
models} \index{Varying coefficients} \index{Bayesian semiparametric
regression} \index{model choice} \index{variable selection}

%First steps with {\em remlreg objects} can be done with the example
%in chapter \ref*{remlregzambiaanalysis} of the tutorial manual which
%provides a self-contained demonstrating example.

\section{Method regress}
\index{Regress function}\index{Stepwisereg object!Regress
function}  \label{stepwiseregregress}

\subsection{Syntax}
\index{Regression syntax}\index{Stepwisereg object!Regression syntax}
\label{stepwiseregregresssyntax}

#># {\em objectname}.#regress# {\em model} [#weight# {\em weightvar}] [#if# {\em expression}] [{\em , options}] #using# {\em dataset}

Method #regress# estimates the regression model specified in {\em
model} using the data specified in {\em dataset}. {\em dataset} has
to be the name of a {\em dataset object} created before. The details
of correct model specification are covered in
\autoref{stepwiseregmodelsyntax}. The distribution of the response
variable can be either Gaussian, binomial, multinomial or Poisson.
The response distribution is specified using
option #family#, see \autoref{stepwiseregfamilysyntax} below. The
default is #family=binomial# with a logit link. An #if# statement
can be specified to analyze only parts of the data set, i.e. the
observations where {\em expression} is true.

\subsubsection{Optional weight variable}
\index{Weighted regression}
\label{stepwiseregweightspecification}

An optional weight variable {\em weightvar} can be specified to
estimate weighted regression models. For Gaussian responses, {\em
BayesX} assumes that $y_i|\eta_i,\sigma^2 \sim
N(\eta_i,\sigma^2/weightvar_i)$. Thus, for grouped Gaussian
responses the weights represent the number of observations in the
groups if the $y_i$'s are the average of individual responses. If
the $y_i$s are the sum of responses in every group, the weights have
to be the reciprocal of the number of observations in the groups. Of
course, estimation of usual weighted regression models with
heteroscedastic errors is also possible. In this case, the weights
should be proportional to the reciprocal of the heteroscedastic
variances. If the response distribution is binomial, the weight
variable should correspond to the number of replications while the
values of the response variable should represent the number of
successes. If weight is omitted, {\em BayesX} assumes that the
number of replications is one, i.e. the values of the response must
be either zero or one. For grouped Poisson data, the weights have to
specify the number of observations in a group while the $y_i$s are
assumed to be the average of individual responses. Weights are not
allowed for models with multicategorical responses.

\subsubsection{Syntax of possible model terms}
\label{stepwiseregmodelsyntax}
\index{Model terms}
\index{Stepwisereg object!Model terms}

The general syntax of models for {\em stepwisereg objects} is:

$depvar = term_1 + term_2 + \cdots + term_r$

{\em depvar} specifies the dependent variable whereas
$term_1$,\dots,$term_r$ define the form of covariate influences. The
different terms must be separated by '+' signs. A constant intercept
is automatically included in the model and has not to be
specified by the user.

This section reviews all possible model terms supported in the
current version of {\em stepwisereg objects} and provides some specific
examples. Note that all terms may be combined in arbitrary
order. An overview about the capabilities of {\em stepwisereg objects}
is given in \autoref{stepwiseregterms}. \autoref{stepwisereginteractions}
shows how interactions between covariates are specified. Full
details about all available options are given in
\autoref{stepwisereglocaloptions}.

Throughout this section Y denotes the dependent variable.


\begin{longtable}{p{2.2cm} p{13.3cm}}
{\tt algorithm} & specifies the selection procedure that is to be used. \\
            & \\
{\tt steps}     & defines the maximum number of iterations that can be used during the selection process. If the value {\tt steps}
                   is reached before the selection process is finished, the process stops and the results of the current model
                   are written to the results files. If that happens, a warning is written to the output window.
                   By setting {\tt steps=0} it is possible to estimate a certain model
                   without performing any selection. \\
            & \\
{\tt criterion} & specifies the selection criterion that is to be used. \\
            & \\
{\tt proportion} & If the selection is based on a criterion using a training and a validation data set,
                  i.e.~on {\tt MSEP}, {\tt proportion} defines the fraction of the original data used as training data. \\
            & \\
{\tt startmodel} & defines the model that is used as starting model. \\
%{\tt hierarchcal} \> hoffentlich \"{u}berfl\"{u}ssig
            & \\
{\tt number} & defines the number of different smoothing parameters to be used for the nonlinear terms. This number can be
overwritten using the local option {\tt number}. \\
            & \\
{\tt trace} & specifies how detailed the output in the {\it output window} will be. \\
            & \\
{\tt CI} & specifies if confidence intervals are to be calculated. The default value is {\tt CI=none}
so that no confidence intervals are obtained. {\tt CI=MCMCselect} yields confidence intervals which are
estimated by MCMC techniques conditional on the selected model, i.e.~scale parameter
and smoothing parameters are fixed on the values chosen by the preceding selection procedure.
Unconditional confidence intervals can be obtained by {\tt CI=MCMCbootstrap} where several models
are selected on the basis of bootstrap samples. For each of the selected models
samples are drawn by MCMC techniques conditional on the respective model and based on the original data set.
{\tt CI=bootstrap} yields unconditional confidence intervals by selecting many models
on the basis of bootstrap samples. \\
            & \\
{\tt bootstrap-} \newline {\tt samples} & defines the number of bootstrap samples used for {\tt CI=bootstrap} or {\tt CI=MCMCbootstrap}. \\
            & \\
{\tt iterations} & defines the number of MCMC iterations used for {\tt CI=MCMCselect} or
{\tt CI=MCMCbootstrap}. With {\tt CI=MCMCbootstrap}, option {\tt iterations} specifies the total number
of iterations, i.e.~the sum of iterations used for the individual conditional MCMC estimations.
Here, {\tt iterations} is divided equally between the individual conditional estimations so that the number of iterations
used for one model is {\tt iterations / (bootstrap + 1)}. Hence, {\tt iterations} should be chosen appropriately.  \\
            & \\
{\tt step} & is a thinning parameter and specifies that only every {\tt step}--th MCMC--sample is used for the calculation of credible intervals
with {\tt CI=MCMCselect} or {\tt CI=MCMCbootstrap}.
Since the samples are correlated, the thinning out of MCMC samples is used to obtain
approximately independent samples. \\
            & \\
{\tt burnin} & defines the number of MCMC iterations used for the burn--in phase
at the beginning of each conditional MCMC estimation.
Hence it is meaningful for {\tt CI=MCMCbootstrap} and {\tt CI=MCMCselect}. The burn--in phase usually
is needed to achieve convergence of the Markov chain regarding its stationary (i.e.~the posterior) distribution.
In our case, the initial estimates for each conditional MCMC estimation are the posterior mode estimates. That
means, the Markov chain already starts in its stationary distribution.
Hence, the burn--in phase usually is not needed here and it is possible to define {\tt burnin=0} what
saves a lot of computing time. \\
            & \\
{\tt level1} & defines the first significance level for confidence intervals. \\
            & \\
{\tt level2} & defines the second significance level for confidence intervals. \\
            & \\
{\tt predict} & By specifying {\tt predict} an additional results file is created containing estimates for the predictor and
                for the conditional expectation of the response variable. \\
            & \\
{\tt family} & specifies response distribution and link function. \\
            & \\
{\tt reference} & specifies the reference category for multinomial logit models.
\end{longtable}


\begin{table}[ht] \footnotesize
\begin{center}
\begin{tabular}{|p{2.5cm}|p{1.5cm}|p{1.5cm}|p{2.5cm}|p{6cm}|}
\hline
{\bf global option} & {\bf type} & {\bf default} & {\bf values} & {\bf meaning} \\
\hline \hline
{\tt algorithm}  & string  & cdescent1 & cdescent1 & adaptive search \\
                 &         &           & cdescent2 & exact search \\
                 &         &           & cdescent3 & adaptive/exact search \\
                 &         &           & stepwise  & stepwise algorithm \\
\hline
{\tt steps}      & numeric \newline (integer) & 1000 & $\{0;10000\}$ & maximum number of iterations \\
\hline
{\tt criterion}  & string  & GCV       & GCV      & GCV (for non--Gaussian response (\ref{GCV_general2})
                                                      based on deviance residuals)\\
                 &         &           & GCVrss   & only meaningful for non--Gaussian response:
                                                    GCV (\ref{GCV_general}) based on residual sum of squares \\
                 &         &           & AIC      & AIC \\
                 &         &           & AIC\_imp & improved AIC \\
                 &         &           & BIC      & BIC \\
                 &         &           & MSEP     & MSEP \\
                 &         &           & CV5      & 5--fold cross validation \\
                 &         &           & CV10     & 10--fold cross validation \\
                 &         &           & AUC      & area under the ROC curve \newline
                                                    (only for binary response) \\
\hline
{\tt proportion} & numeric \newline (real)    & 0.75 & $(0;1)$ & for MSEP (see description above) \\
\hline
{\tt startmodel} & string  & linear    & linear      & base model with degrees of freedom equal to one \\
                 &         &           & empty       & empty model containing the intercept term only \\
                 &         &           & full        & most complex possible model \\
                 &         &           & userdefined & base model specified by the user; \newline
                                                       otherwise the linear one \\
\hline
{\tt number}     & numeric \newline (integer) & 20   & $\{1;50\}$ & number of smoothing parameters \\
\hline
{\tt trace}      & string  & trace\_on & trace\_on   & output shows every new model during iterations \\
                 &         &           & trace\_half & output shows the starting models of all iterations \\
                 &         &           & trace\_off  & no output except starting and final model \\
\hline
{\tt CI}         & string  & none      & none          & no confidance intervals \\
                 &         &           & MCMCselect    & conditional MCMC confidance bands \\
                 &         &           & MCMCbootstrap & unconditional confidance bands based on bootstrap and MCMC \\
                 &         &           & bootstrap     & unconditional MCMC confidance intervals based on bootstrap \\
\hline
{\tt bootstrap-} \newline {\tt samples}  & numeric \newline (integer) & 0 & $\{100;10000\}$ & number of bootstrap samples \\
\hline
\end{tabular}
{\em\caption {\label{stewpisereg_globaloptions} Possible global options
for stepwisereg objects.}}
\end{center}
\end{table}
%
\begin{table}[ht] \footnotesize
\begin{center}
\begin{tabular}{|p{2.5cm}|p{1.5cm}|p{1.5cm}|p{2cm}|p{6cm}|}
\hline
{\bf global option} & {\bf type} & {\bf default} & {\bf values} & {\bf meaning} \\
\hline \hline
{\tt iterations}  & numeric \newline (integer) & 52000 & $\{1;10000000\}$ & total number of MCMC iterations \\
\hline
{\tt step}  & numeric \newline (integer) & 50 & $\{1;1000\}$ & thinning parameter for MCMC samples \\
\hline
{\tt burnin}  & numeric \newline (integer) & 2000 & $\{0;500000\}$ & number of MCMC iterations used for each burnin phase \\
\hline
{\tt level1}  & numeric \newline (real) & 95 & $[40;99]$ & first significance level \\
\hline
{\tt level2}  & numeric \newline (real) & 80 & $[40;99]$ & second significance level \\
\hline
{\tt predict}  & boolean  & false & false & no estimates for predictor / expectation of response \\
               &          &       & true  & estimates for the predictor and expectation are obtained \\
\hline
{\tt family}     & string  & logit     & gaussian    & Gaussian distribution with identity link \\
                 &         &           & binomial    & Binomial distribution with logit link \\
                 &         &           & binomialprobit & Binomial distribution with probit link \\
                 &         &           & poisson     & Poisson distribution with log link \\
                 &         &           & gamma       & Gamma distribution with log link \\
                 &         &           & multinomial & Multinomial distribution with logit link \\
\hline
{\tt reference}  & numeric \newline (real)  & 0     & $[-10000;10000]$    & reference category for multinomial logit models \\
\hline
\end{tabular}
{\em\caption {\label{stewpisereg_globaloptions2} Possible global options
for stepwisereg objects.}}
\end{center}
\end{table}


\begin{longtable}{p{2.2cm} p{13.3cm}}
{\tt dfmin} & Option {\tt dfmin} defines the smallest possible degree of freedom
              for a nonlinear function (besides the linear effect).
              The largest smoothing parameter is calculated according to {\tt dfmin}.
              Possible values depend on the number of regression parameters
              and on the prior distribution
              (compare section \ref{section_df}). In order to avoid numerical problems
              the smoothing parameter may not become larger than $10^9$. In this
              case, {\tt dfmin} is enlarged by ({\tt dfmax} - {\tt dfmin}) / {\tt number}
              (and {\tt number} is reduced by one).
              Additionally, this ascertains that {\tt dfmin} is redefined to
              a possible value. \\
            & \\
{\tt dfmax} & Option {\tt dfmax} defines the largest possible degree of freedom for a
              nonlinear function. The smallest smoothing parameter is calculated
              according to {\tt dfmax}. Possible values depend on the number of regression
              parameters and on the prior distribution
              (compare section \ref{section_df}). In order to avoid numerical problems
              the smoothing parameter may not become smaller than $10^{-9}$. In this
              case, {\tt dfmax} is reduced by ({\tt dfmax} - {\tt dfmin}) / {\tt number}
              (and {\tt number} is reduced by one).
              Additionally, this ascertains that {\tt dfmax} is redefined to
              a possible value. \\
            & \\
{\tt dfstart} & Option {\tt dfstart} defines the complexity of the function used in
                the base model. This option is only meaningful if
                {\tt startmodel=userdefined} is specified. In this case, the default value for
                {\tt dfstart} is either the fixed effect, if possible, or otherwise the degree of freedom nearest to one. \\
            & \\
{\tt logscale} & This option causes the smoothing parameters to lie on a logarithmic
                 scale instead of being specified according to equidistant
                 degrees of freedom. In this case, only the smallest and largest smoothing parameters
                 are calculated according to {\tt dfmin} and {\tt dfmax}. This option is only meaningful
                 if option {\tt sp} is not specified (see below). \\
            & \\
{\tt df\_accuracy} & This option specifies the maximal absolute difference in terms of
                     degrees of freedom that is allowed when calculating smoothing parameters
                     according to user--specified degrees of freedom. \\
            & \\
{\tt sp} & Option {\tt sp} causes the smoothing parameters to be chosen directly according
           to values specified by options {\tt spmin}, {\tt spmax} and {\tt spstart}.
           All other values are chosen according to a logarithmic scale. \\
            & \\
{\tt spmin} & This option defines the smallest smoothing parameter but is only valid if
              {\tt sp} is specified. \\
            & \\
{\tt spmax} & Option {\tt spmax} defines the largest smoothing parameter but is only valid if
              {\tt sp} is specified. \\
            & \\
{\tt spstart} & This option is only meaningful if {\tt startmodel=userdefined} and {\tt sp}
                are specified. It defines the smoothing parameter used for the base model. Note, that {\tt spstart}
                can not only take positive values but can also take the values {\tt spstart=0} for
                excluding the function in the base model and {\tt spstart=-1} for using the fixed effect. \\
            & \\
{\tt number} & {\tt number} specifies the number of different smoothing parameters
               (besides the linear effect and exclusion from the model). For {\tt number=0} the
               global option {\tt number} is used. \\
            & \\
{\tt forced\_into} & This option drops the possibility to exclude the function from the model. That means the respective function
                     is always included in the model. \\
            & \\
{\tt nofixed} & This option drops the possibility to use a linear fit. Hence, only possibilities
for a nonlinear effect and for the removal from the model remain. \\
            & \\
{\tt center} & {\tt center} has to be specified with varying coefficients if the
               coefficients must get centered with regard to the interacting variable, i.e., if
               there are several varying coefficients modifying the same interacting variable.
               Hence, {\tt center} is only meaningful for varying coefficients and random slopes. \\
            & \\
{\tt coding} & Option {\tt coding} is only meaningful for factor variables. It determines wether
               dummy variables ({\tt coding=dummy}) or effect variables ({\tt coding=effect})
               are used to represent the factor. \\
            & \\
{\tt reference} & Option {\tt reference} is again only meaningful for factor variables. It defines
                  the value for the reference category. \\
            & \\
{\tt degree} & Specifies the degree of B-spline basis functions. \\
            & \\
{\tt nrknots} & Specifies the number of inner knots for a P-spline term. \\
            & \\
{\tt monotone} & Option {\tt monotone} specifies additional constraints for univariate
                 P--spline terms. Possible are the estimation of an unrestricted function,
                 a monotonically increasing or decreasing function (i.e. positive/negative first derivative)
                 or a convex or concave function (i.e. positive/negative second derivative).
                 Note, that both type and direction of the constraint have to be defined by the user and are
                 not determined by the selection procedure. \\
            & \\
{\tt gridsize} & The option {\tt gridsize} can be used to restrict the
                 number of points (at the x-axis) for which estimates are computed.
                 By default, estimates are computed at every distinct covariate
                 value in the data set (indicated by {\tt gridsize=-1}). This may be
                 relatively time consuming in situations where the number of
                 distinct covariate values is large. If {\tt gridsize=nrpoints} is
                 specified, estimates are computed
                 on an equidistant grid with {\tt nrpoints} knots. \\
            & \\
{\tt period} & The period of the seasonal effect can be specified with
               option {\tt period}. The default is {\tt period=12} which corresponds
               to monthly data. \\
            & \\
{\tt map} & The map object for a spatial function is defined by option {\tt map}.
\end{longtable}

\begin{table}[ht] \footnotesize
\begin{center}
\begin{tabular}{|p{2.8cm}|p{5cm}|p{7cm}|}
\hline
{\bf Type}     & {\bf Syntax example} & {\bf Description} \\
\hline \hline
offset         & {\tt offs(offset)}  & Variable {\tt offs} is an offset term. \\
\hline
linear effect  & {\tt W1}  & Linear effect for {\tt W1}. \\
\hline
factor         & {\tt F1(factor)} & Effect of categorical variable {\tt F1} \\
\hline
first or second order random walk &   {\tt X1(rw1)} \newline {\tt X1(rw2)}  & Nonlinear effect of {\tt X1}. \\
\hline
P-spline       &  {\tt X1(psplinerw1)} \newline {\tt X1(psplinerw2)}  & Nonlinear effect of {\tt X1}.  \\
\hline
seasonal prior & {\tt time(season)} & Varying seasonal effect of {\tt time}. \\
\hline
Markov random \newline field &  {\tt region(spatial,map=m)}  & Spatial effect of {\tt region} where {\tt region} indicates the region an
observation pertains to. The boundary information and the
neighborhood structure are stored in the {\em map object}
{\tt m} . \\
\hline
Two dimensional \newline P-spline & {\tt region(geosplinerw1,map=m)} \newline {\tt region(geosplinerw2,map=m)}
& Spatial effect of {\tt region}. Estimates a two dimensional P-spline
based on the centroids of the regions. The centroids are stored in the {\em map object} {\tt m}. \\
\hline
random intercept &  {\tt grvar(random)}  & I.i.d.~Gaussian random effect of the group indicator {\tt grvar} ,
e.g.~{\tt grvar}  may be an individuum indicator when analyzing longitudinal data.  \\
\hline
\end{tabular}
{\em\caption {\label{stepwisereg_terms} Overview over different model terms
for stepwisereg objects.}}
\end{center}
\end{table}


\begin{table}[ht] \footnotesize
\begin{center}
\begin{tabular}{|p{3.6cm}|p{4.5cm}|p{6.7cm}|}
\hline
{\bf Type of interaction} & {\bf Syntax example} & {\bf Description} \\
\hline \hline
Varying coefficient term & {\tt X1*X2(rw1)} \newline {\tt X1*X2(rw2)} \newline {\tt X1*X2(psplinerw1)} \newline {\tt X1*X2(psplinerw2)}
%\newline {\tt X1*time(season)}
& Effect of {\tt X1} varies smoothly over the range of the continuous covariate {\tt X2}. \\
\hline
random slope & {\tt X1*grvar(random)}  &  The regression
coefficient of {\tt X1} varies with respect
to the unit- or cluster-index variable {\tt grvar}. \\
\hline
Geographically weighted regression & {\tt X1*region(spatial,map=m)}  & Effect of {\tt X1} varies
geographically. Covariate
{\tt region} indicates the region an observation pertains to. \\
\hline
Two dimensional surface &  {\tt X1*X2(pspline2dimrw1)} \newline {\tt X1*X2(pspline2dimrw2)}
& Two dimensional surface for the continuous
covariates {\tt X1} and {\tt X2}. \\
\hline
ANOVA type interaction &  {\tt X1*X2(psplineinteract) + } \newline {\tt X1(psplinerw?) + X2(psplinerw?)}
& ANOVA type interaction for the continuous
covariates {\tt X1} and {\tt X2}. \\
\hline

\end{tabular}
{\em\caption {\label{stepwisereg_interactions} Possible interaction
terms for stepwisereg objects.}}
\end{center}
\end{table}

 \begin{sidewaystable}[ht] \footnotesize
 \begin{center}
 \begin{tabular}{|l|l|l|l|l|}%{|p{2.5cm}|p{1.5cm}|p{2cm}|p{2cm}|p{7cm}|}
 \hline
 {\bf local option} & {\bf type} & {\bf default} & {\bf values} & {\bf meaning} \\
 \hline \hline
 {\tt dfmin}     & numeric (real) & --    & compare section \ref{section_df} & minimal degree of freedom \\
\hline
 {\tt dfmax}     & numeric (real) & --    & compare section \ref{section_df} & maximal degree of freedom \\
\hline
 {\tt dfstart}   & numeric (real) & 1     & $\{0,1\} \cup [dfmin;dfmax]$ & degree of freedom used in the base model \\
\hline
 {\tt logscale}  & boolean                 & false & false & equidistant degrees of freedom \\
                 &                         &       & true  & smoothing parameters on a logarithmic scale \\
\hline
 {\tt sp}        & boolean                 & false & false & smoothing parameters are specified in terms of $df$ \\
                 &                         &       & true  & smoothing parameters are directly specified \\
\hline
 {\tt spmin}     & numeric (real) & $10^{-4}$ & $[10^{-6};10^{8}]$ & minimal smoothing parameter \\
\hline
 {\tt spmax}     & numeric (real) & $10^4$    & $[10^{-6};10^{8}]$ & maximal smoothing parameter \\
\hline
 {\tt spstart}   & numeric (real) & --    & $\{-1,0\} \cup [10^{-6};10^{8}]$ & smoothing parameter for the base model \\
\hline
 {\tt number}    & numeric (integer) & 0  & $\{0;100\}$ & number of different smoothing parameters \\
\hline
 {\tt forced\_into} & boolean              & false & false & term may be excluded from the model \\
                 &                         &       & true  & term may not be excluded from the model \\
\hline
 {\tt nofixed}   & boolean                 & false & false & Possibility of linear fit \\
                 &                         &       & true  & A linear fit is not possible \\
\hline
 {\tt center}    & boolean                 & false & false & no centering of varying coefficient terms \\
                 &                         &       & true  &  \\
\hline
 {\tt coding}    & string                  & dummy & dummy & dummy coding of categorical variables \\
                 &                         &       & effect & effect coding of categorical variables \\
\hline
 {\tt reference} & numeric (real)    & 1  & $(-100;100)$ & reference category \\
\hline
 {\tt degree}    & numeric (integer) & 3  & $\{0;\ldots;5\}$ & degree of B--spline basis functions \\
\hline
 {\tt nrknots}   & numeric (integer) & 20 & $\{5;\ldots;500\}$ & number of inner knots for a P--spline term \\
\hline
 {\tt monotone}  & string                  & unrestricted & unrestricted & no constraint on the spline function \\
                 &                         &              & increasing   & monotonically increasing function \\
                 &                         &              & decreasing   & monotonically increasing function \\
                 &                         &              & convex       & convex function, i.e. positive second derivative \\
                 &                         &              & concave      & concave function, i.e. negative second derivative \\
\hline
 {\tt gridsize}  & numeric (integer) & -1 & $\{-1;10;\ldots;500\}$ & \\
\hline
 {\tt period}    & numeric (integer) & 12 & $\{2;\ldots;72\}$ & period for a seasonal effect \\
\hline
 {\tt map}       & {\it map object}        & --    & --                & specifies the map object used \\
 \hline
 \end{tabular}
 {\em\caption {\label{stepwisereg_localoptions} Possible local options
 for stepwisereg objects. Note, that boolean options are specified without supplying a value.}}
 \end{center}
 \end{sidewaystable}

%\begin{table}[ht] \footnotesize \centering
%\begin{tabular}{|l|p{0.6\linewidth}|c|}
%
%\hline
%optionname & description & default \\
%\hline
%\hline
%
%{\tt dfmin} & Option {\tt dfmin} defines the smallest possible degree of freedom for a nonlinear function
%(besides the linear effect). The largest smoothing parameter is then calculated according to {\tt dfmin}. & \\
%\hline
%
%{\tt dfmax} & Option {\tt dfmax} defines the largest possible degree of freedom for a nonlinear function.
%The smallest smoothing parameter is then calculated according to {\tt dfmax}. & \\
%\hline
%
%{\tt dfstart} & Option {\tt dfstart} defines the complexity of the function used in the base model.
%This option is only meaningful if {\tt startmodel=userdefined} is specified. &  \\
%\hline
%
%{\tt logscale} & This option causes the smoothing parameters to lie on a logarithmic scale instead of
%being specified according to equidistant degrees of freedom. Only the smallest and largest smoothing
%parameters are calculated according to {\tt dfmin} and {\tt dfmax}. & \\
%\hline
%
%{\tt df\_accuracy} & This option specifies the maximal absolute difference in terms of degrees of freedom allowed when
%calculating smoothing parameters according to user--specified degrees of freedom. & \\
%\hline
%
%{\tt sp} & Option {\tt sp} causes the smoothing parameters to be chosen directly according
%to values specified by options {\tt spmin}, {\tt spmax} and {\tt spstart}. All other values
%are chosen according to a logarithmic scale. & -- \\
%\hline
%
%{\tt spmin} & This option specifies the smallest smoothing parameter. & \\
%\hline
%
%{\tt spmax} & Option {\tt spmax} specifies the largest smoothing parameter. & \\
%\hline
%
%{\tt spstart} & This option is only meaningful if {\tt startmodel=userdefined} is specified. It defines
%the smoothing parameter used for the base model. & \\
%\hline
%
%{\tt number} & {\tt number} specifies the number of different smoothing parameters
%(besides the linear effect and exclusion from the model). For {\tt number=0} the
%global option {\tt number} is used. & {\tt number=20} \\
%\hline
%
%{\tt forced\_into} & This option drops the possibility to exclude the function from the model. & -- \\
%\hline
%
%{\tt nofixed} & {\tt nofixed} has to be specified with varying coefficients if the coefficients must get centered
%with regard to the interacting variable. & -- \\
%\hline
%
%{\tt coding} & Option {\tt coding} is only meaningful for factor variables. It determines wether
%dummy variables ({\tt coding=dummy}) or effect variables ({\tt coding=effect})
%are used to represent the factor. & {\tt coding=dummy} \\
%\hline
%
%{\tt reference} & Option {\tt reference} is again only meaningful for factor variables. It defines
%the value for the reference category. & \\
%\hline
%
%{\tt degree} & Specifies the degree of the B-spline basis functions. & {\tt degree=3} \\
%\hline
%
%{\tt nrknots} & Specifies the number of inner knots for a P-spline term. & {\tt nrknots=20} \\
%\hline
%
%{\tt gridsize} & The option {\tt gridsize} can be used to restrict the
%number of points (at the x-axis) for which estimates are computed.
%By default, estimates are computed at every distinct covariate
%value in the data set (indicated by {\tt gridsize=-1}). This may be
%relatively time consuming in situations where the number of
%distinct covariate values is large. If {\tt gridsize=nrpoints} is
%specified, estimates are computed
%on an equidistant grid with {\tt nrpoints} knots. & {\tt gridsize=-1} \\
%\hline
%
%%#derivative# & The option #derivative# causes that first order
%%derivatives of the estimation are computed. & - \\ \hline
%
%{\tt period} & The period of the seasonal effect can be specified with
%the option {\tt period}. The default is {\tt period=12} which corresponds
%to monthly data. & {\tt period=12} \\
%\hline
%
%\end{tabular}
%{\em\caption{\label{options} Optional arguments for stepwisereg
%object terms}}
%\end{table}


\begin{sidewaystable} \footnotesize
\begin{tabular}{|l||p{1.5cm}|p{1.5cm}|p{1.5cm}|p{1.5cm}|p{2cm}|p{1.5cm}|p{2cm}|p{2.5cm}|}

\hline
                     & factor & rw1 \newline rw2 & season & psplinerw1 \newline psplinerw2 & spatial & random & geosplinerw1 \newline geosplinerw2
                     & pspline2dimrw1 \newline pspline2dimrw2 \newline psplineinteract \\
\hline\hline
 {\tt dfmin}        & -----   & real             & real   & real                           & real    & real   & real
                    & real \\
\hline
 {\tt dfmax}        & -----   & real             & real   & real                           & real    & real   & real
                    & real \\
\hline
 {\tt dfstart}      & integer & real             & real   & real                           & real    & real   & real
                    & real \\
\hline
 {\tt logscale}     & -----   & boolean          & boolean & boolean                       & boolean & boolean & boolean
                    & boolean \\
\hline
 {\tt sp}           & boolean & boolean          & boolean & boolean                       & boolean & boolean & boolean
                    & boolean \\
\hline
 {\tt spmin}        & -----   & real             & real   & real                           & real    & real   & real
                    & real \\
\hline
 {\tt spmax}        & -----   & real             & real   & real                           & real    & real   & real
                    & real \\
\hline
 {\tt spstart}      & integer & real             & real   & real                           & real    & real   & real
                    & real \\
\hline
 {\tt number}       & -----   & integer          & integer & integer                       & integer & integer & integer
                    & integer \\
\hline
 {\tt forced\_into} & boolean & boolean          & boolean & boolean                       & boolean & boolean & boolean
                    & boolean \\
\hline
 {\tt nofixed}      & boolean & boolean          & boolean?? & boolean                       & boolean & boolean & boolean
                    & boolean \\
\hline
 {\tt center}       & -----   & boolean          & -----?? & boolean                       & boolean & boolean & -----
                    & ----- \\
\hline
 {\tt coding}       & string  & -----            & -----   & -----                         & -----   & -----   & -----
                    & ----- \\
\hline
 {\tt reference}    & real    & -----            & -----   & -----                         & -----   & -----   & -----
                    & ----- \\
\hline
 {\tt degree}       & -----   & -----            & -----   & integer                       & -----   & -----   & integer
                    & integer \\
\hline
 {\tt nrknots}      & -----   & -----            & -----   & integer                       & -----   & -----   & integer
                    & integer \\
\hline
 {\tt monotone}     & -----   & -----            & -----   & string                        & -----   & -----   & -----
                    & ----- \\
\hline
 {\tt gridsize}     & -----   & -----            & -----   & integer                       & -----   & -----   & -----
                    & integer \\
\hline
 {\tt period}       & -----   & -----            & integer & -----                         & -----   & -----   & -----
                    & ----- \\
\hline
 {\tt map}          & -----   & -----            & -----   & -----                         & {\it map object} & -----   & {\it map object}
                    & ----- \\
\hline
\end{tabular}
{\em\centering \caption{\label{termsoptions} Terms and options for
stepwisereg objects. For possible values to each of the local options compare table \ref{stepwisereg_localoptions}.
Note, that boolean options are specified without supplying a value.}}
\end{sidewaystable}



%
%
%\section{Commands for variable and smoothing parameter selection} %---------------------------------------
%
%In order to perform a variable and smoothing parameter selection in {\it BayesX}, we
%start with creating a {\it stepwisereg object} which we simply call {\tt s}:
%
%\begin{verbatim}
%> stepwisereg s
%\end{verbatim}
%
%The next step is to specify the output directory and a base filename
%for the files containing the estimation results.
%This is done via the {\tt outfile} command of {\it stepwisereg objects}:
%
%\begin{verbatim}
%> s.outfile = c:\results\car
%\end{verbatim}
%
%Now, all results files created by {\it BayesX} after the selection procedure are stored in the directory
%'{\tt c:/results}' and their names begin with the characters '{\tt car}'. If the user does not specify
%an output directory, the results files are written to the subdirectory '{\tt output}' of the
%installation directory. In this case, the name of the {\it stepwisereg object}, i.e.~'{\tt s}' in our example,
%is used as base filename. \\
%The selection is performed using the {\tt regress} command for {\it stepwisereg objects}. Its general structure is
%
%{\tt > s.regress} {\it depvar = term$_1$ + term$_2$ +} $\ldots$ {\it + term$_r$} [{\tt weight} {\it weightvar}]
%$[${\tt if} {\it expression}$]$ \\
%$[$, {\it options}$]$ {\tt using d}
%
%where {\it depvar} is the dependent variable, i.e.~the logarithmic claim size in our example, and
%term$_1$, etc.~specifies the type of function for the respective covariate.
%An intercept term is automatically included in the model and is not specified by the user.
%The part {\tt using d}
%indicates that data stored in {\it dataset object} {\tt d} is used for the selection.
%In the Belgian car insurance example we want to perform a variable and smoothing parameter selection
%using the dependent variable {\it logs}, weight variable {\it nclaims} and independent variables
%{\it ageph}, {\it bm} and {\it sex}. A simple linear model based on these variables can be selected and
%estimated by command
%
%\begin{verbatim}
%> s.regress logs = sex + ageph + bm weight nclaims,
%  criterion=AIC_imp family=gaussian using d
%\end{verbatim}
%
%But as we want to investigate wether the continuous variables
%{\it ageph} and {\it bm} possess nonlinear effects, we have to specify the semiparametric predictor
%%
%$$
%\eta = \gamma_0 + \gamma_{sex} sex + f_{ageph}(ageph) + f_{bm}(bm),
%$$
%%
%where the two nonlinear functions are represented by P--splines.
%The selection for this semiparametric predictor can be performed using the command
%
%\begin{verbatim}
%> s.regress logs = sex + ageph(psplinerw2,dfmin=2,dfmax=16,number=15) +
%                   bm(psplinerw2,dfmin=2,dfmax=16,number=15) weight nclaims,
%  criterion=AIC_imp family=gaussian using d
%\end{verbatim}
%
%For the selection, there are several global options available whose meanings are described in the following list.
%Possible values and default values are given in tables \ref{stewpisereg_globaloptions} and \ref{stewpisereg_globaloptions2}.
%
%\begin{longtable}{p{2.2cm} p{13.3cm}}
%{\tt algorithm} & specifies the selection procedure that is to be used. \\
%            & \\
%{\tt steps}     & defines the maximum number of iterations that can be used during the selection process. If the value {\tt steps}
%                   is reached before the selection process is finished, the process stops and the results of the current model
%                   are written to the results files. If that happens, a warning is written to the output window.
%                   By setting {\tt steps=0} it is possible to estimate a certain model
%                   without performing any selection. \\
%            & \\
%{\tt criterion} & specifies the selection criterion that is to be used. \\
%            & \\
%{\tt proportion} & If the selection is based on a criterion using a training and a validation data set,
%                  i.e.~on {\tt MSEP}, {\tt proportion} defines the fraction of the original data used as training data. \\
%            & \\
%{\tt startmodel} & defines the model that is used as starting model. \\
%%{\tt hierarchcal} \> hoffentlich \"{u}berfl\"{u}ssig
%            & \\
%{\tt number} & defines the number of different smoothing parameters to be used for the nonlinear terms. This number can be
%overwritten using the local option {\tt number}. \\
%            & \\
%{\tt trace} & specifies how detailed the output in the {\it output window} will be. \\
%            & \\
%{\tt CI} & specifies if confidence intervals are to be calculated. The default value is {\tt CI=none}
%so that no confidence intervals are obtained. {\tt CI=MCMCselect} yields confidence intervals which are
%estimated by MCMC techniques conditional on the selected model, i.e.~scale parameter
%and smoothing parameters are fixed on the values chosen by the preceding selection procedure.
%Unconditional confidence intervals can be obtained by {\tt CI=MCMCbootstrap} where several models
%are selected on the basis of bootstrap samples. For each of the selected models
%samples are drawn by MCMC techniques conditional on the respective model and based on the original data set.
%{\tt CI=bootstrap} yields unconditional confidence intervals by selecting many models
%on the basis of bootstrap samples. \\
%            & \\
%{\tt bootstrap-} \newline {\tt samples} & defines the number of bootstrap samples used for {\tt CI=bootstrap} or {\tt CI=MCMCbootstrap}. \\
%            & \\
%{\tt iterations} & defines the number of MCMC iterations used for {\tt CI=MCMCselect} or
%{\tt CI=MCMCbootstrap}. With {\tt CI=MCMCbootstrap}, option {\tt iterations} specifies the total number
%of iterations, i.e.~the sum of iterations used for the individual conditional MCMC estimations.
%Here, {\tt iterations} is divided equally between the individual conditional estimations so that the number of iterations
%used for one model is {\tt iterations / (bootstrap + 1)}. Hence, {\tt iterations} should be chosen appropriately.  \\
%            & \\
%{\tt step} & is a thinning parameter and specifies that only every {\tt step}--th MCMC--sample is used for the calculation of credible intervals
%with {\tt CI=MCMCselect} or {\tt CI=MCMCbootstrap}.
%Since the samples are correlated, the thinning out of MCMC samples is used to obtain
%approximately independent samples. \\
%            & \\
%{\tt burnin} & defines the number of MCMC iterations used for the burn--in phase
%at the beginning of each conditional MCMC estimation.
%Hence it is meaningful for {\tt CI=MCMCbootstrap} and {\tt CI=MCMCselect}. The burn--in phase usually
%is needed to achieve convergence of the Markov chain regarding its stationary (i.e.~the posterior) distribution.
%In our case, the initial estimates for each conditional MCMC estimation are the posterior mode estimates. That
%means, the Markov chain already starts in its stationary distribution.
%Hence, the burn--in phase usually is not needed here and it is possible to define {\tt burnin=0} what
%saves a lot of computing time. \\
%            & \\
%{\tt level1} & defines the first significance level for confidence intervals. \\
%            & \\
%{\tt level2} & defines the second significance level for confidence intervals. \\
%            & \\
%{\tt predict} & By specifying {\tt predict} an additional results file is created containing estimates for the predictor and
%                for the conditional expectation of the response variable. \\
%            & \\
%{\tt family} & specifies response distribution and link function. \\
%            & \\
%{\tt reference} & specifies the reference category for multinomial logit models.
%\end{longtable}
%
%\begin{table}[ht] \footnotesize
%\begin{center}
%\begin{tabular}{|p{2.5cm}|p{1.5cm}|p{1.5cm}|p{2.5cm}|p{6cm}|}
%\hline
%{\bf global option} & {\bf type} & {\bf default} & {\bf values} & {\bf meaning} \\
%\hline \hline
%{\tt algorithm}  & string  & cdescent1 & cdescent1 & adaptive search \\
%                 &         &           & cdescent2 & exact search \\
%                 &         &           & cdescent3 & adaptive/exact search \\
%                 &         &           & stepwise  & stepwise algorithm \\
%\hline
%{\tt steps}      & numeric \newline (integer) & 1000 & $\{0;10000\}$ & maximum number of iterations \\
%\hline
%{\tt criterion}  & string  & GCV       & GCV      & GCV (for non--Gaussian response (\ref{GCV_general2})
%                                                      based on deviance residuals)\\
%                 &         &           & GCVrss   & only meaningful for non--Gaussian response:
%                                                    GCV (\ref{GCV_general}) based on residual sum of squares \\
%                 &         &           & AIC      & AIC \\
%                 &         &           & AIC\_imp & improved AIC \\
%                 &         &           & BIC      & BIC \\
%                 &         &           & MSEP     & MSEP \\
%                 &         &           & CV5      & 5--fold cross validation \\
%                 &         &           & CV10     & 10--fold cross validation \\
%                 &         &           & AUC      & area under the ROC curve \newline
%                                                    (only for binary response) \\
%\hline
%{\tt proportion} & numeric \newline (real)    & 0.75 & $(0;1)$ & for MSEP (see description above) \\
%\hline
%{\tt startmodel} & string  & linear    & linear      & base model with degrees of freedom equal to one \\
%                 &         &           & empty       & empty model containing the intercept term only \\
%                 &         &           & full        & most complex possible model \\
%                 &         &           & userdefined & base model specified by the user; \newline
%                                                       otherwise the linear one \\
%\hline
%{\tt number}     & numeric \newline (integer) & 20   & $\{1;50\}$ & number of smoothing parameters \\
%\hline
%{\tt trace}      & string  & trace\_on & trace\_on   & output shows every new model during iterations \\
%                 &         &           & trace\_half & output shows the starting models of all iterations \\
%                 &         &           & trace\_off  & no output except starting and final model \\
%\hline
%{\tt CI}         & string  & none      & none          & no confidance intervals \\
%                 &         &           & MCMCselect    & conditional MCMC confidance bands \\
%                 &         &           & MCMCbootstrap & unconditional confidance bands based on bootstrap and MCMC \\
%                 &         &           & bootstrap     & unconditional MCMC confidance intervals based on bootstrap \\
%\hline
%{\tt bootstrap-} \newline {\tt samples}  & numeric \newline (integer) & 0 & $\{100;10000\}$ & number of bootstrap samples \\
%\hline
%\end{tabular}
%{\em\caption {\label{stewpisereg_globaloptions} Possible global options
%for stepwisereg objects.}}
%\end{center}
%\end{table}
%%
%\begin{table}[ht] \footnotesize
%\begin{center}
%\begin{tabular}{|p{2.5cm}|p{1.5cm}|p{1.5cm}|p{2cm}|p{6cm}|}
%\hline
%{\bf global option} & {\bf type} & {\bf default} & {\bf values} & {\bf meaning} \\
%\hline \hline
%{\tt iterations}  & numeric \newline (integer) & 52000 & $\{1;10000000\}$ & total number of MCMC iterations \\
%\hline
%{\tt step}  & numeric \newline (integer) & 50 & $\{1;1000\}$ & thinning parameter for MCMC samples \\
%\hline
%{\tt burnin}  & numeric \newline (integer) & 2000 & $\{0;500000\}$ & number of MCMC iterations used for each burnin phase \\
%\hline
%{\tt level1}  & numeric \newline (real) & 95 & $[40;99]$ & first significance level \\
%\hline
%{\tt level2}  & numeric \newline (real) & 80 & $[40;99]$ & second significance level \\
%\hline
%{\tt predict}  & boolean  & false & false & no estimates for predictor / expectation of response \\
%               &          &       & true  & estimates for the predictor and expectation are obtained \\
%\hline
%{\tt family}     & string  & logit     & gaussian    & Gaussian distribution with identity link \\
%                 &         &           & binomial    & Binomial distribution with logit link \\
%                 &         &           & binomialprobit & Binomial distribution with probit link \\
%                 &         &           & poisson     & Poisson distribution with log link \\
%                 &         &           & gamma       & Gamma distribution with log link \\
%                 &         &           & multinomial & Multinomial distribution with logit link \\
%\hline
%{\tt reference}  & numeric \newline (real)  & 0     & $[-10000;10000]$    & reference category for multinomial logit models \\
%\hline
%\end{tabular}
%{\em\caption {\label{stewpisereg_globaloptions2} Possible global options
%for stepwisereg objects.}}
%\end{center}
%\end{table}
%
%The commands for specifying different term types for univariate covariates are listed in table
%\ref{stepwisereg_terms}. Possibilities for interactions and the respective
%commands are shown in table \ref{stepwisereg_interactions}. For all term types
%apart from fixed effects, there are various options which are described below.
%In the following, we will refer to these options as local options (in constrast to the
%global options affecting the whole selection procedure). Possible values for the local options
%are described in table \ref{stepwisereg_localoptions} whereas table \ref{termsoptions} gives
%a short overview of possible combinations of function terms and local options.
%
%\begin{longtable}{p{2.2cm} p{13.3cm}}
%{\tt dfmin} & Option {\tt dfmin} defines the smallest possible degree of freedom
%              for a nonlinear function (besides the linear effect).
%              The largest smoothing parameter is calculated according to {\tt dfmin}.
%              Possible values depend on the number of regression parameters
%              and on the prior distribution
%              (compare section \ref{section_df}). In order to avoid numerical problems
%              the smoothing parameter may not become larger than $10^9$. In this
%              case, {\tt dfmin} is enlarged by ({\tt dfmax} - {\tt dfmin}) / {\tt number}
%              (and {\tt number} is reduced by one).
%              Additionally, this ascertains that {\tt dfmin} is redefined to
%              a possible value. \\
%            & \\
%{\tt dfmax} & Option {\tt dfmax} defines the largest possible degree of freedom for a
%              nonlinear function. The smallest smoothing parameter is calculated
%              according to {\tt dfmax}. Possible values depend on the number of regression
%              parameters and on the prior distribution
%              (compare section \ref{section_df}). In order to avoid numerical problems
%              the smoothing parameter may not become smaller than $10^{-9}$. In this
%              case, {\tt dfmax} is reduced by ({\tt dfmax} - {\tt dfmin}) / {\tt number}
%              (and {\tt number} is reduced by one).
%              Additionally, this ascertains that {\tt dfmax} is redefined to
%              a possible value. \\
%            & \\
%{\tt dfstart} & Option {\tt dfstart} defines the complexity of the function used in
%                the base model. This option is only meaningful if
%                {\tt startmodel=userdefined} is specified. In this case, the default value for
%                {\tt dfstart} is either the fixed effect, if possible, or otherwise the degree of freedom nearest to one. \\
%            & \\
%{\tt logscale} & This option causes the smoothing parameters to lie on a logarithmic
%                 scale instead of being specified according to equidistant
%                 degrees of freedom. In this case, only the smallest and largest smoothing parameters
%                 are calculated according to {\tt dfmin} and {\tt dfmax}. This option is only meaningful
%                 if option {\tt sp} is not specified (see below). \\
%            & \\
%{\tt df\_accuracy} & This option specifies the maximal absolute difference in terms of
%                     degrees of freedom that is allowed when calculating smoothing parameters
%                     according to user--specified degrees of freedom. \\
%            & \\
%{\tt sp} & Option {\tt sp} causes the smoothing parameters to be chosen directly according
%           to values specified by options {\tt spmin}, {\tt spmax} and {\tt spstart}.
%           All other values are chosen according to a logarithmic scale. \\
%            & \\
%{\tt spmin} & This option defines the smallest smoothing parameter but is only valid if
%              {\tt sp} is specified. \\
%            & \\
%{\tt spmax} & Option {\tt spmax} defines the largest smoothing parameter but is only valid if
%              {\tt sp} is specified. \\
%            & \\
%{\tt spstart} & This option is only meaningful if {\tt startmodel=userdefined} and {\tt sp}
%                are specified. It defines the smoothing parameter used for the base model. Note, that {\tt spstart}
%                can not only take positive values but can also take the values {\tt spstart=0} for
%                excluding the function in the base model and {\tt spstart=-1} for using the fixed effect. \\
%            & \\
%{\tt number} & {\tt number} specifies the number of different smoothing parameters
%               (besides the linear effect and exclusion from the model). For {\tt number=0} the
%               global option {\tt number} is used. \\
%            & \\
%{\tt forced\_into} & This option drops the possibility to exclude the function from the model. That means the respective function
%                     is always included in the model. \\
%            & \\
%{\tt nofixed} & This option drops the possibility to use a linear fit. Hence, only possibilities
%for a nonlinear effect and for the removal from the model remain. \\
%            & \\
%{\tt center} & {\tt center} has to be specified with varying coefficients if the
%               coefficients must get centered with regard to the interacting variable, i.e., if
%               there are several varying coefficients modifying the same interacting variable.
%               Hence, {\tt center} is only meaningful for varying coefficients and random slopes. \\
%            & \\
%{\tt coding} & Option {\tt coding} is only meaningful for factor variables. It determines wether
%               dummy variables ({\tt coding=dummy}) or effect variables ({\tt coding=effect})
%               are used to represent the factor. \\
%            & \\
%{\tt reference} & Option {\tt reference} is again only meaningful for factor variables. It defines
%                  the value for the reference category. \\
%            & \\
%{\tt degree} & Specifies the degree of B-spline basis functions. \\
%            & \\
%{\tt nrknots} & Specifies the number of inner knots for a P-spline term. \\
%            & \\
%{\tt monotone} & Option {\tt monotone} specifies additional constraints for univariate
%                 P--spline terms. Possible are the estimation of an unrestricted function,
%                 a monotonically increasing or decreasing function (i.e. positive/negative first derivative)
%                 or a convex or concave function (i.e. positive/negative second derivative).
%                 Note, that both type and direction of the constraint have to be defined by the user and are
%                 not determined by the selection procedure. \\
%            & \\
%{\tt gridsize} & The option {\tt gridsize} can be used to restrict the
%                 number of points (at the x-axis) for which estimates are computed.
%                 By default, estimates are computed at every distinct covariate
%                 value in the data set (indicated by {\tt gridsize=-1}). This may be
%                 relatively time consuming in situations where the number of
%                 distinct covariate values is large. If {\tt gridsize=nrpoints} is
%                 specified, estimates are computed
%                 on an equidistant grid with {\tt nrpoints} knots. \\
%            & \\
%{\tt period} & The period of the seasonal effect can be specified with
%               option {\tt period}. The default is {\tt period=12} which corresponds
%               to monthly data. \\
%            & \\
%{\tt map} & The map object for a spatial function is defined by option {\tt map}.
%\end{longtable}
%
%\begin{table}[ht] \footnotesize
%\begin{center}
%\begin{tabular}{|p{2.8cm}|p{5cm}|p{7cm}|}
%\hline
%{\bf Type}     & {\bf Syntax example} & {\bf Description} \\
%\hline \hline
%offset         & {\tt offs(offset)}  & Variable {\tt offs} is an offset term. \\
%\hline
%linear effect  & {\tt W1}  & Linear effect for {\tt W1}. \\
%\hline
%factor         & {\tt F1(factor)} & Effect of categorical variable {\tt F1} \\
%\hline
%first or second order random walk &   {\tt X1(rw1)} \newline {\tt X1(rw2)}  & Nonlinear effect of {\tt X1}. \\
%\hline
%P-spline       &  {\tt X1(psplinerw1)} \newline {\tt X1(psplinerw2)}  & Nonlinear effect of {\tt X1}.  \\
%\hline
%seasonal prior & {\tt time(season)} & Varying seasonal effect of {\tt time}. \\
%\hline
%Markov random \newline field &  {\tt region(spatial,map=m)}  & Spatial effect of {\tt region} where {\tt region} indicates the region an
%observation pertains to. The boundary information and the
%neighborhood structure are stored in the {\em map object}
%{\tt m} . \\
%\hline
%Two dimensional \newline P-spline & {\tt region(geosplinerw1,map=m)} \newline {\tt region(geosplinerw2,map=m)}
%& Spatial effect of {\tt region}. Estimates a two dimensional P-spline
%based on the centroids of the regions. The centroids are stored in the {\em map object} {\tt m}. \\
%\hline
%random intercept &  {\tt grvar(random)}  & I.i.d.~Gaussian random effect of the group indicator {\tt grvar} ,
%e.g.~{\tt grvar}  may be an individuum indicator when analyzing longitudinal data.  \\
%\hline
%\end{tabular}
%{\em\caption {\label{stepwisereg_terms} Overview over different model terms
%for stepwisereg objects.}}
%\end{center}
%\end{table}
%
%
%\begin{table}[ht] \footnotesize
%\begin{center}
%\begin{tabular}{|p{3.6cm}|p{4.5cm}|p{6.7cm}|}
%\hline
%{\bf Type of interaction} & {\bf Syntax example} & {\bf Description} \\
%\hline \hline
%Varying coefficient term & {\tt X1*X2(rw1)} \newline {\tt X1*X2(rw2)} \newline {\tt X1*X2(psplinerw1)} \newline {\tt X1*X2(psplinerw2)}
%%\newline {\tt X1*time(season)}
%& Effect of {\tt X1} varies smoothly over the range of the continuous covariate {\tt X2}. \\
%\hline
%random slope & {\tt X1*grvar(random)}  &  The regression
%coefficient of {\tt X1} varies with respect
%to the unit- or cluster-index variable {\tt grvar}. \\
%\hline
%Geographically weighted regression & {\tt X1*region(spatial,map=m)}  & Effect of {\tt X1} varies
%geographically. Covariate
%{\tt region} indicates the region an observation pertains to. \\
%\hline
%Two dimensional surface &  {\tt X1*X2(pspline2dimrw1)} \newline {\tt X1*X2(pspline2dimrw2)}
%& Two dimensional surface for the continuous
%covariates {\tt X1} and {\tt X2}. \\
%\hline
%ANOVA type interaction &  {\tt X1*X2(psplineinteract) + } \newline {\tt X1(psplinerw?) + X2(psplinerw?)}
%& ANOVA type interaction for the continuous
%covariates {\tt X1} and {\tt X2}. \\
%\hline
%
%\end{tabular}
%{\em\caption {\label{stepwisereg_interactions} Possible interaction
%terms for stepwisereg objects.}}
%\end{center}
%\end{table}
%
% \begin{sidewaystable}[ht] \footnotesize
% \begin{center}
% \begin{tabular}{|l|l|l|l|l|}%{|p{2.5cm}|p{1.5cm}|p{2cm}|p{2cm}|p{7cm}|}
% \hline
% {\bf local option} & {\bf type} & {\bf default} & {\bf values} & {\bf meaning} \\
% \hline \hline
% {\tt dfmin}     & numeric (real) & --    & compare section \ref{section_df} & minimal degree of freedom \\
%\hline
% {\tt dfmax}     & numeric (real) & --    & compare section \ref{section_df} & maximal degree of freedom \\
%\hline
% {\tt dfstart}   & numeric (real) & 1     & $\{0,1\} \cup [dfmin;dfmax]$ & degree of freedom used in the base model \\
%\hline
% {\tt logscale}  & boolean                 & false & false & equidistant degrees of freedom \\
%                 &                         &       & true  & smoothing parameters on a logarithmic scale \\
%\hline
% {\tt sp}        & boolean                 & false & false & smoothing parameters are specified in terms of $df$ \\
%                 &                         &       & true  & smoothing parameters are directly specified \\
%\hline
% {\tt spmin}     & numeric (real) & $10^{-4}$ & $[10^{-6};10^{8}]$ & minimal smoothing parameter \\
%\hline
% {\tt spmax}     & numeric (real) & $10^4$    & $[10^{-6};10^{8}]$ & maximal smoothing parameter \\
%\hline
% {\tt spstart}   & numeric (real) & --    & $\{-1,0\} \cup [10^{-6};10^{8}]$ & smoothing parameter for the base model \\
%\hline
% {\tt number}    & numeric (integer) & 0  & $\{0;100\}$ & number of different smoothing parameters \\
%\hline
% {\tt forced\_into} & boolean              & false & false & term may be excluded from the model \\
%                 &                         &       & true  & term may not be excluded from the model \\
%\hline
% {\tt nofixed}   & boolean                 & false & false & Possibility of linear fit \\
%                 &                         &       & true  & A linear fit is not possible \\
%\hline
% {\tt center}    & boolean                 & false & false & no centering of varying coefficient terms \\
%                 &                         &       & true  &  \\
%\hline
% {\tt coding}    & string                  & dummy & dummy & dummy coding of categorical variables \\
%                 &                         &       & effect & effect coding of categorical variables \\
%\hline
% {\tt reference} & numeric (real)    & 1  & $(-100;100)$ & reference category \\
%\hline
% {\tt degree}    & numeric (integer) & 3  & $\{0;\ldots;5\}$ & degree of B--spline basis functions \\
%\hline
% {\tt nrknots}   & numeric (integer) & 20 & $\{5;\ldots;500\}$ & number of inner knots for a P--spline term \\
%\hline
% {\tt monotone}  & string                  & unrestricted & unrestricted & no constraint on the spline function \\
%                 &                         &              & increasing   & monotonically increasing function \\
%                 &                         &              & decreasing   & monotonically increasing function \\
%                 &                         &              & convex       & convex function, i.e. positive second derivative \\
%                 &                         &              & concave      & concave function, i.e. negative second derivative \\
%\hline
% {\tt gridsize}  & numeric (integer) & -1 & $\{-1;10;\ldots;500\}$ & \\
%\hline
% {\tt period}    & numeric (integer) & 12 & $\{2;\ldots;72\}$ & period for a seasonal effect \\
%\hline
% {\tt map}       & {\it map object}        & --    & --                & specifies the map object used \\
% \hline
% \end{tabular}
% {\em\caption {\label{stepwisereg_localoptions} Possible local options
% for stepwisereg objects. Note, that boolean options are specified without supplying a value.}}
% \end{center}
% \end{sidewaystable}
%
%%\begin{table}[ht] \footnotesize \centering
%%\begin{tabular}{|l|p{0.6\linewidth}|c|}
%%
%%\hline
%%optionname & description & default \\
%%\hline
%%\hline
%%
%%{\tt dfmin} & Option {\tt dfmin} defines the smallest possible degree of freedom for a nonlinear function
%%(besides the linear effect). The largest smoothing parameter is then calculated according to {\tt dfmin}. & \\
%%\hline
%%
%%{\tt dfmax} & Option {\tt dfmax} defines the largest possible degree of freedom for a nonlinear function.
%%The smallest smoothing parameter is then calculated according to {\tt dfmax}. & \\
%%\hline
%%
%%{\tt dfstart} & Option {\tt dfstart} defines the complexity of the function used in the base model.
%%This option is only meaningful if {\tt startmodel=userdefined} is specified. &  \\
%%\hline
%%
%%{\tt logscale} & This option causes the smoothing parameters to lie on a logarithmic scale instead of
%%being specified according to equidistant degrees of freedom. Only the smallest and largest smoothing
%%parameters are calculated according to {\tt dfmin} and {\tt dfmax}. & \\
%%\hline
%%
%%{\tt df\_accuracy} & This option specifies the maximal absolute difference in terms of degrees of freedom allowed when
%%calculating smoothing parameters according to user--specified degrees of freedom. & \\
%%\hline
%%
%%{\tt sp} & Option {\tt sp} causes the smoothing parameters to be chosen directly according
%%to values specified by options {\tt spmin}, {\tt spmax} and {\tt spstart}. All other values
%%are chosen according to a logarithmic scale. & -- \\
%%\hline
%%
%%{\tt spmin} & This option specifies the smallest smoothing parameter. & \\
%%\hline
%%
%%{\tt spmax} & Option {\tt spmax} specifies the largest smoothing parameter. & \\
%%\hline
%%
%%{\tt spstart} & This option is only meaningful if {\tt startmodel=userdefined} is specified. It defines
%%the smoothing parameter used for the base model. & \\
%%\hline
%%
%%{\tt number} & {\tt number} specifies the number of different smoothing parameters
%%(besides the linear effect and exclusion from the model). For {\tt number=0} the
%%global option {\tt number} is used. & {\tt number=20} \\
%%\hline
%%
%%{\tt forced\_into} & This option drops the possibility to exclude the function from the model. & -- \\
%%\hline
%%
%%{\tt nofixed} & {\tt nofixed} has to be specified with varying coefficients if the coefficients must get centered
%%with regard to the interacting variable. & -- \\
%%\hline
%%
%%{\tt coding} & Option {\tt coding} is only meaningful for factor variables. It determines wether
%%dummy variables ({\tt coding=dummy}) or effect variables ({\tt coding=effect})
%%are used to represent the factor. & {\tt coding=dummy} \\
%%\hline
%%
%%{\tt reference} & Option {\tt reference} is again only meaningful for factor variables. It defines
%%the value for the reference category. & \\
%%\hline
%%
%%{\tt degree} & Specifies the degree of the B-spline basis functions. & {\tt degree=3} \\
%%\hline
%%
%%{\tt nrknots} & Specifies the number of inner knots for a P-spline term. & {\tt nrknots=20} \\
%%\hline
%%
%%{\tt gridsize} & The option {\tt gridsize} can be used to restrict the
%%number of points (at the x-axis) for which estimates are computed.
%%By default, estimates are computed at every distinct covariate
%%value in the data set (indicated by {\tt gridsize=-1}). This may be
%%relatively time consuming in situations where the number of
%%distinct covariate values is large. If {\tt gridsize=nrpoints} is
%%specified, estimates are computed
%%on an equidistant grid with {\tt nrpoints} knots. & {\tt gridsize=-1} \\
%%\hline
%%
%%%#derivative# & The option #derivative# causes that first order
%%%derivatives of the estimation are computed. & - \\ \hline
%%
%%{\tt period} & The period of the seasonal effect can be specified with
%%the option {\tt period}. The default is {\tt period=12} which corresponds
%%to monthly data. & {\tt period=12} \\
%%\hline
%%
%%\end{tabular}
%%{\em\caption{\label{options} Optional arguments for stepwisereg
%%object terms}}
%%\end{table}
%
%
%\begin{sidewaystable} \footnotesize
%\begin{tabular}{|l||p{1.5cm}|p{1.5cm}|p{1.5cm}|p{1.5cm}|p{2cm}|p{1.5cm}|p{2cm}|p{2.5cm}|}
%
%\hline
%                     & factor & rw1 \newline rw2 & season & psplinerw1 \newline psplinerw2 & spatial & random & geosplinerw1 \newline geosplinerw2
%                     & pspline2dimrw1 \newline pspline2dimrw2 \newline psplineinteract \\
%\hline\hline
% {\tt dfmin}        & -----   & real             & real   & real                           & real    & real   & real
%                    & real \\
%\hline
% {\tt dfmax}        & -----   & real             & real   & real                           & real    & real   & real
%                    & real \\
%\hline
% {\tt dfstart}      & integer & real             & real   & real                           & real    & real   & real
%                    & real \\
%\hline
% {\tt logscale}     & -----   & boolean          & boolean & boolean                       & boolean & boolean & boolean
%                    & boolean \\
%\hline
% {\tt sp}           & boolean & boolean          & boolean & boolean                       & boolean & boolean & boolean
%                    & boolean \\
%\hline
% {\tt spmin}        & -----   & real             & real   & real                           & real    & real   & real
%                    & real \\
%\hline
% {\tt spmax}        & -----   & real             & real   & real                           & real    & real   & real
%                    & real \\
%\hline
% {\tt spstart}      & integer & real             & real   & real                           & real    & real   & real
%                    & real \\
%\hline
% {\tt number}       & -----   & integer          & integer & integer                       & integer & integer & integer
%                    & integer \\
%\hline
% {\tt forced\_into} & boolean & boolean          & boolean & boolean                       & boolean & boolean & boolean
%                    & boolean \\
%\hline
% {\tt nofixed}      & boolean & boolean          & boolean?? & boolean                       & boolean & boolean & boolean
%                    & boolean \\
%\hline
% {\tt center}       & -----   & boolean          & -----?? & boolean                       & boolean & boolean & -----
%                    & ----- \\
%\hline
% {\tt coding}       & string  & -----            & -----   & -----                         & -----   & -----   & -----
%                    & ----- \\
%\hline
% {\tt reference}    & real    & -----            & -----   & -----                         & -----   & -----   & -----
%                    & ----- \\
%\hline
% {\tt degree}       & -----   & -----            & -----   & integer                       & -----   & -----   & integer
%                    & integer \\
%\hline
% {\tt nrknots}      & -----   & -----            & -----   & integer                       & -----   & -----   & integer
%                    & integer \\
%\hline
% {\tt monotone}     & -----   & -----            & -----   & string                        & -----   & -----   & -----
%                    & ----- \\
%\hline
% {\tt gridsize}     & -----   & -----            & -----   & integer                       & -----   & -----   & -----
%                    & integer \\
%\hline
% {\tt period}       & -----   & -----            & integer & -----                         & -----   & -----   & -----
%                    & ----- \\
%\hline
% {\tt map}          & -----   & -----            & -----   & -----                         & {\it map object} & -----   & {\it map object}
%                    & ----- \\
%\hline
%\end{tabular}
%{\em\centering \caption{\label{termsoptions} Terms and options for
%stepwisereg objects. For possible values to each of the local options compare table \ref{stepwisereg_localoptions}.
%Note, that boolean options are specified without supplying a value.}}
%\end{sidewaystable}
%
%Some information about the progression of the selection algorithm and some results are shown in the
%{\it output window} whereas other results are only available from external ASCII--files.
%%
%%s.regress LOGS = SEX2 +  AGEPH(psplinerw2,dfmin=2,dfmax=16,number=15) +    BM(psplinerw2,dfmin=2,dfmax=16,number=15) weight NCLAIMS,  procedure=coorddescent minimum=adaptiv startmodel=userdefined criterion=AIC_imp trace=trace_minim family=gaussian predict using d
%%
%%STEPWISE OBJECT s: stepwise procedure
%%
%%GENERAL OPTIONS:
%%
%%  Performance criterion: AIC_imp
%%  Maximal number of iterations: 1000
%%
%%  RESPONSE DISTRIBUTION:
%%
%%  Family: Gaussian
%%  Number of observations: 18139
%%
%%OPTIONS FOR STEPWISE PROCEDURE:
%%
%%  OPTIONS FOR FIXED EFFECTS TERM: SEX2
%%
%%  Prior: diffuse prior
%%  Startvalue of the 1. startmodel is the fixed effect
%%
%%  OPTIONS FOR NONPARAMETRIC TERM: AGEPH
%%
%%  Minimal value for the smoothing parameter: 2.0480375
%%  This is equivalent to degrees of freedom: approximately 16, exact 16.0369
%%  Maximal value for the smoothing parameter: 62500
%%  This is equivalent to degrees of freedom: approximately 2, exact 1.95119
%%  Number of different smoothing parameters with equidistant degrees of freedom: 15
%%  Startvalue of the 1. startmodel is the fixed effect
%%
%%  OPTIONS FOR NONPARAMETRIC TERM: BM
%%
%%  Minimal value for the smoothing parameter: 1.0240375
%%  This is equivalent to degrees of freedom: approximately 16, exact 16.0487
%%  Maximal value for the smoothing parameter: 45000
%%  This is equivalent to degrees of freedom: approximately 2, exact 2.02502
%%  Number of different smoothing parameters with equidistant degrees of freedom: 15
%%  Startvalue of the 1. startmodel is the fixed effect
%%
%%
%%STEPWISE PROCEDURE STARTED
%%
%%
%%  Startmodel:
%%
%%  LOGS = const + SEX2 + AGEPH + BM
%%  AIC_imp = 14821.315
%%
%%  SEX2
%%
%%  Lambda   Testvalue (approx):
%%       -1   14821.3
%%        0   14820.6
%%
%%
%%  Trial:
%%
%%  LOGS = const + AGEPH + BM
%%  AIC_imp = 14820.56
%%
%%  AGEPH
%%
%%  Lambda   Testvalue (approx):
%%  2.04804   14780.4
%%  4.01452   14778.6
%%  6.99482   14777.1
%%  11.5539   14775.7
%%  19.3413   14774.4
%%  32.8874   14773.2
%%  54.1414   14772.2
%%  93.4288   14771.3
%%   171.57   14770.7
%%   318.55   14770.2
%%  666.043   14770.1
%%   1392.6   14770.9
%%  3733.38   14774.5
%%  12012.3   14782.2
%%    62500   14792.9
%%       -1   14820.5
%%        0   14865.8
%%
%%
%%  Trial:
%%
%%  LOGS = const + BM + AGEPH(psplinerw2,df=5.96466,(lambda=666.043))
%%  AIC_imp = 14770.105
%%
%%  BM
%%
%%  Lambda   Testvalue (approx):
%%  1.02404   14770.1
%%  2.19761   14768.7
%%  4.12662   14767.4
%%  7.35352   14766.1
%%  12.8113   14765
%%  22.1905   14763.8
%%  38.0606   14762.6
%%  67.9218   14761.4
%%  131.282   14760.1
%%  246.389   14759
%%  503.474   14758
%%  1188.99   14757.6
%%  3091.62   14759.2
%%  9603.95   14762.8
%%    45000   14766.4
%%       -1   14768
%%        0   14853
%%
%%
%%  Trial:
%%
%%  LOGS = const + AGEPH(psplinerw2,df=5.96466,(lambda=666.043)) + BM(psplinerw2,df=4.96696,(lambda=1188.99))
%%  AIC_imp = 14757.645
%%
%%  ------------------------------------------------------------------------
%%  ------------------------------------------------------------------------
%%
%%  Startmodel:
%%
%%  LOGS = const + AGEPH(psplinerw2,df=5.96466,(lambda=666.043)) + BM(psplinerw2,df=4.96696,(lambda=1188.99))
%%  AIC_imp = 14757.645
%%
%%  SEX2
%%
%%  Lambda   Testvalue (approx):
%%       -1   14758.6
%%        0   14757.6
%%
%%
%%  AGEPH
%%
%%  Lambda   Testvalue (approx):
%%  2.04804   14768.3333106
%%  4.01452   14766.5095849
%%  6.99482   14764.9238302
%%  11.5539   14763.5164049
%%  19.3413   14762.155637
%%  32.8874   14760.8830989
%%  54.1414   14759.842513
%%  93.4288   14758.908342
%%   171.57   14758.1286187
%%   318.55   14757.6137573
%%  666.043   14757.4846875
%%   1392.6   14758.2899702
%%  3733.38   14761.9155524
%%  12012.3   14769.5821669
%%    62500   14780.4506147
%%       -1   14810.2710016
%%        0   14851.9085539
%%
%%
%%  BM
%%
%%  Lambda   Testvalue (approx):
%%  1.02404   14769.979268
%%  2.19761   14768.5680298
%%  4.12662   14767.2727716
%%  7.35352   14766.0405477
%%  12.8113   14764.8471118
%%  22.1905   14763.6699765
%%  38.0606   14762.5277055
%%  67.9218   14761.3278293
%%  131.282   14760.012711
%%  246.389   14758.8535965
%%  503.474   14757.8212049
%%  1188.99   14757.4848975
%%  3091.62   14758.9815463
%%  9603.95   14762.5502326
%%    45000   14766.1142236
%%       -1   14767.5920845
%%        0   14849.3012707
%%
%%
%%  ------------------------------------------------------------------------
%%  ------------------------------------------------------------------------
%%
%%  Startmodel:
%%
%%  LOGS = const + AGEPH(psplinerw2,df=5.96466,(lambda=666.043)) + BM(psplinerw2,df=4.96696,(lambda=1188.99))
%%  AIC_imp = 14757.485
%%
%%  SEX2
%%
%%  Lambda   Testvalue (approx):
%%       -1   14758.5
%%        0   14757.5
%%
%%
%%  AGEPH
%%
%%  Lambda   Testvalue (approx):
%%  2.04804   14768.3117913
%%  4.01452   14766.4878015
%%  6.99482   14764.9017027
%%  11.5539   14763.4938744
%%  19.3413   14762.132582
%%  32.8874   14760.8593192
%%  54.1414   14759.8178341
%%  93.4288   14758.8825054
%%   171.57   14758.1016164
%%   318.55   14757.586302
%%  666.043   14757.4586799
%%   1392.6   14758.2695722
%%  3733.38   14761.9155247
%%  12012.3   14769.625298
%%    62500   14780.5847697
%%       -1   14810.8412796
%%        0   14851.4457103
%%
%%
%%  BM
%%
%%  Lambda   Testvalue (approx):
%%  1.02404   14769.9792694
%%  2.19761   14768.5677484
%%  4.12662   14767.272393
%%  7.35352   14766.0400104
%%  12.8113   14764.8461721
%%  22.1905   14763.668301
%%  38.0606   14762.5249636
%%  67.9218   14761.3236143
%%  131.282   14760.0065583
%%  246.389   14758.8453927
%%  503.474   14757.8099719
%%  1188.99   14757.4676549
%%  3091.62   14758.954133
%%  9603.95   14762.5085125
%%    45000   14766.0512995
%%       -1   14767.50008
%%        0   14848.3747078
%%
%%
%%  ------------------------------------------------------------------------
%%  ------------------------------------------------------------------------
%%
%%  Startmodel:
%%
%%  LOGS = const + AGEPH(psplinerw2,df=5.96466,(lambda=666.043)) + BM(psplinerw2,df=4.96696,(lambda=1188.99))
%%  AIC_imp = 14757.468
%%
%%  SEX2
%%
%%  Lambda   Testvalue (approx):
%%       -1   14758.5
%%        0   14757.5
%%
%%
%%  AGEPH
%%
%%  Lambda   Testvalue (approx):
%%  2.04804   14768.3164709
%%  4.01452   14766.4924112
%%  6.99482   14764.9062226
%%  11.5539   14763.498292
%%  19.3413   14762.1368726
%%  32.8874   14760.8634425
%%  54.1414   14759.8217557
%%  93.4288   14758.8861705
%%   171.57   14758.1050229
%%   318.55   14757.5896092
%%  666.043   14757.4623533
%%   1392.6   14758.2747345
%%  3733.38   14761.9262653
%%  12012.3   14769.647904
%%    62500   14780.6317888
%%       -1   14811.0025959
%%        0   14851.3413183
%%
%%
%%  BM
%%
%%  Lambda   Testvalue (approx):
%%  1.02404   14769.9812004
%%  2.19761   14768.5696078
%%  4.12662   14767.2742295
%%  7.35352   14766.0418079
%%  12.8113   14764.8478663
%%  22.1905   14763.6698039
%%  38.0606   14762.5261883
%%  67.9218   14761.3244548
%%  131.282   14760.006895
%%  246.389   14758.8451982
%%  503.474   14757.808995
%%  1188.99   14757.4651245
%%  3091.62   14758.9489668
%%  9603.95   14762.4996258
%%    45000   14766.0368818
%%       -1   14767.4781304
%%        0   14848.1364689
%%
%%
%%
%%  There are no new models for this iteration!
%%
%%  ------------------------------------------------------------------------
%%  ------------------------------------------------------------------------
%%
%%  Final Model:
%%
%%  LOGS = const + AGEPH(psplinerw2,df=5.96466,(lambda=666.043)) + BM(psplinerw2,df=4.96696,(lambda=1188.99))
%%  AIC_imp = 14757.465
%%
%%  Used number of iterations: 4
%%
%%  ------------------------------------------------------------------------
%%  ------------------------------------------------------------------------
%%
%%  Final Model:
%%
%%  LOGS = const + AGEPH(psplinerw2,df=5.96466,(lambda=666.043)) + BM(psplinerw2,df=4.96696,(lambda=1188.99))
%%  AIC_imp = 14757.464
%%
%%RESPONSE DISTRIBUTION:
%%
%%  Gaussian
%%  Number of observations: 18139
%%
%%
%%ESTIMATION RESULTS:
%%
%%
%%  Predicted values:
%%
%%  Estimated mean of predictors, expectation of response and
%%  individual deviances are stored in file
%%  C:\bayesx\output\s_predictmean.raw
%%
%%
%%  Saturated deviance: 20058
%%
%%  Estimation results for the scale parameter:
%%
%%  sigma2:         2.03744
%%
%%
%%
%%  FixedEffects1
%%
%%
%%
%%  Variable  mean           Std. Dev.      2.5% quant.    median         97.5% quant.
%%  const     10.0619        0              0              0              0
%%
%%  Results for fixed effects are also stored in file
%%  C:\bayesx\output\s_FixedEffects1.res
%%
%%
%%  f_AGEPH
%%
%%
%%  Results are stored in file
%%  C:\bayesx\output\s_f_AGEPH_pspline.res
%%
%%  Results may be visualized using the R / S-Plus function 'plotnonp'
%%  Type for example:
%%  plotnonp("C:\\bayesx\\output\\s_f_AGEPH_pspline.res")
%%
%%
%%
%%  f_BM
%%
%%
%%  Results are stored in file
%%  C:\bayesx\output\s_f_BM_pspline.res
%%
%%  Results may be visualized using the R / S-Plus function 'plotnonp'
%%  Type for example:
%%  plotnonp("C:\\bayesx\\output\\s_f_BM_pspline.res")
%%
%%
%The {\it output window} shows all specified covariates and terms together with the respective
%number of different smoothing parameters and the way in which they were specified. Furthermore,
%even by specifying option {\tt trace=trace\_off},
%starting model and final model are shown together with the respective values of the selection criterion.
%The total number of iterations is also given in the output.
%By using the default value {\tt trace=trace\_on}, the {\it output window} additionally shows
%every model that was tried during iterations. Option {\tt trace=trace\_half} reduces the output to the starting
%models of the individual iterations. With {\tt trace=trace\_off}, the information
%given in the {\it output window} is
%
%\begin{small}
%\begin{verbatim}
%STEPWISE OBJECT s: stepwise procedure
%
%GENERAL OPTIONS:
%
% Performance criterion: AIC_imp
% Maximal number of iterations: 1000
%
% RESPONSE DISTRIBUTION:
%
% Family: Gaussian
% Number of observations: 18139
%
%OPTIONS FOR STEPWISE PROCEDURE:
%
% OPTIONS FOR FIXED EFFECTS TERM: sex
%
% Prior: diffuse prior
% Startvalue of the 1. startmodel is the fixed effect
%
% OPTIONS FOR NONPARAMETRIC TERM: ageph
%
% Minimal value for the smoothing parameter: 2.0480375
% This is equivalent to degrees of freedom: approximately 16, exact 16.0369
% Maximal value for the smoothing parameter: 62500
% This is equivalent to degrees of freedom: approximately 2, exact 1.95119
% Number of different smoothing parameters with equidistant degrees of freedom: 15
% Startvalue of the 1. startmodel is the fixed effect
%
% OPTIONS FOR NONPARAMETRIC TERM: bm
%
% Minimal value for the smoothing parameter: 1.0240375
% This is equivalent to degrees of freedom: approximately 16, exact 16.0487
% Maximal value for the smoothing parameter: 45000
% This is equivalent to degrees of freedom: approximately 2, exact 2.02502
% Number of different smoothing parameters with equidistant degrees of freedom: 15
% Startvalue of the 1. startmodel is the fixed effect
%
%
%STEPWISE PROCEDURE STARTED
%
% Startmodel:
%
% LOGS = const + sex + ageph + bm
% AIC_imp = 14821.315
%
% ------------------------------------------------------------------------
% ------------------------------------------------------------------------
%
% Final Model:
%
% LOGS = const + ageph(psplinerw2,df=5.96466,(lambda=666.043)) +
%        bm(psplinerw2,df=4.96696,(lambda=1188.99))
% AIC_imp = 14757.465
%
% Used number of iterations: 4
%\end{verbatim}
%\end{small}
%
%The estimation results are stored in several external ASCII-files whose names start with the base filename {\tt car\_}. The file
%{\tt car\_FixedEffects1.res} contains the estimated coefficients for the linear effects in tabular form, including the estimated intercept term and
%coefficients of factor variables.
%The results for linear effects are additionally shown in the {\it output window}.
%For each nonlinear function, e.g.~for $f_{ageph}(ageph)$, there exists one file in form of a data frame,
%here called {\tt car\_f\_ageph\_pspline.res},
%containing the function estimates at all distinct covariate values. The first lines of the file are:
%
%\begin{small}
%\begin{verbatim}
%intnr  ageph    pmean   pqu2p5  pqu10  pmed  pqu90  pqu97p5  pcat95  pcat80
%1       18   -0.0003835   0       0     0      0      0        0       0
%2       19   -0.0226083   0       0     0      0      0        0       0
%3       20   -0.0445334   0       0     0      0      0        0       0
%4       21   -0.0659971   0       0     0      0      0        0       0
%\end{verbatim}
%\end{small}
%
%Column {\it pmean} contains the function estimates. Columns {\it pqu2p5} to {\it pcat80} are only meaningful if
%credible intervals are constructed. In this case, columns {\it pqu2p5} and {\it pqu97p5} build the credible interval
%corresponding to {\tt level1=95}, whereas {\it pqu10} and {\it pqu90} belong to the credible interval with {\tt level2=80}. Columns
%{\it pcat95} and {\it pcat80} indicate wether the credible interval is strictly negative (-1), contains zero (0) or is strictly positive (1)
%with (posterior) probabilities of nominal levels 95\% and 80\%. The first column {\it intnr} is merely a parameter index.
%These results files can be read into any general purpose statistics program
%(e.g.~STATA, R, S-plus) to further analyse and/or visualise the results.
%The names of the respective files are shown in the {\it output window}.
%BayesX has also some facilities for the plotting of nonlinear and spatial functions. The respective
%commands {\tt plotnonp} and {\tt drwamap} are described in the manuals . \\
%Additional to the files containing estimated effects, there are two files containing information about
%the progression of the selection: the file {\tt car\_models.raw} displays the models chosen after every iteration (i.e.~after having passed
%once through all variables and terms). It looks like this:
%
%\begin{small}
%\begin{verbatim}
%step   AIC_imp   model
%
%0   14821.315     LOGS = const + sex + ageph + bm
%
%1   14757.645     LOGS = const + ageph(psplinerw2,df=5.96466,(lambda=666.043)) +
%                         bm(psplinerw2,df=4.96696,(lambda=1188.99))
%...
%3   14757.468     LOGS = const + ageph(psplinerw2,df=5.96466,(lambda=666.043)) +
%                         bm(psplinerw2,df=4.96696,(lambda=1188.99))
%
%4   14757.465
%B   14757.464
%\end{verbatim}
%\end{small}
%%
%%2   14757.485     LOGS = const + AGEPH(psplinerw2,df=5.96466,(lambda=666.043)) + BM(psplinerw2,df=4.96696,(lambda=1188.99))
%%
%In this example, variable {\it sex} has been removed from the model during the first iteration, whereas the effects of {\it ageph}
%and of {\it bm} are modelled by nonlinear effects.
%Column {\it step} shows the number of the current iteration with {\it step=0} indicating the starting model.
%The information {\it step=B} is peculiar to the adaptive search where the final model is estimated by backfitting after the
%selection process is finished what usually changes the value of the selection criterion once more.
%The largest number of {\it steps} indicates the total number of iterations.
%Using this file, it is possible to detect changes in the model that were made during an iteration. Furthermore, it is possible to
%observe changes in the selection criterion. Changes in the selection criterion can also be observed using
%file {\tt car\_criterium.raw}:
%
%\begin{small}
%\begin{verbatim}
%step  var   AIC_imp
%0      0   14821.315
%0      1   14820.56
%0      2   14770.105
%0      3   14757.645
%1      0   14757.645
%1      1   14757.645
%1      2   14757.485
%1      3   14757.485
%...
%4      0   14757.465
%B      0   14757.464
%\end{verbatim}
%\end{small}
%
%This file displays the current value of the selection criterion after the respective covariate or term was updated. Variable {\it step}
%indicates the number of iterations again whereas column {\it var} gives the number of the covariates / terms. In each iteration,
%{\it var=0} indicates the starting model. \\
%If the global option {\tt predict} is specified, BayesX creates a file {\tt car\_predictmean.raw} containing estimates
%for the predictor $\eta_i$ in column {\it linpred}
%and for the conditional expectations of the response $\mu_i$ in column {\it mu}.
%If {\tt CI=MCMCbootstrap} is specified the
%file contains the estimates for the original data in columns {\it linpred} and {\it mu}
%and, additionally, contains average estimates for $\eta_i$ and $\mu_i$ calculated from
%the samples of all selected models (columns {\it average\_linpred} and {\it average\_mu}).
%In this case the first few lines of {\tt car\_predictmean.raw} are given by
%
%\begin{small}
%\begin{verbatim}
%logs    sex ageph bm nclaims linpred average_linpred   mu   average_mu  sat_dev
%11.086   1   50    5   1      9.8551    9.85614      9.8551   9.8561    0.74395
%8.7470  -1   28    9   1      9.9052    9.90306      9.9052   9.9031    0.65828
%8.7470   1   26   11   1      10.016    10.0125      10.016   10.013    0.79044
%\end{verbatim}
%\end{small}
%
%If unconditional confidence bands were constructed by using one of the options {\tt CI=MCMCboostrap}
%or {\tt CI=bootstrap}, BayesX creates one additional results file for each nonlinear term and
%for the linear effects. Those files contain the possible degrees of freedom for the term
%together with the frequency distribution, i.e.~the number of bootstrap samples
%in which the individual degrees of freedom were selected plus the degree of freedom selected for the original data.
%For the P--spline effect of {\it agph} the file is called {\tt car\_f\_ageph\_pspline\_df.res}
%and looks like
%
%\begin{small}
%\begin{verbatim}
%df_value  sp_value  frequency  pmean
%4.04862    3447.83     3        -
%4.99322    1438.86    26        -
%6.03377    632.898    48     selected
%6.98883    325.333    14        -
%7.97817    173.922     1        -
%9.96079    56.4291     2        -
%11.033     32.1334     2        -
%11.9617    19.9828     2        -
%13.0304    11.5881     1        -
%\end{verbatim}
%\end{small}
%
%{\it BayesX} automatically creates a file {\tt car\_model\_summary.tex} summarising the most important results which can be compiled
%using \LaTeX. Among the displayed results are graphics for the nonparametric and spatial effects. These graphics are also created
%automatically and stored in postscript format.
%The effect of {\it ageph}, for example, is contained in file {\tt car\_f\_ageph\_pspline.ps}. \\
%When credible intervals are constructed by using one of the hybrid MCMC methods ({\tt CI=MCMCselect}
%or {\tt CI=MCMCbootstrap}), {\it BayesX} stores the MCMC samples for the regression parameters of linear effects
%and for the nonlinear function evaluations. These samples can be obtained using the post estimation command
%
%{\tt s.getsample}
%
%and used for an analysis of the sampling paths. For further information regarding the
%command {\tt getsample} and the analysis of MCMC output compare the BayesX manuals.
%
%
%\subsection{Specific commands for multinomial logit models}
%
%Finally, we explain the specifics of the commands for multinomial logit models:
%For multinomial logit models, there are two different commands in order to perform a variable and smoothing parameter selection.
%If the data consists of observations with merely one trial per observation, the dependent variable $Y$ is supposed to specify the
%chosen category, e.g. $Y \in \{1,\ldots,k+1\}$. In this case, a selection can be performed using the {\tt regress} command
%like for univariate response variables:
%
%{\tt > s.regress} {\it Y = term$_1$ + term$_2$ +} $\ldots$ {\it + term$_r$}
%$[${\tt if} {\it expression}$]$ $[$, {\it options}$]$ {\tt using d}
%
%Here, an important option is {\tt reference} specifying the category that is to be chosen as reference category.
%A weight variable is not allowed with {\tt regress}. \\
%The second possibility is given by the command {\tt mregress}. Here, it is possible to deal with grouped data with several trials
%per observation. In this case, the command needs $k$ response variables, e.g. $Y_1,\ldots,Y_k$, each specifying the numbers of cases
%in which the respective category was chosen. One category, here $k+1$ serves as reference. The command is
%
%\begin{tabbing}
%{\tt > s.mregress} \= {\it $Y_1$ = term$_{11}$ + term$_{12}$ +} $\ldots$ {\it + term$_{1r}$}: \\
%                   \> {\it $Y_2$ = term$_{21}$ + term$_{22}$ +} $\ldots$ {\it + term$_{2r}$}: \\
%                   \> $\vdots$ \\
%                   \> {\it $Y_k$ = term$_{k1}$ + term$_{k2}$ +} $\ldots$ {\it + term$_{kr}$} \\
%$[${\tt weight} {\it weightvar}$]$ $[${\tt if} {\it expression}$]$ $[$, {\it options}$]$ {\tt using d}
%\end{tabbing}
%
%The weight variable defines the number of trials per observation. The command {\tt mregress} assumes the same fixed effects
%for each of the categories and, regarding all other effects, it requires the same number of terms for all categories but not
%necessarily the same terms.
%The global and local options are the same as for the {\tt regress} command and local options can be individually specified
%for each term and category. \\
%With both commands, BayesX creates one results file for each nonlinear term (in every category) containing the estimated
%effects like in the univariate case. For the linear effects, there exists one results file per category containing all
%respective parameter estimates. The names of results files for the first category
%are identical to the names used for univariate response models, e.g.~{\tt s\_f\_varname\_pspline.res}
%for the P--sline effect of variable {\it varname}. For the $j$--th category with $j=2,\ldots,k$ the
%names additionally contain number $j$ and, thus, the P--spline effect of
%variable {\it varname} is stored in {\tt s\_f\_varname\_j\_pspline.res}.
%
%
%
%
%
%
%
%%
