\chapter{stepwisereg objects}
\normalsize
\label{stepwisereg} \index{Stepwisereg object}

{\em stepwisereg objects} are used to fit models with {\em structured
additive predictor} subsumed in the class of {\em structured
additive regression (STAR)} models, see Belitz and Lang (2007) and
Fahrmeir, Kneib, and Lang (2004, 2007). In contrary to {\em bayesreg} and
{\em remlreg} objects described in the previous two chapters, {\em stepwisereg}
objects are also able perform model choice and variable selection. The algorithms are able to
\begin{itemize}
\item decide whether a particular covariate enters the model,
\item decide whether a continuous covariate enters the model linearly or nonlinearly,
\item decide whether a spatial effect enters the model,
\item decide whether a unit- or cluster specific heterogeneity effect enters the model,
\item select complex interaction effects (two dimensional surfaces, varying coefficient terms),
\item select the degree of smoothness of  nonlinear covariate, spatial or cluster specific heterogeneity effects.
\end{itemize}
Inference is based on penalized likelihood in combination with fast algorithms for selection relevant covariates
and model terms. Different models are compared via various goodness of fit criteria., e.g. AIC, BIC, GCV and 5 or 10 fold cross
validation. Models with structured additive predictor are described in considerable detail in
the methodology manual. Details on the algorithms for model choice and variable selection are given in Belitz and Lang (2007) and Belitz (2007).

\index{Generalized linear models} \index{Generalized additive
models} \index{Varying coefficients} \index{Bayesian semiparametric
regression} \index{model choice} \index{variable selection}

%First steps with {\em remlreg objects} can be done with the example
%in chapter \ref*{remlregzambiaanalysis} of the tutorial manual which
%provides a self-contained demonstrating example.

\section{Method regress}
\index{Regress function}\index{Stepwisereg object!Regress
function}  \label{stepwiseregregress}

\subsection{Syntax}
\index{Regression syntax}\index{Stepwisereg object!Regression syntax}
\label{stepwiseregregresssyntax}

#># {\em objectname}.#regress# {\em model} [#weight# {\em weightvar}] [#if# {\em expression}] [{\em , options}] #using# {\em dataset}

Method #regress# estimates the regression model specified in {\em
model} using the data specified in {\em dataset}. {\em dataset} is
the name of a {\em dataset object} created before. The details
of correct model specification are covered in
\autoref{stepwiseregmodelsyntax}. The distribution of the response
variable can be either Gaussian, binomial, multinomial or Poisson.
The response distribution is specified using
option #family#, see \autoref{stepwiseregfamilysyntax} below. The
default is #family=binomial# with a logit link. An #if# statement
can be specified to analyze only parts of the data set, i.e. the
observations where {\em expression} is true.

\subsubsection{Optional weight variable}
\index{Weighted regression}
\label{stepwiseregweightspecification}

An optional weight variable {\em weightvar} can be specified to
estimate weighted regression models. For Gaussian responses, {\em
BayesX} assumes that $y_i|\eta_i,\sigma^2 \sim
N(\eta_i,\sigma^2/weightvar_i)$. Thus, for grouped Gaussian
responses the weights represent the number of observations in the
groups if the $y_i$'s are the average of individual responses. If
the $y_i$s are the sum of responses in every group, the weights have
to be the reciprocal of the number of observations in the groups. Of
course, estimation of usual weighted regression models with
heteroscedastic errors is also possible. In this case, the weights
should be proportional to the reciprocal of the heteroscedastic
variances. If the response distribution is binomial, the weight
variable should correspond to the number of replications while the
values of the response variable should represent the number of
successes. If weight is omitted, {\em BayesX} assumes that the
number of replications is one, i.e. the values of the response must
be either zero or one. For grouped Poisson data, the weights have to
specify the number of observations in a group while the $y_i$s are
assumed to be the average of individual responses. Weights are not
allowed for models with multicategorical responses.

\subsubsection{Syntax of possible model terms}
\label{stepwiseregmodelsyntax}
\index{Model terms}
\index{Stepwisereg object!Model terms}

The general syntax of models for {\em stepwisereg objects} is:

$depvar = term_1 + term_2 + \cdots + term_r$

{\em depvar} specifies the dependent variable whereas
$term_1$,\dots,$term_r$ define the form of covariate influences. The
different terms must be separated by '+' signs. A constant intercept
is automatically included in the model and has not to be
specified by the user.

This section reviews all possible model terms supported in the
current version of {\em stepwisereg objects} and provides some specific
examples. Note that all terms may be combined in arbitrary
order. An overview about the capabilities of {\em stepwisereg objects}
is given in \autoref{stepwiseregterms}. \autoref{stepwisereginteractions}
shows how interactions between covariates are specified. Full
details about all available options are given in
\autoref{stepwisereglocaloptions}.

Throughout this section Y denotes the dependent variable.

\subsubsection*{Offset}\index{Offset}

\begin{itemize}
\item[] {\em Description}: Adds an offset term to the predictor.
\item[] {\em Predictor}: $\eta =  \cdots + offs + \cdots$
\item[] {\em Syntax}:

#offs(offset)#
\item[] {\em Example}:

For instance, the following model statement can be used to estimate
a Poisson model with #offs# as offset term and #W1# and #W2# as
linear effects (if #family=poisson# is specified in addition):

\texttt{Y = offs(offset) + W1 + W2}
\end{itemize}

\subsubsection*{Linear effects}\index{Fixed effects}

\begin{itemize}
\item[] {\em Description}: Incorporates covariate #W1# as a linear effect into the model.
\item[] {\em Predictor}: $\eta =  \cdots + \gamma_1 W1 + \cdots$
\item[] {\em Syntax}:

#W1#
\item[] {\em Example}:

The following model statement causes #regress# to estimate a model
with $q$ linear effects:

\texttt{Y = W1 + W2 + $\cdots$ + Wq}
\end{itemize}

\subsubsection*{Effects of categorical covariates}
\index{Categorical covariates}



\subsubsection*{Nonlinear effects of continuous covariates and time
scales}\index{Nonlinear effects}\index{Random
walks}\index{P-splines}

\begin{itemize}
\item[] {\bf\sffamily P-spline with first or second order difference penalty}
\label{psplines_stepwise}

\item[] {\em Description}: Defines a P-spline with first or second
order difference penalty for the parameters of the spline.
\item[] {\em Predictor}: $\eta =  \cdots + f_1(X1) + \cdots$
\item[] {\em Syntax}:

#X1(psplinerw1#[, {\em options}]#)#

#X1(psplinerw2#[, {\em options}]#)#
\item[] {\em Example}:

For example, a P-spline with second order random walk penalty is
obtained using the following model statement:

#Y = X1(psplinerw2)#

By default, the degree of the spline is 3 and the number of inner
knots is 20. The following model term defines a quadratic P-spline
with 30 knots:

#Y = X1(psplinerw2,degree=2,nrknots=30)#

{\em stepwisereg objects} provide three alternatives for specifying the ordered list of
smoothing parameters of a model term:

\begin{itemize}
\item {\em Degrees of freedom:} The default is to define the list of  smoothing parameters via the concept of
equivalent degrees of freedom of a model term. Zero (covariate excluded from the model) and one (linear fit)
degrees of freedom are always included in the list. The remaining smoothing parameters in the list are chosen such that they
correspond to an equidistant grid of degrees of freedom between {\tt dfmin} and {\tt dfmax}. The total
number of smoothing parameters respectively degrees of freedom (except 0 and 1) is supplied via the option {\tt number}.
For instance,

#Y = X1(psplinerw2,dfmin=2,dfmax=5,number=4)#

defines the smoothing parameters such that they correspond to 0,1,2,3,4 and 5 degrees of freedom. Specifying

#Y = X1(psplinerw2,dfmin=2,dfmax=5,number=8)#

results in $0,1,2,2.5,3,3.5,\dots,5$ possible degrees of freedom.

The default for P-splines is {\tt dfmin=2}, {\tt dfmax=10} and {\tt number=20}. However, we strongly recommend not to rely on
the default values because reasonable values may change from application to application.
Instead, the values for {\tt dfmin} ,{\tt dfmax} and {\tt number} should always be specified by the user. In particular if the
degree of freedom of the selected best model is close to or even equal to {\tt dfmax} the selection should be rerun with an increased
value for {\tt dfmax}.
\item {\em Degrees of freedom, smoothing parameters on a log-scale:}
This alternative  also defines the list of smoothing parameters via the equivalent degrees of freedom. The difference to the
first alternative is that the corresponding smoothing parameters between {\tt dfmin} and {\tt dfmax}  are chosen on a log-scale resulting
in non-equidistant degrees of freedom. A log-scale is specified using the option {\tt logscale}. An example is the term

#Y = X1(psplinerw2,dfmin=2,dfmax=5,number=4,logscale)#

\item {\em Smoothing parameters on a log-scale:}
The third alternative allows to specify the smoothing parameters directly using the options {\tt sp}, {\tt spmin}, {\tt spmax}
and {\tt number}. The smoothing parameters are chosen between {\tt spmin} and {\tt spmax} on a log-scale. Again exclusion of the covariate from the
model as well as a linear fit is additionally included in the list of modeling alternatives. As an example consider the term

#Y = X1(psplinerw2,sp,spmin=10,spmax=10000,number=10)#

which defines a list of 10 smoothing parameters on a log-scale between 10 and 10000.
\end{itemize}
\item[]{\bf\sffamily First or second order random walk}

\item[] {\em Description}: Defines a first or second order random walk prior for the effect of #X1#.
\item[] {\em Predictor}: $\eta = \cdots + f_1(X1) + \cdots $
\item[] {\em Syntax}:

#X1(rw1#[, {\em options}]#)#

#X1(rw2#[, {\em options}]#)#
\item[] {\em Example}:

Suppose that #X1# is a continuous covariate with possibly
nonlinear effect. The following model statement defines a second
order random walk prior for $f_1$:

#Y = X1(rw2)#

The term #X1(rw2)# indicates, that the effect of
#X1# should be incorporated nonparametrically using a second order
random walk prior. A first order random walk can be requested by
specifying #X1(rw1)# instead.

The list of smoothing parameters is specified in the same way as for P-splines, see the entry above (page \pageref{psplines_stepwise}).

%default 20 verschiedene parameter
%
%degrees of freedom
%kleinster und gr\"{o}{\ss}ter, dazwischen lambdas auf logskala
%
%(logscale)
%
%degrees of freedom (default)
%kleinster und gr\"{o}{\ss}ter, abstand
%
%df_accuracy 0.05   zw 0.01 und 0.5
%
%smoothing parameter
%kleinster und gr\"{o}{\ss}ter, rest auf log skala gew\"{a}hlt
%
%sp angeben, spmin und spmax


\item[]{\bf\sffamily Seasonal component for time scales}

\item[] {\em Description}: Defines a time-varying seasonal effect
of #time#. \item[] {\em Predictor}: $\eta =  \cdots +
f_{season}(time) + \cdots $ \item[] {\em Syntax}:

#time(season#[, {\em options}]#)#
\item[] {\em Example}:

A seasonal component for a time scale #time# is specified by

#Y = time(season,period=12)#

where the second argument indicates the period of the seasonal
effect. In the example above, the period is 12 corresponding to
monthly data.

The list of smoothing parameters is specified in the same way as for P-splines, see the entry above (page \pageref{psplines_stepwise}).
\end{itemize}

\subsubsection*{Spatial Covariates}\index{Spatial effects}\index{Markov random fields}
\index{Two-dimensional P-spline}\index{Kriging}

\begin{itemize}
\item[]{\bf\sffamily Markov random field}

\item[] {\em Description}:

Defines a Markov random field prior for the spatial covariate
#region#. {\em BayesX} allows to incorporate spatial covariates
with geographical information stored in the {\em map object}
specified in option #map#.
\item[] {\em Predictor}: $\eta = \cdots
+ f_{spat}(region) + \cdots$ \item[] {\em Syntax}:

#region(spatial,map=#{\em characterstring}[, {\em options}]#)#
\item[] {\em Example}:

For the specification of a Markov random field prior, #map# is an
obligatory argument that represents the name of a {\em map object}
(see \autoref{map}) containing all necessary spatial information
about the geographical map, i.e.~the neighbors of each region and
the weights associated with the neighbors. For example the
statement

#Y = region(spatial,map=germany)#

defines a Markov random field prior for #region# where the
geographical information is stored in the {\em map object}
#germany#. An error will be raised if #germany# is not existing.

The list of smoothing parameters is specified in the same way as for P-splines, see the entry above (page \pageref{psplines_stepwise}).

\item[]{\bf\sffamily Two-dimensional P-spline with first order
random walk penalty}

\item[] {\em Description}:

Defines a two-dimensional P-spline for the spatial covariate
#region# with a two-dimensional first order random walk penalty
for the parameters of the spline. Estimation is based on the
coordinates of the centroids of the regions. The centroids are
computed using the geographical information stored in the {\em map
object} specified in the option #map#.
\item[] {\em Predictor}:
$\eta= \cdots + f(centroids) + \cdots$ \item[] {\em Syntax}:

#region(geospline,map=#{\em characterstring}[, {\em options}]#)#
\item[] {\em Example}:

For the specification of a two-dimensional P-spline ({\em
geospline}) #map# is an obligatory argument indicating the name of
a {\em map object} (see \autoref{map}) that contains all necessary
spatial information about the geographical map, i.e.~the neighbors
of each region and the weights associated with the neighbors. The
model term

#Y = region(geospline,map=germany)#

specifies a two-dimensional cubic P-spline with first order random
walk penalty where the geographical information is stored in the
{\em map object} #germany#.

The list of smoothing parameters is specified in the same way as for P-splines, see the entry above (page \pageref{psplines_stepwise}).
\end{itemize}

\subsubsection*{Unordered group indicators}\index{Unordered group
indicators}\index{Random effects}\index{Random intercept}

\begin{itemize}
\item[]{\bf\sffamily Unit- or cluster specific unstructured
effect}

\item[] {\em Description}: Defines an unstructured (uncorrelated)
random effect with respect to grouping variable #grvar#. \item[]
{\em Predictor}: $\eta = \cdots + f(grvar) + \cdots$ \item[] {\em
Syntax}:

#grvar(random#[, {\em options}]#)#
\item[] {\em Example}:

Gaussian i.i.d.~random effects allow to cope with unobserved
heterogeneity among units or clusters of observations. Suppose the
analyzed data set contains a group indicator #grvar# that gives
information about the individual or cluster a particular
observation belongs to. Then an individual-specific uncorrelated
random effect is defined by

#Y = grvar(random)#

The inclusion of more than one random effect term in the model is
possible, allowing the estimation of multilevel models. However,
we have only limited experience with multilevel models so that it
is not clear how well these models can be estimated using {\em
stepwisereg objects}.

The list of smoothing parameters is specified in the same way as for P-splines, see the entry above (page \pageref{psplines_stepwise}).
\end{itemize}

\subsubsection*{Varying coefficients with continuous covariates as
effect modifier}\index{Varying coefficients}

\begin{itemize}
\item[]{\bf\sffamily P-spline with first or second order random
walk penalty}

\item[] {\em Description}:

Defines a varying coefficient term, where the effect of #X1#
varies smoothly over the range of #X2#. The smoothness prior for
$f$ is a P-spline with first or second order random walk penalty.
\item[] {\em Predictor}: $\eta= \cdots + f(X2)X1 + \cdots$ \item[]
{\em Syntax}:

#X1*X2(psplinerw1#[, {\em options}]#)#

#X1*X2(psplinerw2#[, {\em options}]#)#
\item[] {\em Example}:

For example, a varying coefficient term with a second order random
walk smoothness prior is defined as follows:

#Y = X1*X2(psplinerw2)#

If the effect of a covariate should vary according to different
types of effect modifiers, this leads to similar identification
problems as in usual additive models. To avoid such problems,
option #center# can be specified to request the estimation of
centered effects. For example, if both #X2# and #Z2# are assumed
to modify the effect of #X1#, the specification of

#Y = X1*X2(psplinerw2) + X1*Z2(psplinerw2)#

yields a non-identifiable model. In contrast

#Y = X1 + X1*X2(psplinerw2, center) + X1*Z2(psplinerw2, center)#

is well-identified. Note that the main effect of #X1# has to be
included separately. Equivalently, we could absorb the main effect
into the first term, yielding

#Y = X1*X2(psplinerw2) + X1*Z2(psplinerw2, center)#

However, the former specification has the advantage that the model
terms are clearly separated.

Models of the type just discussed arise for example if #X1# is a
binary dummy-variable indicating two different groups of data. In
this case the model

 #Y = X1 + X2(psplinerw2) + X1*X2(psplinerw2, center) + Z2(psplinerw2) +#
 #X1*Z2(psplinerw2, center)#

assumes different effects of both #X2# and #Z2# in the groups.

The list of smoothing parameters is specified in the same way as for P-splines,
see the entry above (page \pageref{psplines_stepwise}).

\item[]{\bf\sffamily First or second order random walk}

\item[] {\em Description}:

Defines a varying coefficient term, where the effect of #X1#
varies smoothly over the range of #X2#. Therefore covariate #X2#
is called the effect modifier. The smoothness prior for $f(X2)$ is
a first or second order random walk.
\item[] {\em Predictor}:
$\eta= \cdots + f(X2)X1 + \cdots$ \item[] {\em Syntax}:

#X1*X2(rw1#[, {\em options}]#)#

#X1*X2(rw2#[, {\em options}]#)#
\item[] {\em Example}:

For example, a varying coefficient term with a second order random
walk smoothness prior is defined as follows:

#Y = X1*X2(rw2)#

The list of smoothing parameters is specified in the same way as for P-splines,
see the entry above (page \pageref{psplines_stepwise}).

\item[]{\bf\sffamily Seasonal prior}

\item[] {\em Description}:

Defines a varying coefficients term where the effect of #X1#
varies over the range of the effect modifier #time#. A seasonal
prior is assumed for the effect of #time#.

\item[] {\em Predictor}: $\eta= \cdots + f_{season}(time)X1 +
\cdots $ \item[] {\em Syntax}:

#X1*time(season#[, {\em options}]#)#
\item[] {\em Example}:

The inclusion of a varying coefficients term with a seasonal prior
may be meaningful if we expect a different seasonal effect with
respect to a binary variable #X1#. In this case we can include an
additional seasonal effects for observations with #X1#=1 by

#Y = X1*time(season) #

The list of smoothing parameters is specified in the same way as for P-splines,
see the entry above (page \pageref{psplines_stepwise}).
\end{itemize}

\subsubsection*{ Varying coefficients with spatial covariates as
effect modifiers}

\begin{itemize}
\item[]{\bf\sffamily Markov random field}

\item[] {\em Description}:

Defines a varying coefficient term where the effect of #X1# varies
smoothly over the range of the spatial covariate #region#. A
Markov random field is estimated for $f_{spat}$. The geographical
information is stored in the {\em map object} specified through the
option #map#.
\item[] {\em Predictor}: $\eta = \cdots + f_{spat}(region)X1 + \cdots$
\item[] {\em Syntax}:

#X1*region(spatial,map=#{\it characterstring} #[,# {\it options}#])#
\item[] {\em Example}:

For example the statement

#Y = X1*region(spatial,map=germany)#

defines a varying coefficient term with the spatial covariate
#region# as the effect modifier and a Markov random field as spatial
smoothness prior. Weighted Markov random fields can be estimated by
including an appropriate weight definition when creating the {\em
map object} #germany# (see \autoref{mapinfile}).

Similarly as for varying coefficient terms with continuous effect
modifiers, varying coefficients with spatial effect modifier can be
centered to avoid identifiability problems:

#Y = X1*region(spatial, map=germany, center)#

The list of smoothing parameters is specified in the same way as for P-splines, see the entry above (page \pageref{psplines_stepwise}).
\end{itemize}

\subsubsection*{Varying coefficients with unordered group indicators
as effect modifiers (random slopes)}\index{Random
effects}\index{Random slope}

\begin{itemize}
\item[]{\bf\sffamily Unit- or cluster specific unstructured
effect}

\item[] {\em Description}:

Defines a varying coefficient term where the effect of #X1# varies
over the range of the group indicator #grvar#. Models of this type
are usually referred to as models with random slopes. A Gaussian
i.i.d.~random effect with respect to grouping variable #grvar# is
assumed for $f$.
\item[] {\em Predictor}: $\eta = \cdots + f(grvar)X1 + \cdots$
\item[] {\em Syntax}:

#X1*grvar(random#[, {\em options}]#)#
\item[] {\em Example}:

For example, a random slope is specified as follows:

#Y = X1*grvar(random)#

Note, that in contrast to {\em bayesreg objects}, the main effects
are {\em not} included automatically. If main effects should be
included in the model, they have to be specified as additional
fixed effects. The syntax for obtaining the predictor

$\eta = \cdots + \gamma X1 + f(grvar)X1 + \cdots$

would be

#X1 + X1*grvar(random#[, {\em options}]#)#

The list of smoothing parameters is specified in the same way as for P-splines, see the entry above (page \pageref{psplines_stepwise}).
\end{itemize}

\subsubsection*{Surface estimators}\index{Surface
estimators}\index{Two-dimensional P-spline}\index{Kriging}

\begin{itemize}
\item[]{\bf\sffamily Two-dimensional P-spline with first order
random walk penalty}

\item[] {\em Description}:

Defines a two-dimensional P-spline based on the tensor product of
one-dimensional P-splines with a two-dimensional first order
random walk penalty for the parameters of the spline. \item[] {\em
Predictor}: $\eta= \cdots + f(X1,X2) + \cdots$ \item[] {\em
Syntax}:

#X1*X2(pspline2dimrw1#[, {\em options}]#)#
\item[] {\em Example}:

The model term

#Y = X1*X2(pspline2dimrw1)#

specifies a tensor product cubic P-spline with first order random
walk penalty.

In many applications it is favorable to additionally incorporate
the one-dimensional main effects of #X1# and #X2# into the models.
In this case the two-dimensional surface can be seen as the
deviation from the main effects. Note, that in contrast to {\em
bayesreg objects} the number of inner knots and the degree of the
spline may be different for the main effects and for the
interaction. For example, a model with 20 inner knots for the main
effects and 10 inner knots for the two-dimensional P-spline is
estimated by

 #Y = X1(psplinerw2,nrknots=20) + X2(psplinerw2,nrknots=20)#\\
 #    + X1*X2(pspline2dimrw1,nrknots=10)#
\end{itemize}


\begin{table}[ht] \footnotesize
\begin{center}
\begin{tabular}{|p{2.8cm}|p{5cm}|p{7cm}|}
\hline
{\bf Type}     & {\bf Syntax example} & {\bf Description} \\
\hline \hline
offset         & {\tt offs(offset)}  & Variable {\tt offs} is an offset term. \\
\hline
linear effect  & {\tt W1}  & Linear effect for {\tt W1}. \\
\hline
factor         & {\tt F1(factor)} & Effect of categorical variable {\tt F1} \\
\hline
first or second order random walk &   {\tt X1(rw1)} \newline {\tt X1(rw2)}  & Nonlinear effect of {\tt X1}. \\
\hline
P-spline       &  {\tt X1(psplinerw1)} \newline {\tt X1(psplinerw2)}  & Nonlinear effect of {\tt X1}.  \\
\hline
seasonal prior & {\tt time(season)} & Varying seasonal effect of {\tt time}. \\
\hline
Markov random \newline field &  {\tt region(spatial,map=m)}  & Spatial effect of {\tt region} where {\tt region} indicates the region an
observation pertains to. The boundary information and the
neighborhood structure are stored in the {\em map object}
{\tt m} . \\
\hline
Two dimensional \newline P-spline & {\tt region(geosplinerw1,map=m)} \newline {\tt region(geosplinerw2,map=m)}
& Spatial effect of {\tt region}. Estimates a two dimensional P-spline
based on the centroids of the regions. The centroids are stored in the {\em map object} {\tt m}. \\
\hline
random intercept &  {\tt grvar(random)}  & I.i.d.~Gaussian random effect of the group indicator {\tt grvar} ,
e.g.~{\tt grvar}  may be an individuum indicator when analyzing longitudinal data.  \\
\hline
\end{tabular}
{\em\caption {\label{stepwiseregterms} Overview over different model terms
for stepwisereg objects.}}
\end{center}
\end{table}


\begin{table}[ht] \footnotesize
\begin{center}
\begin{tabular}{|p{3.6cm}|p{4.5cm}|p{6.7cm}|}
\hline
{\bf Type of interaction} & {\bf Syntax example} & {\bf Description} \\
\hline \hline
Varying coefficient term & {\tt X1*X2(rw1)} \newline {\tt X1*X2(rw2)} \newline {\tt X1*X2(psplinerw1)} \newline {\tt X1*X2(psplinerw2)}
%\newline {\tt X1*time(season)}
& Effect of {\tt X1} varies smoothly over the range of the continuous covariate {\tt X2}. \\
\hline
random slope & {\tt X1*grvar(random)}  &  The regression
coefficient of {\tt X1} varies with respect
to the unit- or cluster-index variable {\tt grvar}. \\
\hline
Geographically weighted regression & {\tt X1*region(spatial,map=m)}  & Effect of {\tt X1} varies
geographically. Covariate
{\tt region} indicates the region an observation pertains to. \\
\hline
Two dimensional surface &  {\tt X1*X2(pspline2dimrw1)} \newline {\tt X1*X2(pspline2dimrw2)}
& Two dimensional surface for the continuous
covariates {\tt X1} and {\tt X2}. \\
\hline
ANOVA type interaction &  {\tt X1*X2(psplineinteract) + } \newline {\tt X1(psplinerw2) + X2(psplinerw2)}
& ANOVA type interaction for the continuous covariates {\tt X1} and {\tt X2}. Note that P-splines with first order difference
penalty for the main effects are possible.  \\
\hline

\end{tabular}
{\em\caption {\label{stepwisereginteractions} Possible interaction
terms for stepwisereg objects.}}
\end{center}
\end{table}


\subsubsection{Description of additional options for terms of {\em stepwisereg objects}}
\label{stepwisereglocaloptions}

All arguments described in this section are optional and may be
omitted. Generally, options are specified by adding the option name
to the specification of the model term type in the parentheses,
separated by commas. All options may be specified in arbitrary
order. \autoref{stepwisereg_localoptions} provides explanations and the
default values of all possible options. All reasonable combinations
of model terms and options can be found in
\autoref{stepwiseregtermsoptions}.

\begin{sidewaystable}[ht] \footnotesize
 \begin{center}
 \begin{tabular}{|l|l|l|l|l|}%{|p{2.5cm}|p{1.5cm}|p{2cm}|p{2cm}|p{7cm}|}
 \hline
 {\bf local option} & {\bf type} & {\bf default} & {\bf values} & {\bf description} \\
 \hline \hline
 {\tt dfmin}     & numeric (real) & --    & compare section \ref{section_df} & minimum degree of freedom \\
\hline
 {\tt dfmax}     & numeric (real) & --    & compare section \ref{section_df} & maximum degree of freedom \\
\hline
 {\tt dfstart}   & numeric (real) & 1     & $\{0,1\} \cup [dfmin;dfmax]$ & degree of freedom used in the start model \\
\hline
 {\tt logscale}  & boolean                 & false & false & equidistant degrees of freedom \\
                 &                         &       & true  & smoothing parameters on a logarithmic scale \\
\hline
 {\tt sp}        & boolean                 & false & false & smoothing parameters are specified in terms of $df$ \\
                 &                         &       & true  & smoothing parameters are directly specified \\
\hline
 {\tt spmin}     & numeric (real) & $10^{-4}$ & $[10^{-6};10^{8}]$ & minimum smoothing parameter \\
\hline
 {\tt spmax}     & numeric (real) & $10^4$    & $[10^{-6};10^{8}]$ & maximum smoothing parameter \\
\hline
 {\tt spstart}   & numeric (real) & --    & $\{-1,0\} \cup [10^{-6};10^{8}]$ & smoothing parameter for the start model \\
\hline
 {\tt number}    & numeric (integer) & 0  & $\{0;100\}$ & number of different smoothing parameters \\
\hline
 {\tt forced\_into} & boolean              & false & false & term may be excluded from the model \\
                 &                         &       & true  & term may not be excluded from the model \\
\hline
 {\tt nofixed}   & boolean                 & false & false & Possibility of linear fit \\
                 &                         &       & true  & A linear fit is not possible \\
\hline
 {\tt center}    & boolean                 & false & false & varying coefficient term is not centered  \\
                 &                         &       & true  & varying coefficient term is centered \\
\hline
 {\tt coding}    & string                  & dummy & dummy & dummy coding of categorical variables \\
                 &                         &       & effect & effect coding of categorical variables \\
\hline
 {\tt reference} & numeric (real)    & 1  & $(-100;100)$ & reference category \\
\hline
 {\tt degree}    & numeric (integer) & 3  & $\{0;\ldots;5\}$ & degree of B--spline basis functions \\
\hline
 {\tt nrknots}   & numeric (integer) & 20 & $\{5;\ldots;500\}$ & number of inner knots for a P--spline term \\
\hline
 {\tt monotone}  & string                  & unrestricted & unrestricted & no constraint on the spline function \\
                 &                         &              & increasing   & monotonically increasing function \\
                 &                         &              & decreasing   & monotonically increasing function \\
                 &                         &              & convex       & convex function, i.e. positive second derivative \\
                 &                         &              & concave      & concave function, i.e. negative second derivative \\
\hline
 {\tt gridsize}  & numeric (integer) & -1 & $\{-1;10;\ldots;500\}$ & \\
\hline
 {\tt period}    & numeric (integer) & 12 & $\{2;\ldots;72\}$ & period for a seasonal effect \\
\hline
 {\tt map}       & {\it map object}        & --    & --                & specifies the map object used for a spatial effect \\
 \hline
 \end{tabular}
 {\em\caption {\label{stepwisereg_localoptions} Possible local options
 for stepwisereg objects. Note, that boolean options are specified without supplying a value.}}
 \end{center}
 \end{sidewaystable}

\begin{sidewaystable} \footnotesize
\begin{tabular}{|l||p{1.5cm}|p{1.5cm}|p{1.5cm}|p{1.5cm}|p{2cm}|p{1.5cm}|p{2cm}|p{2.5cm}|}

\hline
                     & factor & rw1 \newline rw2 & season & psplinerw1 \newline psplinerw2 & spatial & random & geosplinerw1 \newline geosplinerw2
                     & pspline2dimrw1 \newline pspline2dimrw2 \newline psplineinteract \\
\hline\hline
 {\tt dfmin}        & -----   & real             & real   & real                           & real    & real   & real
                    & real \\
\hline
 {\tt dfmax}        & -----   & real             & real   & real                           & real    & real   & real
                    & real \\
\hline
 {\tt dfstart}      & integer & real             & real   & real                           & real    & real   & real
                    & real \\
\hline
 {\tt logscale}     & -----   & boolean          & boolean & boolean                       & boolean & boolean & boolean
                    & boolean \\
\hline
 {\tt sp}           & boolean & boolean          & boolean & boolean                       & boolean & boolean & boolean
                    & boolean \\
\hline
 {\tt spmin}        & -----   & real             & real   & real                           & real    & real   & real
                    & real \\
\hline
 {\tt spmax}        & -----   & real             & real   & real                           & real    & real   & real
                    & real \\
\hline
 {\tt spstart}      & integer & real             & real   & real                           & real    & real   & real
                    & real \\
\hline
 {\tt number}       & -----   & integer          & integer & integer                       & integer & integer & integer
                    & integer \\
\hline
 {\tt forced\_into} & boolean & boolean          & boolean & boolean                       & boolean & boolean & boolean
                    & boolean \\
\hline
 {\tt nofixed}      & boolean & boolean          & boolean?? & boolean                       & boolean & boolean & boolean
                    & boolean \\
\hline
 {\tt center}       & -----   & boolean          & -----?? & boolean                       & boolean & boolean & -----
                    & ----- \\
\hline
 {\tt coding}       & string  & -----            & -----   & -----                         & -----   & -----   & -----
                    & ----- \\
\hline
 {\tt reference}    & real    & -----            & -----   & -----                         & -----   & -----   & -----
                    & ----- \\
\hline
 {\tt degree}       & -----   & -----            & -----   & integer                       & -----   & -----   & integer
                    & integer \\
\hline
 {\tt nrknots}      & -----   & -----            & -----   & integer                       & -----   & -----   & integer
                    & integer \\
\hline
 {\tt monotone}     & -----   & -----            & -----   & string                        & -----   & -----   & -----
                    & ----- \\
\hline
 {\tt gridsize}     & -----   & -----            & -----   & integer                       & -----   & -----   & -----
                    & integer \\
\hline
 {\tt period}       & -----   & -----            & integer & -----                         & -----   & -----   & -----
                    & ----- \\
\hline
 {\tt map}          & -----   & -----            & -----   & -----                         & {\it map object} & -----   & {\it map object}
                    & ----- \\
\hline
\end{tabular}
{\em\centering \caption{\label{stepwiseregtermsoptions} Terms and options for
stepwisereg objects. Note, that boolean options are specified without supplying a value.}}
\end{sidewaystable}

\clearpage


\subsubsection{Specifying the response distribution}
\index{Response distribution} \label{stepwiseregfamilysyntax}

Supported univariate distributions are Gaussian, binomial (with
logit or probit link) and Poisson. For
multivariate responses, {\em stepwisereg} objects currently support only
multinomial logit models for categorical responses with unordered categories.
Continuous time survival models may be estimated by assuming a piecewise exponential model which
is estimated via an equivalent Poisson model.
An overview over
the supported models is given in \autoref{stepwiseregfamilyopt}. The
distribution of the response is specified by adding the additional
option #family# to the (global) options list of the regression call.
For instance, #family=gaussian# defines the response to be Gaussian.
In some cases, one or more additional options
associated with the specified response distribution can be
specified. An example is the #reference# option for multinomial
responses, which defines the reference category. In the following we
give details on how to specify the models.

\begin{table}[ht]
\begin{center}
\begin{tabular} {|l|p{5cm}|p{2.7cm}|p{1.7cm}|}
 \hline
 value of #family# & response distribution & link & options \\
 \hline
 \hline
 #family=gaussian#            & Gaussian              & identity & \\
 \hline
 #family=binomial#            & binomial              & logit & \\
 #family=binomialprobit#      & binomial              & probit & \\
 \hline
 #family=multinomial#         & unordered multinomial & logit & #reference#\\
 \hline
 #family=poisson#             & Poisson               & log & \\
\hline
\end{tabular}
{\em \caption {\label{stepwiseregfamilyopt} Summary of supported response distributions.}}
\end{center}
\end{table}

\subsubsection*{Gaussian responses}

For Gaussian responses {\em BayesX} assumes $y_i | \eta_i,\sigma^2
\sim N(\eta_i,\sigma^2/weightvar_i)$ or, equivalently, in matrix
notation $y | \eta, \sigma^2 \sim N(\eta,\sigma^2C^{-1})$, where
$C=diag(weightvar_1,\dots,weightvar_n)$ is a known weight matrix.
Gaussian regression models are obtained by adding

#family=gaussian#

to the options list.

An optional weight variable {\em weightvar} can be specified to
estimate weighted regression models, see
\autoref{stepwiseregweightspecification} for details. For grouped Gaussian
responses, the weights represent the number of observations in the
groups if the $y_i$'s are the average of individual responses. If
the $y_i$s are the sum of responses in every group, the weights are
given by the reciprocal of the number of observations in the groups.
Of course, estimation of usual weighted regression models with
heteroscedastic errors is also possible. In this case, the weights
should be proportional to the reciprocal of the heteroscedastic
variances. If no weight variable is specified, {\em BayesX} assumes
$weightvar_i = 1$, $i=1,\dots,n$.

\subsubsection*{Binomial logit and probit models}

A binomial logit model is requested by the option

#family=binomial#

while a probit model is obtained with

#family=binomialprobit#.

A additional weight variable may be specified, see
\autoref{stepwiseregweightspecification} for the syntax. {\em BayesX} assumes
that the weight variable corresponds to the number of replications
and the response variable to the number of successes. If the weight
variable is omitted, {\em BayesX} assumes that the number of
replications is one, i.e.~the values of the response must be either
zero or one.

\subsubsection*{Multinomial logit models}

So far, {\em stepwisereg objects} support only multinomial logit models
and no probit models. A multinomial logit model is specified by adding
the option

#family=multinomial#

to the options list. A second
option (#reference#) may be added to the options list to define the
reference category.  If the response variable has three categories
1, 2 and 3, the reference category can be set to 2, by adding

#reference=2#

to the options list. If the option is omitted, the {\em smallest}
number will be used as the reference category.


\subsubsection*{Poisson regression}

A Poisson regression model is specified by adding

#family=poisson#

to the options list.

A weight variable may be specified in addition, see
\autoref{stepwiseregweightspecification} for the syntax. For grouped Poisson
data, the weights must be the number of observations in a group and
the responses are assumed to be the average of individual responses.


\subsubsection*{Piecewise exponential model
(p.e.m.)}\index{Piecewise exponential model}

In subsection \ref*{continuoustime} of the methodology manual we
demonstrated how continuous time survival data are manipulated
to transform it to a Poisson model for estimation. Suppose that the
following modified data set is available
\vspace{0.5cm}\\
\begin{tabular}{c|c|c|c|c|c|c}
#y# & #indnr# & #a# & $\delta$ &  $\Delta$ &   #x1# &
#x#2\\\hline\hline
0 &  1 &   0.1 &   1  &  log(0.1) & 0  & 3\\
0  & 1   & 0.2  &  1  &  log(0.1) & 0 &  3\\
1  & 1   & 0.3  &  1  &  log(0.05)& 0  & 3\\\hline
0 &  2 &   0.1 &   0 &   log(0.1) & 1 &  5\\
0  & 2  &  0.2 &   0  &  log(0.02)& 1 &  5\\\hline
$\vdots$ & $\vdots$ & $\vdots$ & $\vdots$ & $\vdots$ & $\vdots$& $\vdots$\\
\end{tabular}
\vspace{0.5cm}\\
with indicator #y#, interval limit #a#, indicator of non-censoring
$\delta$ and offset $\Delta$ defined as in subsection
\ref*{continuoustime} of the methodology manual. Let #x1# be a
covariate with linear effect and #x2# a continuous covariate with
nonlinear effect. Then the correct syntax for estimating a
p.e.m.~with a {\em stepwisereg object} named #s# is e.g.~as follows:

 #> s.regress y = a(rw1) + Delta(offset) + x1 + x2(psplinerw2), family=poisson# $\ldots$

or

 #> s.regress y = a(rw2) + Delta(offset) + x1 + x2(psplinerw2), family=poisson# $\ldots$

Note that a time-varying effect of an additional covariate #X# may
be estimated by simply adding the term

#X*a(rw1) or X*a(rw2)#

to the model statement.

\subsection{Options}
\label{stepwiseregregressoptions}

\subsubsection*{Options for controlling the selection algorithms}
\label{stepwise_options_algorithm}

\begin{itemize}
\item {\tt algorithm = stringvalue} \\
Specifies the selection algorithm. Possible values are {\tt cdescent1} (adaptive algorithms 1 and 2 in the methodology manual, see ??),
{\tt cdescent2} (adaptive algorithms  1 and 2 with backfitting, see remark ?? in the methodology manual), {\tt cdescent3} (search according to
{\tt cdescent1}  followed by {\tt cdescent2}  using the selected model in the first step as the start model) and {\tt stepwise}
(stepwise algorithm implemented in the gam routine of S-plus, see Chambers and Hastie, 1991).
This option will be rarely specified by the user. \\
DEFAULT: {\tt algorithm = cdescent1}
\item {\tt criterion = stringvalue} \\
Specifies the goodness of  fit criterion. Possible values are listed in table \ref{stewpisereg_globaloptions}. If {\tt criterion = MSEP} is specified
the data are randomly divided into a test- and validation data set. The test data set is used to estimate the models and the validation data set is used to
estimate the mean squared prediction error (MSEP) which serves as the goodness of fit criterion to compare different models. The proportion of data used for
the test and validation sample can be specified using option {\tt proportion}, see below. The default is to use 75\% of the data for the training sample. \\
DEFAULT: {\tt criterion = GCV}
\item {\tt proportion = realvalue} \\
This option may be used in combination with option {\tt criterion=MSEP}, see above. In this case the data are randomly divided into
a training and a validation sample. {\tt proportion} defines the fraction (between 0 and 1)  of the original data used as training sample. \\
DEFAULT: {\tt proportion = 0.75}
\item {\tt startmodel = stringvalue} \\
Defines the start model for variable selection. Possible values are listed in table \ref{stewpisereg_globaloptions}.
DEFAULT: {\tt startmodel = empty}
\item {\tt trace=stringvalue} \\
Specifies how detailed the output in the {\it output window} will be. Possible values are given in table \ref{stewpisereg_globaloptions}. \\
DEFAULT: {\tt trace = trace\_on}
\item {\tt steps=integervalue}   \\
Defines the maximum number of iterations.  If the selection process  has not converged after after {\tt steps} iterations the algorithm terminates and a warning is
raised. Setting {\tt steps=0} allows the user to estimate a certain model without any model choice. This option will rarely be specified by the user.\\
DEFAULT: {\tt steps = 1000}
\end{itemize}


\begin{table}[ht] \footnotesize
\begin{center}
\begin{tabular}{|p{2.2cm}|p{1.3cm}|p{1.5cm}|p{1.6cm}|p{7.4cm}|}
\hline
{\bf global option} & {\bf type} & {\bf default} & {\bf values} & {\bf description} \\
\hline \hline
{\tt algorithm}  & string  & cdescent1 & cdescent1 & adaptive search \\
                 &         &           & cdescent2 & exact search \\
                 &         &           & cdescent3 & adaptive/exact search \\
                 &         &           & stepwise  & stepwise algorithm \\
\hline
{\tt criterion}  & string  & GCV       & GCV      & Generalized Cross Validation based on deviance residuals, see e.g. Wood (2006a) \\
                 &         &           & GCVrss   & Generalized Cross Validation based on residual sum of squares
                                                    (for Gaussian responses GCV and GCVrss coincide), see e.g. Wood (2006a) \\
                 &         &           & AIC      & Akaike Information Criterion, see e.g. Burnham and Anderson (1998) \\
                 &         &           & AIC\_imp & improved AIC with bias correction for regression models, see e.g. Burnham and Anderson (1998) \\
                 &         &           & BIC      & Bayesian information criterion, see e.g. Hastie et al. (2001)  \\
                 &         &           & MSEP     & Mean Squared Error Prediction \\
                 &         &           & CV5      & 5--fold cross validation, see e.g. Hastie et al. (2001)\\
                 &         &           & CV10     & 10--fold cross validation, see e.g. Hastie et al. (2001) \\
                 &         &           & AUC      & area under the ROC curve
                                                    (binary response only) \\
\hline
{\tt proportion} & numeric \newline (real)    & 0.75 & $(0;1)$ & in combination with {\tt criterion=MSEP}  (see description above) \\
\hline
{\tt startmodel} & string  & linear    & linear      & start model with degrees of freedom equal to one for model terms \\
                 &         &           & empty       & empty model containing only an intercept  \\
                 &         &           & full        & most complex possible model \\
                 &         &           & userdefined & start model is specified by the user; \newline
                                                       otherwise the linear one \\
\hline
{\tt trace}      & string  & trace\_on & trace\_on   & output shows full selection path \\
                 &         &           & trace\_half & output shows only the best model of each iteration \\
                 &         &           & trace\_off  & no output except start and final model \\
\hline
{\tt steps}      & numeric \newline (integer) & 1000 & $\{0;10000\}$ & maximum number of iterations \\
\hline
\end{tabular}
{\em\caption {\label{stewpisereg_globaloptions} Global options controlling the selection algorithm.}}
\end{center}
\end{table}


\subsubsection*{Options for computing confidence intervals}
\label{stepwise_options_ci}

\begin{itemize}
\item {\tt CI = stringvalue} \\
By default confidence intervals for linear and nonlinear terms are not computed. Option {\tt CI} allows to compute confidence intervals. {\em stepwisereg objects}
provide two alternatives for confidence interval estimation. These are
\begin{itemize}
\item confidence intervals conditional on the selected model, i.e. model uncertainty is not taken into account. The confidence intervals are
based on MCMC simulations from the posterior and pointwise (Bayesian) confidence intervals are simply obtained by computing the respective
quantiles of simulated parameters and function evaluations. Conditional confidence intervals are specified by {\tt CI = MCMCselect}.
\item unconditional confidence intervals where model uncertainty is taken into account. The computation is based on bootstrap confidence intervals
proposed by Wood (2006b), see page ?? of the methodology manual for details. Unconditional confidence intervals are specified by {\tt CI = MCMCbootstrap}.
The number of bootstrap samples is specified using option {\tt bootstrapsamples}, see below.
\end{itemize}
Both alternatives are computer intensive. Conditional confidence intervals take much less computing time than unconditional intervals. The advantage of
unconditional confidence intervals is that sampling distributions for the degrees of freedom or smoothing parameters are obtained. \\
DEFAULT: CI = none
\item {\tt bootstrapsamples = integervalue} \\
Defines the number of bootstrap samples used for {\tt CI=MCMCbootstrap}. \\
DEFAULT: bootstrapsamples=99
\item {\tt iterations=integervalue} \\
 Defines the number of MCMC iterations used for {\tt CI=MCMCselect} or
{\tt CI=MCMCbootstrap}. With {\tt CI=MCMCbootstrap}, option {\tt iterations} specifies the total number
of iterations, i.e.~the sum of iterations used for the individual conditional MCMC estimations.
The number of  {\tt iterations} are then divided equally between the individual conditional estimations so that the number of iterations
used for one model is {\tt iterations / (bootstrapsamples + 1)}. Typically 99 bootstrap samples (plus the original data set yields 100)
are used  to approximate the sampling
distribution of the smoothing parameters. If we wish for every bootstrap replication 200 samples from the posterior
{\tt iterations=20000} is required.     \\
DEFAULT: {\tt iterations=20000}
\item {\tt step=integervalue} \\
Defines the thinning parameter for MCMC simulation with {\tt CI=MCMCselect} or {\tt CI=MCMCbootstrap}. For example,
#step = 20# means, that only every 20th sampled parameter will be
stored and used to compute characteristics of the posterior
distribution. The aim of thinning is to reach a considerable reduction of disk storing and computing time.\\
DEFAULT: #step=20#
\item {\tt burnin = integervalue} \\
Defines the number of MCMC iterations used for the burn--in iterations
at the beginning of each conditional MCMC estimation.
Usually a certain number of burn--in iterations are required
to achieve convergence of the Markov chain towards its stationary (i.e.~the posterior) distribution.
In our case, the initial estimates for each conditional MCMC estimation are the posterior mode estimates, i.e.
the Markov chain already starts in its stationary distribution.
Hence,  burn--in iterations are not necessary  needed here and we can define {\tt burnin=0} which
saves considerable computing time.
Anyway, specifying this option is meaningful only in combination with {\tt CI=MCMCbootstrap} or {\tt CI=MCMCselect}. \\
DEFAULT: {\tt burnin=0}
\item \label{stepwisereglevel1} #level1 = #{\em integer} \\
By default, {\em BayesX} computes confidence intervals for
nominal levels of 80\% and 95\%. The option #level1# allows to
redefine one of the nominal levels (95\%). Adding, for instance,

#level1=99 #

to the options list leads to the computation of confidence intervals
for a nominal level of 99\% rather than 95\%. \\
DEFAULT: #level1 = #{\em 95}
\item \label{remlreglevel2} #level2 = #{\em integer} \\
By default, {\em BayesX} computes credible intervals for
nominal levels of 80\% and 95\%. The option #level2# allows to
redefine one of the nominal levels (80\%). Adding, for instance,

#level2=70#

to the options list leads to the computation of credible intervals
for a nominal level of 70\% rather than 80\%. \\
DEFAULT: #level2 = #{\em 80}
\end{itemize}


\subsubsection*{Further options}
\label{stepwisereg_further_options}

\begin{itemize}
\item #family = # {\em stringvalue} \\
 Specifies the response distribution and link function, see section \ref{stepwiseregfamilysyntax} for details. \\
DEFAULT: #family = # {\em binomial}
\item  #predict# \\
By specifying {\tt predict} an
additional file with ending #predictmean.raw# is created that contains for
every observation the estimated predictor $\hat \eta_i$ and
expectation $\hat E(y_i | \eta_i) = \hat \mu_i$ as well as the
deviance $D_i$.  If bootstrap replications are available (see option #CI# for details) model averaged estimates for $\eta_i$ and $\mu_i$
are additionally computed.
\item #reference = #{\em realvalue} \\
Option #reference# is meaningful only if  #family=multinomial# is
specified as the response distribution. In this case #reference#
defines the reference category to be chosen. Suppose, for
instance, that the response is three categorical with categories
1, 2 and 3. Then #reference=2# defines the value 2 to be the reference category. \\
DEFAULT: #reference = # 1
\end{itemize}







\begin{longtable}{p{2.2cm} p{13.3cm}}
{\tt dfmin} & Option {\tt dfmin} defines the smallest possible degree of freedom
              for a nonlinear function (besides the linear effect).
              The largest smoothing parameter is calculated according to {\tt dfmin}.
              Possible values depend on the number of regression parameters
              and on the prior distribution
              (compare section \ref{section_df}). In order to avoid numerical problems
              the smoothing parameter may not become larger than $10^9$. In this
              case, {\tt dfmin} is enlarged by ({\tt dfmax} - {\tt dfmin}) / {\tt number}
              (and {\tt number} is reduced by one).
              Additionally, this ascertains that {\tt dfmin} is redefined to
              a possible value. \\
            & \\
{\tt dfmax} & Option {\tt dfmax} defines the largest possible degree of freedom for a
              nonlinear function. The smallest smoothing parameter is calculated
              according to {\tt dfmax}. Possible values depend on the number of regression
              parameters and on the prior distribution
              (compare section \ref{section_df}). In order to avoid numerical problems
              the smoothing parameter may not become smaller than $10^{-9}$. In this
              case, {\tt dfmax} is reduced by ({\tt dfmax} - {\tt dfmin}) / {\tt number}
              (and {\tt number} is reduced by one).
              Additionally, this ascertains that {\tt dfmax} is redefined to
              a possible value. \\
{\tt df\_accuracy} & This option specifies the maximal absolute difference in terms of
                     degrees of freedom that is allowed when calculating smoothing parameters
                     according to user--specified degrees of freedom. \\
            & \\
{\tt spstart} & This option is only meaningful if {\tt startmodel=userdefined} and {\tt sp}
                are specified. It defines the smoothing parameter used for the base model. Note, that {\tt spstart}
                can not only take positive values but can also take the values {\tt spstart=0} for
                excluding the function in the base model and {\tt spstart=-1} for using the fixed effect. \\
            & \\
{\tt number} & {\tt number} specifies the number of different smoothing parameters
               (besides the linear effect and exclusion from the model). For {\tt number=0} the
               global option {\tt number} is used. \\
            & \\
{\tt forced\_into} & This option drops the possibility to exclude the function from the model. That means the respective function
                     is always included in the model. \\
            & \\
{\tt nofixed} & This option drops the possibility to use a linear fit. Hence, only possibilities
for a nonlinear effect and for the removal from the model remain. \\
            & \\
{\tt center} & {\tt center} has to be specified with varying coefficients if the
               coefficients must get centered with regard to the interacting variable, i.e., if
               there are several varying coefficients modifying the same interacting variable.
               Hence, {\tt center} is only meaningful for varying coefficients and random slopes. \\
            & \\
{\tt coding} & Option {\tt coding} is only meaningful for factor variables. It determines wether
               dummy variables ({\tt coding=dummy}) or effect variables ({\tt coding=effect})
               are used to represent the factor. \\
            & \\
{\tt reference} & Option {\tt reference} is again only meaningful for factor variables. It defines
                  the value for the reference category. \\
            & \\
{\tt degree} & Specifies the degree of B-spline basis functions. \\
            & \\
{\tt nrknots} & Specifies the number of inner knots for a P-spline term. \\
            & \\
{\tt monotone} & Option {\tt monotone} specifies additional constraints for univariate
                 P--spline terms. Possible are the estimation of an unrestricted function,
                 a monotonically increasing or decreasing function (i.e. positive/negative first derivative)
                 or a convex or concave function (i.e. positive/negative second derivative).
                 Note, that both type and direction of the constraint have to be defined by the user and are
                 not determined by the selection procedure. \\
            & \\
{\tt gridsize} & The option {\tt gridsize} can be used to restrict the
                 number of points (at the x-axis) for which estimates are computed.
                 By default, estimates are computed at every distinct covariate
                 value in the data set (indicated by {\tt gridsize=-1}). This may be
                 relatively time consuming in situations where the number of
                 distinct covariate values is large. If {\tt gridsize=nrpoints} is
                 specified, estimates are computed
                 on an equidistant grid with {\tt nrpoints} knots. \\
            & \\
{\tt period} & The period of the seasonal effect can be specified with
               option {\tt period}. The default is {\tt period=12} which corresponds
               to monthly data. \\
            & \\
{\tt map} & The map object for a spatial function is defined by option {\tt map}.
\end{longtable}


\subsection{Estimation output}

The estimation output depends on the estimated
model. Estimation results for linear effects are displayed in a
tabular form in the {\em output window} and/or in a log file (if
created before). This table always contains the estimated coefficient.
Standard deviations and  95\% confidence intervals are available only
if option #CI = # {\em MCMCbootstrap} or
#CI = # {\em MCMCselect} have been specified.
A different confidence level may be obtained by specifying the
#level1# option, see \autoref{stepwise_options_ci} for details.
Additionally, a file replicating results for the fixed effects is
created. The name of this file is supplied in the {\em output
window} and/or in a log file.

Estimated nonparametric effects are presented in a different way.
Here, results are stored in external ASCII-files that can be read
into any general purpose statistics program (e.g. STATA, R, S-plus)
to further analyze and/or visualize the results. The structure of
these files is as follows: There will be one file for every
nonparametric effect in the model. The names of the files and the
storing directory are displayed in the {\em output window} and/or a
log file. The files contain ten columns (for main effects) or eleven
columns (for interaction effects). The first column contains a
parameter index (starting with one), the second column (and the
third column if the estimated effect is an interaction) contain the
values of the covariate(s) whose effect has been estimated. In the
following columns the estimation results are given in form of the point estimate,
the lower boundaries of the  95\% and
80\% credible intervals, the standard deviation and the upper
boundaries of the 80\% and 95\% credible intervals. The last two
columns contain approximations to the posterior probabilities based
on nominal levels of 95\% and 80\%. A value of 1 corresponds to a
strictly positive 95\% or 80\% credible interval while a value of -1
to a strictly negative credible interval. A value of 0 indicates
that the corresponding credible interval contains zero. Other
credible intervals and posterior probabilities may be obtained by
specifying the #level1# and/or #level2# option, see
\autoref{stepwise_options_ci} for details. As an example, compare
the following lines, which are the beginning of a file containing
the results for a nonparametric effect of a particular covariate, x
say:

\footnotesize
 intnr \,\, x \,\, pmean \,\, pqu2p5  \,\, pqu10 \,\, pmed \,\, pqu90 \,\, pq97p5 \,\, pcat95 \,\, pcat80\\
 1 \,\, -2.87694 \,\, -0.307921 \,\, -0.886815 \,\, -0.686408 \,\, 0.295295 \,\, 0.070567   \,\, 0.270973 \,\, 0 \,\, 0\\
 2 \,\, -2.86203 \,\, -0.320479 \,\, -0.885375 \,\, -0.689815 \,\, 0.288154 \,\, 0.0488558  \,\, 0.244416 \,\, 0 \,\, 0\\
 3 \,\, -2.8515  \,\, -0.329367 \,\, -0.88473  \,\, -0.69247  \,\, 0.283292 \,\, 0.0337362  \,\, 0.225997 \,\, 0 \,\, 0\\
 4 \,\, -2.85066 \,\, -0.330072 \,\, -0.884692 \,\, -0.692689 \,\, 0.282913 \,\, 0.0325457  \,\, 0.224549 \,\, 0 \,\, 0\\
 5 \,\, -2.82295 \,\, -0.3535   \,\, -0.884544 \,\, -0.700703 \,\, 0.270887 \,\,-0.00629671 \,\, 0.177545 \,\, 0 \,\, -1\\
 6 \,\, -2.79856 \,\, -0.37418  \,\, -0.886192 \,\, -0.708939 \,\, 0.261178 \,\,-0.0394208  \,\, 0.137832 \,\, 0 \,\, -1\\
 7 \,\, -2.79492 \,\, -0.377272 \,\, -0.886579 \,\, -0.710263 \,\, 0.259798 \,\,-0.0442813  \,\, 0.132035 \,\, 0 \,\, -1\\
 8 \,\, -2.79195 \,\, -0.379788 \,\, -0.886921 \,\, -0.711358 \,\, 0.258689 \,\,-0.0482183  \,\, 0.127345 \,\, 0 \,\, -1\\
 9 \,\, -2.78837 \,\, -0.382834 \,\, -0.887367 \,\, -0.712704 \,\, 0.257363 \,\,-0.0529641  \,\, 0.1217   \,\, 0 \,\, -1
\normalsize

Note that credible intervals, standard deviations etc. are available only
if option #CI = # {\em MCMCbootstrap} or
#CI = # {\em MCMCselect} have been specified.

The estimated nonlinear effects can be visualized using either the
graphics capabilities of {\em BayesX} or a couple of R and S-plus
functions,  see \autoref{bayesxplot} and \autoref{splus},
respectively. Of course, any other (statistics) software package
with plotting facilities can be used as well.

Estimation results for the variances and the smoothing parameters
of nonparametric effects are printed in the {\em output window}
and/or a log file. Additionally, a file is created containing the
same information. For example, the file corresponding to the
nonparametric effect presented above contains:

\footnotesize
 variance \,\, smoothpar \,\, stopped\\
 0.0492324 \,\, 20.3118 \,\, 0
\normalsize

The value in the last row indicates whether the estimation of the
variance has been stopped before convergence. A value of 1
corresponds to a 'stopped' variance.

\subsection{Examples}

Here we give only a few examples about the usage of method
#regress#. A more detailed, tutorial like example can be found in
chapter \ref*{remlregzambiaanalysis} of the tutorial manual.

Suppose that we have a data set #test# with a binary response
variable #y#, and covariates #x1#, #x2#, #x3#, #t# and #region#,
where #t# is assumed to be a time scale measured in months and
#region# indicates the geographical region an observation belongs
to. Suppose further that we have already created a {\em remlreg
object} #r#.

\subsubsection*{Fixed effects}

We first specify a model with #y# as the response variable and
fixed effects for the covariates #x1#, #x2# and #x3#. Hence the
predictor is

$$
\eta = \gamma_0 + \gamma_1 x1 + \gamma_2 x2 + \gamma_3 x3
$$

This model is estimated by typing:

#> r.regress y = x1 + x2 + x3, family=binomial using test#

By specifying option #family=binomial#, a binomial logit model is
estimated. A probit model can be obtained by specifying
#family=binomialprobit#.

\subsubsection*{Additive models}

Suppose now that we want to allow for possibly nonlinear effects
of #x2# and #x3#. Defining cubic P-splines with second order
random walk penalty as smoothness priors, we obtain

 #> r.regress y = x1 + x2(psplinerw2) + x3(psplinerw2), family=binomial using test#

which corresponds to the predictor

$$
\eta = \gamma_0 + \gamma_1 x1 + f_1(x2) + f_2(x3).
$$

If the response is not binary but categorical with unordered
categories 1, 2 and 3, we can estimate a multinomial logit model by
typing:

 #> r.regress y = x1 + x2(psplinerw2) + x3(psplinerw2), family=multinomial#\\
 #  reference=2 using test#

In this case, #family=binomial# has to be altered to
#family=multinomial#, and the option #reference=2# was added to
define the value 2 as the reference category.

\subsubsection*{Time scales}

In the next step we extend the model by incorporating an additional
trend and a flexible seasonal component for the time scale #t#:

 #> r.regress y = x1 + x2(psplinerw2) + x3(psplinerw2) +  #\\
 #  t(psplinerw2) + t(season,period=12), family=binomial using test#

Note that we passed the period of the seasonal component as a
second argument.

\subsubsection*{Spatial covariates}

To incorporate a structured spatial effect, we have to create a {\em
map object} first. Afterwards we read the boundary information of
the different regions (polygons that form the regions, neighbors
etc.). If you are unfamiliar with {\em map objects} please read
\autoref{map} first.

#> map m# \\
#> m.infile using c:\maps\map.bnd#

Since we usually need the map again in further sessions, we store
it in {\em graph file} format, because reading {\em graph files}
is much faster than reading {\em boundary files}.

#> m.outfile , graph using c:\maps\mapgraph.gra#

We can now augment our predictor with a spatial effect:

 #> r.regress y = x1 + x2(psplinerw2) + x3(psplinerw2) + t(psplinerw2)#\\
 #  + t(season,period=12) + region(spatial,map=m), family=binomial using test#

In some situations it may be reasonable to incorporate  an
additional unstructured  random effect into the model in order to
split the total spatial effect into a structured and an unstructured
component. This is achieved by

#> r.regress y = x1 + x2(psplinerw2) + x3(psplinerw2) + t(psplinerw2)#\\
#  + t(season,period=12) + region(spatial,map=m) + region(random),#\\
#  family=binomial using test#

\section{Global options}
\label{stepwiseregglobopt} \index{Stepwisereg object!Global options}

The purpose of global options is to affect the global behavior of
a {\em stepwisereg object}. The main characteristic of global options
is, that they are not associated with a certain method.

The syntax for specifying global options is

{\em objectname}.{\em optionname} = {\em newvalue}

where {\em newvalue} is the new value of the option. The type of
the value depends on the respective option.

Currently only one global option is available for {\em stepwisereg
objects}:

\begin{itemize}
\item #outfile = #{\em filename} \\
By default, the estimation output produced by the #regress#
procedure will be written to the default output directory, which
is

{\em$<$INSTALLDIRECTORY$>$}#\output#

The default file name is composed of the name of the {\em remlreg
object} and the type of the file. For example, if you estimated a
nonparametric effect for a covariate #X#, say, using a P-spline,
then the estimation output will be written to

{\em$<$INSTALLDIRECTORY$>$}#\output\r_f_X_pspline.res#

where #r# is the name of the {\em stepwisereg object}. In most cases,
however, it may be necessary to save estimation results into a
different directory and/or under a different file name than the
default. This can be achieved using the #outfile# option. Here, you
have to specify the directory where the output should be stored and
a base file name. This base file name should not be a complete file
name. For example specifying

#outfile = c:\data\res1#

would cause {\em BayesX} to store the estimation result for the
nonparametric effect of #X# in file

#c:\data\res1_f_X_pspline.res#
\end{itemize}



\section{Visualizing estimation results}

Visualization of estimation results is described in
\autoref{visualization}

\section{References}
\label{stepwiseregreferences}

\begin{description}
\item[Belitz, C. (2007):] {\tt Model Selection in Generalized Structured Additive Regression Models.} PhD Thesis, University of Munich.

\item [Belitz, C. and Lang, S. (2007):] Simultaneous selection of variables and smoothing
parameters in structured additive regression models. {\it Technical Report} University of Innsbruck.

\item[Burnham, K.P. and Anderson, D.R., (1998):] {\it Model Selection and Multimodel Inference.} Springer, New York.

\item[Chambers, J.M. and Hastie, T. (1991):] {\it Statistical Models in S.} Chapman and Hall.

\item[Fahrmeir, L., Kneib, T. and Lang, S. (2004):] Penalized
structured additive regression for space-time data: A Bayesian
perspective. {\it Statistica Sinica}, 14, 715-745.

\item[Fahrmeir, L., Kneib, T. and Lang, S. (2007):] {\it Regression. Modelle, Methoden und Anwendungen.} Springer Verlag, Berlin.

\item[Hastie, T.J. and Tibshirani, R.J. and Friedman, J. (2001):] The Elements of Statistical Learning. Springer, New York.

\item[Wood, S.N. (2006a):] {\it Generalized Additive Models: An Introduction with R.} Chapman and Hall.

\item[Wood, S.N. (2006b):] {\it On confidence intervals for GAMs based on penalized regression splines.}
{\it Australian and New Zealand Journal of Statistics}, 48, 445-464.
\end{description}





%\begin{table}[ht] \footnotesize \centering
%\begin{tabular}{|l|p{0.6\linewidth}|c|}
%
%\hline
%optionname & description & default \\
%\hline
%\hline
%
%{\tt dfmin} & Option {\tt dfmin} defines the smallest possible degree of freedom for a nonlinear function
%(besides the linear effect). The largest smoothing parameter is then calculated according to {\tt dfmin}. & \\
%\hline
%
%{\tt dfmax} & Option {\tt dfmax} defines the largest possible degree of freedom for a nonlinear function.
%The smallest smoothing parameter is then calculated according to {\tt dfmax}. & \\
%\hline
%
%{\tt dfstart} & Option {\tt dfstart} defines the complexity of the function used in the base model.
%This option is only meaningful if {\tt startmodel=userdefined} is specified. &  \\
%\hline
%
%{\tt logscale} & This option causes the smoothing parameters to lie on a logarithmic scale instead of
%being specified according to equidistant degrees of freedom. Only the smallest and largest smoothing
%parameters are calculated according to {\tt dfmin} and {\tt dfmax}. & \\
%\hline
%
%{\tt df\_accuracy} & This option specifies the maximal absolute difference in terms of degrees of freedom allowed when
%calculating smoothing parameters according to user--specified degrees of freedom. & \\
%\hline
%
%{\tt sp} & Option {\tt sp} causes the smoothing parameters to be chosen directly according
%to values specified by options {\tt spmin}, {\tt spmax} and {\tt spstart}. All other values
%are chosen according to a logarithmic scale. & -- \\
%\hline
%
%{\tt spmin} & This option specifies the smallest smoothing parameter. & \\
%\hline
%
%{\tt spmax} & Option {\tt spmax} specifies the largest smoothing parameter. & \\
%\hline
%
%{\tt spstart} & This option is only meaningful if {\tt startmodel=userdefined} is specified. It defines
%the smoothing parameter used for the base model. & \\
%\hline
%
%{\tt number} & {\tt number} specifies the number of different smoothing parameters
%(besides the linear effect and exclusion from the model). For {\tt number=0} the
%global option {\tt number} is used. & {\tt number=20} \\
%\hline
%
%{\tt forced\_into} & This option drops the possibility to exclude the function from the model. & -- \\
%\hline
%
%{\tt nofixed} & {\tt nofixed} has to be specified with varying coefficients if the coefficients must get centered
%with regard to the interacting variable. & -- \\
%\hline
%
%{\tt coding} & Option {\tt coding} is only meaningful for factor variables. It determines wether
%dummy variables ({\tt coding=dummy}) or effect variables ({\tt coding=effect})
%are used to represent the factor. & {\tt coding=dummy} \\
%\hline
%
%{\tt reference} & Option {\tt reference} is again only meaningful for factor variables. It defines
%the value for the reference category. & \\
%\hline
%
%{\tt degree} & Specifies the degree of the B-spline basis functions. & {\tt degree=3} \\
%\hline
%
%{\tt nrknots} & Specifies the number of inner knots for a P-spline term. & {\tt nrknots=20} \\
%\hline
%
%{\tt gridsize} & The option {\tt gridsize} can be used to restrict the
%number of points (at the x-axis) for which estimates are computed.
%By default, estimates are computed at every distinct covariate
%value in the data set (indicated by {\tt gridsize=-1}). This may be
%relatively time consuming in situations where the number of
%distinct covariate values is large. If {\tt gridsize=nrpoints} is
%specified, estimates are computed
%on an equidistant grid with {\tt nrpoints} knots. & {\tt gridsize=-1} \\
%\hline
%
%%#derivative# & The option #derivative# causes that first order
%%derivatives of the estimation are computed. & - \\ \hline
%
%{\tt period} & The period of the seasonal effect can be specified with
%the option {\tt period}. The default is {\tt period=12} which corresponds
%to monthly data. & {\tt period=12} \\
%\hline
%
%\end{tabular}
%{\em\caption{\label{options} Optional arguments for stepwisereg
%object terms}}
%\end{table}




%
%
%\section{Commands for variable and smoothing parameter selection} %---------------------------------------
%
%In order to perform a variable and smoothing parameter selection in {\it BayesX}, we
%start with creating a {\it stepwisereg object} which we simply call {\tt s}:
%
%\begin{verbatim}
%> stepwisereg s
%\end{verbatim}
%
%The next step is to specify the output directory and a base filename
%for the files containing the estimation results.
%This is done via the {\tt outfile} command of {\it stepwisereg objects}:
%
%\begin{verbatim}
%> s.outfile = c:\results\car
%\end{verbatim}
%
%Now, all results files created by {\it BayesX} after the selection procedure are stored in the directory
%'{\tt c:/results}' and their names begin with the characters '{\tt car}'. If the user does not specify
%an output directory, the results files are written to the subdirectory '{\tt output}' of the
%installation directory. In this case, the name of the {\it stepwisereg object}, i.e.~'{\tt s}' in our example,
%is used as base filename. \\
%The selection is performed using the {\tt regress} command for {\it stepwisereg objects}. Its general structure is
%
%{\tt > s.regress} {\it depvar = term$_1$ + term$_2$ +} $\ldots$ {\it + term$_r$} [{\tt weight} {\it weightvar}]
%$[${\tt if} {\it expression}$]$ \\
%$[$, {\it options}$]$ {\tt using d}
%
%where {\it depvar} is the dependent variable, i.e.~the logarithmic claim size in our example, and
%term$_1$, etc.~specifies the type of function for the respective covariate.
%An intercept term is automatically included in the model and is not specified by the user.
%The part {\tt using d}
%indicates that data stored in {\it dataset object} {\tt d} is used for the selection.
%In the Belgian car insurance example we want to perform a variable and smoothing parameter selection
%using the dependent variable {\it logs}, weight variable {\it nclaims} and independent variables
%{\it ageph}, {\it bm} and {\it sex}. A simple linear model based on these variables can be selected and
%estimated by command
%
%\begin{verbatim}
%> s.regress logs = sex + ageph + bm weight nclaims,
%  criterion=AIC_imp family=gaussian using d
%\end{verbatim}
%
%But as we want to investigate wether the continuous variables
%{\it ageph} and {\it bm} possess nonlinear effects, we have to specify the semiparametric predictor
%%
%$$
%\eta = \gamma_0 + \gamma_{sex} sex + f_{ageph}(ageph) + f_{bm}(bm),
%$$
%%
%where the two nonlinear functions are represented by P--splines.
%The selection for this semiparametric predictor can be performed using the command
%
%\begin{verbatim}
%> s.regress logs = sex + ageph(psplinerw2,dfmin=2,dfmax=16,number=15) +
%                   bm(psplinerw2,dfmin=2,dfmax=16,number=15) weight nclaims,
%  criterion=AIC_imp family=gaussian using d
%\end{verbatim}
%
%For the selection, there are several global options available whose meanings are described in the following list.
%Possible values and default values are given in tables \ref{stewpisereg_globaloptions} and \ref{stewpisereg_globaloptions2}.
%
%\begin{longtable}{p{2.2cm} p{13.3cm}}
%{\tt algorithm} & specifies the selection procedure that is to be used. \\
%            & \\
%{\tt steps}     & defines the maximum number of iterations that can be used during the selection process. If the value {\tt steps}
%                   is reached before the selection process is finished, the process stops and the results of the current model
%                   are written to the results files. If that happens, a warning is written to the output window.
%                   By setting {\tt steps=0} it is possible to estimate a certain model
%                   without performing any selection. \\
%            & \\
%{\tt criterion} & specifies the selection criterion that is to be used. \\
%            & \\
%{\tt proportion} & If the selection is based on a criterion using a training and a validation data set,
%                  i.e.~on {\tt MSEP}, {\tt proportion} defines the fraction of the original data used as training data. \\
%            & \\
%{\tt startmodel} & defines the model that is used as starting model. \\
%%{\tt hierarchcal} \> hoffentlich \"{u}berfl\"{u}ssig
%            & \\
%{\tt number} & defines the number of different smoothing parameters to be used for the nonlinear terms. This number can be
%overwritten using the local option {\tt number}. \\
%            & \\
%{\tt trace} & specifies how detailed the output in the {\it output window} will be. \\
%            & \\
%{\tt CI} & specifies if confidence intervals are to be calculated. The default value is {\tt CI=none}
%so that no confidence intervals are obtained. {\tt CI=MCMCselect} yields confidence intervals which are
%estimated by MCMC techniques conditional on the selected model, i.e.~scale parameter
%and smoothing parameters are fixed on the values chosen by the preceding selection procedure.
%Unconditional confidence intervals can be obtained by {\tt CI=MCMCbootstrap} where several models
%are selected on the basis of bootstrap samples. For each of the selected models
%samples are drawn by MCMC techniques conditional on the respective model and based on the original data set.
%{\tt CI=bootstrap} yields unconditional confidence intervals by selecting many models
%on the basis of bootstrap samples. \\
%            & \\
%{\tt bootstrap-} \newline {\tt samples} & defines the number of bootstrap samples used for {\tt CI=bootstrap} or {\tt CI=MCMCbootstrap}. \\
%            & \\
%{\tt iterations} & defines the number of MCMC iterations used for {\tt CI=MCMCselect} or
%{\tt CI=MCMCbootstrap}. With {\tt CI=MCMCbootstrap}, option {\tt iterations} specifies the total number
%of iterations, i.e.~the sum of iterations used for the individual conditional MCMC estimations.
%Here, {\tt iterations} is divided equally between the individual conditional estimations so that the number of iterations
%used for one model is {\tt iterations / (bootstrap + 1)}. Hence, {\tt iterations} should be chosen appropriately.  \\
%            & \\
%{\tt step} & is a thinning parameter and specifies that only every {\tt step}--th MCMC--sample is used for the calculation of credible intervals
%with {\tt CI=MCMCselect} or {\tt CI=MCMCbootstrap}.
%Since the samples are correlated, the thinning out of MCMC samples is used to obtain
%approximately independent samples. \\
%            & \\
%{\tt burnin} & defines the number of MCMC iterations used for the burn--in phase
%at the beginning of each conditional MCMC estimation.
%Hence it is meaningful for {\tt CI=MCMCbootstrap} and {\tt CI=MCMCselect}. The burn--in phase usually
%is needed to achieve convergence of the Markov chain regarding its stationary (i.e.~the posterior) distribution.
%In our case, the initial estimates for each conditional MCMC estimation are the posterior mode estimates. That
%means, the Markov chain already starts in its stationary distribution.
%Hence, the burn--in phase usually is not needed here and it is possible to define {\tt burnin=0} what
%saves a lot of computing time. \\
%            & \\
%{\tt level1} & defines the first significance level for confidence intervals. \\
%            & \\
%{\tt level2} & defines the second significance level for confidence intervals. \\
%            & \\
%{\tt predict} & By specifying {\tt predict} an additional results file is created containing estimates for the predictor and
%                for the conditional expectation of the response variable. \\
%            & \\
%{\tt family} & specifies response distribution and link function. \\
%            & \\
%{\tt reference} & specifies the reference category for multinomial logit models.
%\end{longtable}
%
%\begin{table}[ht] \footnotesize
%\begin{center}
%\begin{tabular}{|p{2.5cm}|p{1.5cm}|p{1.5cm}|p{2.5cm}|p{6cm}|}
%\hline
%{\bf global option} & {\bf type} & {\bf default} & {\bf values} & {\bf meaning} \\
%\hline \hline
%{\tt algorithm}  & string  & cdescent1 & cdescent1 & adaptive search \\
%                 &         &           & cdescent2 & exact search \\
%                 &         &           & cdescent3 & adaptive/exact search \\
%                 &         &           & stepwise  & stepwise algorithm \\
%\hline
%{\tt steps}      & numeric \newline (integer) & 1000 & $\{0;10000\}$ & maximum number of iterations \\
%\hline
%{\tt criterion}  & string  & GCV       & GCV      & GCV (for non--Gaussian response (\ref{GCV_general2})
%                                                      based on deviance residuals)\\
%                 &         &           & GCVrss   & only meaningful for non--Gaussian response:
%                                                    GCV (\ref{GCV_general}) based on residual sum of squares \\
%                 &         &           & AIC      & AIC \\
%                 &         &           & AIC\_imp & improved AIC \\
%                 &         &           & BIC      & BIC \\
%                 &         &           & MSEP     & MSEP \\
%                 &         &           & CV5      & 5--fold cross validation \\
%                 &         &           & CV10     & 10--fold cross validation \\
%                 &         &           & AUC      & area under the ROC curve \newline
%                                                    (only for binary response) \\
%\hline
%{\tt proportion} & numeric \newline (real)    & 0.75 & $(0;1)$ & for MSEP (see description above) \\
%\hline
%{\tt startmodel} & string  & linear    & linear      & base model with degrees of freedom equal to one \\
%                 &         &           & empty       & empty model containing the intercept term only \\
%                 &         &           & full        & most complex possible model \\
%                 &         &           & userdefined & base model specified by the user; \newline
%                                                       otherwise the linear one \\
%\hline
%{\tt number}     & numeric \newline (integer) & 20   & $\{1;50\}$ & number of smoothing parameters \\
%\hline
%{\tt trace}      & string  & trace\_on & trace\_on   & output shows every new model during iterations \\
%                 &         &           & trace\_half & output shows the starting models of all iterations \\
%                 &         &           & trace\_off  & no output except starting and final model \\
%\hline
%{\tt CI}         & string  & none      & none          & no confidance intervals \\
%                 &         &           & MCMCselect    & conditional MCMC confidance bands \\
%                 &         &           & MCMCbootstrap & unconditional confidance bands based on bootstrap and MCMC \\
%                 &         &           & bootstrap     & unconditional MCMC confidance intervals based on bootstrap \\
%\hline
%{\tt bootstrap-} \newline {\tt samples}  & numeric \newline (integer) & 0 & $\{100;10000\}$ & number of bootstrap samples \\
%\hline
%\end{tabular}
%{\em\caption {\label{stewpisereg_globaloptions} Possible global options
%for stepwisereg objects.}}
%\end{center}
%\end{table}
%%
%\begin{table}[ht] \footnotesize
%\begin{center}
%\begin{tabular}{|p{2.5cm}|p{1.5cm}|p{1.5cm}|p{2cm}|p{6cm}|}
%\hline
%{\bf global option} & {\bf type} & {\bf default} & {\bf values} & {\bf meaning} \\
%\hline \hline
%{\tt iterations}  & numeric \newline (integer) & 52000 & $\{1;10000000\}$ & total number of MCMC iterations \\
%\hline
%{\tt step}  & numeric \newline (integer) & 50 & $\{1;1000\}$ & thinning parameter for MCMC samples \\
%\hline
%{\tt burnin}  & numeric \newline (integer) & 2000 & $\{0;500000\}$ & number of MCMC iterations used for each burnin phase \\
%\hline
%{\tt level1}  & numeric \newline (real) & 95 & $[40;99]$ & first significance level \\
%\hline
%{\tt level2}  & numeric \newline (real) & 80 & $[40;99]$ & second significance level \\
%\hline
%{\tt predict}  & boolean  & false & false & no estimates for predictor / expectation of response \\
%               &          &       & true  & estimates for the predictor and expectation are obtained \\
%\hline
%{\tt family}     & string  & logit     & gaussian    & Gaussian distribution with identity link \\
%                 &         &           & binomial    & Binomial distribution with logit link \\
%                 &         &           & binomialprobit & Binomial distribution with probit link \\
%                 &         &           & poisson     & Poisson distribution with log link \\
%                 &         &           & gamma       & Gamma distribution with log link \\
%                 &         &           & multinomial & Multinomial distribution with logit link \\
%\hline
%{\tt reference}  & numeric \newline (real)  & 0     & $[-10000;10000]$    & reference category for multinomial logit models \\
%\hline
%\end{tabular}
%{\em\caption {\label{stewpisereg_globaloptions2} Possible global options
%for stepwisereg objects.}}
%\end{center}
%\end{table}
%
%The commands for specifying different term types for univariate covariates are listed in table
%\ref{stepwisereg_terms}. Possibilities for interactions and the respective
%commands are shown in table \ref{stepwisereg_interactions}. For all term types
%apart from fixed effects, there are various options which are described below.
%In the following, we will refer to these options as local options (in constrast to the
%global options affecting the whole selection procedure). Possible values for the local options
%are described in table \ref{stepwisereg_localoptions} whereas table \ref{termsoptions} gives
%a short overview of possible combinations of function terms and local options.
%
%\begin{longtable}{p{2.2cm} p{13.3cm}}
%{\tt dfmin} & Option {\tt dfmin} defines the smallest possible degree of freedom
%              for a nonlinear function (besides the linear effect).
%              The largest smoothing parameter is calculated according to {\tt dfmin}.
%              Possible values depend on the number of regression parameters
%              and on the prior distribution
%              (compare section \ref{section_df}). In order to avoid numerical problems
%              the smoothing parameter may not become larger than $10^9$. In this
%              case, {\tt dfmin} is enlarged by ({\tt dfmax} - {\tt dfmin}) / {\tt number}
%              (and {\tt number} is reduced by one).
%              Additionally, this ascertains that {\tt dfmin} is redefined to
%              a possible value. \\
%            & \\
%{\tt dfmax} & Option {\tt dfmax} defines the largest possible degree of freedom for a
%              nonlinear function. The smallest smoothing parameter is calculated
%              according to {\tt dfmax}. Possible values depend on the number of regression
%              parameters and on the prior distribution
%              (compare section \ref{section_df}). In order to avoid numerical problems
%              the smoothing parameter may not become smaller than $10^{-9}$. In this
%              case, {\tt dfmax} is reduced by ({\tt dfmax} - {\tt dfmin}) / {\tt number}
%              (and {\tt number} is reduced by one).
%              Additionally, this ascertains that {\tt dfmax} is redefined to
%              a possible value. \\
%            & \\
%{\tt dfstart} & Option {\tt dfstart} defines the complexity of the function used in
%                the base model. This option is only meaningful if
%                {\tt startmodel=userdefined} is specified. In this case, the default value for
%                {\tt dfstart} is either the fixed effect, if possible, or otherwise the degree of freedom nearest to one. \\
%            & \\
%{\tt logscale} & This option causes the smoothing parameters to lie on a logarithmic
%                 scale instead of being specified according to equidistant
%                 degrees of freedom. In this case, only the smallest and largest smoothing parameters
%                 are calculated according to {\tt dfmin} and {\tt dfmax}. This option is only meaningful
%                 if option {\tt sp} is not specified (see below). \\
%            & \\
%{\tt df\_accuracy} & This option specifies the maximal absolute difference in terms of
%                     degrees of freedom that is allowed when calculating smoothing parameters
%                     according to user--specified degrees of freedom. \\
%            & \\
%{\tt sp} & Option {\tt sp} causes the smoothing parameters to be chosen directly according
%           to values specified by options {\tt spmin}, {\tt spmax} and {\tt spstart}.
%           All other values are chosen according to a logarithmic scale. \\
%            & \\
%{\tt spmin} & This option defines the smallest smoothing parameter but is only valid if
%              {\tt sp} is specified. \\
%            & \\
%{\tt spmax} & Option {\tt spmax} defines the largest smoothing parameter but is only valid if
%              {\tt sp} is specified. \\
%            & \\
%{\tt spstart} & This option is only meaningful if {\tt startmodel=userdefined} and {\tt sp}
%                are specified. It defines the smoothing parameter used for the base model. Note, that {\tt spstart}
%                can not only take positive values but can also take the values {\tt spstart=0} for
%                excluding the function in the base model and {\tt spstart=-1} for using the fixed effect. \\
%            & \\
%{\tt number} & {\tt number} specifies the number of different smoothing parameters
%               (besides the linear effect and exclusion from the model). For {\tt number=0} the
%               global option {\tt number} is used. \\
%            & \\
%{\tt forced\_into} & This option drops the possibility to exclude the function from the model. That means the respective function
%                     is always included in the model. \\
%            & \\
%{\tt nofixed} & This option drops the possibility to use a linear fit. Hence, only possibilities
%for a nonlinear effect and for the removal from the model remain. \\
%            & \\
%{\tt center} & {\tt center} has to be specified with varying coefficients if the
%               coefficients must get centered with regard to the interacting variable, i.e., if
%               there are several varying coefficients modifying the same interacting variable.
%               Hence, {\tt center} is only meaningful for varying coefficients and random slopes. \\
%            & \\
%{\tt coding} & Option {\tt coding} is only meaningful for factor variables. It determines wether
%               dummy variables ({\tt coding=dummy}) or effect variables ({\tt coding=effect})
%               are used to represent the factor. \\
%            & \\
%{\tt reference} & Option {\tt reference} is again only meaningful for factor variables. It defines
%                  the value for the reference category. \\
%            & \\
%{\tt degree} & Specifies the degree of B-spline basis functions. \\
%            & \\
%{\tt nrknots} & Specifies the number of inner knots for a P-spline term. \\
%            & \\
%{\tt monotone} & Option {\tt monotone} specifies additional constraints for univariate
%                 P--spline terms. Possible are the estimation of an unrestricted function,
%                 a monotonically increasing or decreasing function (i.e. positive/negative first derivative)
%                 or a convex or concave function (i.e. positive/negative second derivative).
%                 Note, that both type and direction of the constraint have to be defined by the user and are
%                 not determined by the selection procedure. \\
%            & \\
%{\tt gridsize} & The option {\tt gridsize} can be used to restrict the
%                 number of points (at the x-axis) for which estimates are computed.
%                 By default, estimates are computed at every distinct covariate
%                 value in the data set (indicated by {\tt gridsize=-1}). This may be
%                 relatively time consuming in situations where the number of
%                 distinct covariate values is large. If {\tt gridsize=nrpoints} is
%                 specified, estimates are computed
%                 on an equidistant grid with {\tt nrpoints} knots. \\
%            & \\
%{\tt period} & The period of the seasonal effect can be specified with
%               option {\tt period}. The default is {\tt period=12} which corresponds
%               to monthly data. \\
%            & \\
%{\tt map} & The map object for a spatial function is defined by option {\tt map}.
%\end{longtable}
%
%\begin{table}[ht] \footnotesize
%\begin{center}
%\begin{tabular}{|p{2.8cm}|p{5cm}|p{7cm}|}
%\hline
%{\bf Type}     & {\bf Syntax example} & {\bf Description} \\
%\hline \hline
%offset         & {\tt offs(offset)}  & Variable {\tt offs} is an offset term. \\
%\hline
%linear effect  & {\tt W1}  & Linear effect for {\tt W1}. \\
%\hline
%factor         & {\tt F1(factor)} & Effect of categorical variable {\tt F1} \\
%\hline
%first or second order random walk &   {\tt X1(rw1)} \newline {\tt X1(rw2)}  & Nonlinear effect of {\tt X1}. \\
%\hline
%P-spline       &  {\tt X1(psplinerw1)} \newline {\tt X1(psplinerw2)}  & Nonlinear effect of {\tt X1}.  \\
%\hline
%seasonal prior & {\tt time(season)} & Varying seasonal effect of {\tt time}. \\
%\hline
%Markov random \newline field &  {\tt region(spatial,map=m)}  & Spatial effect of {\tt region} where {\tt region} indicates the region an
%observation pertains to. The boundary information and the
%neighborhood structure are stored in the {\em map object}
%{\tt m} . \\
%\hline
%Two dimensional \newline P-spline & {\tt region(geosplinerw1,map=m)} \newline {\tt region(geosplinerw2,map=m)}
%& Spatial effect of {\tt region}. Estimates a two dimensional P-spline
%based on the centroids of the regions. The centroids are stored in the {\em map object} {\tt m}. \\
%\hline
%random intercept &  {\tt grvar(random)}  & I.i.d.~Gaussian random effect of the group indicator {\tt grvar} ,
%e.g.~{\tt grvar}  may be an individuum indicator when analyzing longitudinal data.  \\
%\hline
%\end{tabular}
%{\em\caption {\label{stepwisereg_terms} Overview over different model terms
%for stepwisereg objects.}}
%\end{center}
%\end{table}
%
%
%\begin{table}[ht] \footnotesize
%\begin{center}
%\begin{tabular}{|p{3.6cm}|p{4.5cm}|p{6.7cm}|}
%\hline
%{\bf Type of interaction} & {\bf Syntax example} & {\bf Description} \\
%\hline \hline
%Varying coefficient term & {\tt X1*X2(rw1)} \newline {\tt X1*X2(rw2)} \newline {\tt X1*X2(psplinerw1)} \newline {\tt X1*X2(psplinerw2)}
%%\newline {\tt X1*time(season)}
%& Effect of {\tt X1} varies smoothly over the range of the continuous covariate {\tt X2}. \\
%\hline
%random slope & {\tt X1*grvar(random)}  &  The regression
%coefficient of {\tt X1} varies with respect
%to the unit- or cluster-index variable {\tt grvar}. \\
%\hline
%Geographically weighted regression & {\tt X1*region(spatial,map=m)}  & Effect of {\tt X1} varies
%geographically. Covariate
%{\tt region} indicates the region an observation pertains to. \\
%\hline
%Two dimensional surface &  {\tt X1*X2(pspline2dimrw1)} \newline {\tt X1*X2(pspline2dimrw2)}
%& Two dimensional surface for the continuous
%covariates {\tt X1} and {\tt X2}. \\
%\hline
%ANOVA type interaction &  {\tt X1*X2(psplineinteract) + } \newline {\tt X1(psplinerw?) + X2(psplinerw?)}
%& ANOVA type interaction for the continuous
%covariates {\tt X1} and {\tt X2}. \\
%\hline
%
%\end{tabular}
%{\em\caption {\label{stepwisereg_interactions} Possible interaction
%terms for stepwisereg objects.}}
%\end{center}
%\end{table}
%
% \begin{sidewaystable}[ht] \footnotesize
% \begin{center}
% \begin{tabular}{|l|l|l|l|l|}%{|p{2.5cm}|p{1.5cm}|p{2cm}|p{2cm}|p{7cm}|}
% \hline
% {\bf local option} & {\bf type} & {\bf default} & {\bf values} & {\bf meaning} \\
% \hline \hline
% {\tt dfmin}     & numeric (real) & --    & compare section \ref{section_df} & minimal degree of freedom \\
%\hline
% {\tt dfmax}     & numeric (real) & --    & compare section \ref{section_df} & maximal degree of freedom \\
%\hline
% {\tt dfstart}   & numeric (real) & 1     & $\{0,1\} \cup [dfmin;dfmax]$ & degree of freedom used in the base model \\
%\hline
% {\tt logscale}  & boolean                 & false & false & equidistant degrees of freedom \\
%                 &                         &       & true  & smoothing parameters on a logarithmic scale \\
%\hline
% {\tt sp}        & boolean                 & false & false & smoothing parameters are specified in terms of $df$ \\
%                 &                         &       & true  & smoothing parameters are directly specified \\
%\hline
% {\tt spmin}     & numeric (real) & $10^{-4}$ & $[10^{-6};10^{8}]$ & minimal smoothing parameter \\
%\hline
% {\tt spmax}     & numeric (real) & $10^4$    & $[10^{-6};10^{8}]$ & maximal smoothing parameter \\
%\hline
% {\tt spstart}   & numeric (real) & --    & $\{-1,0\} \cup [10^{-6};10^{8}]$ & smoothing parameter for the base model \\
%\hline
% {\tt number}    & numeric (integer) & 0  & $\{0;100\}$ & number of different smoothing parameters \\
%\hline
% {\tt forced\_into} & boolean              & false & false & term may be excluded from the model \\
%                 &                         &       & true  & term may not be excluded from the model \\
%\hline
% {\tt nofixed}   & boolean                 & false & false & Possibility of linear fit \\
%                 &                         &       & true  & A linear fit is not possible \\
%\hline
% {\tt center}    & boolean                 & false & false & no centering of varying coefficient terms \\
%                 &                         &       & true  &  \\
%\hline
% {\tt coding}    & string                  & dummy & dummy & dummy coding of categorical variables \\
%                 &                         &       & effect & effect coding of categorical variables \\
%\hline
% {\tt reference} & numeric (real)    & 1  & $(-100;100)$ & reference category \\
%\hline
% {\tt degree}    & numeric (integer) & 3  & $\{0;\ldots;5\}$ & degree of B--spline basis functions \\
%\hline
% {\tt nrknots}   & numeric (integer) & 20 & $\{5;\ldots;500\}$ & number of inner knots for a P--spline term \\
%\hline
% {\tt monotone}  & string                  & unrestricted & unrestricted & no constraint on the spline function \\
%                 &                         &              & increasing   & monotonically increasing function \\
%                 &                         &              & decreasing   & monotonically increasing function \\
%                 &                         &              & convex       & convex function, i.e. positive second derivative \\
%                 &                         &              & concave      & concave function, i.e. negative second derivative \\
%\hline
% {\tt gridsize}  & numeric (integer) & -1 & $\{-1;10;\ldots;500\}$ & \\
%\hline
% {\tt period}    & numeric (integer) & 12 & $\{2;\ldots;72\}$ & period for a seasonal effect \\
%\hline
% {\tt map}       & {\it map object}        & --    & --                & specifies the map object used \\
% \hline
% \end{tabular}
% {\em\caption {\label{stepwisereg_localoptions} Possible local options
% for stepwisereg objects. Note, that boolean options are specified without supplying a value.}}
% \end{center}
% \end{sidewaystable}
%
%%\begin{table}[ht] \footnotesize \centering
%%\begin{tabular}{|l|p{0.6\linewidth}|c|}
%%
%%\hline
%%optionname & description & default \\
%%\hline
%%\hline
%%
%%{\tt dfmin} & Option {\tt dfmin} defines the smallest possible degree of freedom for a nonlinear function
%%(besides the linear effect). The largest smoothing parameter is then calculated according to {\tt dfmin}. & \\
%%\hline
%%
%%{\tt dfmax} & Option {\tt dfmax} defines the largest possible degree of freedom for a nonlinear function.
%%The smallest smoothing parameter is then calculated according to {\tt dfmax}. & \\
%%\hline
%%
%%{\tt dfstart} & Option {\tt dfstart} defines the complexity of the function used in the base model.
%%This option is only meaningful if {\tt startmodel=userdefined} is specified. &  \\
%%\hline
%%
%%{\tt logscale} & This option causes the smoothing parameters to lie on a logarithmic scale instead of
%%being specified according to equidistant degrees of freedom. Only the smallest and largest smoothing
%%parameters are calculated according to {\tt dfmin} and {\tt dfmax}. & \\
%%\hline
%%
%%{\tt df\_accuracy} & This option specifies the maximal absolute difference in terms of degrees of freedom allowed when
%%calculating smoothing parameters according to user--specified degrees of freedom. & \\
%%\hline
%%
%%{\tt sp} & Option {\tt sp} causes the smoothing parameters to be chosen directly according
%%to values specified by options {\tt spmin}, {\tt spmax} and {\tt spstart}. All other values
%%are chosen according to a logarithmic scale. & -- \\
%%\hline
%%
%%{\tt spmin} & This option specifies the smallest smoothing parameter. & \\
%%\hline
%%
%%{\tt spmax} & Option {\tt spmax} specifies the largest smoothing parameter. & \\
%%\hline
%%
%%{\tt spstart} & This option is only meaningful if {\tt startmodel=userdefined} is specified. It defines
%%the smoothing parameter used for the base model. & \\
%%\hline
%%
%%{\tt number} & {\tt number} specifies the number of different smoothing parameters
%%(besides the linear effect and exclusion from the model). For {\tt number=0} the
%%global option {\tt number} is used. & {\tt number=20} \\
%%\hline
%%
%%{\tt forced\_into} & This option drops the possibility to exclude the function from the model. & -- \\
%%\hline
%%
%%{\tt nofixed} & {\tt nofixed} has to be specified with varying coefficients if the coefficients must get centered
%%with regard to the interacting variable. & -- \\
%%\hline
%%
%%{\tt coding} & Option {\tt coding} is only meaningful for factor variables. It determines wether
%%dummy variables ({\tt coding=dummy}) or effect variables ({\tt coding=effect})
%%are used to represent the factor. & {\tt coding=dummy} \\
%%\hline
%%
%%{\tt reference} & Option {\tt reference} is again only meaningful for factor variables. It defines
%%the value for the reference category. & \\
%%\hline
%%
%%{\tt degree} & Specifies the degree of the B-spline basis functions. & {\tt degree=3} \\
%%\hline
%%
%%{\tt nrknots} & Specifies the number of inner knots for a P-spline term. & {\tt nrknots=20} \\
%%\hline
%%
%%{\tt gridsize} & The option {\tt gridsize} can be used to restrict the
%%number of points (at the x-axis) for which estimates are computed.
%%By default, estimates are computed at every distinct covariate
%%value in the data set (indicated by {\tt gridsize=-1}). This may be
%%relatively time consuming in situations where the number of
%%distinct covariate values is large. If {\tt gridsize=nrpoints} is
%%specified, estimates are computed
%%on an equidistant grid with {\tt nrpoints} knots. & {\tt gridsize=-1} \\
%%\hline
%%
%%%#derivative# & The option #derivative# causes that first order
%%%derivatives of the estimation are computed. & - \\ \hline
%%
%%{\tt period} & The period of the seasonal effect can be specified with
%%the option {\tt period}. The default is {\tt period=12} which corresponds
%%to monthly data. & {\tt period=12} \\
%%\hline
%%
%%\end{tabular}
%%{\em\caption{\label{options} Optional arguments for stepwisereg
%%object terms}}
%%\end{table}
%
%
%\begin{sidewaystable} \footnotesize
%\begin{tabular}{|l||p{1.5cm}|p{1.5cm}|p{1.5cm}|p{1.5cm}|p{2cm}|p{1.5cm}|p{2cm}|p{2.5cm}|}
%
%\hline
%                     & factor & rw1 \newline rw2 & season & psplinerw1 \newline psplinerw2 & spatial & random & geosplinerw1 \newline geosplinerw2
%                     & pspline2dimrw1 \newline pspline2dimrw2 \newline psplineinteract \\
%\hline\hline
% {\tt dfmin}        & -----   & real             & real   & real                           & real    & real   & real
%                    & real \\
%\hline
% {\tt dfmax}        & -----   & real             & real   & real                           & real    & real   & real
%                    & real \\
%\hline
% {\tt dfstart}      & integer & real             & real   & real                           & real    & real   & real
%                    & real \\
%\hline
% {\tt logscale}     & -----   & boolean          & boolean & boolean                       & boolean & boolean & boolean
%                    & boolean \\
%\hline
% {\tt sp}           & boolean & boolean          & boolean & boolean                       & boolean & boolean & boolean
%                    & boolean \\
%\hline
% {\tt spmin}        & -----   & real             & real   & real                           & real    & real   & real
%                    & real \\
%\hline
% {\tt spmax}        & -----   & real             & real   & real                           & real    & real   & real
%                    & real \\
%\hline
% {\tt spstart}      & integer & real             & real   & real                           & real    & real   & real
%                    & real \\
%\hline
% {\tt number}       & -----   & integer          & integer & integer                       & integer & integer & integer
%                    & integer \\
%\hline
% {\tt forced\_into} & boolean & boolean          & boolean & boolean                       & boolean & boolean & boolean
%                    & boolean \\
%\hline
% {\tt nofixed}      & boolean & boolean          & boolean?? & boolean                       & boolean & boolean & boolean
%                    & boolean \\
%\hline
% {\tt center}       & -----   & boolean          & -----?? & boolean                       & boolean & boolean & -----
%                    & ----- \\
%\hline
% {\tt coding}       & string  & -----            & -----   & -----                         & -----   & -----   & -----
%                    & ----- \\
%\hline
% {\tt reference}    & real    & -----            & -----   & -----                         & -----   & -----   & -----
%                    & ----- \\
%\hline
% {\tt degree}       & -----   & -----            & -----   & integer                       & -----   & -----   & integer
%                    & integer \\
%\hline
% {\tt nrknots}      & -----   & -----            & -----   & integer                       & -----   & -----   & integer
%                    & integer \\
%\hline
% {\tt monotone}     & -----   & -----            & -----   & string                        & -----   & -----   & -----
%                    & ----- \\
%\hline
% {\tt gridsize}     & -----   & -----            & -----   & integer                       & -----   & -----   & -----
%                    & integer \\
%\hline
% {\tt period}       & -----   & -----            & integer & -----                         & -----   & -----   & -----
%                    & ----- \\
%\hline
% {\tt map}          & -----   & -----            & -----   & -----                         & {\it map object} & -----   & {\it map object}
%                    & ----- \\
%\hline
%\end{tabular}
%{\em\centering \caption{\label{termsoptions} Terms and options for
%stepwisereg objects. For possible values to each of the local options compare table \ref{stepwisereg_localoptions}.
%Note, that boolean options are specified without supplying a value.}}
%\end{sidewaystable}
%
%Some information about the progression of the selection algorithm and some results are shown in the
%{\it output window} whereas other results are only available from external ASCII--files.
%%
%%s.regress LOGS = SEX2 +  AGEPH(psplinerw2,dfmin=2,dfmax=16,number=15) +    BM(psplinerw2,dfmin=2,dfmax=16,number=15) weight NCLAIMS,  procedure=coorddescent minimum=adaptiv startmodel=userdefined criterion=AIC_imp trace=trace_minim family=gaussian predict using d
%%
%%STEPWISE OBJECT s: stepwise procedure
%%
%%GENERAL OPTIONS:
%%
%%  Performance criterion: AIC_imp
%%  Maximal number of iterations: 1000
%%
%%  RESPONSE DISTRIBUTION:
%%
%%  Family: Gaussian
%%  Number of observations: 18139
%%
%%OPTIONS FOR STEPWISE PROCEDURE:
%%
%%  OPTIONS FOR FIXED EFFECTS TERM: SEX2
%%
%%  Prior: diffuse prior
%%  Startvalue of the 1. startmodel is the fixed effect
%%
%%  OPTIONS FOR NONPARAMETRIC TERM: AGEPH
%%
%%  Minimal value for the smoothing parameter: 2.0480375
%%  This is equivalent to degrees of freedom: approximately 16, exact 16.0369
%%  Maximal value for the smoothing parameter: 62500
%%  This is equivalent to degrees of freedom: approximately 2, exact 1.95119
%%  Number of different smoothing parameters with equidistant degrees of freedom: 15
%%  Startvalue of the 1. startmodel is the fixed effect
%%
%%  OPTIONS FOR NONPARAMETRIC TERM: BM
%%
%%  Minimal value for the smoothing parameter: 1.0240375
%%  This is equivalent to degrees of freedom: approximately 16, exact 16.0487
%%  Maximal value for the smoothing parameter: 45000
%%  This is equivalent to degrees of freedom: approximately 2, exact 2.02502
%%  Number of different smoothing parameters with equidistant degrees of freedom: 15
%%  Startvalue of the 1. startmodel is the fixed effect
%%
%%
%%STEPWISE PROCEDURE STARTED
%%
%%
%%  Startmodel:
%%
%%  LOGS = const + SEX2 + AGEPH + BM
%%  AIC_imp = 14821.315
%%
%%  SEX2
%%
%%  Lambda   Testvalue (approx):
%%       -1   14821.3
%%        0   14820.6
%%
%%
%%  Trial:
%%
%%  LOGS = const + AGEPH + BM
%%  AIC_imp = 14820.56
%%
%%  AGEPH
%%
%%  Lambda   Testvalue (approx):
%%  2.04804   14780.4
%%  4.01452   14778.6
%%  6.99482   14777.1
%%  11.5539   14775.7
%%  19.3413   14774.4
%%  32.8874   14773.2
%%  54.1414   14772.2
%%  93.4288   14771.3
%%   171.57   14770.7
%%   318.55   14770.2
%%  666.043   14770.1
%%   1392.6   14770.9
%%  3733.38   14774.5
%%  12012.3   14782.2
%%    62500   14792.9
%%       -1   14820.5
%%        0   14865.8
%%
%%
%%  Trial:
%%
%%  LOGS = const + BM + AGEPH(psplinerw2,df=5.96466,(lambda=666.043))
%%  AIC_imp = 14770.105
%%
%%  BM
%%
%%  Lambda   Testvalue (approx):
%%  1.02404   14770.1
%%  2.19761   14768.7
%%  4.12662   14767.4
%%  7.35352   14766.1
%%  12.8113   14765
%%  22.1905   14763.8
%%  38.0606   14762.6
%%  67.9218   14761.4
%%  131.282   14760.1
%%  246.389   14759
%%  503.474   14758
%%  1188.99   14757.6
%%  3091.62   14759.2
%%  9603.95   14762.8
%%    45000   14766.4
%%       -1   14768
%%        0   14853
%%
%%
%%  Trial:
%%
%%  LOGS = const + AGEPH(psplinerw2,df=5.96466,(lambda=666.043)) + BM(psplinerw2,df=4.96696,(lambda=1188.99))
%%  AIC_imp = 14757.645
%%
%%  ------------------------------------------------------------------------
%%  ------------------------------------------------------------------------
%%
%%  Startmodel:
%%
%%  LOGS = const + AGEPH(psplinerw2,df=5.96466,(lambda=666.043)) + BM(psplinerw2,df=4.96696,(lambda=1188.99))
%%  AIC_imp = 14757.645
%%
%%  SEX2
%%
%%  Lambda   Testvalue (approx):
%%       -1   14758.6
%%        0   14757.6
%%
%%
%%  AGEPH
%%
%%  Lambda   Testvalue (approx):
%%  2.04804   14768.3333106
%%  4.01452   14766.5095849
%%  6.99482   14764.9238302
%%  11.5539   14763.5164049
%%  19.3413   14762.155637
%%  32.8874   14760.8830989
%%  54.1414   14759.842513
%%  93.4288   14758.908342
%%   171.57   14758.1286187
%%   318.55   14757.6137573
%%  666.043   14757.4846875
%%   1392.6   14758.2899702
%%  3733.38   14761.9155524
%%  12012.3   14769.5821669
%%    62500   14780.4506147
%%       -1   14810.2710016
%%        0   14851.9085539
%%
%%
%%  BM
%%
%%  Lambda   Testvalue (approx):
%%  1.02404   14769.979268
%%  2.19761   14768.5680298
%%  4.12662   14767.2727716
%%  7.35352   14766.0405477
%%  12.8113   14764.8471118
%%  22.1905   14763.6699765
%%  38.0606   14762.5277055
%%  67.9218   14761.3278293
%%  131.282   14760.012711
%%  246.389   14758.8535965
%%  503.474   14757.8212049
%%  1188.99   14757.4848975
%%  3091.62   14758.9815463
%%  9603.95   14762.5502326
%%    45000   14766.1142236
%%       -1   14767.5920845
%%        0   14849.3012707
%%
%%
%%  ------------------------------------------------------------------------
%%  ------------------------------------------------------------------------
%%
%%  Startmodel:
%%
%%  LOGS = const + AGEPH(psplinerw2,df=5.96466,(lambda=666.043)) + BM(psplinerw2,df=4.96696,(lambda=1188.99))
%%  AIC_imp = 14757.485
%%
%%  SEX2
%%
%%  Lambda   Testvalue (approx):
%%       -1   14758.5
%%        0   14757.5
%%
%%
%%  AGEPH
%%
%%  Lambda   Testvalue (approx):
%%  2.04804   14768.3117913
%%  4.01452   14766.4878015
%%  6.99482   14764.9017027
%%  11.5539   14763.4938744
%%  19.3413   14762.132582
%%  32.8874   14760.8593192
%%  54.1414   14759.8178341
%%  93.4288   14758.8825054
%%   171.57   14758.1016164
%%   318.55   14757.586302
%%  666.043   14757.4586799
%%   1392.6   14758.2695722
%%  3733.38   14761.9155247
%%  12012.3   14769.625298
%%    62500   14780.5847697
%%       -1   14810.8412796
%%        0   14851.4457103
%%
%%
%%  BM
%%
%%  Lambda   Testvalue (approx):
%%  1.02404   14769.9792694
%%  2.19761   14768.5677484
%%  4.12662   14767.272393
%%  7.35352   14766.0400104
%%  12.8113   14764.8461721
%%  22.1905   14763.668301
%%  38.0606   14762.5249636
%%  67.9218   14761.3236143
%%  131.282   14760.0065583
%%  246.389   14758.8453927
%%  503.474   14757.8099719
%%  1188.99   14757.4676549
%%  3091.62   14758.954133
%%  9603.95   14762.5085125
%%    45000   14766.0512995
%%       -1   14767.50008
%%        0   14848.3747078
%%
%%
%%  ------------------------------------------------------------------------
%%  ------------------------------------------------------------------------
%%
%%  Startmodel:
%%
%%  LOGS = const + AGEPH(psplinerw2,df=5.96466,(lambda=666.043)) + BM(psplinerw2,df=4.96696,(lambda=1188.99))
%%  AIC_imp = 14757.468
%%
%%  SEX2
%%
%%  Lambda   Testvalue (approx):
%%       -1   14758.5
%%        0   14757.5
%%
%%
%%  AGEPH
%%
%%  Lambda   Testvalue (approx):
%%  2.04804   14768.3164709
%%  4.01452   14766.4924112
%%  6.99482   14764.9062226
%%  11.5539   14763.498292
%%  19.3413   14762.1368726
%%  32.8874   14760.8634425
%%  54.1414   14759.8217557
%%  93.4288   14758.8861705
%%   171.57   14758.1050229
%%   318.55   14757.5896092
%%  666.043   14757.4623533
%%   1392.6   14758.2747345
%%  3733.38   14761.9262653
%%  12012.3   14769.647904
%%    62500   14780.6317888
%%       -1   14811.0025959
%%        0   14851.3413183
%%
%%
%%  BM
%%
%%  Lambda   Testvalue (approx):
%%  1.02404   14769.9812004
%%  2.19761   14768.5696078
%%  4.12662   14767.2742295
%%  7.35352   14766.0418079
%%  12.8113   14764.8478663
%%  22.1905   14763.6698039
%%  38.0606   14762.5261883
%%  67.9218   14761.3244548
%%  131.282   14760.006895
%%  246.389   14758.8451982
%%  503.474   14757.808995
%%  1188.99   14757.4651245
%%  3091.62   14758.9489668
%%  9603.95   14762.4996258
%%    45000   14766.0368818
%%       -1   14767.4781304
%%        0   14848.1364689
%%
%%
%%
%%  There are no new models for this iteration!
%%
%%  ------------------------------------------------------------------------
%%  ------------------------------------------------------------------------
%%
%%  Final Model:
%%
%%  LOGS = const + AGEPH(psplinerw2,df=5.96466,(lambda=666.043)) + BM(psplinerw2,df=4.96696,(lambda=1188.99))
%%  AIC_imp = 14757.465
%%
%%  Used number of iterations: 4
%%
%%  ------------------------------------------------------------------------
%%  ------------------------------------------------------------------------
%%
%%  Final Model:
%%
%%  LOGS = const + AGEPH(psplinerw2,df=5.96466,(lambda=666.043)) + BM(psplinerw2,df=4.96696,(lambda=1188.99))
%%  AIC_imp = 14757.464
%%
%%RESPONSE DISTRIBUTION:
%%
%%  Gaussian
%%  Number of observations: 18139
%%
%%
%%ESTIMATION RESULTS:
%%
%%
%%  Predicted values:
%%
%%  Estimated mean of predictors, expectation of response and
%%  individual deviances are stored in file
%%  C:\bayesx\output\s_predictmean.raw
%%
%%
%%  Saturated deviance: 20058
%%
%%  Estimation results for the scale parameter:
%%
%%  sigma2:         2.03744
%%
%%
%%
%%  FixedEffects1
%%
%%
%%
%%  Variable  mean           Std. Dev.      2.5% quant.    median         97.5% quant.
%%  const     10.0619        0              0              0              0
%%
%%  Results for fixed effects are also stored in file
%%  C:\bayesx\output\s_FixedEffects1.res
%%
%%
%%  f_AGEPH
%%
%%
%%  Results are stored in file
%%  C:\bayesx\output\s_f_AGEPH_pspline.res
%%
%%  Results may be visualized using the R / S-Plus function 'plotnonp'
%%  Type for example:
%%  plotnonp("C:\\bayesx\\output\\s_f_AGEPH_pspline.res")
%%
%%
%%
%%  f_BM
%%
%%
%%  Results are stored in file
%%  C:\bayesx\output\s_f_BM_pspline.res
%%
%%  Results may be visualized using the R / S-Plus function 'plotnonp'
%%  Type for example:
%%  plotnonp("C:\\bayesx\\output\\s_f_BM_pspline.res")
%%
%%
%The {\it output window} shows all specified covariates and terms together with the respective
%number of different smoothing parameters and the way in which they were specified. Furthermore,
%even by specifying option {\tt trace=trace\_off},
%starting model and final model are shown together with the respective values of the selection criterion.
%The total number of iterations is also given in the output.
%By using the default value {\tt trace=trace\_on}, the {\it output window} additionally shows
%every model that was tried during iterations. Option {\tt trace=trace\_half} reduces the output to the starting
%models of the individual iterations. With {\tt trace=trace\_off}, the information
%given in the {\it output window} is
%
%\begin{small}
%\begin{verbatim}
%STEPWISE OBJECT s: stepwise procedure
%
%GENERAL OPTIONS:
%
% Performance criterion: AIC_imp
% Maximal number of iterations: 1000
%
% RESPONSE DISTRIBUTION:
%
% Family: Gaussian
% Number of observations: 18139
%
%OPTIONS FOR STEPWISE PROCEDURE:
%
% OPTIONS FOR FIXED EFFECTS TERM: sex
%
% Prior: diffuse prior
% Startvalue of the 1. startmodel is the fixed effect
%
% OPTIONS FOR NONPARAMETRIC TERM: ageph
%
% Minimal value for the smoothing parameter: 2.0480375
% This is equivalent to degrees of freedom: approximately 16, exact 16.0369
% Maximal value for the smoothing parameter: 62500
% This is equivalent to degrees of freedom: approximately 2, exact 1.95119
% Number of different smoothing parameters with equidistant degrees of freedom: 15
% Startvalue of the 1. startmodel is the fixed effect
%
% OPTIONS FOR NONPARAMETRIC TERM: bm
%
% Minimal value for the smoothing parameter: 1.0240375
% This is equivalent to degrees of freedom: approximately 16, exact 16.0487
% Maximal value for the smoothing parameter: 45000
% This is equivalent to degrees of freedom: approximately 2, exact 2.02502
% Number of different smoothing parameters with equidistant degrees of freedom: 15
% Startvalue of the 1. startmodel is the fixed effect
%
%
%STEPWISE PROCEDURE STARTED
%
% Startmodel:
%
% LOGS = const + sex + ageph + bm
% AIC_imp = 14821.315
%
% ------------------------------------------------------------------------
% ------------------------------------------------------------------------
%
% Final Model:
%
% LOGS = const + ageph(psplinerw2,df=5.96466,(lambda=666.043)) +
%        bm(psplinerw2,df=4.96696,(lambda=1188.99))
% AIC_imp = 14757.465
%
% Used number of iterations: 4
%\end{verbatim}
%\end{small}
%
%The estimation results are stored in several external ASCII-files whose names start with the base filename {\tt car\_}. The file
%{\tt car\_FixedEffects1.res} contains the estimated coefficients for the linear effects in tabular form, including the estimated intercept term and
%coefficients of factor variables.
%The results for linear effects are additionally shown in the {\it output window}.
%For each nonlinear function, e.g.~for $f_{ageph}(ageph)$, there exists one file in form of a data frame,
%here called {\tt car\_f\_ageph\_pspline.res},
%containing the function estimates at all distinct covariate values. The first lines of the file are:
%
%\begin{small}
%\begin{verbatim}
%intnr  ageph    pmean   pqu2p5  pqu10  pmed  pqu90  pqu97p5  pcat95  pcat80
%1       18   -0.0003835   0       0     0      0      0        0       0
%2       19   -0.0226083   0       0     0      0      0        0       0
%3       20   -0.0445334   0       0     0      0      0        0       0
%4       21   -0.0659971   0       0     0      0      0        0       0
%\end{verbatim}
%\end{small}
%
%Column {\it pmean} contains the function estimates. Columns {\it pqu2p5} to {\it pcat80} are only meaningful if
%credible intervals are constructed. In this case, columns {\it pqu2p5} and {\it pqu97p5} build the credible interval
%corresponding to {\tt level1=95}, whereas {\it pqu10} and {\it pqu90} belong to the credible interval with {\tt level2=80}. Columns
%{\it pcat95} and {\it pcat80} indicate wether the credible interval is strictly negative (-1), contains zero (0) or is strictly positive (1)
%with (posterior) probabilities of nominal levels 95\% and 80\%. The first column {\it intnr} is merely a parameter index.
%These results files can be read into any general purpose statistics program
%(e.g.~STATA, R, S-plus) to further analyse and/or visualise the results.
%The names of the respective files are shown in the {\it output window}.
%BayesX has also some facilities for the plotting of nonlinear and spatial functions. The respective
%commands {\tt plotnonp} and {\tt drwamap} are described in the manuals . \\
%Additional to the files containing estimated effects, there are two files containing information about
%the progression of the selection: the file {\tt car\_models.raw} displays the models chosen after every iteration (i.e.~after having passed
%once through all variables and terms). It looks like this:
%
%\begin{small}
%\begin{verbatim}
%step   AIC_imp   model
%
%0   14821.315     LOGS = const + sex + ageph + bm
%
%1   14757.645     LOGS = const + ageph(psplinerw2,df=5.96466,(lambda=666.043)) +
%                         bm(psplinerw2,df=4.96696,(lambda=1188.99))
%...
%3   14757.468     LOGS = const + ageph(psplinerw2,df=5.96466,(lambda=666.043)) +
%                         bm(psplinerw2,df=4.96696,(lambda=1188.99))
%
%4   14757.465
%B   14757.464
%\end{verbatim}
%\end{small}
%%
%%2   14757.485     LOGS = const + AGEPH(psplinerw2,df=5.96466,(lambda=666.043)) + BM(psplinerw2,df=4.96696,(lambda=1188.99))
%%
%In this example, variable {\it sex} has been removed from the model during the first iteration, whereas the effects of {\it ageph}
%and of {\it bm} are modelled by nonlinear effects.
%Column {\it step} shows the number of the current iteration with {\it step=0} indicating the starting model.
%The information {\it step=B} is peculiar to the adaptive search where the final model is estimated by backfitting after the
%selection process is finished what usually changes the value of the selection criterion once more.
%The largest number of {\it steps} indicates the total number of iterations.
%Using this file, it is possible to detect changes in the model that were made during an iteration. Furthermore, it is possible to
%observe changes in the selection criterion. Changes in the selection criterion can also be observed using
%file {\tt car\_criterium.raw}:
%
%\begin{small}
%\begin{verbatim}
%step  var   AIC_imp
%0      0   14821.315
%0      1   14820.56
%0      2   14770.105
%0      3   14757.645
%1      0   14757.645
%1      1   14757.645
%1      2   14757.485
%1      3   14757.485
%...
%4      0   14757.465
%B      0   14757.464
%\end{verbatim}
%\end{small}
%
%This file displays the current value of the selection criterion after the respective covariate or term was updated. Variable {\it step}
%indicates the number of iterations again whereas column {\it var} gives the number of the covariates / terms. In each iteration,
%{\it var=0} indicates the starting model. \\
%If the global option {\tt predict} is specified, BayesX creates a file {\tt car\_predictmean.raw} containing estimates
%for the predictor $\eta_i$ in column {\it linpred}
%and for the conditional expectations of the response $\mu_i$ in column {\it mu}.
%If {\tt CI=MCMCbootstrap} is specified the
%file contains the estimates for the original data in columns {\it linpred} and {\it mu}
%and, additionally, contains average estimates for $\eta_i$ and $\mu_i$ calculated from
%the samples of all selected models (columns {\it average\_linpred} and {\it average\_mu}).
%In this case the first few lines of {\tt car\_predictmean.raw} are given by
%
%\begin{small}
%\begin{verbatim}
%logs    sex ageph bm nclaims linpred average_linpred   mu   average_mu  sat_dev
%11.086   1   50    5   1      9.8551    9.85614      9.8551   9.8561    0.74395
%8.7470  -1   28    9   1      9.9052    9.90306      9.9052   9.9031    0.65828
%8.7470   1   26   11   1      10.016    10.0125      10.016   10.013    0.79044
%\end{verbatim}
%\end{small}
%
%If unconditional confidence bands were constructed by using one of the options {\tt CI=MCMCboostrap}
%or {\tt CI=bootstrap}, BayesX creates one additional results file for each nonlinear term and
%for the linear effects. Those files contain the possible degrees of freedom for the term
%together with the frequency distribution, i.e.~the number of bootstrap samples
%in which the individual degrees of freedom were selected plus the degree of freedom selected for the original data.
%For the P--spline effect of {\it agph} the file is called {\tt car\_f\_ageph\_pspline\_df.res}
%and looks like
%
%\begin{small}
%\begin{verbatim}
%df_value  sp_value  frequency  pmean
%4.04862    3447.83     3        -
%4.99322    1438.86    26        -
%6.03377    632.898    48     selected
%6.98883    325.333    14        -
%7.97817    173.922     1        -
%9.96079    56.4291     2        -
%11.033     32.1334     2        -
%11.9617    19.9828     2        -
%13.0304    11.5881     1        -
%\end{verbatim}
%\end{small}
%
%{\it BayesX} automatically creates a file {\tt car\_model\_summary.tex} summarising the most important results which can be compiled
%using \LaTeX. Among the displayed results are graphics for the nonparametric and spatial effects. These graphics are also created
%automatically and stored in postscript format.
%The effect of {\it ageph}, for example, is contained in file {\tt car\_f\_ageph\_pspline.ps}. \\
%When credible intervals are constructed by using one of the hybrid MCMC methods ({\tt CI=MCMCselect}
%or {\tt CI=MCMCbootstrap}), {\it BayesX} stores the MCMC samples for the regression parameters of linear effects
%and for the nonlinear function evaluations. These samples can be obtained using the post estimation command
%
%{\tt s.getsample}
%
%and used for an analysis of the sampling paths. For further information regarding the
%command {\tt getsample} and the analysis of MCMC output compare the BayesX manuals.
%
%
%\subsection{Specific commands for multinomial logit models}
%
%Finally, we explain the specifics of the commands for multinomial logit models:
%For multinomial logit models, there are two different commands in order to perform a variable and smoothing parameter selection.
%If the data consists of observations with merely one trial per observation, the dependent variable $Y$ is supposed to specify the
%chosen category, e.g. $Y \in \{1,\ldots,k+1\}$. In this case, a selection can be performed using the {\tt regress} command
%like for univariate response variables:
%
%{\tt > s.regress} {\it Y = term$_1$ + term$_2$ +} $\ldots$ {\it + term$_r$}
%$[${\tt if} {\it expression}$]$ $[$, {\it options}$]$ {\tt using d}
%
%Here, an important option is {\tt reference} specifying the category that is to be chosen as reference category.
%A weight variable is not allowed with {\tt regress}. \\
%The second possibility is given by the command {\tt mregress}. Here, it is possible to deal with grouped data with several trials
%per observation. In this case, the command needs $k$ response variables, e.g. $Y_1,\ldots,Y_k$, each specifying the numbers of cases
%in which the respective category was chosen. One category, here $k+1$ serves as reference. The command is
%
%\begin{tabbing}
%{\tt > s.mregress} \= {\it $Y_1$ = term$_{11}$ + term$_{12}$ +} $\ldots$ {\it + term$_{1r}$}: \\
%                   \> {\it $Y_2$ = term$_{21}$ + term$_{22}$ +} $\ldots$ {\it + term$_{2r}$}: \\
%                   \> $\vdots$ \\
%                   \> {\it $Y_k$ = term$_{k1}$ + term$_{k2}$ +} $\ldots$ {\it + term$_{kr}$} \\
%$[${\tt weight} {\it weightvar}$]$ $[${\tt if} {\it expression}$]$ $[$, {\it options}$]$ {\tt using d}
%\end{tabbing}
%
%The weight variable defines the number of trials per observation. The command {\tt mregress} assumes the same fixed effects
%for each of the categories and, regarding all other effects, it requires the same number of terms for all categories but not
%necessarily the same terms.
%The global and local options are the same as for the {\tt regress} command and local options can be individually specified
%for each term and category. \\
%With both commands, BayesX creates one results file for each nonlinear term (in every category) containing the estimated
%effects like in the univariate case. For the linear effects, there exists one results file per category containing all
%respective parameter estimates. The names of results files for the first category
%are identical to the names used for univariate response models, e.g.~{\tt s\_f\_varname\_pspline.res}
%for the P--sline effect of variable {\it varname}. For the $j$--th category with $j=2,\ldots,k$ the
%names additionally contain number $j$ and, thus, the P--spline effect of
%variable {\it varname} is stored in {\tt s\_f\_varname\_j\_pspline.res}.
%
%
%
%
%
%
%
%%
