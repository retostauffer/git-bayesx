\chapter{DAG Objects}
\index{model selection} \index{dag objects} \label{dag}

{\em Author: Eva--Maria Fronk} \\
{\em email: \href{mailto:fronk@stat.uni-muenchen.de}{fronk@stat.uni-muenchen.de}}\\
\vspace{0.3cm}



%%%%%%%%%%%%%%%%%%%%%%%%%%%%%%%%%%%%%%%%%%%%%%%%%%%%%%%%%%%%%%%%%%%%%%%%%%%%%%%%%%%%%%%%%%%%%%%%%%%%%%%%%%%%%%%%%%%%%%%
Dag objects are needed to estimate dag models using reversible
jump MCMC. The considered variables may be Gaussian or binary,
even the mixed case of a conditional Gaussian distribution is
possible. A general introduction into graphical models can be
found in Lauritzen (1996). For a description of the more
particular Gaussian dags see for instance Geiger and Heckerman
(1994). We refer to Brooks (1998) or Gilks (1996) for an
introduction into MCMC simulation techniques. For the more general
reversible jump MCMC have a look at Green (1995); for reversible
jump MCMC in context of graphical models at Giudici and Green
(1999). The following explanations to the statistical background
of the program can be found in more detail in Fronk and Giudici
(2000).



\section{Method estimate}



\subsection{Description}

The method #estimate# estimates the dependency structure of the
given variable which is represented by a dag. Furthermore, the
parameters of this model are estimated. This is done within a
Bayesian framework; we assume prior distributions for the unknown
parameters and use MCMC techniques for estimation. In the
following we first focus on the Gaussian case  and describe the
statistical model which is assumed for the variables. Some
factorizations which result from the properties of dags are also
given. To represent the dags we rely on the concept of adjacency
matrices which is briefly explained and necessary to understand
the output. We finally give some brief information about the used
algorithm without going into details. Finally, we address the
situation of binary and mixed (i.e.~continuous and binary)
variables, too, which is reduced to the Gaussian case by
introducing latent variables.

\subsubsection*{Model Assumptions }
\index{dag object!assumptions}

A Gaussian dag $d$ can be represented as a regression model for
each variable $X_i, \, i=0, \dots, p-1$, given the parents of
$X_i$, denoted by $\X_{pa(i)}$,
%
\begin{eqnarray}
X_i \mid \x_{pa(i)},  \mbeta_{i \mid pa(i)}, \sigma_{i \mid
pa(i)}^2, d \sim  \N(\beta_{i0} + \sum_{x_l \in pa(x_i)}
\beta_{il}x_l, \sigma_{i \mid pa(i)}^2). \nonumber
\end{eqnarray}
%
The joint distribution of all variables ${\bf X} = (X_0, \dots,
X_{p-1})'$ is then given by
%
\begin{eqnarray}
%p(\x \mid  \mmu, \mSigma) =
p(\x \mid  \mbeta, \msigma^2) = \prod_{i=0}^{p-1} p(x_i \mid
\x_{pa(i)}, \mbeta_{i \mid pa(i)}, \sigma_{i \mid pa(i)}^2),
\nonumber
\end{eqnarray}
%
where $\mbeta_{i \mid pa(i)}$ is the $|pa(i)|+1$-dimensional
vector of the intercept $\beta_{i0}$ and the $|pa(i)|$ regression
coefficients of $X_i$. Furthermore, $\sigma_{i \mid pa(i)}^2$ is
the partial variance of $X_i$ given its parents $\x_{pa(i)}$. Let
$\mbeta = (\mbeta'_{0 \mid pa(1)}, \dots, \mbeta'_{p-1 \mid
pa(p)})'$ denote the vector of the $\mbeta_{i \mid pa(i)}$'s and
accordingly $\msigma^2 = (\sigma_{0 \mid pa(1)}^2, \dots,
\sigma_{p-1 \mid pa(p)}^2)'$ the vector of the conditional
variances $\sigma_{i \mid pa(i)}^2$.

\bigskip

The vector $\mbeta_{i \mid pa(i)}$ is assumed to be normally
distributed with mean $\mb_{i \mid pa(i)}$ and covariance matrix $
\frac{1}{\alpha_i} \sigma_{i \mid pa(i)}^2 \I $, where $\alpha_i$
is a known scaling factor. For the sake of simplicity, we shall
assume $\alpha_i=\alpha$. Formally:
%
$$\mbeta_{i \mid pa(i)} \mid \sigma_{i \mid pa(i)}^2, d   \sim
\N_{|pa(i)|+1} \left ( \mb_{i \mid pa(i)}, \,\frac{1}{\alpha}
\sigma_{i \mid pa(i)}^2 \I  \right ).$$
%
This implies that the coefficients of a regression model are
assumed to be mutually independent. For the partial variance
$\sigma_{i \mid pa(i)}^2$ we use an inverse gamma prior with
parameters $\delta_{i \mid pa(i)}$ and $\lambda_{i \mid pa(i)}$:
%
$$\sigma_{i \mid pa(i)}^2 \mid  d \sim \mbox{IG} \left ( \delta_{i
\mid pa(i)}, \,\lambda_{i \mid pa(i)} \right).$$
%
Finally, by supposing that there exist $D$ possible dags, which,
in the absence of subject-matter information, have all the same
probability, we get a discrete uniform distribution for $d$: $p(d)
=  1/D.$ Taking advantage of the well-known factorization property
of the joint distribution
\begin{eqnarray}
p(\x \mid \mbeta, \msigma^2, d) &=&  \prod_{i=0}^{p-1} p(x_i \mid
\x_{pa(i)}, \mbeta_{i \mid pa(i)}, \sigma_{i\mid pa(i)}^2)
\nonumber
\end{eqnarray}
and the "global parameter independences"
\begin{eqnarray}
p(\mbeta \mid  \msigma^2, d) &=& \prod_{i=0}^{p-1} p(\mbeta_{i
\mid pa(i)} \mid \msigma_{i \mid pa(i)}^2),  \nonumber \\ [0.1cm]
\mbox{and \hspace{1cm}} p( \msigma^2 \mid d) &=& \prod_{i=0}^{p-1}
p(\sigma_{i \mid pa(i)}^2) \nonumber
\end{eqnarray}
(for a detailed description see Geiger and Heckerman, 1999), we
get the joint distribution:
%
\begin{eqnarray}
\lefteqn{ p(\x, \mbeta, \msigma^2, d)  }  \nonumber \\  [0.3cm]
&=&   p(\x \mid \mbeta, \msigma^2, d) \,p(\mbeta \mid \msigma^2, d)  \, p( \msigma^2 \mid d)  \, p(d) \nonumber\\
&=&    \prod_{i=0}^{p-1} p( \x_i \mid \x_{pa(i)}, \mbeta_{i \mid
pa(i)}, \sigma^2_{i \mid pa(i)})
    \, \prod_{i=0}^{p-1} p(\mbeta_{i \mid pa(i)} \mid \sigma_{i \mid pa(i)}^2)  \nonumber  \\
& &  \, \prod_{i=0}^{p-1} p(\sigma_{i \mid pa(i)}^2) \, p(d)
\nonumber
\end{eqnarray}

%joint distribution
%Taking advantage of the well-known factorization property of the
%joint distribution in (\ref{modularity_1}) and the "global
%parameter independence" in (\ref{modularity_2}) and
%(\ref{modularity_3}) (for a detailed description see Geiger and
%Heckerman, 1999):
%
%\begin{eqnarray}
%p(\x \mid \mbeta, \msigma^2, d)
%&=&  \prod_{i=1}^p p(x_i \mid \x_{pa(i)}, \mbeta_{i \mid pa(i)}, \sigma_{i\mid pa(i)}^2), \label{modularity_1}  \\ [0.1cm]
%p(\mbeta \mid  \msigma^2, d)
%&=& \prod_{i=1}^p p(\mbeta_{i \mid pa(i)} \mid \msigma_{i \mid pa(i)}^2), \label{modularity_2}   \\ [0.1cm]
%p( \msigma^2 \mid d) &=&  \prod_{i=1}^p p(\sigma_{i \mid pa(i)}^2). \label{modularity_3}
%\end{eqnarray}

\newpage

\subsubsection*{Representation of DAGs }
\index{dag object!representation}

To represent dags we rely on the concept of adjacency matrices.
For a given graph ${\cal G} =(V,E)$ with $|V|=p$, the adjacency
matrix of ${\cal G}$ is defined as the $(p \times p)$-matrix $A$,
$[A]_{ij}=a_{ij}$, with
%
\begin{eqnarray} a_{ij} = \left \{
\begin{array} {l}
            1, \mbox{ if } (v_i,v_j) \in E \hspace*{1cm} \\
            0, \mbox{ if } (v_i,v_j) \not\in E. \hspace*{1cm}
            \end{array} \right. \nonumber
\end{eqnarray}
%
In general, all three types of graphs (undirected, directed and
chain graphs) can be uniquely represented by the corresponding
adjacency matrix. Note that regarding dags, as we do, the parents
of the vertex $i$ are indicated by the $i$-th column, while its
children are given in the $i$-th row. We use the representation
via adjacency matrices also to check the acyclicity of the graph.
For an illustration of this concept consider the graph in
\autoref{adja}.
%
%
\begin{figure}[ht]
\renewcommand{\baselinestretch}{1.0}
{\small \hspace*{2cm}
\parbox {3cm}
{ \setlength{\unitlength}{1cm}
\begin{picture}(3, 5)
\put(0.5,3.5) {\circle*{0.2}}
\put(0.6,3.8){\makebox(0,0)[l]{$x_1$}} \put(0.5,1.5)
{\circle*{0.2}} \put(0.6,1.4){\makebox(0,0)[tl] {$x_2$}}
\put(2.5,1.5){\circle*{0.2}} \put(2.6,1.4){\makebox(0,0)[tl]
{$x_3$}} \put(2.5,3.5) {\circle*{0.2}}
\put(2.6,3.8){\makebox(0,0)[l] {$x_4$}}
\put(0.5,3.3) {\vector(0,-1){1.6}}      %von x_1 nach x_2
\put(0.7,1.5) {\vector(1,0){1.6}}       %von x_2 nach x_3
\put(0.7,3.3) {\vector(1,-1){1.6}}      %von x_1 nach x_3
\put(2.5,1.7)   {\vector(0,1){1.6}}     %von x_3 nach x_4
\end{picture}
} \hspace{3cm}
\parbox {6cm}
{ $ A = \left( \begin{array} {cccc}
0&  1&  0&  0   \\
0&  0&  1&  0   \\
1&  0&  0&  1   \\
0&  0&  0&  0
\end{array}
\right )$
 }
\vspace{-1cm} {\em\caption{\label{adja}A directed acyclic graph
containing and the corresponding adjacency matrix $A$.}}}
\end{figure}

\subsubsection*{Reversible Jump Algorithm for Continuous Variables}
\index{model selection! Gaussian variables}

We are not only interested in estimating the parameters for a
given dag $d$ but also want to learn about the structure of $d$
itself. So we need to construct a Markov chain which has $
\pi(d,\mmu, \mbeta, \msigma^2\mid \x)$ as its invariant
distribution. Changing the dag like adding or deleting a directed
edge implies also a changing in the dimension of the parameter
space. To deal with this situation we use a reversible jump
algorithm. Reversible jump MCMC was proposed and described by
Green (1995); it can be regarded as a generalization of the usual
MCMC and allows to sample simultaneously from parameter spaces of
different dimensions.

Our algorithm can be briefly summarized by the following moves,
which produce a Markov chain in the state space that is made up by
the vector of unknowns $(d,\mbeta,\msigma^2)$:
%
\begin{enumerate}
\item Updating the dag $d$ by adding, switching or deleting a
directed edge, remaining always in the class of directed acyclic
graphs. When adding or deleting an edge this move involves a
change in dimensionality of the parameter space.
\item Update  $\mbeta_{i\mid pa(i)}$, $i=0, \dots, p-1$.
\item Update $\sigma^2_{i\mid pa(i)}$, $i=0, \dots, p-1$.
\end{enumerate}
%
For a detailed explanation of the different steps in the
continuous case, see Fronk and Giudici (2000). For a
simplification of the oftentimes crucial switch step, see Fronk
(2002) and also the explanations of the option {\it switch} in
\autoref{switch_step}.

\newpage

\subsubsection*{Reversible Jump Algorithm for Binary Variables}
\index{model selection! Binary variables}

Now we consider the situation of $p$ binary variables of which the
joint distribution is assumed to be multinomial. The influence to
a variable $X_i$ from its known parents $\x_{pa(i)}$ shall be
given by a probit model, i.e.~
\begin{eqnarray} \label{probit}
p_{i} \,=\,  E(X_{i} \mid \x_{pa(i)})
      \,=\,  \Phi (  \x_{pa(i)}'\mbeta_{i \mid pa(i)}, \sigma_{i \mid pa(i)}^2),
\end{eqnarray}
where $i=1, \dots , p$ and  $\Phi (\mu ,\sigma^2)$ denotes the cdf
of the normal distribution. This binary situation is reduced to
the continuous one by sampling a latent variable, a so-called
utility, $Z_i$ for each binary variable $X_i$. The general idea is
found in Albert and Chib (1993). Here, we first focus on the
situation that we are only interested in the main effects. I.e.~we
do not take any interactions into account although they now of
course can occur as we do not longer consider the Gaussian case.
The algorithm, which does not account for interactions, can be
briefly summarized as:
%
\begin{enumerate}
\item  For $X_i$, $i=0, \dots, p-1$, draw $Z_{i}$ from its full conditional
        $\N(\x_{pa(i)}' \mbeta_{i \mid pa(i)},1)$, which is truncated at the left by 0 if $x_{i}=1$
        and at the  right if $x_{i}=0$.
\item Add, delete, or switch a directed edge like in the Gaussian case, but take the utility $Z_i$ instead of
$X_i$ as response in $i$th regression model; the covariables
$\x_{pa(i)}$ of the $i$th model remain unchanged.
\item Update  $\mbeta_{i\mid pa(i)}$, $i=0, \dots, p-1$.
\item Update $\sigma^2_{i\mid pa(i)}$, $i=0, \dots, p-1$.
\end{enumerate}

To be able to take interactions into account, inside the algorithm
interactions are treated as own variables. Due to the enormous
complexity we restrict ourselves to two way interactions which
seem sufficient for most situations in practice. For details, see
Fronk (2002).

\subsubsection*{Reversible Jump Algorithm for Mixed Case}
\index{model selection! Mixed case}

For the mixed case, we assume the considered continuous and binary
variables to follow a conditional Gaussian (CG) distribution. For
a general introduction we refer to Lauritzen (1996). The
univariate conditioned distribution of $f(x_i \mid \x_{pa(i)})$
are then CG regressions (see Lauritzen and Wermuth, 1989) and can
be represented by a normal regression resp.~a probit model with
mixed covariables.
\begin{enumerate}
\item For all variables $X_i$, $i=0, \dots, p-1$,
    \begin{enumerate}
       \item[] If $X_i$ is discrete,
            \begin{enumerate}
            \item[] For all observations $X_{ki}$, $k=1, \dots, n$,  \\
                    draw utility $Z_{ki}$ from full conditional $Z_{ki} \mid x_{ki}, \x_{k pa(i)} \mbeta_{i \mid pa(i)}$
            \end{enumerate}
    \end{enumerate}
\item Update $d$, i.e.~cancel, add, or switch the directed edge $X_j \rightarrow X_i$;
      thereby distinguish
    \begin{itemize}
    \item Response $X_i$ is continuous:
    \begin{enumerate}
     \item Take the algorithm for the Gaussian case, where now the covariables
            $pa(X_i)$ can be continuous or binary
     \end{enumerate}
    \item Response $X_i$ is discrete
     \begin{enumerate}
     \item Replace binary response $X_i$ by continuous utility $Z_i$
     \item Consider the new or vanishing interactions among the parents of
     $X_i$ and possibly~$X_j$, that can be pairwise discrete or mixed
     \item Carry out birth, death or switch step
     \end{enumerate}
    \end{itemize}
\item Update $\mbeta_{i\mid pa(i)}$, $i=0, \dots, p-1$.
\item Update $\sigma^2_{i\mid pa(i)}$, $i=0, \dots, p-1$.
\end{enumerate}
%
For detailed explanations, see again Fronk (2002).

\subsubsection*{Remark about Markov-equivalence}

Our algorithm does not take care about the so-called Markov
equivalence, which describes the fact that different dags can
represent the same statistical model. Equivalent dags can be
summarized to equivalent classes which again can be represented by
one single graph, the essential graph. Of course model selection
could be done in a more effective way if only the space of those
essential graphs would be considered. This will be a task of our
research in future. For more details concerning Markov-equivalence
we refer to papers of Andersson et al. (1997a, b) and Chickering
(1995).


%%%%%%%%%%%%%%%%%%%%%%%%%%%%%%%%%%%%%%%%%%%%%%%%%%%%%%%%%%%%%%%%%%%%%%%%%%%%%%%%%%%%%%%%%%%%%%%%%%%%%%%%%%%%%%%%%%%%%%%


%{\bf  \large Syntax}\\[0.3cm]

\subsection{Syntax}
%
\index{dag object!estimate command} \index{dag object!create}
%
The creation of objects has been described in general in
\autoref{createobject}; in the context of dag models the
corresponding dag object is created by:

\begin{center}
{\texttt{dag} {\em objectname}},\\
\end{center}

To perform a model selection as described above call:

\begin{center}
{{\em objectname}.\texttt{estimate} {\em variables} [\texttt{if}
{\em expression}], [{\em options}]\texttt{ using} {\em dataset} }
\bigskip
\end{center}

Then the method #estimate# estimates the dag for the variables
given in {\it variables} which have to be defined in {\it
dataset}. The parameters of the via the dag defined regression
models are also estimated. An if-statement may be specified to
analyze only a part of the data set, i.e.~only those observations
where {\it expression} is true. There are several facultative {\it
options} concerning the (start) parameters of the algorithm or the
kind of output at the end. They are listed in the next paragraphs.

%{\bf  \large Options}\\[0.3cm]
\subsection{Options}

\subsubsection*{Options for controlling MCMC simulations}

The following options correspond to those given on
\hyperref[mcmc_options]{page~\pageref*{mcmc_options}} and are
therefore only briefly explained.
\begin{itemize}
\item #burnin = #{\em b} \\
Changes the number of burnin iterations from 2000 to {\em b}; it is a positive integer number with $0<b<500001$.\\
DEFAULT:# burnin = 2000#
\item #iterations = #{\em i} \\
Changes the number of MCMC iterations from 52000 to {\em i}; it is a positive integer number with $0< i<10000000$.\\
DEFAULT: #iterations = 52000#
\item #step = #{\em s} \\
Changes the thinning parameter of MCMC iterations from 50 to {\em s}; it is a positive integer number with $0<s<1000$.\\
DEFAULT: #step = 50#
\end{itemize}

\subsubsection*{Options for initial values of algorithm}
\begin{itemize}
%\item priori\_sigma
\item {\bf\sffamily Changing hyperparameters of partial variances} \\
As already mentioned we assume $\sigma_{i \mid pa(i)}^2 \sim IG
(\delta_{i \mid pa(i)},\lambda_{i \mid pa(i)} )$ for $i=0, \dots,
p-1$. By the following two commands the values of the two
hyperparameters $\delta_{i \mid pa(i)}$ and $\lambda_{i \mid
pa(i)}$ can be freely chosen. If this is not done the default
values correspond to a non-informative gamma distribution.
\begin{itemize}
\item #delta = #{\em c} \\
    Specifies the first parameter of the inverse gamma distribution of the partial
    variances, $\delta_{i \mid pa(i)}$, is set equal to $d$. Otherwise it is equal to 1. The value
    {\em c} has to be of type realvalue with $0< c<20$.\\
    DEFAULT: #delta =  1#
\item #lambda = #{\em l}\\
    Specifies the second parameter of the inverse gamma distribution of the partial
    variances, $\lambda_{i \mid pa(i)}$, is set equal to $l$. Otherwise it is equal to 0.005.
    The value {\em d} has to be of type realvalue with $0< d <20$.\\
    DEFAULT: #lambda = 0.005#
\end{itemize}
\item {\bf\sffamily Choosing special graph to start from} \\
Usually the algorithm starts from the independent model, that
means from a dag without any edges. This can be changed by the
command
\begin{center}
#type = #0/1/2/3/4\\
DEFAULT: #type = 0#
\end{center}
where the different values have the following meanings:
\begin{itemize}
\item #type=0# \\
    Algorithm starts from an {\bf independent} model with no edges.
\item #type=1# \\
    Algorithm starts from a {\bf complete} model where all edges are directed
    from "lower" variables to "higher" ones, i.e.~$x_i \rightarrow x_j, \, \forall i<j$.
\item #type=2# \\
    Algorithm starts from a {\bf complete} model where all edges are directed
    from "higher" variables to "lower" ones, i.e.~$x_j \rightarrow x_i, \, \forall i<j$.
\item #type=3# \\
    Algorithm starts from a model where there is an edge from each variable to
    the next "higher" one, like a {\bf chain}, i.e.~$x_i \rightarrow x_j, \, \forall i=j+1$.
\item #type=4# \\
    Algorithm starts from a model where there is an edge from each variable to
    the next "lower" one, like a {\bf chain}, i.e.~$x_j \rightarrow x_i, \, \forall i=j+1$.
\end{itemize}
\end{itemize}

\subsubsection*{Options concerning the way of model selection}

\begin{itemize}
\item {\bf\sffamily Kind of switch step} \label{switch_step} \\
There are three ways how the switch step can be carried out in the
rj-algorithm. The first one is similar to the performance of a
birth or death step (i.e.~adding or deleting an edge): A proposal
is made and then accepted by its corresponding acceptance ratio.
As it may be very complicate to calculate a good proposal, a
simplification can be achieved by the consideration if the switch
step leads to an equivalent dag model. If this holds true, the
given and the proposed dag should be statistically
indistinguishable. The proposed dag can therefore be accepted with
probability 0.5. The kind of switch step can be chosen by the
command
\begin{center}
    #switch = normal#/#equi#/#mix#, \\
    \hspace*{-1.8cm}  DEFAULT: #switch = normal#
\end{center}
which differ in the following way:
\begin{itemize}
\item #switch=normal#\\
The switch step is carried out by proposing the new dag and
accepting it with the corresponding acceptance probability. The
transformation into equivalent model may occur very seldom and,
consequently, the acceptance ratio very low.
\item #switch=equi# \\
The switch step is only allowed if it results into an equivalent
model. In this case, it is performed with a probability of 0.5.
Transformations into a non-equivalent model can only occur by a
birth or death step.
\item #switch=mix#\\
This command causes a mixture of both procedures described above:
If the proposed switch step leads to an equivalent model it is
accepted with probability 0.5. If it results into a non-equivalent
model a proposal is made and accepted by the corresponding
acceptance ratio.
\end{itemize}
\item {\bf\sffamily Kind of distribution family / interactions }\\
There are three different types of data sets as they can consists
of continuous, binary, or mixed variables which results in the
assumption of a Gaussian, a multinomial, or a conditional Gaussian
distribution. Dependent on the kind of data set the rj-algorithm
for the model selection changes as described above.This can be
indicated by the optional command
\begin{center}
    #family = continuous#/#discrete#/#mixed#.\\
\end{center}
In the case that the model selection for a binary data set shall
be carried out accounting for interactions the command
\begin{center}
    #family = discrete_ia#\\
\end{center}
is needed instead of  #family = discrete#. In this case, a special
option concerning the output is given by the command #detail_ia#
which is explained below.
\item {\bf\sffamily Restriction to the search space}\\
It is possible to restrict the search space, i.e.~to state an
(missing) edge as fix or determinate the orientation of an edge.
This is done by writing the restrictions into a file {\em
restrict} which is then read by the command
\begin{center}
    #fix_file = #{\em path\_of\_restrict}\\
\end{center}
The restriction is then given by a $p \times p$ matrix that lies
under the path {\em path\_of\_restrict}. The matrix is allowed to
have three possible entries, namely  0,1, and 2 which have the
following meaning: An entry of 2 corresponds to no restriction of
the corresponding edge, it may occur or not. An entry of 1
indicates that this edge has to exist in each graph of the Markov
chain, whereas 0 denotes that the corresponding edge must not
occur.
\end{itemize}

\subsubsection*{Options concerning the output}
\begin{itemize}
\item {\bf\sffamily Estimated regression coefficients}\\
As already mentioned, the parameters of each regression model are
estimated in every iteration. Because of the fact that the
qualitative structure of the dag is usually of greater interest
than the quantitative estimations of the regression coefficients,
in the standard output these estimated parameters are omitted.
Nevertheless
\begin{center}
#print_dags#
\end{center}
gives the mean, the 10\%, 50\% and the 90\% quantile of every
parameter of all regression models.
%averaged over {\bf all} dags of the Markov chain.
As the model space for dags is very huge we abandon the
possibility to store the estimated values for each dag. To perform
the necessary calculation {\em BayesX} creates a temporary file
under the device {\em c:$\setminus\setminus$...}. For this
purpose, its important to ensure that a device with this name
exists. Otherwise the user has to provide an alternative path for
the storage file by the command
\begin{center}
#store_file = #{\em alternative\_path}
\end{center}
\item {\bf\sffamily Estimated coefficients of interactions}
\end{itemize}

\begin{itemize}
\item {\bf\sffamily Criteria for the listed models} \\
    As model selection for dags is performed in an extremely huge search space one might not want
    to get a list of all models which have been visited during MCMC estimation regardless of the
    relative frequency of their appearance. The option
    \begin{center}
    #print_models = all#/#prob#/#limit#/#normal#, \\
    \hspace*{-1.8cm}  DEFAULT: #print_models = normal#
    \end{center}
    %where "all", "prob" and  "limit" are strings,
    allows to focus
    on special criterions for the models printed in the output.
\begin{itemize}
\item #print_models = all#  \\
    All models which have been visited by the Markov chain are printed.
\item #print_models = prob# \\
    The most frequent models of the chain are printed except for those which are the less frequent
    ones and have altogether a probability of alpha=0.05. The value of alpha can be changed as
    it is explained a few lines below.
\item #print_models = limit# \\
    Here, the first 10 models with the highest frequencies are printed. The number of listed models
    can be made different from 10 as it is explained a few lines below.
\item #print_models = normal# \\
    The option #normal# is a mixture of #limit# and #prob#, as it chooses the one
    which produces less models. The default parameters are again alpha=0.05 and number=10.
\end{itemize}
The default setting of  #print_models# is #print_models=normal#.
\item #number = #{\em n} \\
    Changes the number of printed models in the option  #print_models = limit# to $n$.
    The variable number has to be of type realvalue with $0 \leq n \leq 10000$. \\
    DEFAULT = 10
\item #alpha = #{\em a} \\
    Sets alpha in the option {\tt print\_models = prob} equal to $a$. That means when using
    {\tt print\_models = prob} the most frequent models which unify 1-a  of the posterior
    probability are printed. The variable alpha has to be of type intvalue with $a \in [0,1]$. \\
    DEFAULT = 0.05
\item #printit = #{\em p} \\
    Prints every {\em p}-th iteration in the output window instead of every 100-th. The printing of the
    iterations can be suppressed by setting $p$ higher than the number of iterations. The variable printit has to be of
    type intvalue with $0 < p< 10000001$. \\
    DEFAULT = 100
\end{itemize}

%%%%%%%%%%%%%%%%%%%%%%%%%%%%%%%%%%%%%%%%%%%%%%%%%%%%%%%%%%%%%%%%%%%%%%%%%%%%%%%%%%%%%%%%%%%%%%%%%%%%%%%%%%%%%%%%%%%%%%%


%{\bf  \large Estimation Output}\\[0.3cm]
\subsection{Estimation Output}

The output can be written to a file by opening a logfile before
using the estimation command. It has to be closed afterwards. The
use of logfiles is described in detail in \autoref{logfile}. The
output itself is structured as follows:
\begin{itemize}
\item {\bf\sffamily Listing of different dags:}\\
The different models are listed by their adjacency matrices. In
order to save space, the different rows are printed in one line
with a blank indicating the beginning of a new one. The number of
edges is given as well as the absolute and relative frequency of
the model. For example the first line of the exemplifying output
in \autoref{output}
%
\begin{figure}
\begin{center}
{\it

Number of different dags visited by the algorithm: 16 \\[0.5cm]
******** DIFFERENT MODELS sorted by frequencies  ******** \\
********************** all models ********************** \\[0.2cm]

\begin{tabular}{llllll}
010 \hspace{0.2cm}& 000 \hspace{0.2cm}& 000 \hspace{0.5cm}&    1  \hspace{0.5cm}& 3894  \hspace{0.5cm}&  0.3890 \\
000 \hspace{0.2cm}& 100 \hspace{0.2cm}& 000 \hspace{0.5cm}&    1  \hspace{0.5cm}& 3814  \hspace{0.5cm}&  0.3810 \\
000 \hspace{0.2cm}& 100 \hspace{0.2cm}& 100 \hspace{0.5cm}&    2
\hspace{0.5cm}& \,\,806   \hspace{0.5cm}&  0.0806
\end{tabular}
}


$\vdots$ \\[-0.15cm]
$\vdots$

\end{center}
{\em\caption{\label{output} Example for model listing in the
output.}}
\end{figure}
%
gives the information that the most frequent model is represented
by the adjacency matrix $ A = \left( \begin{array} {cccc}
0&1&0   \\
0&0&0   \\
0&0&0
\end{array}
\right )$ and contains 1 edge. It occurred in the thinned out
Markov chain for 3894 times which corresponds to a relative
frequency of about 0.389. Notice, that different dags may
represent the same statistical model as they may be
Markov-equivalent.
%
%
\item {\bf\sffamily Listing of essential graphs}\\
\begin{figure}
%\begin{center}
{\it

Number of different equivalent classes visited by the algorithm: 6 \\[0.5cm]
******* DIFFERENT EQUIVALENCE CLASSES sorted by frequencies  ********\\
**************************** all models ******************************** \\[0.5cm]

Skeleton: 010 000 000\\
No immoralities.\\
Number of edges: 1 \hspace{0.3cm} Abs.freq.: 7708 \hspace{0.3cm} Rel.freq.: 0.771 \\[0.3cm]

Skeleton: 011 000 000 \\
Immoralities: (0;1,2) \\
Number of edges: 2 \hspace{0.3cm} Abs.freq.: 806 \hspace{0.3cm} Rel.freq.: 0.0806 \\[0.3cm]

Skeleton: 011 000 000 \\
No immoralities. \\
Number of edges: 2 \hspace{0.3cm} Abs.freq.: 523 \hspace{0.3cm} Rel.freq.: 0.0523\\

$\vdots$ \\[-0.15cm]
$\vdots$

}
%\end{center}
{\em\caption{\label{output_ess} Example for listing of equivalence
classes in the output.}}
\end{figure}
Additional to the dags, the essential graphs are printed, too.
I.e.~those dags, which are equivalent to each other, are
summarized and represented by their essential graph. The
representation of the essential graph, which can contain
undirected as well as directed edges, is as follows. First the
underlying graph, the skeleton, is given by the adjacency matrix
as described above. But now, the entries indicate always an
undirected edge. (E.g.~the undirected graph $a$--$b$ of the two
variables $a$ and $b$ is given by 01 00.) Then the immoralities of
the essential graph are listed. Remember that within an essential
graph an oriented edge can only occur as a part of an immorality
$b \rightarrow a \leftarrow c$ which is here represented by the
triple (a;b,c). The example output of Figure \autoref{output_ess}
shows that the first two dag models of \autoref{output} which are
equivalent have been summarized to their representing essential
graph $X_0$--$X_1$ $X_2$. The next most frequent statistical model
is represented by the essential graph $X_0 \rightarrow X_2
\leftarrow X_1$ which is given by our representation as the
skeleton matrix 011 000 000 and the immorality (0;1,2).
%
%
\item {\bf\sffamily Averaged adjacency matrix:}\\
The $(i,j)$-th element of the averaged adjacency matrix gives the
estimated posterior probability of the presence of the edge $i
\rightarrow j$ in the true dag.
\item {\bf\sffamily Mean of skeletons:}\\
The skeleton of a dag is defined as the same graph without
regarding the directions of the edges. Equivalent dags have at
least the same skeleton. So it may be helpful to have also a look
at the averaged matrix of the skeletons, which is of course
symmetric.
\item {\bf\sffamily Correlation:}\\
The marginal and the partial correlation matrices of the regarded
data set is given, too.
\item {\bf\sffamily Ratios:}\\
We give some short information about the acceptance ratios for the
birth-, death- and switch-steps which denotes the cases where an
edge is added, dropped or switched. The first two cases imply a
change in dimension and are therefore sampled by a reversible jump
step.
\item {\bf\sffamily Estimated parameters:}\\
If the option {\tt print\_dags} is used, the estimated regression
coefficients $\mbeta_{i \mid -i}$, $i=0, \dots, p-1$, are listed
at the end. The notation $-i$ denotes all variables except for
$i$. Besides the mean of the sampled Markov chain for each
parameter there is also the 10\%, the 50\% and the 90\% quantile
given. As in equivalent models the direction of edges and, thus,
also the regression models vary, in most cases the estimated
regression coefficients do not give a deeper insight into the
model and have to be interpreted in a very careful way.
%\item {\bf Estimated interactions:}\\
\end{itemize}

%%%%%%%%%%%%%%%%%%%%%%%%%%%%%%%%%%%%%%%%%%%%%%%%%%%%%%%%%%%%%%%%%%%%%%%%%%%%%%%%%%%%%%%%%%%%%%%%%%%%%%%%%%%%%%%%%%%%%%%
\subsection{References}

\begin{description}

{\small

\item [Andersson, S.~A., Madigan, D., and Perlman, M.~D.~(1997a).] A
Characterization of Markov equivalence Classes for Acyclic
Digraphs. {\it The Annals of Statistics}, {\bf 25}, 505--541.
%
\item[Andersson, S.~A., Madigan, D., and Perlman, M.~D.~(1997b).] On the
Markov equivalence of Chain Graphs, Undirected Graphs, and Acyclic
Digraphs. { \it Scandinavian Journal of Statistics}, {\bf  24},
81--102.

\item[Chickering, D.~M.~(1995).] A Transformational Characterization of Equivalent Bayesian Network Structures.
In P.~Besnard and S.~Hanks (Eds.), {\it Uncertainty in Artificial
Intelligence, Proceedings of the Eleventh Conference}, San
Francisco: Morgan Kaufmann, pp. 87 -- 98
%
\item[Brooks, S.~P.~(1998).] Markov Chain Monte Carlo Method and its Application. {\it The Statistician},
{\bf 47}, 69--100.
%
\item[Fronk, E.--M. and Giudici, P. (2000).] Markov Chain Monte Carlo model selection for DAG Models.
       {\it Discussion Paper}, Universit{\"a}t M{\"u}nchen.
%
%\item{Gelman, A., Carlin, J.~B., Stern, H.~S., and Rubin,
%D.~B.~(1995).} {\it Bayesian Data Analysis}, Chapman and Hall,
%London.
%
%
\item[Geiger, D. and Heckerman, D.~(1994).] Learning Gaussian Networks.
{\it Proceedings of the Tenth Conference on Uncertainty in
Artificial Intelligence.} Morgan Kaufmann, 235--243.
%
\item[Geiger, D. and Heckerman, D.~(1999).] Parameter priors for
directed acyclic graphical models and the characterisation of
several probability distributions. Submitted for publication.
%
\item [Gilks, W.~R., Richardson, S., and Spiegelhalter, D.~J. (1996).]
        {\it Markov Chain Monte Carlo in Practice.},Chapman and Hall, London.
%
\item[Giudici, P.~and Green, P.~J.~(1999).] Decomposable Graphical
Gaussian Model Determination. {\it Biometrika}, {\bf 86},
785--801.
%
%\item{Giudici, P., Green, P.~J., and Tarantola, C.~(1999).} Efficient
%Model Determination for Discrete Graphical Models. Submitted for
%publication.
%
\item [Green, P.~J.~(1995).]
Reversible Jump Markov Chain Monte Carlo Computation and Bayesian
Model Determination. {\it Biometrika}, {\bf  82}, 711--32.
%
\item[Lauritzen, S.~L.~(1996).] {\it Graphical Models}, Clarendon Press, Oxford.

%\item{Lang, S.~and Brezger, A.~(2000).} BayesX--Software for
%Bayesian Inference Based on Markov Chain Monte Carlo Simulation Techniques.
%Discussion Paper 187,
%Sonderforschungsbereich 386, Ludwig--Maximilians--Universit\"at, M\"unchen, Germany.
%
%\item{Schachter, R.~and Kenley, C.~(1989).} Gaussian Influence Diagrams.
%{\it Management Science}, {\bf 35}, 527--550.
}

\end{description}


%{\bf  \large Example}\\[0.3cm]
%\subsection{Example}
%Blablabla...  ist das notwendig ?????
