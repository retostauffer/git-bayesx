\documentclass[10pt,a4paper]{article}
\usepackage[dvips]{graphicx}
\usepackage{rawfonts}
\usepackage{german}
\sloppy
\parindent0em
\parskip0.2em
\topmargin-1.5cm \textheight24cm \textwidth15cm
\oddsidemargin0.5cm
%\pagestyle{empty}

\renewcommand{\baselinestretch}{2}



\def \re {{\bf R}}
\def \beq {\begin{equation}}
\def \eeq {\end{equation}}
\def \bdis {\begin{displaymath}}
\def \edis {\end{displaymath}}
\def \ds {\displaystyle}

\newcommand{\reell}{{I\hspace{-0.18cm} R}}


\begin{document}



{\bf \Huge TESTDATENS\"{A}TZE}

\tableofcontents

\section{Gaussian response}

\subsection{Eine r\"{a}umliche Funktion, nicht hierarchisch}

{\bf Simulationsfile}

gaussian\_spatial\_1fkt.do

{\bf Datensatz}

gaussian\_spatial\_1fkt.raw

{\bf Modell}

$y1  \sim N(0.4(xcenter+ycenter),0.3^2)$

wobei $xcenter$ und $ycenter$ die (zentrierten) Zentroide der
Regionen sind. Die Zentroide findet man in der Datei
kreisecentroid.raw, das bnd-file unter kreisesim.bnd.

{\bf mcmcreg: batch-file}

gaussian\_spatial\_1fkt\_work.prg

{\bf bayesreg: batch-file}

gaussian\_spatial\_1fkt\_alt.prg

\end{document}
