\documentclass[10pt,a4paper]{article}
\usepackage[dvips]{graphicx}
\usepackage{rawfonts}
\usepackage{german}
\sloppy
\parindent0em
\parskip0.2em
\topmargin-1.5cm \textheight24cm \textwidth15cm
\oddsidemargin0.5cm
%\pagestyle{empty}

\renewcommand{\baselinestretch}{2}



\def \re {{\bf R}}
\def \beq {\begin{equation}}
\def \eeq {\end{equation}}
\def \bdis {\begin{displaymath}}
\def \edis {\end{displaymath}}
\def \ds {\displaystyle}

\newcommand{\reell}{{I\hspace{-0.18cm} R}}


\begin{document}



{\bf \Huge TESTDATENS\"{A}TZE}

\tableofcontents

\section{Gaussian response}

\subsection{Lineare Effekte}

{\bf Simulationsfile}

hgaussian\_linear.do

{\bf Datensatz}

hgaussian\_linear.raw \\[0.2cm]
hgaussian\_linear.dta


{\bf Modell}


$y  \sim N(2+0.8 x1 - 0.9 x2 ,0.7^2)$

$x1 \sim U(-3,3)$

$x2 \sim U(-3,3)$

{\bf mcmcreg: batch-file}

hgaussian\_linear.prg

{\bf bayesreg: batch-file}

hgaussian\_linear\_alt.prg



\subsection{Eine nichtlineare Funktion}

{\bf Simulationsfile}

hgaussian\_nonp\_1fkt.do

kleiner Effekt:

hgaussian\_nonp\_1fkt\_small.do

{\bf Datensatz}

hgaussian\_nonp\_1fkt.raw \\[0.2cm]
hgaussian\_nonp\_1fkt.dta

kleiner Effekt:

hgaussian\_nonp\_1fkt\_small.raw \\[0.2cm]
hgaussian\_nonp\_1fkt\_small.dta


{\bf Modell}

$y  \sim N(sin(x1),0.7^2)$

bzw.

$y  \sim N(0.3 \cdot sin(x1),0.7^2)$

bei kleinem Effekt

$x1 \sim U(-3,3)$


{\bf mcmcreg: batch-file}

hgaussian\_nonp\_1fkt.prg

bzw.

hgaussian\_nonp\_1fkt\_small.prg

bei kleinem Effekt.


{\bf bayesreg: batch-file}

hgaussian\_nonp\_1fkt\_alt.prg

bzw.

hgaussian\_nonp\_1fkt\_alt\_small.prg

bei kleinem Effekt


\subsection{Hierarchisches Modell mit zwei Stufen}

{\bf Simulationsfile}

hgaussian\_nonp.do

{\bf Datensatz}

hgaussian\_nonp\_re.raw \\[0.2cm]
hgaussian\_nonp\_re.dta

und

hgaussian\_nonp.raw \\[0.2cm]
hgaussian\_nonp.dta


{\bf Modell}

$y  \sim N(x1+sin(x2)+re,0.75^2)$

$re \sim N(cos(x3)+log(x4),0.66^2)$


{\bf mcmcreg: batch-file}

hgaussian\_nonp.prg


{\bf bayesreg: batch-file}

hgaussian\_nonp\_alt.prg





\section{Binomial response: Logit}

\subsection{Lineare Effekte}

{\bf Simulationsfile}

hbinomial\_linear.do

{\bf Datensatz}

hbinomial\_linear.raw \\[0.2cm]
hbinomial\_linear.dta


{\bf Modell}


$y  \sim B(1,\pi)$

$\eta = -0.25+0.4 \cdot x1-0.6 \cdot x2$


$x1 \sim U(-3,3)$

$x2 \sim U(-3,3)$

{\bf mcmcreg: batch-file}

hbinomial\_linear.prg

{\bf bayesreg: batch-file}

hbinomial\_linear\_alt.prg






\end{document}
